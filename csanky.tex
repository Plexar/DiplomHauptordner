%
% Datei: csanky.tex (Textteile nach 'Csan74' und 'Csan76')
%

\MyChapter{Die Algorithmen von Csanky}
\label{ChapCsanky}

In diesem Kapitel werden zwei von L. Csanky\footnote{ \cite{Csan74} und 
\cite{Csan76} } vorgeschlagene Algorithmen 
behandelt. Der erste davon verwendet eine relativ bekannte Methode zur
Determinantenberechnung\footnote{Entwicklungssatz von Laplace}. 
Die Effizienz dieser Methode ist jedoch nicht befriedigend. Ihre 
Darstellung ist als Einstieg gedacht.

\MySection{Die Stirling'schen Ungleichungen}

Als Vorarbeit f"ur den ersten Algorithmus werden in diesem Unterkapitel 
weitere Grundlagen behandelt.

\begin{satz}[Stirling'sche Ungleichungen]
\label{SatzStirling}
\index{Stirling'sche Ungleichungen}
    Sei \[ n \in \Nat \] Dann gilt
    \begin{eqnarray}
        \label{Equ1SatzStirling}
        n! & \geq & \sqrt{2\pi n}
                    \left( \frac{n}{\MathE} \right)^n
                    \MathE^{\frac{1}{12n+1}}
    \\  
        \label{Equ2SatzStirling}
        n! & \leq & \sqrt{2\pi n}
                    \left( \frac{n}{\MathE} \right)^n
                    \MathE^{\frac{1}{12n}}
    \end{eqnarray}
\end{satz}
\begin{beweis}
    Die Ungleichungen folgen aus \cite{Mang67} S. 154.
\end{beweis}

Die Ungleichungen werden in den folgenden Lemmata angewendet. Die Richtung
der Ab\-sch"a\-tzung der jeweiligen Werte wird durch deren Verwendung
im sp"ateren Text bestimmt.

\begin{lemma}
\label{SatzStirling2Anwendung}
    \[ \MyChoose{ n }{ \frac{n}{2} }
           \geq
       \frac{2^n}{ \sqrt{\pi \frac{n}{2}} \: \MathE }
    \]
\end{lemma}
\begin{beweis}
    Es soll nach unten abgesch"atzt werden. Deshalb wird der Z"ahler mit
    Hilfe von \equref{Equ1SatzStirling} und der Nenner mit Hilfe von
    \equref{Equ2SatzStirling} abgesch"atzt. Man erh"alt:
    \begin{eqnarray}
       \MyChoose{ n }{ \frac{n}{2} }
           & = &
       \frac{n!}{ \left( \frac{n}{2} \right)^2 } \nonumber
    \\ \label{Equ1LemmaStirling2}
           & \geq &
       \frac{
            \sqrt{2\pi n}
            \left( \frac{n}{\MathE} \right)^n
            \MathE^{\frac{1}{12n+1}}
       }{
            \left(
                \sqrt{\pi n}
                \left( \frac{ \frac{n}{2} }{\MathE} \right)^{\frac{n}{2}}
                \MathE^{\frac{1}{6n}}
            \right)^2
       }
    \\     & \geq &
       \frac{
           \sqrt{2}
       }{
           \sqrt{\pi n} \left( \frac{1}{2} \right)^n
       }
       \, \MathE^{ \frac{ -18n-2 }{ (12n + 1)6n } }
    \\     & \geq & 
       \frac{2^n}{ \sqrt{\pi \frac{n}{2}} \: \MathE 
       } \nonumber
    \end{eqnarray}
\end{beweis}

Auf die gleiche Weise, wie im vorangegangenen Lemma
\ref{SatzStirlingAnwendung} eine Absch"atzung nach unten erfolgt,
erh"alt man eine Absch"atzung von
\[ \log \MyChoose{ n }{ \frac{n}{2} } \] nach oben:

\begin{lemma}
\label{SatzStirlingAnwendung}
    \[ \log{ n \choose \frac{n}{2} }
           \leq
       n - \frac{\log \left( \frac{n}{2} \right) + 1}{2}
    \]
\end{lemma}
\begin{beweis}
    Aus \equref{Equ1SatzStirling} folgt:
    \[ \left( \frac{n}{2} \right) !
           \geq
       \sqrt{\pi n}
       \left( \frac{ \frac{n}{2} }{\MathE} \right)^{\frac{n}{2}}
       \MathE^{\frac{1}{6n+1}}
    \]
    Also gilt:
    \begin{eqnarray}
    \label{EquStirling1Schaetzung}
       \frac{n!}{\left( \frac{n}{2}! \right)^2 }
           & \leq &
       \frac{
            \sqrt{2\pi n}
            \left( \frac{n}{\MathE} \right)^n
            \MathE^{\frac{1}{12n}}
       }{
            \left(
                \sqrt{\pi n}
                \left( \frac{ \frac{n}{2} }{\MathE} \right)^{\frac{n}{2}}
                \MathE^{\frac{1}{6n+1}}
            \right)^2
       }
    \\ \label{EquStirling2Schaetzung}
           & = &
       \frac{
           \sqrt{2}
       }{
           \sqrt{\pi n} \left( \frac{1}{2} \right)^n
       }
       \MathE^{ \frac{ 1-18n }{ 12n(6n+1) } }
    \\ \nonumber
        & \leq & \frac{2^n}{ \sqrt{ \pi \frac{n}{2} }  }
    \end{eqnarray}
    Bildet man nun auf beiden Seiten der Ungleichungskette den
    Logarithmus, erh"alt man mit Hilfe einiger Logarithmengesetze
    \[ \log{ n \choose \frac{n}{2} }
           \leq
       n - \frac{1}{2} \log\left( \frac{n}{2} \right) -
                                                   \frac{1}{2} \log(\pi)
           \leq
       n - \frac{ \log \left( \frac{n}{2} \right) + 1 }{2}
    \]
\end{beweis}

% **************************************************************************

\MySection{Der Entwicklungssatz von Laplace}
\label{SecLaplace}

F"ur den ersten im Kapitel \ref{ChapCsanky} darzustellenden Algorithmus
wird ein bekannter Satz verwendet. Er wird in diesem Unterkapitel zusammen 
mit einer h"aufig benutzten Folgerung dargestellt.

\begin{satz}[Entwicklungssatz von Laplace]
\label{SatzLaplace}
\index{Laplace!Entwicklungssatz von}
    Sei $k$ eine nat"urliche Zahl mit
    \[ 1 \leq k \leq n-1 \] Sei $D_n$ die Determinante der Matrix $A$.

    F"ur die nat"urliche Zahl $i$ gelte \[ 1 \leq i \leq {n \choose k} \]
    Sei $D_k^{(2i-1)}$ die Determinante einer Untermatrix $A_k^{(2i-1)}$ von
    $A$, die aus $k$ Spalten der ersten $k$ Zeilen gebildet werde.
    von
    $A$, die aus den f"ur $A_k^{(2i-1)}$ nicht gew"ahlten $n-k$ Zeilen und
    Spalten gebildet werde.

    F"ur jedes $i$ werde eine andere der \[ {n \choose k} \] m"oglichen
    Auswahlen f"ur die $k$ Spalten f"ur $A^{(2i-1)}$ getroffen.

    F"ur eine Untermatrix $A_k^{(2i-1)}$ bezeichne \[ f(A_k^{(2i-1)}) \] die
    $n$-Permutation, die die Spalten von $A$ so vertauscht, da"s
    $A_k^{(2i-1)}$ aus den ersten $k$ und $A_{n-k}^{(2i)}$
    aus den weiteren $n-k$ Zeilen und Spalten von $A$ besteht.

    Dann gilt:
    \begin{equation}
    \label{EquSatzLaplace}
    D_n = \sig(f(A_k^1)) D_k^1 D_{n-k}^2 + \sig(f(A_k^3)) D_k^3 D_{n-k}^4
          + \cdots
          + \sig \left( f \left( A_k^{2{n \choose k}-1} \right) \right)
            D_k^{2{n \choose k}-1} D_k^{2{n \choose k}}
    \end{equation}
\end{satz}
\begin{beweis}
    (vergl. \cite{Csan74} Seite 21)
    Es ist zu zeigen, da"s die Berechnung der Determinante nach
    \equref{EquSatzLaplace} mit der Berechnung nach \equref{EquDet}
    "ubereinstimmt.

    Die einzelnen Determinanten in \equref{EquSatzLaplace} werden auf
    die in \equref{EquDet} angegebene Weise berechnet.
    Dazu ist zu beachten, da"s $\permut_n$ genau $n!$
    Elemente besitzt.
    Es gen"ugt zu zeigen, da"s in \equref{EquSatzLaplace}
    ebenso viele Permutationen auf die Indizes von $1$ bis $n$ angewendet
    werden und alle voneinander verschieden sind.

    Berechnet man einen Summanden
    \[ \sig\lb f\lb A_k^{(2i-1)}\rb\rb D_k^{(2i-1)} D_{n-k}^{(2i)} \]
    in \equref{EquSatzLaplace} anhand von \equref{EquDet},
    kann man die zwei Summen durch Ausmultiplizieren zu einer
    zusammenfassen. Den $k!$ Permutationen zur Berechnung von $D_k^{(2i-1)}$
    und den $(n-k)!$ Permutationen zur Berechnung von $D_{n-k}^{(2i)}$
    entsprechen \[ k! \, (n-k)! \] $n$-Permutationen zur Berechnung der
    zusammengefa"sten Summen f"ur diesen einen Summanden. Die Menge dieser
    $n$-Permutationen wird mit $M_i$ bezeichnet.

    Die $k$-Permutationen bzw. $(n-k)$-Permutationen zur Berechnung aller
    Determinanten $D_k$ und $D_{n-k}$ werden jeweils auf
    verschiedene Mengen von Indizes angewendet, d. h. f"ur zwei beliebige
    dieser Mengen von Indizes gibt es in der einen mindestens einen Index,
    der in der anderen nicht enthalten ist. Das bedeutet, da"s alle Mengen
    $M_i$ voneinander verschieden sind.

    Die Anzahl der Mengen $M_i$, die man auf diese Weise f"ur
    \equref{EquSatzLaplace} erh"alt, betr"agt \[ {n \choose k} \]
    Ihre Vereinigung ergibt eine Menge mit 
    \[ {n \choose k} k! \, (n-k)! \]
    also \[ n! \] Elementen, was zu zeigen war.
\end{beweis}

Aus diesem Satz erh"alt man mit $k=1$:

\begin{korollar}[Zeilen- und Spaltenentwicklung]
\label{SatzEntw}
\index{Zeilenentwicklung}
\index{Spaltenentwicklung}
\index{Entwicklung! nach einer Zeile oder Spalte}
     Seien \[ 1 \leq p \leq n \] und \[ 1 \leq q \leq n \] beliebig.
     Dann gilt die {\em Entwicklung nach Zeile $p$}
     \[ \det(A)= \sum_{j=1}^n (-1)^{p+j} a_{p,j} \det(A_{(p|j)}) \]
     und die {\em Entwicklung nach Spalte $q$}
     \[ \det(A)= \sum_{i=1}^n (-1)^{i+q} a_{i,q} \det(A_{(i|q)}) \]
\end{korollar}

% **************************************************************************

\MySection{Determinantenberechnung durch 'Divide and Conquer'}
\label{SecDivCon}
\index{Divide and Conquer}
\index{Csanky!Algorithmus von}
\index{Algorithmus!von Csanky}

Der Algorithmus (\cite{Csan74} S. 21 ff.), der hier betrachtet wird,
berechnet die Determinante
rekursiv mit Hilfe der Methode {\em Divide and Conquer}, d. h. durch
die Berechnung der Determinanten von Untermatrizen der gegebenen Matrix.

Da es sich hierbei nur um ein einleitendes Beispiel handelt, wird zur
Vereinfachung angenommen,
da"s die Anzahl der Zeilen und Spalten $n$ der
Matrix eine Zweierpotenz ist. Falls dies nicht der Fall ist, wird sie um
entsprechend viele Zeilen und Spalten erweitert, so da"s die neuen
Elemente der Hauptdiagonalen jeweils gleich $1$ und alle weiteren
neuen Elemente jeweils gleich $0$ sind.

Zur rekursiven Berechnung der Determinate wird Satz \ref{SatzLaplace}
benutzt. Man w"ahlt
\[ k := \frac{1}{2}n \MyPunkt \]
Somit gilt auch \[ n-k = \frac{1}{2}n \MyPunkt \]

Die Anzahl der Schritte, die der Algorithmus ben"otigt, um mit Hilfe 
dieses Satzes die Determinante einer $n \times n$-Matrix zu berechnen, 
wird mit \[ s(n) \] bezeichnet. Die Anzahl der Prozessoren, die dabei
besch"aftigt werden, wird mit \[ p(n) \] bezeichnet.

Bei der Berechnung wird Gleichung
\equref{EquSatzLaplace} rekursiv ausgewertet. Das f"uhrt dazu, da"s
\[ \MyChoose{n}{ \frac{n}{2} } \] Determinanten von Untermatrizen zu
berechnen sind. Die Berechnung einer Determinanten erfordert
\[ s\left( \frac{n}{2} \right) \] Schritte und
\[ 2 \MyChoose{n}{ \frac{n}{2} } p\left( \frac{n}{2} \right) \] Prozessoren.

Aus \ref{SatzAlgRechnen} folgt, da"s die in
\equref{EquSatzLaplace} auftretenden Additionen in
\[ \log \MyChoose{ n }{ \frac{n}{2} } \] Schritten von
\[ \frac{ \MyChoose{ n }{ \frac{n}{2}} }{2} \] Prozessoren erledigt werden
k"onnen. Die Multiplikationen k"onnen in einem Schritt von ebenfalls
\[ \frac{ \MyChoose{ n }{ \frac{n}{2}} }{2} \] Prozessoren durchgef"uhrt
werden. Also gilt f"ur die Anzahl der Schritte, die der Algorithmus 
ben"otigt, folgende Rekursionsgleichung:
\[
   s(n) = \left\{
              \begin{array}{lcr}
                  0 & : & n = 1
              \\ s\left( \frac{n}{2} \right) + 1 +
                 \log \MyChoose{ n }{ \frac{n}{2} }
                   & : & n > 1
              \end{array}
          \right.
\]
Mit \ref{SatzStirlingAnwendung} folgt
\[ s(n)
       \leq
   s\left( \frac{n}{2} \right) + 1 +
   n - \frac{\log \left( \frac{n}{2} \right) + 1}{2}
\]
Das ist "aquivalent zu
\[
   s(n)
       \leq
   s\left( \frac{n}{2} \right) + 1 +
   n - \frac{\log(n)}{2}
\]
Die Aufl"osung der Rekursion ergibt:
\[
   s(n)
       \leq
   \log(n) + \sum_{j=1}^{\log(n)} \frac{n}{j} -
   \frac{1}{2} \sum_{k=1}^{\log(n)} \log\left( \frac{n}{k} \right)
\]
Man kann nun die folgenden Absch"atzungen vornehmen:
\begin{eqnarray*}
   \sum_{j=1}^{\log(n)} \frac{1}{j}
       & = &
   1 + \frac{1}{2} + \frac{1}{3} + \cdots + \frac{1}{\log(n)}
\\
   & \leq & 1 + \int\limits_1^{\log(n)} \frac{1}{x} \, dx
\\
   & = & [ \ln(x) ]_1^{\log(n)} = 1 + \ln(\log(n))
\\ 
   & \leq & 1 + \log(\log(n))
\end{eqnarray*}
und
\[
   \sum_{k=1}^{\log(n)} \log(k)
       \leq
   \sum_{k=1}^{\log(n)} \log(\log(n))
       =
   \log(n)\log(\log(n))
\]
und kommt so in Verbindung mit einigen Logarithmengesetzen auf
\[
   s(n)
       \leq
   \log(n) + n (1 + \lc \log(\log(n)) \rc ) 
   - \frac{1}{2} \log^2(n) + \log(n) \lc \log(\log(n)) \rc
\] % Probe: s(2) \leq 2.5
Also gilt \[ s(n) = O(n\log(\log(n))) \]

F"ur die Anzahl der besch"aftigten Prozessoren gilt
\[ p(n) = \left\{
              \begin{array}{lcr}
                  0 & : & n = 1
              \\  2 & : & n = 2
              \\  \max\left(
                         2 \MyChoose{ n }{ \frac{n}{2} }
                         p\left( \frac{n}{2} \right)
                      ,  \frac{ \MyChoose{ \scriptstyle n }{ \frac{n}{2} } 
                              }{2}
                      \right)
                    & : & \mbox{sonst}
              \end{array}
          \right.
\]
Bei diesem Beispielalgorithmus interessiert uns nur die Gr"o"senordnung
der Anzahl besch"aftigter Prozessoren. Deshalb wird diese Anzahl grob
nach unten abgesch"atzt durch
\[ p(n) > \MyChoose{ n }{ \frac{n}{2} } \]
Mit \ref{SatzStirling2Anwendung} folgt
\[ p(n) > \frac{ 2^n }{ \sqrt{\pi \frac{n}{2}} \: \MathE } \]
Da nach unten abgesch"atzt wurde, folgt aus dieser Ungleichung
\[ p(n) = \Omega\left( \frac{2^n}{\sqrt{n}} \right) \]
Die Anzahl der Prozessoren ist also von exponentieller Gr"o"senordnung.

Es wird sich bei den noch zu betrachtenden Algorithmen zeigen, da"s 
sowohl f"ur die Anzahl der Schritte als auch f"ur die
Anzahl der besch"aftigten Prozessoren deutlich bessere Werte m"oglich sind.

% **************************************************************************

\MySection{Die Linearfaktorendarstellung}

In diesem Unterkapitel werden einige Aussagen behandelt, die sich in
Verbindung mit einer weiteren Darstellungsm"oglichkeit f"ur das 
charakteristische Polynom ergeben. Diese Aussagen werden f"ur den 
Beweis des Satzes von Frame (siehe \ref{SecFrame}) ben"otigt. Literatur
dazu ist bereits in Kapitel \ref{ChapBase} aufgelistet.

Neben der Matrizendarstellung und der Koeffizientendarstellung gibt es 
noch eine dritte f"ur uns wichtige Darstellung f"ur das charakteristische
Polynom, die 
\index{Linearfaktorendarstellung} {\em Linearfaktorendarstellung}. 
Ein Polynom $n$-ten Grades
kann man auch als Produkt von $n$ Linearfaktoren darstellen, so da"s das
charakteristische Polynom folgendes Aussehen bekommt:
\Beq{TermLinearfaktoren}
    (\lambda_1 - \lambda) (\lambda_2 - \lambda) \ldots
    (\lambda_n - \lambda)
\Eeq
Dabei sind die $\lambda_i$ die Eigenwerte der zugrunde liegenden
$n \times n$-Matrix.
Sie sind {\em nicht} paarweise verschieden.
Diese Tatsache f"uhrt uns zum Begriff der
{\em Vielfachheit}. Man kann in der obigen Linearfaktorendarstellung
gleiche Faktoren mit Hilfe von Potenzen beschreiben. Dazu gelte
\[ k \in \Nat, \: 1 \leq k \leq n \MyPunkt \]
Die Eigenwerte seien
\[ \lambda'_1, \lambda'_2, \ldots, \lambda'_k \MyKomma \]
jedoch alle paarweise voneinander verschieden. So bekommt
\equref{TermLinearfaktoren} folgendes Aussehen:
\[ (\lambda'_1 - \lambda)^{m(\lambda'_1)}
   (\lambda'_2 - \lambda)^{m(\lambda'_2)} \ldots
   (\lambda'_k - \lambda)^{m(\lambda'_k)}
\]
\index{Vielfachheit}
In dieser Darstellung wird $m(\lambda'_i)$ mit {\em Vielfachheit} des
Eigenwertes $\lambda'_i$ bezeichnet.

Die $n$ Eigenwerte einer $n \times n$-Matrix besitzen zur Determinante 
und zur Spur jeweils eine wichtige Beziehung, welche in den n"achsten 
beiden S"atzen dargestellt wird:
\begin{satz}
    \[ \det(A)= \prod_{i=1}^{n} \lambda_i \]
\end{satz}
\begin{beweis}
    Die G"ultigkeit der Behauptung wird bei Betrachtung der Berechnung der
    Koeffizientendarstellung des charakteristischen Polynoms aus dessen
    Linearfaktorendarstellung deutlich. Beim Ausmultiplizieren der 
    Linearfaktoren wird der Wert von $c_0$ nur durch das Produkt der 
    Eigenwerte beeinflu"st.
\end{beweis}

\begin{satz}
\label{SatzTrEigenwerte}
    \[ \tr(A)= \sum_{i=1}^{n} \lambda_i \]
\end{satz}
\begin{beweis}
    Betrachtet man die Berechnung der Koeffizientendarstellung des
    charakteristischen Polynoms aus dessen Linearfaktorendarstellung durch
    Ausmultiplizieren, erkennt man, da"s gilt:
    \[ c_{n-1} = (-1)^{n-1} \sum_{i=1}^n \lambda_i \MyPunkt \]
    Mit \ref{SatzDdurchP} folgt daraus die Behauptung.
\end{beweis}

Die Eigenwerte besitzen weiterhin folgende interessante Eigenschaft:
\begin{satz}
\label{SatzEigenPotenz}
    Seien $\lambda_1, \lambda_2, \ldots, \lambda_n$ die Eigenwerte der
    $n \times n$-Matrix $A$. Sei $k \in \Nat$. Dann gilt:
    \[ \lambda_1^k, \lambda_2^k, \ldots, \lambda_n^k \] sind die
    Eigenwerte von $A^k$.
\end{satz}
\begin{beweis}
    Ein Skalar $\lambda$ ist genau dann Eigenwert von $A$, wenn es einen
    Vektor $x \neq 0_n$ gibt, so da"s gilt\footnote{vgl. 
    Erl"auterungen auf S. \pageref{PageEigenMotiv}}
    \Beq{Equ1EigenPotenz}
        Ax = \lambda x \MyPunkt
    \Eeq
    Mit Hilfe dieser Beziehung wird die Behauptung per Induktion
    bewiesen.

    Nach Voraussetzung ist die Behauptung f"ur $k=1$ erf"ullt. Es gelte
    also nun
    \[ \forall \lambda= \lambda_1, \ldots, \lambda_n \exists x : \:
       A^k x = \lambda^k x \MyPunkt
    \]
    Zu zeigen ist, da"s dann auch gilt
    \[ \forall \lambda= \lambda_1, \ldots, \lambda_n \exists x : \:
       A^{k+1} x = \lambda^{k+1} x
    \]
    Diese Gleichung kann man umformen in
    \[ \forall \lambda= \lambda_1, \ldots, \lambda_n \exists x : \:
       A^k \underbrace{A x}_{(*)}= \lambda^k \underbrace{\lambda x}_{(*)}
    \]
    Nach Voraussetzung sind die Terme $(*)$ gleich. Sie werden mit $y$ 
    bezeichnet. Die Gleichung bekommt dann folgendes Aussehen:
    \[ \forall \lambda= \lambda_1, \ldots, \lambda_n \exists y: \:
       A^k y= \lambda^k y
    \]
    Dies ist wiederum nach Voraussetzung richtig.
\end{beweis}

Aus diesem Satz ergibt sich mit Hilfe von \ref{SatzTrEigenwerte}
eine f"ur uns wichtige Beziehung:
\begin{korollar}
\label{SatzTraceLambda}
    \[ \tr(A^k) = \sum_{i=1}^n \lambda_i^k \]
\end{korollar}

% **************************************************************************

\MySection{Die Newton'schen Gleichungen f"ur Potenzsummen}
\label{SecNewtonPotenz}

Mit den {\em Newton'schen Gleichungen f"ur Potenzsummen} werden in diesem
Unterkapitel weitere Grundlagen f"ur den Beweis der S"atze von Frame
(siehe \ref{SecFrame}) behandelt. Die gesamten Hintergr"unde f"ur diese
Gleichungen werden z. B. in \cite{Haup52} (Kapitel 7 und 8) behandelt.

Eine {\em Potenzsumme} ist eine Summe von Potenzen einer oder mehrerer
Unbestimmter. Auf Seite \ref{SatzSumK} sind weitere Beispiele f"ur 
einfachere Potenzsummengleichungen zu finden.

Da uns diese Gleichungen im Zusammenhang mit dem charakteristischen
Polynom einer Matrix interessieren, werden sie anhand dieses Polynoms
entwickelt. Dazu werden folgende Vereinbarungen getroffen:
\begin{itemize}
\item
      Das charakteristische Polynom der $n \times n$-Matrix $A$ sei
      \[
         p(\lambda) := c_n \lambda^n + c_{n-1} \lambda^{n-1} + \cdots
                       + c_1 \lambda + c_0 \MyPunkt 
      \]
\item
      Die erste Ableitung von $p(\lambda)$ wird mit \[ p'(\lambda) \]
      bezeichnet.
\item Die Eigenwerte von $A$ seien
      \[ \lambda_1, \lambda_2, \ldots, \lambda_n \MyPunkt \]
      Es wird definiert
      \[ s_k := \sum_{i=1}^n \lambda_i^k \MyPunkt \]
\end{itemize}

Zun"achst besch"aftigen wir uns mit der 
Polynomdivision. \index{Polynomdivsion}
Sei dazu ein $i$
mit $ 1 \leq i \leq n $ gegeben. Da $\lambda_i$ Eigenwert von $A$ und somit
Nullstelle von $p(\lambda)$ ist, l"a"st sich $p(\lambda)$ ohne Rest durch
\[ \lambda_i - \lambda \] teilen. Das Ergebnis ist ein Polynom vom Grad
$n-1$. Das folgende Lemma liefert eine Aussage "uber dessen
Koeffizienten:

\begin{lemma}
\label{SatzPolynomDiv}
    Gegeben sei die Gleichung:
    \Beq{Equ1SatzPolynomDiv}
        \frac{ p(\lambda) }{ (\lambda_i - \lambda) }
            =
        \hat{c}_{n-1}\lambda^{n-1} +
        \hat{c}_{n-2}\lambda^{n-2} + \ldots +
        \hat{c}_1 \lambda + \hat{c}_0
    \Eeq
    Dann gilt f"ur $1 \leq k \leq n-1$:
    \Beq{Equ2SatzPolynomDiv}
        \hat{c}_k =
        - \sum_{j=1}^{n-k} \lambda_i^{j-1} c_{k+j}
%        = - ( \lambda_i^{n-k-1} c_n + \lambda_i^{n-k-2} c_{n-2} +
%              \cdots + \lambda c_{k+2} + c_{k+1} )
    \Eeq
\end{lemma}
\begin{beweis}
     Multipliziert man beide Seiten von Gleichung
     \equref{Equ1SatzPolynomDiv} mit \[ (\lambda_i - \lambda) \] und
     multipliziert die rechte Seite der so gewonnenen Gleichung aus,
     ergibt sich:
     \begin{eqnarray*}
         \lefteqn{c_n \lambda^n + c_{n-1} \lambda^{n-1} + \cdots
                       + c_1 \lambda + c_0 }
     \\ & = &
         - \hat{c}_{n-1} \lambda^n +
         (\lambda_i \hat{c}_{n-1} - \hat{c}_{n-2}) \lambda^{n-1} +
         (\lambda_i \hat{c}_{n-2} - \hat{c}_{n-3}) \lambda^{n-2} + \cdots +
         (\lambda_i \hat{c}_1 - \hat{c}_0) \lambda +
         \lambda_i \hat{c}_0
     \end{eqnarray*}
     Setzt man \[ \hat{c}_n = \hat{c}_{-1} = 0 \MyKomma \]
     erh"alt man durch Koeffizientenvergleich:
     \begin{eqnarray}
        \nonumber
            & \forall 1 \leq l \leq n :
            c_k = \lambda_i \hat{c}_k - \hat{c}_{k-1}
     \\ \label{Equ3SatzPolynomDiv}
            \Leftrightarrow &
                \forall 1 \leq l \leq n :
                \hat{c}_{l-1} = \lambda_i \hat{c}_l - c_l
     \end{eqnarray}
     Gleichung \equref{Equ2SatzPolynomDiv} wird durch
     Induktion\footnote{Um die Art und Weise, wie die Koeffizienten der
     Polynome indiziert sind, einheitlich zu halten, verl"auft die
     Induktion etwas ungewohnt.} bewiesen.
     F"ur $k = n-1$ folgt die G"ultigkeit von \equref{Equ2SatzPolynomDiv} 
     aus \equref{Equ3SatzPolynomDiv}. 
     
     Gelte \equref{Equ2SatzPolynomDiv} also nun f"ur $k = l$.
     Es ist zu zeigen, da"s die Gleichung dann auch f"ur $k = l-1$ gilt:
     \[
        \hat{c}_{l-1} = - \sum_{j=1}^{n-(l-1)} \lambda_i^{j-1} c_{l-1+j}
     \]
     Zieht man $c_l$ aus der Summe heraus und indiziert neu, erh"alt man:
     \[
        \hat{c}_{l-1} = - c_l - \sum_{j=1}^{n-l} \lambda_i^{j} c_{l+j}
     \]
     Zieht man nun noch $\lambda_i$ aus der Summe heraus, erh"alt man:
     \[
        \hat{c}_{l-1} = - c_l - \lambda_i 
                                   \sum_{j=1}^{n-l} \lambda_i^{j-1} c_{l+j}
     \]
     Nach Induktionsvoraussetzung ist dies gleichbedeutend mit:
     \[
        \hat{c}_{l-1} = - c_l + \lambda_i \hat{c}_l
     \]
     Die G"ultigkeit dieser Gleichung folgt aus \equref{Equ3SatzPolynomDiv}.
\end{beweis}

Eine an dieser Stelle wichtige Regel f"ur das Differenzieren 
lautet (\cite{Haup52} S. 160):

\begin{bemerkung}
\label{SatzProdDiff}
\index{Differenzierung!von Produkten}
    Seien \[ g_1(x), g_2(x), \ldots, g_n(x) \]
    auf einem Intervall $I$ differenzierbare Funktionen. Es gelte 
    \[ f(x) = \prod_{i=1}^n g_i(x) \MyPunkt \]
    Dann ist auch $f(x)$ auf $I$ differenzierbar mit
    \[ f'(x) = \sum_{i=1}^n g_1(x) g_2(x) \cdots 
                            g_{i-1}(x) g_i'(x) g_{i+1}(x) \cdots
                            g_{n-1}(x) g_n(x) \MyPunkt
    \]
\end{bemerkung}

Falls gilt \[ \forall x \in I, 1 \leq i \leq n: \: g_i(x) \neq 0 \MyKomma \]
l"a"st sich diese Regel auch einfacher formulieren:
\[ f'(x) = \sum_{i=1}^n \frac{f(x)}{g_i(x)} g_i'(x) \]

Betrachtet man das charakteristische Polynom $p(\lambda)$ in 
Linearfaktorendarstellung und beachtet, da"s die Ableitung von
\[ (\lambda_i - \lambda) \] $-1$ ergibt, dann erh"alt man mit Hilfe von 
\ref{SatzProdDiff}:

\begin{korollar}
\label{SatzCharPolyDiff}
    \Beq{EquCharPolyDiff} 
        p'(\lambda)= - \sum_{i=1}^n 
                               \frac{ p(\lambda) }{ (\lambda_i - \lambda) }
    \Eeq
\end{korollar}

Mit Hilfe von \ref{SatzPolynomDiv} und \ref{SatzCharPolyDiff} erh"alt man
nun die gesuchten Gleichungen (\cite{Haup52} S. 181):

\begin{satz}[Newton'sche Gleichungen f"ur Potenzsummen]
\label{SatzNewtonPotenz}
\index{Newton!Gleichungen von}
    \hfill \mbox{\hspace{1cm}} \\ 
    F"ur die Koeffizienten des charakteristischen Polynoms 
    gilt\footnote{Die Terme $s_i$ sind am Beginn des Unterkapitels 
    definiert.}:
    \[
       \forall 0 \leq k \leq n-1 : \:
       - (n - (k+1)
         ) c_{k+1}
       - \sum_{j=2}^{n-k} s_{j-1} c_{k+j}
            = 0
    \]
\end{satz}
\begin{beweis}
    Die $\hat{c}_k$ aus Lemma \ref{SatzPolynomDiv} sind abh"angig vom
    gew"ahlten $\lambda_i$. Deshalb definieren wir diese Koeffizienten
    als Funktionen von $\lambda_i$:
    \[ \hat{c}_k(\lambda_i) :=
           - \sum_{j=1}^{n-k} \lambda_i^{j-1} c_{k+j} \MyPunkt
    \]

    Dann folgt aus Lemma \ref{SatzPolynomDiv}:
    \Beq{Equ4SatzNewtonPotenz}
       \sum_{i=1}^n \hat{c}_k(\lambda_i)
           = \overbrace{- \sum_{j=1}^{n-k} s_{j-1} c_{k+j}
                       }^{ \dot{c}_k := } \MyPunkt
    \Eeq
    "Ubertr"agt man diese Beziehung zwischen den Koeffizienten auf die
    entsprechenden Polynome erh"alt man:
    \Beq{Equ1SatzNewtonPotenz}
       \sum_{i=1}^n \frac{ p(\lambda) }{ \lambda_i - \lambda }
           =
       \sum_{k=0}^{n-1} \dot{c}_k \lambda^k
    \Eeq
    Die erste Ableitung des charakteristischen Polynoms in der
    Koeffizientendarstellung sieht folgenderma"sen aus:
    \Beq{Equ2SatzNewtonPotenz}
       p'(\lambda) = n c_n \lambda^{n-1} + (n-1) c_{n-1} \lambda^{n-2}
                     + \cdots +
                     2 c_2 \lambda + c_1
    \Eeq
    Aus den drei Gleichungen \equref{EquCharPolyDiff},
    \equref{Equ1SatzNewtonPotenz} und \equref{Equ2SatzNewtonPotenz} folgt:
    \[ - \sum_{k=0}^{n-1} \dot{c}_k \lambda^k
           =
       n c_n \lambda^{n-1} + (n-1) c_{n-1} \lambda^{n-2}
           + \cdots +
       2 c_2 \lambda + c_1
    \]
    Durch Koeffizientenvergleich erh"alt man aus dieser Gleichung:
    \[ 
        \forall 0 \leq k \leq n-1 : \: - \dot{c}_k = (k+1) c_{k+1} 
    \]
    Setzt man f"ur $\dot{c}_j$ den entsprechenden Term aus 
    Gleichung \equref{Equ4SatzNewtonPotenz} ein, ergibt sich:
    \begin{MyEqnArray}
       \MT
       \forall 0 \leq k \leq n-1 : \: 
         \sum_{j=1}^{n-k} s_{j-1} c_{k+j}
           \MT = \MT 
       (k+1) c_{k+1}
    \MNl \Leftrightarrow \MT
       \forall 0 \leq k \leq n-1 : \: 
         n c_{k+1} + \sum_{j=2}^{n-k} s_{j-1} c_{k+j}
           & \DS = & \DS
       (k+1) c_{k+1}
    \MNl \Leftrightarrow \MT 
       \forall 0 \leq k \leq n-1 : \: 
         (n-(k+1)) c_{k+1} + \sum_{j=2}^{n-k} s_{j-1} c_{k+j}
           \MT = \MT 0 
    \end{MyEqnArray}
\end{beweis}

% **************************************************************************

\MySection{Die Adjunkte einer Matrix}
\label{SecAdj}

Bei den Beweisen des Satzes von Frame in Unterkapitel \ref{SecFrame} 
spielt die Adjunkte einer Matrix eine bedeutende Rolle und wird
deshalb hier behandelt.

\MyBeginDef
\label{DefAdj}
\index{Adjunkte}
    Sei $A$ eine $n \times n$-Matrix.
    Erh"alt man die Matrix $B$ aus der Matrix $A$ nach
    \[
        b_{i,j} := (-1)^{i+j} \det(A_{(j|i)}) \MyKomma
    \] so hei"st $B$ {\em Adjunkte der Matrix $A$}. Die Adjunkte wird mit
    \[ \adj(A) \] bezeichnet.
\MyEndDef

Zum Beweis einer uns besonders interessierenden Eigenschaft der Adjunkten
ben"otigen wir zun"achst noch ein Lemma. Es
behandelt den Fall einer Zeilen- bzw. Spaltenentwicklung\footnote{ vgl. 
\ref{SatzEntw} }, bei der jedoch
als Faktoren f"ur die Unterdeterminanten die Matrizenelemente nicht aus
der Zeile bzw. Spalte entnommen werden, nach der die Determinante 
entwickelt wird:

\begin{lemma}
\label{SatzFalscheEntw}
    Seien \[ 1 \leq p,p' \leq n \] und \[ 1 \leq q,q' \leq n \] mit
    \[ p \neq p' \] und \[ q \neq q' \] Dann gilt:
    \[ \sum_{j=1}^n (-1)^{p+j} a_{p',j} \det(A_{(p|j)}) = 0 \] und
    \[ \sum_{i=1}^n (-1)^{i+q} a_{i,q'} \det(A_{(i|q)}) = 0 \]
\end{lemma}
\begin{beweis}
    Betrachtet man die Berechnung der Unterdeterminanten in den obigen
    Gleichungen nach \equref{EquDet}, erkennt man beim
    Vergleich mit der Berechnung der Determinante einer Matrix, die
    zwei gleiche Zeilen enth"alt, da"s die Terme beider Berechnungen 
    nach einigen Vereinfachungen "ubereinstimmen. Nach Satz
    \ref{SatzDetPermut} ist die Determinante in diesem Fall gleich $0$.
\end{beweis}

Mit Hilfe von \ref{SatzFalscheEntw} erhalten wir nun:

\begin{satz}
\label{SatzAdj}
    \begin{eqnarray}
        \label{EquSatzAdj1}
        A * \adj(A) = E_n * \det(A) 
     \\ \label{EquSatzAdj2}
        \adj(A) * A = E_n * \det(A)
    \end{eqnarray}
\end{satz}
\begin{beweis}
    Spalte $k$ der Matrix $\adj(A)$ sieht so aus:
    \[ 
        \left[
        \begin{array}{c}
            (-1)^{1+k} \det(A_{(k|1)}) 
         \\ (-1)^{2+k} \det(A_{(k|2)})
         \\ \vdots
         \\ (-1)^{n+k} \det(A_{(k|n)})
        \end{array} 
        \right]
    \]
    Das Element an der Stelle $(i,k)$ der Produktmatrix \[ A * \adj(A) \]
    ist also gleich \[ \sum_{j=1}^n a_{i,j} (-1)^{j+k} A_{(k|j)} \]
    Dies ist nach \ref{SatzEntw} gleich $\det(A)$ f"ur \[ i = k \] und
    nach \ref{SatzFalscheEntw} gleich $0$ f"ur \[ i \neq k \]
    Daraus folgt die G"ultigkeit von \equref{EquSatzAdj1}. 
    Die Argumentation
    f"ur \equref{EquSatzAdj2} verl"auft analog.
\end{beweis}

% **************************************************************************

\MySection{Der Satz von Frame}
\label{SecFrame}

In diesem Unterkapitel wird eine Methode von J. S. Frame (\cite{Fram49};
\cite{Dwye51} S. 225-235) vorgestellt\footnote{ Die
Originalver"offentlichung \cite{Fram49} enth"alt keinen Beweis. Dieser
Beweis ist schwer zu bekommen. Er wird hier deshalb
frei nachvollzogen und d"urfte sich vom Original nicht wesentlich 
unterscheiden. In diesem Zusammenhang m"ochte ich mich bei R. T. Bumby 
f"ur seinen Hinweis auf die Newton'schen Gleichungen f"ur Potenzsummen
bedanken.} , die es u. a. erlaubt, die
Determinante einer Matrix zu berechnen. Diese Methode ist im wesentlichen
eine Neuentdeckung der 
Methode von Leverrier \index{Leverrier!Methode von}
aus dem 19. Jahrhundert zur 
Bestimmung der Koeffizienten des charakteristischen Polynoms 
(z. B. \cite{Hous64} S. 166 ff. ). Die Darstellung wird hier auf die Teile 
beschr"ankt, die f"ur die Determinantenberechnung wichtig sind.

Die Adjunkte von
\Beq{TermFrameAminusLambda} 
    A - \lambda E_n
\Eeq 
besteht aus lauter Determinanten von $(n-1) \times (n-1)$-Matrizen. Sie
kann deshalb durch ein Polynom vom Grad $n-1$ dargestellt werden
(siehe dazu auch \ref{DefCharPoly} und \ref{DefAdj}).
Dies motiviert die folgende Vereinbarung zus"atzlich zu den in
\ref{SecNewtonPotenz} aufgef"uhrten Bezeichnungen:
\begin{quote}
    Seien \[ B_i \: , \: 0 \leq i \leq n-1 \] geeignet gew"ahlte 
    $n \times n$-Matrizen.
    Dann bezeichnet
    \[
       c(\lambda) := B_{n-1} \lambda^{n-1} + B_{n-2} \lambda^{n-2} + 
                     \cdots + B_2 \lambda^2 + B_1 \lambda + B_0 \MyPunkt
    \] die Adjunkte von \equref{TermFrameAminusLambda}.
\end{quote}

\begin{lemma}
\label{SatzFrame1}
    \begin{eqnarray*}
       B_{n-1} & = & (-1) E_n
    \\ \forall n-2 \geq i \geq 0 : \:
       B_{i} & = & A B_{i+1} - c_{i+1} E_n
    \end{eqnarray*}
\end{lemma}
\begin{beweis}
    Aus Satz \ref{SatzAdj} in Verbindung mit Definition \ref{DefCharPoly}
    folgt:
    \[ (A - \lambda E_n) c(\lambda) = p(\lambda) E_n \]
    Setzt man die Koeffizientendarstellungen von $c(\lambda)$ und
    $p(\lambda)$ in diese Gleichung ein, erh"alt man
    \[
    \begin{array}{p{1.8em}p{3.6em}l}
    & \multicolumn{2}{l}{
          \DS (A - \lambda E_n) (B_{n-1} \lambda^{n-1} + 
               B_{n-2} \lambda^{n-2} + \cdots + 
               B_2 \lambda^2 + B_1 \lambda + B_0) \MatStrut
      } 
    \\ & & \DS \MatStrut
        = (c_n \lambda^n + c_{n-1} \lambda^{n-1} + \cdots
                       + c_2 \lambda^2 + c_1 \lambda + c_0 ) E_n
    \\ \mbox{ $\DS \Leftrightarrow$ } & 
       \multicolumn{2}{l}{ \MatStrut
           \DS AB_{n-1} \lambda^{n-1} + AB_{n-2} \lambda^{n-2} +
                       \cdots + AB_2 \lambda^2 + AB_1 \lambda + AB_0
       }
    \\ &
       \multicolumn{2}{l}{ \MatStrut
           \DS - B_{n-1} \lambda^{n} - B_{n-2} \lambda^{n-1} -
                       \cdots - B_2 \lambda^3 - B_1 \lambda^2 - B_0 \lambda
       }
    \\ & & \DS \MatStrut
       = (c_n \lambda^n + c_{n-1} \lambda^{n-1} + \cdots
                   + c_2 \lambda^2 + c_1 \lambda + c_0 ) E_n
    \\ \mbox{ $\DS \Leftrightarrow$ } & 
       \multicolumn{2}{l}{ \MatStrut
           \DS - B_{n-1} \lambda^{n} + (AB_{n-1} - B_{n-2}) \lambda^{n-1} +
               (AB_{n-2} - B_{n-3}) \lambda^{n-2} + \cdots 
       }
    \\ &
       \multicolumn{2}{l}{ \MatStrut
           \DS + (AB_2 - B_1) \lambda^2 + (AB_1 - B_0) \lambda
       }
    \\ & & \DS \MatStrut
       = (c_n \lambda^n + c_{n-1} \lambda^{n-1} + \cdots
                         + c_2 \lambda^2 + c_1 \lambda + c_0 ) E_n
    \end{array}
    \]
    Koeffizientenvergleich ergibt die Behauptung.
\end{beweis}

\begin{lemma}
\label{SatzFrame2}
    Es gilt\footnote{Da in dieser Arbeit verschiedene Arten von hoch und
    tiefgestellten Indizes und Markierungen verwendet werden, sei hiermit
    exiplizit darauf hingewiesen, da"s mit $p'(\lambda)$ ,wie allgemein
    "ublich, die erste Ableitung von $p(\lambda)$ gemeint ist.}
    \[ p'(\lambda) = - \tr(c(\lambda)) \MyPunkt \]
\end{lemma}
\begin{beweis}
    Zu zeigen ist:
    \[ \sum_{j=1}^n j c_j \lambda^{j-1} = 
       - \tr\left( \sum_{j=1}^{n} B_{j-1} \lambda^{j-1} \right)
    \]
    Durch Koeffizientenvergleich erh"alt man:
    \[ 
       \forall 1 \leq j \leq n: \: j c_j = - \tr(B_{j-1}) 
    \]
    Durch wiederholte Anwendung von \ref{SatzFrame1} ergibt sich daraus:
    \begin{eqnarray*}
       j c_j & = & - \tr(A B_{j} - c_{j} E_n)
    \\ \Leftrightarrow
       (n-j) c_j & = & \tr(A B_j)
    \\ \Leftrightarrow
       (n-j) c_j & = & \tr(A (A B_{j+1} - c_{j+1} E_n))
    \\ \Leftrightarrow
       (n-j) c_j & = & \tr(A^{n-j}B_{n-1}) 
                       - \sum_{k=1}^{n-j-1} \tr(A^k) c_{j+k}
    \\ \Leftrightarrow
       (n-j) c_j & = & - \tr(A^{n-j}) 
                       - \sum_{k=1}^{n-j-1} \tr(A^k) c_{j+k}
    \\ \Leftrightarrow
       (n-j) c_j & = & - \tr(A^{n-j}) 
                       - \sum_{k=1}^{n-j-1} \tr(A^k) c_{j+k}
    \end{eqnarray*}
    Nach \ref{SatzTraceLambda} ist dies gleichbedeutend mit
    \[
       (n-j) c_j + s_{n-j}
                   + \sum_{k=1}^{n-j-1} s_k c_{j+k} = 0
    \]
    Da f"ur das charakteristische Polynom $c_n=1$ gilt, ist diese 
    Gleichung nach Satz \ref{SatzNewtonPotenz} richtig.
    \\ \hspace{10em} % damit das Viereck auf der rechten Seite steht
\end{beweis}

\begin{lemma}
\label{SatzFrame3}
    \[ \forall 1 \leq i \leq n: \: c_i = \frac{1}{n-i} \tr(A B_i) \]
\end{lemma}
\begin{beweis}
    Wie in \ref{SatzFrame1} folgt zun"achst aus
    \ref{SatzAdj} in Verbindung mit Definition \ref{DefCharPoly}:
    \begin{MyEqnArray}
       \MT (A - \lambda E_n) c(\lambda) \MT = \MT p(\lambda) E_n 
    \MNl \Leftrightarrow \MT
        A c(\lambda) - \lambda c(\lambda) \MT = \MT p(\lambda) E_n
    \MNl \Leftrightarrow \MT
       \tr(A c(\lambda)) \MT = \MT 
       \tr(\lambda c(\lambda)) + \tr(p(\lambda) E_n) \MatStrut
    \end{MyEqnArray}
    Mit Hilfe von \ref{SatzFrame2} folgt:
    \[
    \begin{array}{p{1.8em}rcl}
       & \multicolumn{3}{l}{ 
             \DS \tr(A c(\lambda)) \mbox{ $\DS =$ }
             \DS n p(\lambda) - \lambda p'(\lambda) 
         }
    \\ \mbox{ $\DS \Leftrightarrow$ } &
       \multicolumn{3}{l}{ \MatStrut
          \DS \tr(AB_{n-1} \lambda^{n-1} + AB_{n-2} \lambda^{n-2} + 
                       \cdots + AB_2 \lambda^2 + AB_1 \lambda + AB_0)
       }
    \\ & \mbox{ $\DS = $ } & \MatStrut
        \DS n ( \lambda^n c_n + c_{n-1} \lambda^{n-1} + \cdots
                       + c_2 \lambda^2 + c_1 \lambda + c_0)
    \\ & & \MatStrut
        \DS - (n \lambda^n c_n + (n-1) c_{n-1} \lambda^{n-1} + \cdots
                       + 2 c_2 \lambda^2 + c_1 \lambda )
    \end{array}
    \]
    Koeffizientenvergleich ergibt
    \[ \forall 1 \leq i \leq n: \: \tr(AB_i) = n c_i - i c_i \MyPunkt \]
\end{beweis}

\begin{satz}[Frame]
\label{SatzFrame}
\index{Frame!Satz von}
    \Beq{Equ3SatzFrame}
        \det(A)= \frac{ \tr(A B_0) }{ n }
    \Eeq
\end{satz}
\begin{beweis}
    Man erh"alt die Behauptung aus \ref{SatzFrame1}, \ref{SatzFrame3} und
    \ref{SatzDdurchP}.
\end{beweis}

% **************************************************************************

\MySection{Determinantenberechnung mit Hilfe des Satzes von Frame}
\label{SecAlgFrame}
\index{Algorithmus!von Csanky}
\index{Csanky!Algorithmus von}

In diesem Unterkapitel wird eine effiziente Methode zur parallelen
Determinantenberechnung \cite{Csan76} vorgestellt (abgek"urzt mit 
{\em C-Alg.};
vgl. Unterkapitel \ref{SecBez}). Sie benutzt Divisionen und kann deshalb
nur angewendet werden, wenn die Berechnungen in einem K"orper stattfinden.
Dies ist problematisch,
weil in realen Rechnern nur mit begrenzter Genauigkeit gearbeitet werden 
kann und somit immer auf die eine oder andere Weise modulo gerechnet wird. 
Z. B. besitzt 6 im Ring $ \Integers_8 $ kein multiplikatives Inverses.

Dies motiviert den Entwurf von Algorithmen, die ohne Divisionen
auskommen, und somit auch in Ringen anwendbar sind, wie BGH-Alg. und
B-Alg. . P-Alg. benutzt wie C-Alg. ebenfalls Divisionen.

Die Determinantenberechnung erfolgt in dem Algorithmus, der in diesem
Unterkapitel vorgestellt wird, mit Hilfe des Satzes von Frame
(Satz \ref{SatzFrame}). Der Satz nutzt die bereits in \ref{SatzDdurchP}
erw"ahnte Tatsache aus, da"s sich unter den Koeffizienten des
charakteristischen Polynoms auch die Determiante befindet. Diese Eigenschaft
des charakteristischen Polynoms wird auch in B-Alg. und P-Alg. in
Verbindung mit anderen Verfahren zur Bestimmung der Koeffizienten
verwendet.

Die Berechnung der Determinante nach Satz \ref{SatzFrame} erfolgt mit Hilfe
einer Rekursionsgleichung. F"ur eine effiziente parallele Berechnung ist
dies nicht befriedigend. Deshalb ist sind einige Umformungen \cite{Csan76}
erforderlich. F"ur diese Umformungen wird ein Operator ben"otigt. Dazu seien
$M$ und $N$ jeweils $n \times n$-Matrizen:

\MyBeginDef
\index{Spuroperator}
\label{DefSpurOp}
    Der Operator $T$ wird definiert durch: \[ {T}{N} := \tr(N) \]
    Er wird {\em Spuroperator} genannt.
\MyEndDef

Es gilt also \[ (E + MT)N = N + {M}{T}{N} = N + M \tr(N) \MyPunkt \]

F"ur die Determinantenberechnung nach Satz \ref{SatzFrame} ist die dort
auftretende Matrix $B_0$ zu berechnen. Mit Hilfe der Lemmata
\ref{SatzFrame1} und \ref{SatzFrame2} sowie des soeben definierten
Operators $T$ erh"alt diese Berechnung folgendes Aussehen:
\begin{eqnarray*}
   B_0 & = & A B_1 - c_1 E_n
\\     & = & A B_1 - \frac{E_n}{n-1} \tr(A B_1)
\\     & = & \left( E_n - \frac{E_n}{n-1} T \right) A B_1
\\     & \vdots & 
\\     & = & \left( E_n - \frac{E_n}{n-1}T \right)
             \left\{A
                 \left[
                     \left(E_n - \frac{E_n}{n-2}T \right)
                     \left\{A
                         \left[
                             \cdots
                             (E_n - {E_n}{T})
                             \{A[E_n]\}
                             \cdots
                         \right]
                     \right\}
                 \right]
             \right\}
\end{eqnarray*}
Mit Hilfe der Assoziativit"at der Matrizenmultiplikation erh"alt man:
\Beq{Equ1SatzCsanky}
    B_0 = \left(
              \underbrace{
                  A - \overbrace{ \frac{E}{n-1} }^{\mbox{Term 1}}
                  \overbrace{ {T}{A} }^{\mbox{Term 2}}
              }_{\mbox{Term 3}}
          \right)
          \left(A - \frac{E}{n-2} {T}{A} \right)
          \cdots
          \left(A - \frac{E}{2} {T}{A} \right)
          (A - {E}{T}{A})
\end{equation}
Da $E_n$ nur in der Hauptdiagonalen von $0$ verschiedene Elemente
besitzt,
l"a"st sich Term 1 in einem Schritt von \[ n \] Prozessoren berechnen.
Parallel dazu l"a"st sich Term 2 nach Satz \ref{SatzAlgRechnen} in
\[ \left\lceil \log(n) \right\rceil \] Schritten von
\[ \left\lfloor \frac{n}{2} \right\rfloor \] Prozessoren berechnen.

Anschlie"send ist die Ergebnismatrix von Term 1 mit dem Ergebnis von
Term 2 zu multiplizieren. Dies kann, wie bei der Berechnung von Term 1
in einem Schritt von \[ n \] Prozessoren durchgef"uhrt werden. Die
darauf folgende Matrizensubtraktion zur Berechnung von Term 3 kann in
einem Schritt von \[ n^2 \] Prozessoren erledigt werden.

Insgesamt kann Term 3 also in
\[ \left\lceil \log(n) \right\rceil + 2 \] Schritten von
\[ n^2 \] Prozessoren berechnet werden.

Zur Berechnung von $B_0$ sind $n$ Terme auf die gleiche Weise wie
Term 3 zu berechnen. Term 2 braucht f"ur all diese Terme nur einmal
berechnet zu werden. Insgesamt kann die Berechnung der $n$ Terme in
\[ \left\lceil \log(n) \right\rceil + 2 \] Schritten von
\[ n^3 \] Prozessoren erledigt werden.

Um das Endergebnis $B_0$ zu erhalten sind schlie"slich noch die
Ergebnismatrizen der $n$ Terme miteinander zu multiplizieren. Die
Anzahl der Schritte und Prozessoren daf"ur folgt aus den S"atzen
\ref{SatzAlgBinaerbaum} und \ref{SatzAlgMatMult}.
Zu beachten ist dabei,
da"s eine Verkn"upfung nicht in einem Schritt von
einem Prozessor durchgef"uhrt wird, sondern nach \ref{SatzAlgMatMult} in
\[ \lceil \log(n) \rceil + 1 \] Schritten von \[ n^3 \] Prozessoren.
Deshalb werden diese Matrizenmultiplikationen in
\[ (\lceil \log(n) \rceil + 1) \lceil \log(n) \rceil \] Schritten von
\[ n^3 \left\lfloor \frac{n}{2} \right\rfloor \] Prozessoren
durchgef"uhrt.

Insgesamt wird die Berechnung von $B_0$ also in
\[ \lceil \log(n) \rceil^2 + 2 \lceil \log(n) \rceil + 2 \] Schritten
von \[ n^3 \left\lfloor \frac{n}{2} \right\rfloor \] Prozessoren
durchgef"uhrt.

Um die Determinante zu berechnen, sind noch nacheinander durchzuf"uhren:
\begin{enumerate}
\item
      eine Matrizenmultiplikation nach Satz \ref{SatzAlgMatMult} in
      \[ \lceil \log(n) \rceil + 1 \] Schritten von \[ n^3 \] 
      Prozessoren,
\item 
      die Berechnung der Spur\footnote{siehe Definition
      \ref{DefTr} } einer Matrix nach Satz \ref{SatzAlgBinaerbaum} in
      \[ \lc \log(n) \rc \] Schritten von
      \[ \lf \frac{n}{2} \rf \]
      Prozessoren und
\item
      eine Division in einem Schritt von einem Prozessor.
\end{enumerate}
Diese drei Berechnungsstufen werden insgesamt in
\[ 2 \lceil \log(n) \rceil + 2 \] Schritten von \[ n^3 \] Prozessoren
durchgef"uhrt.

Die Berechnung der Determinanten mit Hilfe von Satz \ref{SatzFrame}
kann also in
\[ \lc \log(n) \rc^2 + 4 \lc \log(n) \rc + 4 \] Schritten
durchgef"uhrt werden. Die Anzahl der Prozessoren betr"agt
\begin{eqnarray*}
   &      & n^3 \lf \frac{n}{2} \rf
\\ & \leq & \lc \frac{n^4}{2} \rc
\end{eqnarray*}

Man erkennt, da"s C-Alg. keine Fallunterscheidungen verwendet. Dies ist
ein Vorteil bei der Konstruktion von Schaltkreisen, da somit keine
Teilschaltkreise f"ur einzelne Zweige entworfen werden m"ussen. Dadurch
wird ein Schaltkreis zur Determinantenberechnung mit Hilfe von C-Alg.
nicht unn"otig vergr"o"sert. Es wird sich zeigen, da"s die anderen
Algorithmen (BGH-Alg., B-Alg. und P-Alg.) die Eigenschaft fehlender 
Fallunterscheidungen ebenfalls besitzen. In dieser Hinsicht besitzt keiner
der Algorithmen einen Vorteil gegen"uber den anderen.

Vergleicht man den Aufwand an Schritten und Prozessoren mit dem der anderen
Algorithmen, zeigt sich, da"s C-Alg. bereits sehr effizient ist.

