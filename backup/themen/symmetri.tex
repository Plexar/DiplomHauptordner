\MyBeginDef
\label{DefSymmetrisch}
\index{symmetrisch} \index{Matrix!symmetrische}
    Gilt f"ur eine Matrix $A$ \[ A = A^T \MyKomma \] dann wird sie als
    {\em symmetrisch} bezeichnet.
\MyEndDef

\begin{lemma}
\label{SatzTransponiertNorm}
    F"ur jede Matrix $A$ gilt:
    \[ ||A||_2 = ||A^T||_2 \MyPunkt \]
\end{lemma}

\begin{lemma}
\label{SatzBeliebigSymmetrisch}
    Sei $A$ eine beliebige Matrix. Dann ist \[ A^T A \] symmetrisch.
\end{lemma}
\begin{beweis}
   Es gilt \[ (A^T A)^T = A^T(A^T)^T = A^T A \MyPunkt \]
\end{beweis}
 
\begin{lemma}
\label{SatzSymmetrischSpektral}
    F"ur jede symmetrische Matrix $A$ gilt:
    \[ ||A||_2 = \rho(A) \MyPunkt \]
\end{lemma}

