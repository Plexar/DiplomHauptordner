\begin{satz}[Logarithmus zur Basis $b$]
\label{SatzAlgLog}
\index{Algorithmus!parallele Logarithmusberechnung}
    Die Funktion
    \[ \lceil \log_b(n) \rceil \] l"a"st sich in
    \[ \max(2\lceil \log(n/b) \rceil, \, 1) \] Schritten von
    \[ \lf \frac{3n}{4b} \rf \] Prozessoren berechnen.
\end{satz}
\begin{beweis}
    Falls \[ n \leq b \] lautet das Ergebnis $1$.
    Andernfalls berechne 
    \[ b_i := b^i , \: i = 1, \ldots, \lceil n/b \rceil  \MyPunkt\]
    Dies kann nach Satz \ref{SatzAlgPraefix}
    von \[ \lf \frac{3n}{4b} \rf \] Prozessoren
    in \[ \lc \log(n/b) \rc \] Schritten geleistet werden.
    Suche dann unter den $b_i$
    dasjenige, f"ur das gilt
    \[ \forall j\neq i: (b_j < n) \vee (b_j > b_i) \]
    Dies kann nach \ref{SatzAlgSuche} von
    \[ \lf \frac{3n}{4b} \rf \] Prozessoren in
    \[ \lc \log(n/b) \rc \] Schritten durchgef"uhrt werden.
    Die Werte f"ur Schritte und
    Prozessoren zusammengenommen ergeben die Behauptung. 
    \mbox{\hspace{10em} \hfill}
\end{beweis}

