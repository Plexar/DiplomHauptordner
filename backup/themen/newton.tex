Dazu beobachten wir zun"achst, da"s unter der Bedingung, da"s 
\equref{EquResidualLessOne} erf"ullt ist, auch gilt: 
\Beq{EquPanInverseSumme}
    A^{-1} = \lb \sum_{i=0}^{\infty}R^i(B) \rb B \MyKomma
\Eeq
Diese Gleichung erh"alt man folgenderma"sen:
\[ A^{-1} = (A^{-1}B^{-1})B = (BA)^{-1}B = (E_n - R(B))^{-1}B =
   \lb \sum_{i=0}^{\infty}R^i(B) \rb B \MyPunkt
\]

Betrachten wir nun:
\begin{eqnarray}
    B_k^{*} & := & 
                \lb \prod_{h=0}^{k-1}(E_n + R(B))^{2^h} \rb B \nonumber \\
            & = & (\:(E_n + R^2(B))\,(E_n + R^4(B))\,\dots \, 
                  (E_n + R^{2^{k-1}}(B))\: ) B  \nonumber \\
            & = & \lb \sum_{i=0}^{2^k-1} R^i(B) \rb B \label{TermPanBStern}
\end{eqnarray}
Subtrahiert man $B_k^{*}$ bzw. Term \equref{TermPanBStern} auf beiden 
Seiten von Gleichung \equref{EquPanInverseSumme}, erh"alt man:
\Beq{EquPanInverseDifferenz}
    A^{-1} - B_k^{*} = \lb \sum_{i=2^k}^{\infty} R^i(B) \rb B \MyPunkt 
\Eeq
Zur Abk"urzung wird definiert
\[ q:= ||R(B)|| \MyPunkt \]
Bildet man die Norm von beiden Seiten von \equref{EquPanInverseDifferenz},
erh"alt man mit Hilfe der in Definition \ref{DefMatrixNorm} erw"ahnten
Eigenschaften von Matrixnormen:
\Beq{EquPanDifferenzUngleichung}
    ||A^{-1} - B_k^{*}|| \leq \sum_{i=2^k}^{\infty} q^i ||B|| 
    = q^{2^k} \frac{ ||B|| }{ 1 - q }  \MyPunkt 
\Eeq
$$$$$

Vergleicht man \equref{EquNewtonBedingung} und \ref{SatzNormNaheInverse},
erkennt man eine Bedingung, die man nicht "ubersehen darf:
um eine Matrix $A$ durch Berechnung einer N"aherungsinversen nach dem
in Unterkapitel \ref{SecGuessInverse} und anschlie"sender Anwendung der
Newton-Methode n"aherungsweise zu invertieren, mu"s diese Matrix 
die Bedingung
\Beq{EquWellConditioned}
     \cond(A) < n^{O(1)} 
\Eeq erf"ullen.

