 Fehler in diesem Textteil nicht behoben !

\MySection{Determinantenberechnung durch L"osung verwandter Probleme}

Determinanten werden bei verschiedenen Verfahren dadurch berechnet, da"s
ein verwandtes Problem gel"ost wird, um dann aus dem
erhaltenen Ergebnis die Determinante zu gewinnen. In diesem Unterkapitel
wird untersucht, auf welche Weise die Probleme $G$, $I$ und $P$ zur
Determinantenberechnung verwendet werden k"onnen.

Sei $C$ ein Problem der Gr"o"se $n$
und $A$ ein Algorithmus zur L"osung dieses Problems.
Dann wird die parallele Zeitkomplexit"at des Algorithmus mit
\[ s_{C,A}(n) \] bezeichnet und sein Parallelisierungsgrad mit
\[ p_{C,A}(n) \] Wenn der Bezug auf den Algorithmus aus dem Zusammenhang
hervorgeht, oder der Algorithmus unbestimmt ist, werden
\[ s_C(n) \] bzw. \[ p_C(n) \] benutzt.

Um die "Ubersichtlichkeit zu erh"ohen und Verweise zu erleichtern, erfolgt
die Formulierung der uns interessierenden Aspekte in Form mathematischer
S"atze.

In der benutzten Schreibweise f"ur Anzahlen von Schritten und
Prozessoren tauchen keine Verweise auf Algorithmen auf. Die Ausdr"ucke auf
den linken Seiten der Gleichungen beziehen jeweils auf den Algorithmus, der
im Beweis zum jeweiligen Satz angegeben wird. Bei den Ausdr"ucken auf den
rechten Seiten der Gleichungen wird f"ur das jeweilige Problem $C$
angenommen, da"s es einen Algorithmus gibt, der $C$ in $s_C(n)$ Schritten
bei einem Parallelisierungsgrad von $p_C(n)$ l"o"st.

\begin{korollar}
\label{SatzDdurchP}
\index{Determinante} \index{charakteristisches Polynom}
    \begin{eqnarray*}
        s_D(n) & = & s_P(n)
    \\  p_D(n) & = & p_P(n)
    \end{eqnarray*}
\end{korollar}
\begin{beweis}
    Wenn \[ p(\lambda) \] das charakteristische Polynom der Matrix $A$ ist,
    gilt \[ \det(A) = p(0) \] Dies wird bei Betrachtung der Definition des
    charakteristischen Polynoms \ref{DefCharPoly} deutlich.
\end{beweis}

Der folgende Satz liefert zwar ein f"ur uns unbefriedigendes Ergebnis, da
die L"osung linearer Gleichungssysteme zur Determinantenberechnung in
enger Beziehung steht \footnote{Literatur dazu ist in
Unterkapitel \ref{SecMatUndDet}) aufgelistet}, wird er hier trotzdem
angegeben.

\begin{satz}
\label{SatzDdurchG}
\index{Determinante} \index{lineares Gleichungssystem}
    \begin{eqnarray*}
        s_D(n)= O(s_G(n))
    \\  p_D(n)= O(n! * p_G(n))
    \end{eqnarray*}
\end{satz}
\begin{beweis}
    Zur Berechnung der Determinante wird Satz \ref{SatzCramer} benutzt.
    Da die Effizienz dieser Methode zu gering ist bez"uglich der
    Anzahl der besch"aftigten Prozessoren, wird der Beweis nur
    angedeutet.

    Setzt man \[ b= E_{|1} \] in Gleichung \equref{Equ1SatzCramer} ein,
    so erh"alt man:
    \begin{equation}
    \label{Equ1SatzDdurchG}
        x_i = \frac{ \det(A_{(1|i)}) }{ \det(A) }
    \end{equation}
    Mit Hilfe einiger Grundlagen "uber lineare Gleichungssysteme
    (z. B. \cite{MM64} ab Seite 30) kann man folgende Aussagen
    "uber Gleichung \equref{Equ1SatzDdurchG} machen:
    \begin{itemize}
    \item Da $A$ invertierbar sein soll, besitzt das
          inhomogene Gleichungssystem \equref{Equ1SatzCramer} eine
          nichttriviale L"osung (d. h. $x$ ist nicht der Nullvektor).
    \item Das bedeutet, da"s es ein $i$ gibt, so f"ur Gleichung
          \equref{Equ1SatzDdurchG} gilt \[ x_i \neq 0 \]
    \item Das bedeutet wiederum, da"s es ein $i$ gibt, so da"s
          \[ \det(A_{(1|i)}) \neq 0 \MyKomma \]
          da ja $A$ invertierbar und somit
          \[ \det(A) \neq 0 \]
          ist.
    \item Dies schlie"slich bedeutet, da"s die erw"ahnte Matrix
          $A_{(1|i)}$ ebenfalls invertierbar ist.
    \item Ein Problem ist es, da"s die Aussage nur {\em es gibt
          ein $i$} lautet und
          keinerlei Annahmen "uber die Gestalt der Matrix gemacht
          werden k"onnen, so da"s zus"atzlicher Aufwand n"otig ist,
          um das $i$ zu bestimmen. An dieser Stelle ist der Algorithmus
          erheblich verbesserungsf"ahig, da er alle M"oglichkeiten f"ur $i$
          ausprobiert, was von exponentiell vielen Prozessoren
          bewerkstelligt wird.
    \end{itemize}
    Man betrachtet nun ein Gleichungssystem, dessen
    Koeffizientenmatrix durch diese Matrix $A_{(1|i)}$ gebildet wird,
    behandelt es ebenso und erh"alt so eine Untermatrix
    $(A_{ (1|i)_{(1|k)} }$ f"ur ein geeignetes $k$, deren Determinante
    wiederum ungleich $0$ ist.

    Setzt man die Betrachtung in dieser Weise fort, erh"alt man
    $n-1$ Untermatrizen von $A$
    \[ \det(A_{(1,\ldots,i|k_1,\ldots,k_i)}) \neq 0 \]
    wobei $i$ die Werte von $1$ bis $n-1$ annimmt und die $k_i$
    Werte zwischen $1$ und $n$. Dabei gilt:
    \begin{itemize}
    \item Jede der Matrizen ist eine Untermatrix der n"achst
          gr"o"seren Matrix. Die Untermatrix wurde durch Streichen
          der ersten Zeile und einer geeigneten Spalte gewonnen.
    \item Jeder der Matrizen entspricht ein Wert $\dot{x}_i$,
          der aus der
          jeweiligen Gleichung \equref{Equ1SatzDdurchG}
          entsprechenden Gleichung berechnet wurde.
    \end{itemize}
    Berechnet man das Produkt der $ \dot{x}_i $,
    erh"alt man nach Vereinfachung die Gleichung:
    \[
        \prod_{i=1}^{n-1} \dot{x}_i =
            \frac{ \det( {A}_{ ( 1,\ldots,n-1|k_1,\ldots,k_{n-1}
                               )
                             }
                       )
                 }{
                   \det( {A}_{(|)} ) }
    \]
    Diese Gleichung ist "aquivalent zu:
    \begin{equation}
    \label{Equ2SatzDdurchG}
        \det(A)=
            \frac{ {a}_{nj} }{
                   \prod_{k=1}^{n-1} \dot{x}_k }
    \end{equation}
    Wobei gilt:
    \[ \{j\}= \{ 1,\ldots,n \} \setminus \{ k_1,\ldots,k_{n-1} \} \]

    Um nach Gleichung \equref{Equ2SatzDdurchG} die Determinante zu
    berechnen, sind also zur Auswertung der rechten Seite die
    Gleichungssysteme zur Berechnung der $\dot{x}_k$ zu l"osen.

    Die beschriebene Vorgehensweise zur Berechnung der Determinante
    hat, wie bereits erw"ahnt, den Nachteil, da"s der Wert von
    $i$ in Gleichung \equref{Equ1SatzDdurchG} nicht im voraus bekannt
    ist. Sobald die Determinante berechnet ist, entspricht ihr im
    obigen Verfahren eine Folge $v$ von Zahlen mit
    \[ v = (f(1),f(2),\ldots,f(n-1)) \]
    f"ur ein \[ f \in \permut_n \]
    Diese Zahlenfolge $v$ gibt die Untermatrizen von $A$ an, die
    als Koeffizientenmatrizen f"ur Gleichungssysteme benutzt
    wurden. Es sind die Matrizen
    \[ A_{(1,\ldots,i)|f(1),\ldots,f(i))} \]
    wobei $i$ die Werte von $1$ bis $n-1$ annimmt. Da
    $\permut_n$ $n!$ viele Elemente enth"alt, f"uhrt das ganze zu dem
    angegebenen Aufwand an Schritten und Prozessoren.
\end{beweis}

\begin{satz}
\label{SatzGdurchI}
\index{lineares Gleichungssystem}
\index{Inverse!einer Matrix}
    \begin{eqnarray*}
        s_G(n) & = & O(s_I(n))
    \\  p_G(n) & = & O(\max(p_I(n),n^2))
    \end{eqnarray*}
\end{satz}
\begin{beweis}
    Gegeben sei ein Gleichungssystem von $n$ Gleichungen mit $n$
    Unbekannten. In Matrizenschreibweise:
    \begin{equation}
    \label{EquNGleiNUnbek}
        Ax=b
    \end{equation}
    Dabei ist
    $A$ die Koeffizientenmatrix, $x$ der Vektor
    der $n$ Unbekannten und
    $b$ der Vektor der $n$ Konstanten auf den rechten Seite der
    Gleichungen.
    Zuerst wird $A$ in $s_I(n)$ Schritten von $p_I(n)$ Prozessoren
    invertiert, wodurch \equref{EquNGleiNUnbek} die Form
    \[ x=A^{-1}b \] bekommt. Als Folgerung aus \ref{SatzAlgMatMult}
    kann die Multiplikation von $A^{-1}$ mit $b$ in
    \[ \lceil \log(n) \rceil + 1 \] Schritten von \[ n^2 \] Prozessoren
    durchgef"uhrt werden, da $b$ nur ein Vektor der L"ange $n$ und keine
    $n \times n$-Matrix ist. Es werden insgesamt
    \[ s_I(n) + \lceil \log(n) \rceil + 1 \] Schritte ben"otigt. Mit
    Satz \ref{SatzUntereSchranke} folgt die Behauptung.
\end{beweis}

\begin{korollar}
\label{SatzDdurchI}
\index{Determinante} \index{Inverse!einer Matrix}
    \begin{eqnarray*}
        s_D(n) & = & O(s_I(n))
    \\  p_D(n) & = & O(n! * p_I(n))
    \end{eqnarray*}
\end{korollar}
\begin{beweis}
    Die Behauptung folgt aus \ref{SatzDdurchG} und \ref{SatzGdurchI}.
\end{beweis}

Die vorangegangenen S"atze liefern insbesondere folgende Ergebnisse:
\begin{itemize}
\item
      Nach \ref{SatzDdurchP} ist die L"osung von $D$ mit Hilfe eines
      Algorithmus f"ur $P$ eine wichtige M"oglichkeit.
\item
      Nach \ref{SatzGdurchI} ist die L"osung von $G$ mit Hilfe eines
      Algorithmus f"ur $I$ ebenfalls eine bemerkenswerte M"oglichkeit.
      Da uns die Berechnung von Determinanten interessiert, ist sie
      f"ur uns jedoch nur indirekt, d. h. in Verbindung mit weiteren
      Algorithmen, von Bedeutung.
\item
      Da nach \ref{SatzDdurchG} die L"osung von $D$ mit Hilfe der L"osung
      von $G$ in der beschriebenen Weise exponentiell viele Prozessoren
      besch"aftigt, ist diese M"oglichkeit f"ur uns nicht von Interesse.
\end{itemize}

