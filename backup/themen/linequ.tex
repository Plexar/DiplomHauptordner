% **************************************************************************

\MySection{L"osbarkeit linearer Gleichungssysteme}
\label{SecLinEqu}
\index{Gleichungssysteme}

Da lineare Gleichungssysteme eine wichtige Rolle bei Verst"andnis
noch folgender Ausf"uhrungen spielen, werden sie in diesem Unterkapitel
n"aher betrachtet. Die nun folgenden Grundlagen aus der linearen 
Algebra sind in der in \ref{SecMatUndDet} aufgelisteten Literatur
ausf"uhrlich behandelt. Wir beschr"anken uns hier auf die f"ur
den nachfolgenden Text wichtigen Sachverhalte.

F"ur die weiteren Beschreibungen ist folgender Begriff von 
Bedeutung:

\MyBeginDef
\label{DefKern}
\index{Kern!einer linearen Abbildung}
    Die Menge aller Vektoren x,
    f"ur die bei einer gegebenen $m \times n$-Matrix $A$ gilt
    \Beq{EquKern}
        A x = 0_n
    \Eeq
    wird als {\em Kern der Matrix A} , kurz $\MyKer(A)$,
    bezeichnet \footnote{ In der Literatur wird h"aufig an dieser
    Stelle der Kern einer linearen Abbildung betrachtet. Da theoretischen
    Hintergr"unde hier nicht von Interesse sind, wird die Betrachtung auf
    Matrizen beschr"ankt.} .
\MyEndDef

F"ur unsere Zwecke ben"otigen wir noch zwei weitere Feststellungen:

\begin{bemerkung}
\label{SatzKernUnterraum}
    Betrachtet man \ref{DefUnterraum}, erkennt man,
    da"s alle $x$, die Gleichung \equref{EquKern} erf"ullen,
    einen Unterraum bilden.
\end{bemerkung}

Der Rang einer Matrix und die Dimension von $\MyKer(A)$ h"angen auf
eine f"ur uns wichtige Weise zusammen:

\begin{lemma}
\label{SatzDimKer}
    Sei $V$ ein $K$-Vektorraum der Dimension $n$ und
    $W$ ein $K$-Vektorraum der Dimension $m$. Sei
    \[ f: V \rightarrow W \] eine lineare Abbildung mit der
    entsprechenden $m \times n$-Matrix $A$. Dann gilt:
    \[ \rg(A) + \dim(\MyKer(A)) = \dim(V) \MyPunkt \]
\end{lemma}
\begin{beweis}
    Zum Beweis der obigen Gleichung wird die Dimension von 
    $\MyKer(A)$ in Abh"angigkeit vom Rang von $A$ betrachtet.

    Die Matrix $A$ habe den Rang $r$. Die maximale Anzahl linear
    unabh"angiger Spaltenvektoren betr"agt also $r$. O. B. d. A. seien
    die ersten $r$ Spaltenvektoren linear unabh"angig.
    
    Es wird in zwei Schritten gezeigt, da"s in $\MyKer(A)$ die maximale 
    Anzahl der linear unabh"angigen Vektoren, also die Dimension von $A$,
    $n-r$ betr"agt.
    
    \begin{itemize}
    \item Bezeichne $a_i$ den $i$-ten Spaltenvektor von $A$. Der Kern 
          von $A$ ist die Menge
          \begin{eqnarray*}
              & & \{ x \in V \MySetProperty A x = 0_m \} \\
              & = & \{ x \in V \MySetProperty
               a_1 x_1 + a_2 x_2 + \cdots + a_n x_n = 0_m \} \MyPunkt
          \end{eqnarray*}
          Die Dimension dieser Menge ist die maximale Anzahl linear
          unabh"angiger Vektoren $x$, die in ihr enthalten sind. Die
          Elemente eines Vektors $x$ kann man, wie an der obigen
          Darstellung ersichtlich ist, als Faktoren in einer
          Linearkombination von Spaltenvektoren von $A$ betrachten.
          Anhand der Voraussetzungen, die f"ur die Spaltenvektoren
          gelten, lassen sich Aussagen f"ur die Vektoren $x$ machen.
          
          Da $r+1$
          Spaltenvektoren von $A$ immer linear abh"angig sind, ist es
          m"oglich, durch Linearkombination von jeweils $r+1$ 
          Spaltenvektoren den Nullvektor zu erhalten. Sind die "ubrigen 
          $n-(r+1)$ Spaltenvektoren nicht an einer Linearkombination 
          beteiligt entspricht das einer Null als entsprechendes Element 
          von $x$.
          
          Es gibt $n-r$ M"oglichkeiten,
          zu den ersten $r$ linear unabh"angigen Spaltenvektoren von $A$
          genau einen weiteren
          auszusuchen. Aufgrund der Verteilung der Nullen in den 
          entsprechenden Vektoren $x$ mu"s es mindestens $n-r$ linear 
          unabh"angige Vektoren im Kern von $A$ geben.
    \item 
          Angenommen, die Anzahl der linear unabh"angigen Vektoren 
          in $\MyKer(A)$ ist gr"o"ser als $n-r$. O. B. d. A. seien die 
          ersten $n-r$ dieser Vektoren
          so gew"ahlt, da"s bei jedem dieser Vektoren unter den
          Vektorelementen $x_{r+1}$ bis $x_n$ genau eines ungleich Null ist.
          Aus den vorangegangenen Ausf"uhrungen folgt, das solche
          Vektoren existieren m"ussen. Die Vektoren werden mit 
          $v_{r+1}, \, \ldots, \, v_n$ bezeichnet. F"ur den Vektor $v_i$
          sei das $i$-te Vektorelement ungleich Null.
          
          Da die ersten $r$ Spaltenvektoren von $A$ linear unabh"angig sind,
          gibt es f"ur die ersten $r$ Elemente jedes Vektors $v_i$
          keine Wahlm"oglichkeit, sobald das $i$-te Element im
          Rahmen der Randbedingungen dem Wert nach festliegt.

          Sei $w$ ein weiterer Vektor aus $\MyKer(A)$, der zu den Vektoren
          $v$ linear unabh"angig ist. Wird das $j$-te Element eines Vektors
          $v_i$ mit $v_{i,j}$ bezeichnet, und das $k$-te Element von $w$
          mit $w_k$, dann gibt es Faktoren $d_{r+1}, \, \ldots , \, d_n$,
          so da"s
          \[ \forall r+1 \leq k \leq n : \: w_k = d_k * v_{k,k} \MyPunkt \]
          
          Ebenso wie f"ur die Vektoren $v$ liegen damit auch die
          ersten $r$ Elemente des Vektors $w$ fest. Sie k"onnen aus anhand
          der obigen Beziehung aus den Elementen $r+1$ bis $n$ der
          Vektoren $v$ und $w$ berechnet werden, sofern $w$ "uberhaupt die
          Rahmenbedinungen erf"ullt und in $\MyKer(A)$ ist.

          Damit $w$ dennoch zu den Vektoren $v$ linear unabh"angig ist,
          mu"s es einen $r+1$-ten Spaltenvektor von $A$ geben, der zu den
          ersten $r$ Spaltenvektoren linear unabh"angig ist, und der
          bei der Zusammenstellung der Vektoren $v$ mit Hilfe der
          Betrachtung von Linearkombinationen (s. o.) noch nicht benutzt
          wurde. Dies ist jedoch im Widerspruch zu den Voraussetzungen.
    \end{itemize}
    Somit folgt:
    \[ \rg(A) + \dim(\MyKer(A)) = r + (n-r) = n = \dim(V) \MyPunkt \]
\end{beweis}
Mit Hilfe von \ref{SatzKernUnterraum} und \ref{SatzDimKer} k"onnen wir
die f"ur uns wichtigen Eigenschaften linearer Gleichungssysteme
betrachten.

Ein lineares Gleichungssystem besteht aus $n$ Gleichungen mit $m$
Unbekannten:
\begin{eqnarray*}
    a_{1,1} x_1 + a_{1,2} x_2 + \cdots + a_{1,n} x_n & = & b_1 \\
    \vdots & \vdots & \vdots \\
    a_{n,1} x_1 + a_{n,2} x_2 + \cdots + a_{m,n} x_n & = & b_m
\end{eqnarray*}
Die $x_i$ sind die zu bestimmenden Unbekannten. Betrachtet man die
gegebenen Konstanten $a_{i,j}$ und $b_i$ als Elemente einer Matrix
bzw. eines Vektors kann man das Gleichungssystem auch kompakter
darstellen:
\[ A x = b \MyPunkt \]
Dabei ist $A$ ein $n \times m$-Matrix und $b$ ein Vektor der L"ange $m$.
Ist $b= 0_m$, so bezeichnet man das Gleichungssystem als
homogen \index{Gleichungsystem!homogenes}.
Ist $b \neq 0_m$, so bezeichnet man das Gleichungssystem als
inhomogen \index{Gleichungssystem!inhomogenes}.

Uns interessieren die L"osungsmengen eines solchen Gleichungssystems.

\begin{satz}
\label{SatzLoesungsraum}
    Sei $V$ der zugrundeliegende $K$-Vektorraum der Dimension $n$.
    Die L"osungsmenge $L(A,0_m)$ des homogenen Gleichungssystems
    \[ A x = 0_m \] ist ein Unterraum von $V$ der Dimension
    \[ n - \rg(A) \MyPunkt \]
\end{satz}
\begin{beweis}
    Die L"osungsmenge $L(A,0_m)$ ist der Kern von $A$. Damit folgt
    die Behauptung aus \ref{SatzKernUnterraum} und
    \ref{SatzDimKer}.
\end{beweis}

Ist $\rg(A)=n$, gilt also $L(A,0_m) = \{ x_n \}$.

Uns interessieren auch die L"osungen von inhomogener Gleichungssysteme.
Dazu betrachtet man die erweiterte Matrix $[A,b]$, die aus der Matrix
$A$ und dem Vektor $b$ als $n+1$-te Spalte besteht:

\begin{satz}
\label{SatzRangGleich}
    Die Gleichung
    \Beq{EquInhomogen}
        A x = b \MyPunkt
    \Eeq
    ist genau dann l"osbar, wenn gilt
    \Beq{EquRangGleich}
        \rg(A) = \rg([A,b]) \MyPunkt
    \Eeq
\end{satz}
\begin{beweis}
    Man kann die linke Seite von \equref{EquInhomogen} als 
    Linearkombination von Spaltenvektoren von $A$ betrachten. Die Faktoren der
    Linearkombination bilden die Elemente des L"osungsvektors $x$.

    Der Vektor $b$ ist genau dann als Linearkombination der
    Spaltenvektoren von $A$ darstellbar, wenn die maximale Anzahl linear
    unabh"angiger Vektoren in $A$ und $[A,b]$ gleich ist. Dies ist
    gleichbedeutend mit der G"ultigkeit von \equref{EquRangGleich}.
\end{beweis}

Dieser Satz f"uhrt zu einer f"ur uns bedeutsamen Charakterisierung
der L"osbarkeit:

\begin{korollar}
\label{SatzGenauEine}
    Ist \equref{EquInhomogen} l"osbar und $\rg(A) = n$,
    also $n \leq m$, dann sind die Spaltenvektoren von $A$
    linear unabh"angig
    und werden alle f"ur die Linearkombination zur Darstellung von $b$
    ben"otigt. Es gibt in diesem Fall genau einen Vektor $x$, der
    \equref{EquInhomogen} l"ost.
\end{korollar}

Ist $\rg(A) = m$, also $m \leq n$, dann ist \equref{EquInhomogen} f"ur
jedes $b$ l"osbar, denn es gilt allgemein
\[ \rg(A) \leq \rg([A,b]) \leq m \MyPunkt \] F"ur $\rg(A) = m$ folgt
daraus mit Hilfe der vorangegangenen Bemerkungen die L"osbarkeit.

% $$$ hier evtl. noch Wegener Satz 7.4 und Hilfssatz 7.7

