
\begin{korollar}[Parallele Suche]
\label{SatzAlgSuche}            % $$$ wird in 'SatzAlgLog' benutzt
\index{Algorithmus!parallele Suche}
    Sei $X$ eine Menge mit $n$ Elementen. Die Elemente seien
    \[ x_1, x_2, \ldots, x_n \]
    Sei \[ \circ : X \times X \rightarrow X \] eine zweistellige Operation,
    genannt {\em Suchoperation}.
    Sie sei assoziativ und liefere als Ergebnis dasjenige
    Element der beiden, auf
    die sie angewendet wird, das eine vorgegebene Bedingung erf"ullt.
    Falls beide die Bedingung erf"ullen oder nicht erf"ullen, ist das
    Ergebnis ein beliebiges der beiden Elemente.

    Es l"a"st sich aus der Menge $X$ parallel in
    \[ \left\lceil \log(n) \right\rceil \] Schritten von
    \[ \left\lfloor \frac{n}{2} \right\rfloor \] Prozessoren ein
    Element heraussuchen, das die Bedingung erf"ullt.
\end{korollar}
\begin{beweis}
    Die Behauptung folgt aus Satz \ref{SatzAlgBinaerbaum}.
\end{beweis}

Es folgen ein paar Beispiele f"ur die Suchoperationen.
\begin{itemize}
\item
     ({\em kleinstes} Element; analog: {\em gr"o"stes} Element)
     \[ x \circ y :=
                 \left\{
                     \begin{array}{lcr}
                         x & : & x < y \\
                         y & : & x \geq y
                     \end{array}
                 \right.
     \]
\item
     (Element, welches {\em n"achst gr"o"ser} ist als ein vorgegebenes $z$;
      analog: {\em gr"o"ser oder gleich})

     In diesem Fall ist die zu pr"ufende Bedingung etwas komplizierter.
     Um die Assoziativit"at sicherzustellen, ist $y$ immer dann das 
     Ergebnis, wenn $x$ die Bedingungen nicht erf"ullt, um das Ergebnis 
     zu sein. Das wird mit folgender Definition ausgedr"uckt:
     \[ x \circ y := \left\{
                     \begin{array}{rcl}
                         x & : & 
                           \begin{array}{rcl}    
                              ((x \leq y) & \und & ((z<x) \oder (y \leq z)))
                           \\ \oder & & 
                           \\ ((y<x) & \und & (y \leq z))
                           \end{array}
                     \\  y & : & \mbox{sonst}
                     \end{array}
                     \right.
     \]
\item
     (Element, welches {\em gleich} einem vorgegebenen $z$ ist;
      analog: {\em ungleich} )
     \[ x \circ y := \left\{
                     \begin{array}{lcr}
                         x & : & x = z \\
                         y & : & x \neq z
                     \end{array}
                 \right.
     \]
\end{itemize}

Das f"ur uns wichtige in \ref{SecModell} beschriebene
Berechnungsmodell erm"oglicht eine einfachere M"oglichkeit zur
{\em Suche nach Gleichheit} --- die {\em Suche nach Ungleichheit}
funktioniert analog --- als mit der obigen Suchoperation, wie der
folgende Satz zeigt:

\begin{satz}[Suche nach Gleichheit]
\label{SatzAlgSucheGleich}
\index{Algorithmus!parallele Suche nach Gleichheit}
    Seien \[ x_1, \ldots , x_n \] sowie $z$ vorgegebene Werte.
    Durch \[ n \] Prozessoren kann in einem Schritt von den $x_i$
    irgendeins herausgesucht werden, das gleich $z$ ist.
\end{satz}
\begin{beweis}
    Prozessor $i$ pr"uft $x_i$ auf Gleichheit zu $z$. Ist dies gegeben,
    so ist $i$ das gesuchte Ergebnis. In dem Fall, da"s die Bedingung f"ur
    mehrere $i$ erf"ullt ist, wird irgendeins dieser $i$ als endg"ultiges
    Ergebnis angesehen \footnote{Das verwendete Rechnermodell (siehe
    \ref{SecModell}) erlaubt dieses Vorgehen sehr effizient.} .
    Es werden offensichtlich \[ n \] Prozessoren besch"aftigt, die das 
    Problem in einem Schritt l"osen.
\end{beweis}

