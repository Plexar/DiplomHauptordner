\begin{lemma}
\label{SatzSumK}
    Sei $n \in \Nat_0$. Dann gilt:
    \[ \sum_{k=1}^n k = \frac{ n(n+1) }{ 2 } \]
\end{lemma}
\begin{beweis}
    Der Beweis erfolgt durch Induktion "uber $n$.
    \begin{MyDescription}
    \MyItem{$n=1$}
        Der Satz besitzt offensichtlich G"ultigkeit.
    \MyItem{$n>1$}
        Der Satz gelte f"ur $n<i$. Es ist zu zeigen, da"s er dann auch
        f"ur $n=i$ gilt:
        \begin{eqnarray*}
            \sum_{k=1}^i k & = & i + \sum_{k=1}^{i-1} k \\
            & = & 
                i + \frac{ (i-1)i }{ 2 } \\
            & = & \frac{ 2i+i^2-i }{ 2 } \\
            & = & \frac{ i(i+1) }{ 2 }
        \end{eqnarray*}
    \end{MyDescription}
\end{beweis}

\begin{lemma}
\label{SatzSumK2}
    Sei $n \in \Nat_0$. Dann gilt:
    \[ \sum_{k=1}^n k^2 = \frac{ n(n+1)(2n+1) }{ 6 } \]
\end{lemma}
\begin{beweis}
    Der Beweis erfolgt durch Induktion "uber $n$.
    \begin{MyDescription}
    \MyItem{$n=1$}
        Der Satz ist offensichlich richtig.
    \MyItem{$n>1$}
        Der Satz gelte f"ur $n<i$. Es ist zu zeigen, da"s er dann auch
        f"ur $n=i$ gilt:
        \begin{eqnarray*}
            \sum_{k=1}^i k^2 & = & i^2 + \sum_{k=1}^{i-1} \\
            & = & 
                i^2 + \frac{ (i-1)i(2(i-1)+1) }{ 6 } \\
            & = & 
                \frac{ 6i^2+(i^2-i)(2i-1) }{ 6 } \\
            & = & 
                \frac{ 2i^3 + 3i^2 + i }{ 6 } \\
            & = & 
                \frac{ i(i+1)(2i+1) }{ 6 }
        \end{eqnarray*}
    \end{MyDescription}
\end{beweis}

