\MySubSubSection{Genauigkeit der Rechnung}
\label{BGH82Par7}
Um die Besprechung der hier vorgeschlagenen Art der Vermeidung von
Divisionen abzuschlie"sen, fehlt noch die Betrachtung der Genauigkeit
der Berechnungen. Zu diesem Zweck wird im vorliegenden Abschnitt eine 
Art der Zahlendarstellung beschrieben und im Hinblick auf die
Rechengenauigkeit hin untersucht.

\MyBeginDef
    Seien $\base$, $\accuracy$ und $\ExpBound$ Elemente von $\Nat$.
    Dann hei"st die Menge \[ \LogRep(\base,\accuracy,\ExpBound) \]
    {\em Menge der logarithmischen Darstellungen} zu $\base$, $\accuracy$
     und $\ExpBound$.
    $\base$ wird als {\em Basis}, $\accuracy$ als {\em Genauigkeit} und
    $\ExpBound$ als {\em Schranke der Exponenten} bezeichnet.
    
    Sei $a$ ein Vektor mit $\accuracy$ vielen Komponenten die mit $a_i$
    bezeichnet werden, wobei gilt \[ 0 \leq i \leq \accuracy-1 \]
    Die Elemente
    von $\LogRep(\base,\accuracy,\ExpBound)$ sind alle Terme der Form
    \[ v \base^m\sum_{i=0}^{\accuracy-1}a_i\base^i \]
    mit
    \begin{eqnarray*}
        m & \in & \Nat_0 \\
        m & \leq & \ExpBound \\
        \forall 0 \leq i \leq \accuracy-1 & : & a_i \in \Nat_0 \\
        \forall 0 \leq i \leq \accuracy-1 & : & 0 \leq a_i \leq \base-1 \\
        v & \in & \{ 1,-1 \}
    \end{eqnarray*}
    Dabei wird $m$ als {\em Exponent der logarithmischen Darstellung}
    und die Summe als {\em Mantisse der logarithmischen
    Darstellung} bezeichnet. $v$ ist das {\em Vorzeichen}.
\MyEndDef
Ein Beispiel f"ur eine Menge {\em logarithmischer Darstellungen} w"are
\[ \LogRep(10,20,200) \]
was folgendes bedeutet:
\begin{itemize}
    \item Die Darstellung erfolgt zur Basis $10$, d. h. im gewohnten
          Dezimalsystem.
    \item Es wird mit $20$ Stellen Genauigkeit dargestellt, d. h. eine
          Zahl kann, abgesehen von mit Hilfe des {\em Exponenten}
          hervorgerufenen f"uhrenden oder folgenden Nullen, aus bis zu
          $20$ Ziffern bestehen.
    \item Die Schranke f"ur {\em Exponenten} ist $200$, d. h. der
          maximale Betrag einer Zahl, die dargestellt werden kann, ist
          \[ 10^{200}*(10^{20}-1) \]
          also eine Zahl mit $220$ Stellen, von denen jedoch die $200$
          nur aus $0$ bestehen.
    \item Ein Element von \[ \LogRep(10,20,200) \] k"onnte z. B. sein
          \[ 49845943345 * 10^{145} \]
          $a_i$ mit \[ \forall j \geq i : a_j= 0 \] werden beim
          Aufschreiben weggelassen.
\end{itemize}

\MyBeginDef
    Sei \[ L = \LogRep(\base,\accuracy,\ExpBound) \] gegeben. Eine Zahl
    \[ a \in \Rationals \] hei"st {\em logarithmisch darstellbar
    bez"uglich $L$}, wenn es ein \[ l \in L \] gibt, so da"s die Auswertung
    des Terms $l$ den Wert von $a$ ergibt. In diesem Fall gilt also
    \[ l = a \]
\MyEndDef

\begin{bemerkung}
    Seien $L$ und $L'$ zwei Mengen logarithmischer Darstellungen. Es gelte
    \[ \accuracy' > \accuracy \] 
    Dabei bezeichne $\accuracy$ die Genauigkeit von $L$ und $\accuracy'$ 
    diejenige von $L'$. Sei $z$ logarithmisch darstellbar 
    bez"uglich $L$. Dann $z$ auch logarithmisch darstellbar bez"uglich
    $L'$.
\end{bemerkung}

\begin{bemerkung}
\label{OhnedivSatz4}
    F"ur jedes \[ z \in \Rationals \] gibt es eine Menge $L$ von
    logarithmischen Darstellungen, so da"s $z$ logarithmisch darstellbar
    bez"uglich $L$ ist. 

    Dazu ist gegebenenfalls die Basis $\base$ so zu w"ahlen, da"s keine
    Periode entsteht, wie z. B. bei \[ \frac{1}{3} \] beim Versuch einer
    Darstellung zur Basis $10$.
\end{bemerkung}

\MyBeginDef
    Sei \[ L = \LogRep(\base,\accuracy,\ExpBound) \] eine Menge 
    logarithmischer Darstellungen.
    Seien $x$, $y$ und $z$ logarithmisch darstellbar.
    Sei \[ f: \Rationals \times \Rationals \rightarrow \Rationals \]
    eine Abbildung und es gelte \[ f(x,y)=z \]
    Sei $z'$ logarithmisch darstellbar. Es gelte \[ n \in \Nat_0 \] und
    \[ j \in \Nat_0 \] $n$ bezeichnet die 
    Anzahl der Stellen, die {\em genau} (s. u.) sollen. $j$ bezeichnet die
    Stelle in der logarithmischen Darstellung, ab der die f"uhrenden 
    (und somit nicht zu beachtenden) Nullen der von $z'$ beginnen.
    D. h. $z'$ besitzt \[ \accuracy-j \] f"uhrende Nullen in der
    logarithmischen Darstellung, die f"ur die Betrachtung der Genauigkeit
    nicht beachtet werden.
    Damit nicht insgesamt von mehr Stellen die Rede ist, als
    eine Zahl in der logarithmischen Darstellung besitzen kann, mu"s
    also gelten \[ n+(\accuracy-j) \leq \accuracy \].
    Bei \[ j = \accuracy \] besitzt die Zahl keine f"uhrenden Nullen.
    $z'$ wird {\em auf n Stellen genau bez"uglich $z$} genannt, wenn gilt
    \begin{eqnarray*}
        \forall i \geq j & : & z'_i = 0 \\
        \frac{ |z - z'| }{ \base^m } & < & \frac{\base^(j-n)}{2} \\
    \end{eqnarray*}
    wobei $m$ der Exponent von $z'$ ist.

    Diese Definition gelte entsprechend auch f"ur $q$-stellige $f$ mit
    \[ q \neq 2 \] 
\MyEndDef
F"ur das $z$ in der obigen Definition erh"alt man ein passendes $z'$, wenn
man $z$ mit Hilfe der $4$-$5$-Rundung unabh"angig von der Position 
des Kommas auf $n$ f"uhrende Stellen rundet. 

\MyBeginDef
    Sei \[ z \in \Rationals \] und \[ n \in \Nat \]
    Sei $L$ eine Menge logarithmischer Darstellungen. Sei $j$ das Element 
    von $\Nat$, f"ur das gilt \[ \base^j \geq z \]
    Dann wird die {\em Rundungsfunktion auf $n$ f"uhrende Stellen} definiert
    durch 
    \[ \round(z,n):= 
           \left(
           \left\lfloor
               \frac{ z }{ \base^{j-n} } + \frac{1}{2}
           \right\rfloor
           \right) \base^{j-n}
    \]
\MyEndDef
Somit ist \[ \round(z,n) \] auf $n$ Stellen genau bez"uglich $z$.

\MyBeginDef
    Sei \[ z \in \Rationals \].
    Sei $L$ eine Menge logarithmischer Darstellungen.
    Dann hei"st $\RepErr(z)$ {\em Darstellungsfehler von $z$ bez"uglich
    $L$} und wird definiert durch \[ \RepErr(z):= |z-\round(z,\accuracy)| \]
\MyEndDef

\MyBeginDef
    Sei $L$ eine Menge logarithmischer Darstellungen.
    Sei \[ z \in \Rationals \].
    Dann ist die  {Anzahl n"otiger Stellen von $z$ bez"uglich $L$}
    definiert durch
    \[ \necess(z,L):= \min\{i: \forall j \geq i: z_j=0\} \]
    Abk"urzend wird auch einfach \[ \necess(z) \] geschrieben, wenn der
    Bezug auf $L$ offensichtlich ist.
\MyEndDef
Das bedeutet, da"s $0$ gar keine n"otigen Stellen besitzt und jede Zahl
ungleich $0$ mindestens eine.

Mit Hilfe von Bemerkung \ref{OhnedivSatz4}
kann man f"ur jedes \[ z \in \Rationals \] die Anzahl n"otiger
Stellen betrachten. Bei einer fest gew"ahlten Basis $\base$ kann von
den Grundrechenarten nur die Division ein Ergebnis liefert, das sich
bez"uglich dieser Basis nicht darstellen l"a"st.

Nach den Definitionen einiger Begriffe nun zur Betrachtung der Genauigkeit
der bis hierhin dargestellten M"oglichkeit zur Determinantenberechnung ohne
Divisionen:

Zun"achst wird betrachtet, wie sich die vier Grundrechenarten Addition,
Subtraktion, Multiplikation und Division im Hinblick auf die Exaktheit
der obigen {\em logarithmischen Darstellung} auswirken. Seien $a$ und $b$
{\em logarithmisch darstellbar} bez"uglich einer gegebenen Menge $L$
logarithmischer Darstellungen. Dann sind folgende Fragen von Interesse
(exemplarisch f"ur die Addition formuliert):
\begin{itemize}
    \item Ist auch \[ a+b \] logarithmisch darstellbar?
    \item Seien $necess(a)$ und $\necess(b)$ bekannt.
          Wie gro"s ist dann \[ necess(a+b) \]
    \item F"ur welches logarithmisch
          darstellbare $d$ ist \[ |a+b-\round(c,\accuracy)|-d \]
          minimal? W"unschenswert ist nat"urlich \[ d=0 \]
    \item Was ist zu tun, um den Darstellungsfehler bei Berechnungen
          minimal zu halten?
\end{itemize}
F"ur die anderen Grundrechenarten ist in den obigen Punkten jeweils das
$+$ durch $-$, $*$ bzw. $/$ zu ersetzen.

Allgemein ist das Ergebnis einer Verkn"upfung dann logarithmisch
darstellbar, wenn die Anzahl seiner n"otigen Stellen die Genauigkeit
$\accuracy$ nicht "ubersteigt. Wie entwickelt sich also die Anzahl der
n"otigen Stellen im Laufe einer Rechung? F"ur die Grundrechenarten wird
dies im folgenden kurz betrachtet. Dazu seinen $a$ und $b$ gegeben und 
$c$ jeweils das Ergebnis der Verkn"upfung von $a$ und $b$. Unter dem 
{\em schlimmsten Fall} wird der Fall verstanden, in dem 
\[ \necess(c)-\max(\necess(a),\necess(b)) \] maximal wird, da eine 
Steigerung der Anzahl der n"otigen Stellen die Rundung es Ergebnisses
erforderlich machen kann und dadurch zu Ungenauigkeiten f"uhrt.
\begin{itemize} 
    \item[Addition und Subtraktion]
         Der schlimmste Fall tritt ein, wenn $a$ und $b$ das gleiche 
         Vorzeichen besitzen und jede Stelle in der logarithmischen 
         Darstellung der beiden Zahlen den Wert $\base-1$ hat. In diesem
         Fall gilt \[ \necess(c) \leq \max(\necess(a),\necess(b))+1 \]
         Ein Beispiel ist
         \begin{eqnarray*}
             a & = & 9 \\
             b & = & 9 
         \end{eqnarray*}
         Somit ist \[ c = 18 \]
    \item[Multiplikation]
         Der schlimmste Fall tritt ein, wenn jede Stelle in der
         logarithmischen Darstellung von $a$ und $b$ den Wert $\base-1$ 
         besitzt. In diesem Fall gilt
         \[ \necess(c) \leq \necess(a)+\necess(b) \]
    \item[Division]
         Es sei \[ \frac{a}{b} \] zu berechnen. Es gibt zwei F"alle:
         \begin{itemize}
             \item[$b$ teilt $a$]
                  Dieser Fall ist das Gegenteil zu \[ a=bc \]. Somit gilt
                  \[ \necess(c) \leq \necess(a)-\necess(b)+1 \]
                  Ein Beispiel ist 
                  \begin{eqnarray*}
                      a & = & 99 \\
                      b & = & 1
                  \end{eqnarray*}
                  Dann ist \[ c = 99 \] was die obige Ungleichung erf"ullt.
             \item[$b$ teilt $a$ nicht]
                  In diesem Fall h"angt die Anzahl der n"otigen Stellen 
                  f"ur $c$ von den genauen Werten f"ur $a$ und $b$ ab. 
                  Da keine Teilbarkeit gegeben ist, gilt jedoch auf jeden
                  Fall \[ \necess(c) > \necess(a)-\necess(b)+1 \]. F"ur
                  einzelne Betrachtungen der Entwicklung der n"otigen 
                  Stellen im Laufe einer Rechnung k"onnte man f"ur den 
                  schlimmsten Fall der Division eine nach intuitiver
                  Beurteilung als ausreichend erscheinende Annahme treffen,
                  wie z. B. 
                  \[ \necess(c) \leq 2\max(\necess(a),\necess(b)) \]
         \end{itemize}
\end{itemize}
Es ist zu beachten, da"s selbst dann, wenn alle Eingabewerte f"ur einen
Algorithmus so gew"ahlt sind, da"s der schlimmste Fall bez"uglich der Anzahl
der n"otigen Stellen des Ergebnisses maximal oft auftritt, es bei l"angeren
Rechnungen nicht m"oglich ist, zu erreichen, da"s der schlimmste Fall bei
jeder Verkn"upfung auftritt. Im allgemeinen tritt er tats"achlich nur bei
einem relativ geringen Prozentsatz aller Vern"upfungen auf, so da"s die 
obigen Werte f"ur $\necess(c)$ als obere Schranke zu betrachten sind.

Darstellungsfehler entstehen dadurch, da"s
logarithmisch nicht darstellbare Ergebnisse durch Rundung darstellbar
gemacht werden. Nach der ersten Rundung im Laufe einer Rechnung ist das
betroffene Zwischenergebnis bez"uglich des tats"achlich richtigen Wertes
noch auf $\accuracy$ Stellen genau.

Man k"onnte jetzt sehr ausf"uhrliche
Betrachtungen dar"uber anstellen, wie sich Ungenauigkeiten bei l"angeren
Rechnungen auswirken. Bei der Betrachtung des im vorliegenden Text
darzustellenden Algorithmus zur Determinantenberechnung erkennt man jedoch,
da"s er, abgesehen von einer Stelle, keine gr"o"seren Probleme bei der
Genauigkeit aufwirft, als allgemein "ublich. Die eine Stelle, die
besondere Probleme macht und hier betrachtet werden soll, ist die Art
und Weise, wie Divisionen vermieden werden.

Dies geschieht mit Hilfe von Reihenentwicklungen. Diese
Reihen erfordern besondere Beachtung bei der Betrachtung der Genauigkeit
der Berechnungen weil sie
unendlich viele Glieder besitzen, die man nat"urlich nicht alle
ausrechnen kann. Somit ist es n"otig die Berechnung auf bestimmte Glieder zu
beschr"anken. Ihre Auswahl bestimmt entscheident die Genauigkeit der
Rechnung.

Dazu sei die Reihenentwicklung von \[ \frac{1}{1-g} \] im Hinblick auf die
oben beschriebene {\em logarithmische Darstellung} betrachtet. Die
Reihe zum obigen Term hat die Form \[ 1 + g + g^2 + g^3 + \ldots \] D. h.
das Glied $n$ der Reihe hat die Form \[ g^n \].

Es wird vorausgesetzt, da"s die Darstellungen zur Basis $10$ erfolgen und
eine Genauigkeit $\accuracy$ (s. o.) gegeben ist.
Wenn man beachtet, da"s im vorliegende Fall die m"oglichen Werte f"ur $g$
aus dem Intervall \[ (-0.1,0.1) \] stammen und unter dieser Voraussetzung
die m"oglichen logarithmischen Darstellungen der Glieder \[ g \] und 
\[ g^{\accuracy+1} \] miteinander vergleicht, erkennt man, da"s f"ur 
die Exponenten $m$ und $m'$ in der logarithmischen Darstellung der beiden 
Glieder gilt: \[ |m-m'| \geq \accuracy \]
Die Verkn"upfung zweier logarithmisch dargestellter Werte (s. o.), 
l"auft so ab,
da"s der Exponent des kleineren Wertes dem des gr"o"seren angeglichen
angeglichen wird, wobei die Mantisse nat"urlich Nachkommastellen bekommt,
die jedoch nach dem Angleichen durch Rundung beseitigt werden. Falls die
Differenz der Exponenten $\accuracy$ betr"agt kann bei einer Addition, wie
z. B. der Addition eines weiteren Reihengliedes zum bisherigen 
Zwischenergebnis, nach der Rundung noch maximal der Wert $1$ hinzu kommen.
Jedes Reihenglied h"oheren Grades w"urde nach Angleichen der Exponenten 
und Runden nur noch $0$ betragen.

Es ist also nicht sehr zweckm"a"sig, "uber \[ g^{\accuracy+1} \] hinaus
noch Glieder auszurechnen, da sie durch die Rundung auf $\accuracy$ Stellen
das Ergebnis in der logarithmischen Darstellung nicht mehr beeinfl"ussen
w"urden. Falls $g$ kein einzelner Wert oder eine einzelne Unbestimmte, 
sondern ein Polynom mit einem Grad gr"o"ser als 1 oder wiederum eine 
Potenzreihe ist, gen"ugt es demnach, alle homogenen Komponenten bis zum
Grad $\accuracy+1$ auszurechnen.
% $$$$ Referenz auf Def. von 'homogene Komponente'

Leider wird somit das Ziel, durch Vermeidung von Divisionen die
Genauigkeit m"oglichst vollkommen exakt zu machen, wiederum nicht erreicht.
Aber man kann bei einer gen"ugend hohen Wahl von $\accuracy$, wie z. B.
\[ 3n \] f"ur die Berechnung der Determinante einer $n \times n$-Matrix,
hoffen, da"s Ergebnis zumindest auf $n$ Stellen genau bez"uglich des
tats"achlichen Ergebnisses ist.


