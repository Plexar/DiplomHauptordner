\MySection{Einfache Voraussetzungen}

F"ur Leser, die mit dem Thema dieser Arbeit nicht vertraut sind, werden
in diesem Unterkapitel Ansatzpunkte zur Beseitigung von Unklarheiten 
angegeben.

Vorausgesetzt wird aus der linearen Algebra wird die Kenntnis der Begriffe
\begin{quote}
    Halbgruppe, Monoid, Gruppe, kommutative Gruppe, Ring, K"orper,
    Vektorraum, Abbildung, injektiv, surjektiv, bijektiv,
    Matrix, Addition von Matrizen, Multiplikation von Matrizen,
    Multiplikation von Matrizen mit Skalaren
\end{quote}
sowie ihrer Eigenschaften und Beziehungen zueinander.

Die Bezeichnung von Gr"o"senordnungen ist in der Informatik so 
selbstverst"andlich, da"s man u. U. Schwierigkeiten hat, eine genaue
Definition der Begriffe zu finden. Deshalb werden die Definitionen hier
noch einmal angegeben.
\MyBeginDef
    Die Menge 
    \[ 
         \{ g | \exists c \in \Rationals: \exists n_0 \in \Nat: 
                \forall n>n_0: c f(n) \geq g(n) \}
    \]
    hei"st 'Menge der Funktionen von {\em h"ochstens} der 
    Gr"o"senordnung $f$' und wird mit \[ O(f) \] bezeichnet.
\MyEndDef
\MyBeginDef
    Die Menge
    \[  \{ g | \forall c \in \Rationals: \exists n_0 \in \Nat: 
               \forall n>n_0: c f(n) \geq g(n) \}
    \]
    hei"st 'Menge der Funktionen von {\em geringerer} Gr"o"senordnung als 
    $f$' und wird mit \[ o(f) \] bezeichnet.
\MyEndDef % g=o(f) => f=\Omega(g)
\MyBeginDef
    Die Menge 
    \[ 
        \{ g | \exists c \in \Rationals: \exists n_0 \in \Nat: 
               \forall n>n_0: c f(n) \leq g(n) \}
    hei"st 'Menge der Funktionen von {\em mindestens} der Gr"o"senordnung
    $f$' und wird mit \[ \Omega(f) \] bezeichnet.
\MyEndDef
\MyBeginDef
    Die Menge
    \[
        \{ g | \exists c_1, c_2 \in \Rationals: \exists n_0 \in \Nat: 
               \forall n>n_0: c_1 f(n) \leq g(n) \leq c_2 f(n) \}
    \]
    hei"st 'Menge der Funktionen von {\em genau} der Gr"o"senordnung
    $f$' und wird mit \[ \Theta(f) \] bezeichnet.
\MyEndDef
Leider ist es in der Literatur oft "ublich, f"ur \[ g \in O(f) \] auch
\[ g = O(f) \] zu schreiben (entsprechend f"ur die anderen
Gr"o"senordnungen).
Da diese Schreibweise mathematisch unsinnig ist, werden wir sie vermeiden.

% **************************************************************************

