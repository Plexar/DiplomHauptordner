$$$$$ noch an neue Analyse von \ref{SatzAlgPraefix} anzupassen:
\begin{satz}[Ganzzahlige Wurzel]
\label{SatzAlgWurzel}
\index{Algorithmus!parallele Wurzelberechnung}
    Seien $b$ und $n$ nat"urliche Zahlen.
    \[ \left\lceil \sqrt[b]{n} \right\rceil \] l"a"st sich in
    \[ \lceil \log(b) \rceil + 1 \] Schritten von
    \[ n \left\lfloor \frac{b}{2} \right\rfloor \] Prozessoren berechnen.
\end{satz}
\begin{beweis}
    Berechne zun"achst
    \[ w_i := \underbrace{ i * i * \ldots * i }_{\mbox{$b$ mal}}
       \, , \: 1 \leq i \leq n
    \]
    Dies kann nach \ref{SatzAlgPraefix} in
    \[ \lceil \log(b) \rceil \] Schritten von
    \[ n \left\lfloor \frac{b}{2} \right\rfloor \] Prozessoren geleistet 
    werden.

    Pr"ufe dann f"ur \[ 1 \leq i \leq n \] die Bedingung
    \[ w_i \geq n > w_{i-1} \] Dabei gelte \[ w_0 := 0 \]
    Das $i$, das diese Bedingung erf"ullt, ist der gesuchte Wert.
    Diese Pr"ufung kann in einem Schritt von $n$ Prozessoren
    durchgef"uhrt werden.
\end{beweis}

