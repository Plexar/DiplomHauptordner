% **************************************************************************

\MySection{L"osbarkeit linearer Gleichungssysteme}
\label{SecLinEqu}
\index{Gleichungssysteme}
$$$$$
Da lineare Gleichungssysteme eine wichtige Rolle beim Verst"andnis
verschiedener Sachverhalte spielen, werden sie in diesem
Unterkapitel n"aher betrachtet. Die nun folgenden Grundlagen aus der
linearen Algebra sind in der in \ref{SecMatUndDet} aufgelisteten
Literatur ausf"uhrlich behandelt. Wir beschr"anken uns hier auf die
f"ur den nachfolgenden Text wichtigen Aspekte.

Ein lineares Gleichungssystem besteht aus $n$ Gleichungen mit $m$
Unbekannten:
\begin{eqnarray*}
    a_{1,1} x_1 + a_{1,2} x_2 + \cdots + a_{1,n} x_n & = & b_1 \\
    \vdots & \vdots & \vdots \\
    a_{n,1} x_1 + a_{n,2} x_2 + \cdots + a_{m,n} x_n & = & b_m
\end{eqnarray*}
Die $x_i$ sind die zu bestimmenden Unbekannten. Betrachtet man die
gegebenen Konstanten $a_{i,j}$ und $b_i$ als Elemente einer Matrix
bzw. eines Vektors kann man das Gleichungssystem auch kompakter
darstellen:
\Beq{EquGleichungssystem} 
    A x = b \MyPunkt 
\Eeq
Dabei ist $A$ ein $n \times m$-Matrix und $b$ ein Vektor der L"ange $m$.
Ist $b= 0_m$, so bezeichnet man das Gleichungssystem als
homogen \index{Gleichungsystem!homogenes}.
Ist $b \neq 0_m$, so bezeichnet man das Gleichungssystem als
inhomogen \index{Gleichungssystem!inhomogenes}.

Bei der Betrachtung von \equref{EquGleichungssystem} erkennt man 
eine Eigenschaft, die besonders in Verbindung mit passenden Algorithmen
zur L"osung anderer Probleme f"ur die Determinantenberechnung von 
Bedeutung ist: 
\begin{bemerkung}
\label{SatzGdurchI}
    Falls $A$ invertierbar ist, l"a"st sich \equref{EquGleichungssystem} 
    "uberf"uhren in die Form
    \[ x=A^{-1}b \MyPunkt \]
    Ein Algorithmus zur Invertierung einer Matrix l"a"st sich also zur 
    L"osung eines linearen Gleichungssystems benutzten, falls die 
    Koeffizientenmatrix invertierbar ist.
\end{bemerkung}

Uns interessieren speziell die L"osungen inhomogener Gleichungssysteme.
Dazu betrachtet man die erweiterte Matrix $[A,b]$, die aus der Matrix
$A$ und dem Vektor $b$ als $n+1$-te Spalte besteht:

\begin{satz}
\label{SatzRangGleich}
    Die Gleichung
    \Beq{EquInhomogen}
        A x = b \MyPunkt
    \Eeq
    ist genau dann l"osbar, wenn gilt
    \Beq{EquRangGleich}
        \rg(A) = \rg([A,b]) \MyPunkt
    \Eeq
\end{satz}
\begin{beweis}
    Man kann die linke Seite von \equref{EquInhomogen} als 
    Linearkombination von Spaltenvektoren von $A$ betrachten. 
    Die Faktoren der Linearkombination bilden die Elemente des 
    L"osungsvektors $x$.

    Der Vektor $b$ ist genau dann als Linearkombination der
    Spaltenvektoren von $A$ darstellbar, wenn die maximale Anzahl linear
    unabh"angiger Vektoren in $A$ und $[A,b]$ gleich ist. Dies ist 
    gleichbedeutend mit der G"ultigkeit von \equref{EquRangGleich}.
\end{beweis}

Aus $$$$$ (Wegener 7.2, 7.3, 7.4) erh"alt man: % 7.2 aus 7.1

\begin{korollar}
\label{SatzGenauEine}
    Ist \equref{EquInhomogen} l"osbar und $\rg(A) = n$, 
    also $n \leq m$, gibt es genau einen Vektor $x$, der die 
    Gleichung erf"ullt.
\end{korollar}

