\begin{lemma}
\label{SatzSumZweiHoch}
    Sei $n \in \Nat_0$. Dann gilt:
    \[ 2^n = \sum_{k=0}^{n-1} 2^k  + 1 \]
\end{lemma}
\begin{beweis}
    Der Beweis erfolgt durch Induktion "uber $n$.
    \begin{MyDescription}
    \MyItem{$n=0$}
        Die Aussage ist offensichtlich richtig.
    \MyItem{$n>0$}
        Die Aussage gelte f"ur $n<i$. Es ist zu zeigen, da"s sie dann 
        auch f"ur $n=i$ gilt.
        \begin{eqnarray*}
            2^i $ = $ 2 * 2^{i-1} $
         \\     $ = $ 2 * \lb \sum_{k=0}^{i-2} 2^k + 1  \rb
         \\     $ = $ \sum_{k=1}^{i-1} 2^k + 2
         \\     $ = $ \sum_{k=0}^{i-1} 2^k + 1
        \end{eqnarray*}
    \end{MyDescription}
\end{beweis}

