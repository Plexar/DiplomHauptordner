%
% Datei: implemen.tex
%
\MyChapter{Implementierung}
\label{ChapImplemen}

In diesem Kapitel wird die Implementierung der in den Kapitel 
\ref{ChapCsanky} bis \ref{ChapPan} behandelten Algorithmen 
beschrieben\footnote{Der Algorithmus nach dem Entwicklungssatz von
Laplace (siehe Unterkapitel \ref{SecDivCon}) wird hier nicht 
ber"ucksichtigt.}.

\MySection{Erf"ullte Anforderungen}
\label{SecAnford}

In diesem Unterkapitel werden die wesentlichen Eigenschaften des
implementierten Programms beschrieben um einen "Uberblick "uber dessen
Leistungsmerkmale zu geben.

Die Algorithmen sind in Modula-2 auf einem Rechner mit einem Prozessor
implementiert\footnote{Megamax Modula-2 auf einem ATARI ST}. 
Als Literatur "uber die Programmiersprache Modula-2 ist
z. B. \cite{DCLR86} empfehlenswert.

Alle Algorithmusteile, die parallel ausgef"uhrt werden sollen,
werden mit Hilfe von Schleifen nacheinander ausgef"uhrt. Mit Hilfe von
Z"ahlprozeduren (siehe Modul `Pram') wird w"ahrend der Programmausf"uhrung
ermittelt, in wieviel Schritten und mit Hilfe von wieviel Prozessoren der
Algorithmus durch eine PRAM abgearbeitet werden kann.

Das Programm erm"oglicht es, Matrizen anhand von Parametervorgaben 
zuf"allig zu erzeugen. Die Parameter sind
\begin{itemize}
\item Gr"o"se der Matrix,
\item
      Rang (um auch Matrizen mit der Determinante Null gezielt 
      erzeugen zu k"onnen),
\item
      Wahl einer der Mengen\footnote{durch geeignete Parameterwahl mit dem 
      Befehl `param'}  $\Nat$, $\Integers$, $\Rationals$ oder
      $\Rationals^+$ \footnote{alle Elemente aus $\Rationals$, die
      gr"o"ser oder gleich Null sind} f"ur die Matrizenelemente,
      da Algorithmen evtl. nicht auf jede dieser Mengen anwendbar sind,
      und
\item 
      Vielfachheiten der Eigenwerte (wichtig f"ur den Algorithmus von Pan).
\end{itemize}

F"ur erzeugte Matrizen kann jeder der implementierten Algorithmen aufgerufen
werden. Eine Matrix wird zusammen mit den durch die verschiedenen 
Algorithmen berechneten Determinanten und den Me"sergebnissen der 
Z"ahlprozeduren unter einem Namen auf dem Hintergrundspeicher abgelegt. 
Diese Verwaltung geschieht automatisch. Dazu mu"s der Benutzer vor jedem
neuen Anlegen eines aus den obigen Daten bestehenden Datensatzes einen
Namen f"ur diesen Datensatz angeben.

Das Programm ist kommandoorientiert. D. h. nach dem Start wird der 
Benutzer aufgefordert, Befehle einzugeben, die nacheinander ausgef"uhrt 
werden.

Die Lesbarkeit des Quelltextes wird durch h"aufige Kommentare gesteigert.

% **************************************************************************

\MySection{Bedienung des Programms}
\label{SecBedienung}

In diesem Unterkapitel wird die Benutzung des Programms beschrieben.

Nach dem Programmstart mu"s zun"achst mit Hilfe des Befehls {\em find}
ein Datensatz festgelegt werden, der bearbeitet werden soll. Dies kann
ein bereits existierender oder ein neu anzulegender sein.

Bei einem neu angelegten Datensatz m"ussen danach mit Hilfe des Befehls 
{\em param} die Parameter f"ur die zu erzeugende Matrix festgelegt werden.

Anschlie"send kann man mit {\em gen} eine neue Matrix erzeugen lassen.
Nachdem mit {\em param} die Parameter festgelegt sind, kann mit mit {\em gen}
zu jeder Zeit eine neue Matrix generieren lassen, wodurch jedoch die 
vorangegangene Matrix verloren geht.

Wenn eine Matrix erzeugt worden ist, kann man mit Hilfe der Befehle
{\em berk, bgh, csanky} und {\em pan} die entsprechenden Algorithmen auf 
die generierte Matrix anwenden.

Mit {\em show} kann man sich zu jeder Zeit den aktuellen mit {\em find}
bestimmten Datensatz auf dem Bildschirm anzeigen lassen. Die generierte
Matrix wird nur ausgegeben, wenn eine Matrixzeile in eine Bildschirmzeile 
pa"st. Gr"o"sere Matrizen kann man mit {\em mshow} trotzdem ausgeben 
lassen.

Falls man einen weiteren Datensatz anlegen oder einen bereits vorhandenen
wieder bearbeiten will, benutzt man erneut den Befehl {\em find}. Die
Speicherung des alten Datensatzes geschieht automatisch.

Weitere Befehle, die man nach dem Programmstart zu jedem Zeitpunkt angeben
kann, sind: {\em del, exit, h, help, hilfe, ?, ls} und {\em q} . Ihre
Bedeutung ist in der folgenden Liste aller erlaubten Befehle erkl"art:

\begin{MyDescription}
\MyItem{\em berk}
    Der Algorithmus von Berkowitz aus Kapitel \ref{ChapBerk} wird auf
    die Matrix des aktuellen Datensatzes angewendet. Die berechnete
    Determinante sowie die Ergebisse der Z"ahlprozeduren 
    (siehe Modul `Pram') werden im aktuellen Datensatz abgelegt.
\MyItem{\em bgh}
    Die Wirkung dieses Befehls ist analog zu der des Befehls {\em berk},
    jedoch bezogen auf den Algorithmus von Borodin, von zur Gathen und
    Hopcroft aus Kapitel \ref{ChapBGH}.
\MyItem{\em csanky}
    Die Wirkung dieses Befehls ist analog zu der des Befehls {\em berk},
    jedoch bezogen auf den Algorithmus von Csanky aus Unterkapitel 
    \ref{SecAlgFrame}.
\MyItem{\em del}
    Es wird nach einem Datensatznamen gefragt. Der zugeh"orige Datensatz
    wird im Hauptspeicher und auf dem Hintergrundspeicher gel"oscht.
\MyItem{\em exit}
    Das Programm wird beendet. Der aktuelle mit {\em find} festgelegte
    Datensatz wird automatisch auf dem Hintergrundspeicher abgelegt.
\MyItem{\em find}
    Es wird nach einem Datensatznamen gefragt. Falls ein Datensatz mit 
    diesem Namen bereits im Hauptspeicher abgelegt ist, wird dieser erneut
    zum aktuellen Datensatz. Falls dies nicht der Fall ist und auf dem 
    Hintergrundspeicher ein Datensatz mit diesem Namen abgelegt ist,
    wird dieser in den Hauptspeicher geladen und zum aktuellen Datensatz.
    Falls beide F"alle nicht zutreffen, wird ein neuer Datensatz mit dem 
    angegebenen Namen im Hauptspeicher angelegt und als aktueller 
    Datensatz betrachtet.
\MyItem{\em gen}
    Entsprechend der mit {\em param} festgelegten Parameter wird eine neue
    Matrix f"ur den aktuellen Datensatz generiert. Die dort durch die
    Befehle {\em berk, bgh, csanky} und {\em pan} abgelegten Daten werden
    gel"oscht.
\MyItem{\em h, help, hilfe, ?}
    Durch diese Befehle wird eine Kurzbeschreibung aller erlaubten Befehle
    auf den Bildschirm ausgegeben.
\MyItem{\em ls}
    Auf dem Bildschirm wird eine Liste der Namen der im Hauptspeicher
    befindlichen Datens"atze ausgegeben.
\MyItem{\em mshow}
    Die Matrix des aktuellen Datensatzes wird auf dem Bildschirm ausgegeben.
\MyItem{\em pan}
    Die Wirkung dieses Befehls ist analog zu der des Befehls {\em berk},
    jedoch bezogen auf den Algorithmus von Pan aus Kapitel \ref{ChapPan}.
\MyItem{\em param}
    Es wird nach den Parametern f"ur die mit Hilfe von {\em gen} zu
    generierende Matrix gefragt. Alle zuvor im aktuellen Datensatz
    abgelegten Daten werden gel"oscht.
\MyItem{\em q}
    Die Wirkung dieses Befehls ist mit der des Befehls {\em exit} identisch.
\MyItem{\em show}
    Der aktuelle Datensatz wird auf dem Bildschirm ausgegeben. Die 
    Matrix wird nur ausgegeben, falls eine Matrixzeile in eine 
    Bildschirmzeile pa"st. Mit dem Befehl {\em mshow} kann die Matrix 
    dennoch ausgegeben werden.
\end{MyDescription}

% **************************************************************************

\MySection{Die Modulstruktur}
\label{SecModule}

In diesem Unterkapitel wird die Struktur des implementierten Programms 
beschrieben. Dazu wird zu jedem Modul dessen Aufgabe und evtl. dessen
Beziehung zu anderen Modulen angegeben. Alle Beschreibungen von Details
der Implementierung, die f"ur die Benutzung des Programms und f"ur 
das Verst"andnis von dessen Gesamtstruktur unwichtig sind, erfolgen durch 
Kommentare innerhalb des Quelltextes (siehe Anhang).

Eine Beschreibung des Programmoduls {\em main} entspricht einer Erkl"arung
der Benutzung des Programms. Diese ist in Unterkapitel \ref{SecBedienung}
zu finden.

Die Beschreibung der Programmodule zum Test einzelner Teile des 
Gesamtprogramms ist von untergeordnetem Interesse und beschr"ankt sich 
deshalb auf kurze Bemerkungen "uber ihren Zweck ihm Rahmen der 
alphabetischen Auflistung (s. u.).

Die vorrangige Aufmerksamkeit des an der Implementierung der Algorithmen
Interessierten sollte sich auf die Module {\em Det} und {\em Pram} sowie
auf das Programmodul {\em algtest} richten. 

Die genannten vorrangig interessanten Module sind im Anhang 
\ref{ChapImplDet} zusammen mit dem Modul {\em main} gesammelt. Die
weniger interessanten Testprogrammodule sind im Anhang \ref{ChapTest}
aufgef"uhrt. Alle weiteren Module sind in alphabetischer Reihenfolge in
Anhang \ref{ChapSupport} zu finden.

Zun"achst wird anhand eines {\em reduzierten Ebenenstrukturbildes}
(Erkl"arung s. u.) ein "Uberblick "uber die Programmstruktur geben. 
Anschlie"send erfolgt eine alphabetische Auflistung der Module und 
ihrer Erkl"arungen.

In das erw"ahnte Strukturbild sind alle Module nach folgenden Regeln
eingetragen:
\begin{itemize}
\item
      Die Eintragung erfolgt ebenenweise. Die niedrigste Ebene ist im
      Bild unten zu finden und die h"ochste oben. Jedes Modul geh"ort 
      genau einer Ebene an.
\item
      Jedes Modul wird im Rahmen der Ma"sgaben durch die anderen Regeln
      in einer m"oglichst niedrigen Ebene eingetragen.
\item
      Von jedem Modul A aus, das ein Modul B benutzt, z. B. durch Aufruf von
      Prozeduren des Moduls B, wird im Rahmen der Einschr"ankungen durch
      andere Regeln ein Pfeil auf dieses Modul $B$ gerichtet.
\item 
      Jedes Modul wird so eingetragen, da"s kein Pfeil von ihm auf ein 
      Modul auf der gleichen oder einer h"oheren Ebene gerichtet ist.
\item
      Alle Pfeile von einem Modul aus auf Module, die nicht genau eine
      Ebene tiefer angeordnet sind, werden weggelassen.
\end{itemize}
Durch die letzte Regel gewinnt das Strukturbild erheblich an
"Ubersichtlichkeit, ohne wesentlichen Informationsgehalt zu verlieren.
Aus dem Quelltext jedes Moduls ist zu entnehmen, welche Module, au"ser
den im Bild angegebenen, sonst noch benutzt werden. Die aufgef"uhrten 
Regeln liefern das in Abbildung \ref{PicModule} angegebene Bild.

\begin{figure}[htb]
\begin{center}
    %
% You need 'epic.sty'
%
\setlength{\unitlength}{1mm}
\makeatletter
\def\Thicklines{\let\@linefnt\tenlnw \let\@circlefnt\tencircw
\@wholewidth4\fontdimen8\tenln \@halfwidth .5\@wholewidth}
\makeatother
\begin{picture}(105.00,115.00)
\put(35.75,9.00){SysMath}
\put(60.75,9.25){Sys}
\put(36.00,20.50){Func}
\put(32.75,30.25){Rnd}
\put(48.00,30.00){Str}
\put(21.00,39.00){rndtest}
\put(42.75,39.00){Type}
\put(60.50,38.75){strtest}
\put(15.00,47.75){Frag}
\put(32.75,48.00){List}
\put(51.00,48.00){Simptype}
\put(78.00,48.00){typetest}
\put(15.00,57.00){Mat}
\put(30.00,57.25){Cali}
\put(44.75,57.00){Inli}
\put(60.00,57.25){Reli}
\put(17.75,66.00){Hash}
\put(30.50,66.00){listtest}
\put(54.00,66.00){Pram}
\put(35.75,74.75){Rema}
\put(54.00,75.00){pramtest}
\put(27.00,84.25){Data}
\put(42.50,84.50){Mali}
\put(42.00,94.00){Det}
\put(40.50,102.00){main}
\put(45.00,96.75){\vector(-1,-4){0}}
\drawline(45.25,102.00)(45.00,96.75)
\put(45.00,88.00){\vector(-1,-4){0}}
\drawline(45.25,93.50)(45.00,88.00)
\put(38.25,78.25){\vector(4,-3){0}}
\drawline(31.00,84.00)(38.25,78.25)
\put(39.50,78.00){\vector(-1,-1){0}}
\drawline(46.00,83.75)(39.50,78.00)
\put(58.75,69.00){\vector(-1,-4){0}}
\drawline(59.00,74.75)(58.75,69.00)
\put(53.75,69.50){\vector(3,-1){0}}
\drawline(39.25,74.25)(53.75,69.50)
\put(31.75,61.00){\vector(2,-1){0}}
\drawline(22.25,65.50)(31.75,61.00)
\put(33.50,60.75){\vector(-1,-3){0}}
\drawline(35.25,65.50)(33.50,60.75)
\put(36.00,61.25){\vector(-4,-1){0}}
\drawline(57.25,66.00)(36.00,61.25)
\put(62.25,61.00){\vector(1,-1){0}}
\drawline(58.50,65.25)(62.25,61.00)
\put(18.25,51.00){\vector(1,-4){0}}
\drawline(18.00,56.50)(18.25,51.00)
\put(36.25,51.00){\vector(1,-3){0}}
\drawline(34.25,57.25)(36.25,51.00)
\put(37.75,51.75){\vector(-2,-1){0}}
\drawline(47.25,56.75)(37.75,51.75)
\put(40.25,51.50){\vector(-4,-1){0}}
\drawline(63.75,56.75)(40.25,51.50)
\put(43.50,42.75){\vector(4,-1){0}}
\drawline(19.25,47.00)(43.50,42.75)
\put(45.75,42.50){\vector(3,-2){0}}
\drawline(37.75,48.00)(45.75,42.50)
\put(47.00,42.75){\vector(-2,-1){0}}
\drawline(57.50,47.50)(47.00,42.75)
\put(48.75,42.50){\vector(-4,-1){0}}
\drawline(82.50,47.75)(48.75,42.50)
\put(34.00,33.75){\vector(1,-1){0}}
\drawline(28.75,38.75)(34.00,33.75)
\put(49.75,33.50){\vector(3,-4){0}}
\drawline(46.00,38.50)(49.75,33.50)
\put(51.00,33.50){\vector(-3,-1){0}}
\drawline(65.25,38.75)(51.00,33.50)
\put(39.00,24.00){\vector(1,-2){0}}
\drawline(35.50,30.25)(39.00,24.00)
\put(40.50,24.00){\vector(-2,-1){0}}
\drawline(50.50,29.50)(40.50,24.00)
\put(40.75,12.00){\vector(1,-4){0}}
\drawline(39.75,20.00)(40.75,12.00)
\end{picture}
%

    \caption{reduziertes Ebenenstrukturbild}
    \label{PicModule}
\end{center}
\end{figure}

Es folgen die Kurzbeschreibungen der einzelnen Module in alphabetischer 
Reihenfolge.

\begin{MyDescription}
\MyItem{\bf algtest (Programmodul)}
    Dieses Modul dient zum Test der Algorithmen zur 
    Determinantenberechnung ohne Behinderung durch Anforderungen 
    irgendwelcher Art, insbesondere
    ohne Beachtung der Parallelisierung und der Ma"sgabe, Matrizen
    beliebiger Gr"o"se zu verarbeiten.
\MyItem{Cali (CArdinal LIst)}
    In diesem Modul sind lineare Listen positiver ganzer Zahlen 
    implementiert. Es st"utzt sich auf das Modul `List'.
\MyItem{Data}
    Das Modul `Data' dient der Verwaltung der Datens"atze bestehend aus 
    Matrizen und ihren Parametern, sowie der berechneten Determinanten 
    und der dabei gez"ahlten Schritte und Prozessoren.
\MyItem{Det}
    In diesem Modul sind die Algorithmen zur parallelen
    Determinantenberechnung implementiert.
\MyItem{Frag (array FRAGments)}
    Im Modul `Frag' sind Felder beliebiger variabler L"ange und beliebigen
    Inhalts implementiert. Das Modul ist erforderlich, um Matrizen 
    verarbeiten zu k"onnen, deren Gr"o"se durch den Benutzer erst w"ahrend 
    der Laufzeit des Programms festgelegt wird.
    
    Das Modul profitiert von der Verwaltung von 
    Elementen beliebiger Typen durch das Modul `Type'.
\MyItem{Func (FUNCtions)}
    In diesem Modul sind verschiedene Prozeduren und Funktionen 
    insbesondere f"ur mathematische Zwecke zusammengefa"st.
\MyItem{Hash}
    Durch dieses Modul werden Prozeduren zur Streuspeicherung, auch unter
    dem Namen `Hashing' bekannt, zur Verf"ugung gestellt. Das Modul wird 
    durch den Algorithmus von Borodin, von zur Gathen und Hopcroft im 
    Modul `Det' ben"otigt, um Zwischenergebnisse bei der parallelen 
    Berechnung von Termen zu speichern.
    
    Das Modul `Hash' erlaubt es, beliebige Daten zu speichern. Dabei wird
    auf das Modul `Type' zur Verwaltung von Elementen beliebiger Typen
    zur"uckgegriffen.
\MyItem{Inli (INteger LIst)}
    In diesem Modul sind lineare Listen ganzer Zahlen 
    implementiert. Es st"utzt sich auf das Modul `List'.
\MyItem{List}
    Das Modul `List' stellt Prozeduren zur Verwaltung von linearen 
    doppelt verketteten Listen bliebiger Elemente zur Verf"ugung. Analog
    zu den Modulen `Frag', `Hash' und `Mat' benutzt `List' das Modul
    'Type` zur Verwaltung von Elementen beliebiger Typen.
    
    Auf dem Modul `List' bauen verschiedene Module zur Implementierung von
    Listen spezieller Typen auf.
\MyItem{\bf listtest (Programmodul)}
    Dieses Programmodul dient zum Test des Moduls `List'. Es verwendet
    dazu das Modul `Cali'.
\MyItem{\bf main (Programmodul)}
    Dies ist das Hauptmodul des gesamten Programms. Es nimmt die Befehle
    des Benutzers entgegen und ruft die entsprechenden Prozeduren auf.
    Die Benutzung ist in Unterkapitel \ref{SecBedienung} beschrieben.
\MyItem{Mali (MAtrix LIst)}
    In diesem Modul sind lineare Listen von Matrizen
    implementiert. Es st"utzt sich auf die Module `List' und `Mat'.
\MyItem{Mat (MATrix)}
    Dieses Modul stellt Prozeduren zur Verwaltung von zweidimensionalen 
    Matrizen beliebiger Gr"o"se f"ur beliebige Elemente zur Verf"ugung.
    Es st"utzt sich auf das Modul `Frag' zur Verwaltung der Felder
    beliebiger Gr"o"se und auf das Modul `Type' zur Verwaltung von 
    Elementen beliebiger Typen.
\MyItem{Pram}
    Das Modul `Pram' stellt die Z"ahlprozeduren zur Verf"ugung, die zur 
    Ermittlung des Aufwandes f"ur
    eine PRAM zur Abarbeitung der verschiedenen Algorithmen zur 
    Determiantenberechnung erforderlich sind. Das Modul wird durch die
    Module 'Det' und 'Rema' benutzt und verwendet seinerseits insbesondere
    das Modul `Cali' f"ur Verwaltungsaufgaben.
\MyItem{\bf pramtest (Programmodul)}
    Dieses Programmodul dient zum Test des Modul `Pram'.
\MyItem{Reli (REal LIst)}
    In diesem Modul sind lineare Listen von Flie"skommazahlen
    implementiert. Es st"utzt sich auf das Modul `List'.
\MyItem{Rema (REal MAtrix)}
    Dieses Modul implementiert Matrizen aus Flie"skommazahlen. Es st"utzt
    sich dazu auf das Modul \nopagebreak[3] `Mat'.
\MyItem{Rnd  (RaNDomize)}
    Das Modul 'Rnd' erlaubt es, Zufallszahlen nach der linearen 
    Kongruenzmethode zu erzeugen. Es wird vom Modul `Rema' dazu benutzt,
    anhand von verschiedenen Parametern zuf"allige Matrizen zu generieren.
\MyItem{\bf rndtest (Programmodul)}
    Dieses Programmodul dient zum Test des Moduls `Rnd'.
\MyItem{simptype (SIMPle TYPE)} \sloppy
    Dieses Modul stellt Verwaltungsprozeduren f"ur die einfachen 
    Datentypen \newline[3] `LONGCARD', `LONGINT' und `LONGREAL' zur 
    Verf"ugung, damit
    sie in Verbindung mit dem Modul `Type' verwendet werden k"onnen.
    \fussy
\MyItem{Str (STRing)}
    Im Modul `Str' sind diverse Prozeduren zur Verarbeitung von 
    Zeichenketten implementiert.
\MyItem{\bf strtest (Programmodul)}
    Dieses Programmodul dient zum Test des Moduls `Str'.
\MyItem{Sys}
    Dieses Modul stellt Prozeduren zum Ablegen von Daten auf dem 
    Hintergrundspeicher zur Verf"ugung. Da die Behandlung der Massenspeicher
    auf den verschiendenen Rechnersystemen unterschiedlich ist, mu"s
    das Modul 'Sys' bei der Portierung des Programms auf einen
    anderen Rechner neu implementiert werden. 
\MyItem{SysMath}
    Die zur Verf"ugung gestellten mathematischen Funktionen sind von 
    System zu System unterschiedlich. Deshalb sind im Modul `SysMath' die
    Funktionen gesammelt, die im Programm benutzt werden. Bei der Portierung
    des Programms auf ein anderes Computersystem mu"s dieses Modul evtl.
    angepa"st werden.
\MyItem{Type}
    Dieses Modul dient der Verwaltung von Elementen beliebiger Datentypen.
    Ein neuer Typ wird im Rahmen dieses Moduls durch die Angabe 
    verschiedener Verwaltungsprozeduren definiert. Das Modul "ubernimmt 
    auf diese Weise die Sammlung der Eigenschaften verschiedener Typen, um 
    so die "Ubersichtlichkeit zu steigern. Ohne dieses Modul mu"s jedes 
    der Module `Frag', `Hash', `List' und `Mat' eine entsprechende 
    Verwaltung separat enthalten.
\MyItem{\bf typetest (Programmodul)}
    Dieses Programmodul dient zum Test des Moduls `Type'.
\end{MyDescription}

% **************************************************************************

\MySection{Anmerkungen zur Implementierung}

An dieser Stelle werden einige praktische Gesichtspunkte der
Implementierung kommentiert.

Vergleicht man das Modul `algtest' mit dem Rest des Quelltextes, so erkennt 
man, da"s insbesondere die Anforderung der {\em Pseudoparallelisierung}
die L"ange des Quelltextes stark vergr"o"sert. Bei den Algorithmen im
Modul `Det' handelt es sich ungef"ahr um eine Vergr"o"serung 
um den Faktor 5.

Weiterhin zeigt sich, da"s eine flexible auch nachtr"aglich 
erweiterbare Programmstruktur, die unter dem Gesichtspunkt sich evtl. 
anschlie"sender Arbeiten w"unschenswert ist, nicht unerheblichen Aufwand
bedeutet. So machen die eigentlich interessierenden Programmteile nur
ca. 30 Prozent des Quelltextes\footnote{Gesamtl"ange ca. 9500 Zeilen} aus. 
Der gesamte weitere Aufwand ergibt sich
einerseits aus verschiedenen Anforderungen an Leistungen und Struktur des 
Programms, andererseits aus der Notwendigkeit, Datentypen zu implementieren,
die die verwendete Sprache nicht standardm"a"sig zur Verf"ugung stellt.

Auf eine Implementierung auf leistungsf"ahigeren Rechnern wurde verzichtet,
da der ben"otigte Speicherplatz quadratisch mit der Anzahl der Zeilen und
Spalten der Matrizen w"achst. Der zus"atzliche Speicherplatz f"uhrt nicht
zu einer Steigerung der Matrizengr"o"se von weitreichendem Interesse.
Diese Beschr"ankung erlaubt es, mit der relativ geringen Rechenleistung
eine ATARI ST auszukommen.

Gr"o"sere Matrizen sind auch aus einem weiteren Grund nicht ohne
erheblichen weiteren Aufwand sinnvoll. Die Standardarithmetiken
verschiedener Implementierungen von Programmiersprachen erlauben eine
Rechengenauigkeit von typischerweise ca. 19 Stellen. Dies reicht nicht
aus, um Determinanten
gr"o"serer Matrizen "uberhaupt darzustellen. Deshalb ist es f"ur eine
deutliche Steigerung der Matrizengr"o"se erforderlich, eine eigene
Flie"skommaarithmetik zu implementieren, die es erm"oglicht, mit
beliebiger Genauigkeit\footnote{im Rahmen der physikalischen Grenzen}
zu rechnen.

Die praktischen Erfahrungen in verschiedenen Bereichen der angewandten
Informatik zeigen, da"s in bestehenden in der Regel zufriedenstellend
laufenden Programmen eine Restquote an Programmierfehlern im
Quelltext von ca. einem Fehler pro 1000 Zeilen Quelltext existiert.
Beim gegenw"artigen Stand der Technik ist es nicht m"oglich, Programme
wesentlich fehlerfreier zu bekommen.

Besonders bei mathematischen Programmen ist es in der Regel erforderlich,
trotzdem nahezu Fehlerfreiheit zu erreichen, was die Implementierung
solcher Programme zus"atzlich erschwert. Ein praktisches Beispiel
f"ur diese Probleme sind die implementierten Algorithmen zur
Determiantenberechnung. Die pseudoparallelen Algorithmen im Modul
`Det' besitzen eine Gesamtl"ange von ca. 2300 Zeilen, erheblich mehr
als die Implementierungen im Programmodul 'algtest'.

Da die Dauer einer Fehlersuche schwer abzusch"atzen ist und nur begrenzte
Zeit zur Verf"ugung stand, haben die genannten Schwierigkeiten dazu
gef"uhrt, da"s zwar alle Algorithmen lauff"ahig sind, jedoch
die im Anhang zu findenden Implementierungen leider keine
Determinanten berechnen:
\begin{itemize}
\item P-Alg. im Programmodul `algtest' sowie
\item BGH-Alg., B-Alg. und P-Alg. im Modul `Det'.
\end{itemize}
Durch umfangreiche Testl"aufe kann ausgeschlossen werden, da"s die Fehler
au"serhalb der Module zu suchen sind. Es mu"s sich jeweils um fehlerhafte
Implementierung der Algorithmusbeschreibungen in den jeweiligen Kapiteln
handeln (z. B. Vorzeichenfehler oder falsche Indizes).

