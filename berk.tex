%
% Datei: berk.tex (Textteile nach 'Berk84')
%
\MyChapter{Der Algorithmus von Berkowitz}
\label{ChapBerk}

Der in diesem Kapitel vorgestellte Algorithmus \cite{Berk84} berechnet
die Determinante mit Hilfe einer rekursiven Beziehung zwischen den
charakteristischen Polynomen einer Matrix und ihren Untermatrizen. Dabei
wird \ref{SatzDdurchP} ausgenutzt. Auf den Algorithmus wird mit {\em B-Alg.} Bezug
genommen\footnote{vgl. Unterkapitel \ref{SecBez}}. 

Wie in BGH-Alg., werden keine Divisionen
verwendet\footnote{vgl. Erl"auterungen in Unterkapitel \ref{SecAlgFrame}}.

%******************************************************************

\MySection{Toepliz-Matrizen}

Im darzustellenden Algorithmus spielen Toepliz-Matrizen (Definition s. u.)
eine wichtige Rolle und werden deshalb in diesem Unterkapitel behandelt.

Eine Matrix $n \times p$-Matrix $A$ hei"st {\em Toepliz-Matrix}, falls
gilt: \index{Toepliz-Matrizen}
\[ a_{i,j} = a_{i-1,j-1} , \: 1 < i \leq n, \: 1 < j \leq p \MyPunkt \]
Sie hat also folgendes Aussehen:
\[
   \left[ \begin{array}{ccccc}
       a_{1,1} & a_{1,2} & a_{1,3} & a_{1,4} & \cdots \MatStrut \\
       a_{2,1} & a_{1,1} & a_{1,2} & a_{1,3} & \ddots \MatStrut \\
       a_{3,1} & a_{2,1} & a_{1,1} & a_{1,2} & \ddots \MatStrut \\
       a_{4,1} & a_{3,1} & a_{2,1} & a_{1,1} & \ddots \MatStrut \\
       \vdots  & \ddots  & \ddots  & \ddots  & \ddots \MatStrut
   \end{array} \right]
\]

Die folgende Eigenschaft von Toepliz-Matrizen ist f"ur uns wichtig:

\begin{satz}
\label{SatzToeplizMult}
\index{Toepliz-Matrizen!Multiplikation von}
    Sei $A$ eine $n \times p$-Matrix und $B$ eine $p \times m$-Matrix.
    Beide seien untere Dreiecks-Toeplitz-Matrizen.
    Falls f"ur die Matrix $C$ gilt
    \[ C = A * B \MyKomma \]
    dann ist $C$ ebenfalls eine untere Dreiecks-Toeplitz-Matrix.
    Sie kann in \[ \lceil \log(p) \rceil + 1 \] Schritten von
    \begin{eqnarray*} 
        & & \frac{ \min(p,\,m)* (\min(p,\,m) +1) }{2}
            + p * \max(n-p,0) \\
        & \leq & n * p
    \end{eqnarray*} Prozessoren berechnet
    werden. 
\end{satz}
\begin{beweis}
    Es sind drei Eigenschaften von $C$ zu zeigen:
    \begin{enumerate}
        \item $C$ ist eine untere Dreiecksmatrix.
        \item $C$ ist eine Toeplitz-Matrix.
        \item $C$ kann mit dem oben angegebenen Aufwand an Schritten und
              Prozessoren berechnet werden.
    \end{enumerate}
    Dies geschieht in drei entsprechenden Beweisschritten. Dazu ist zu
    beachten, da"s die einzelnen Elemente von $C$ nach der Gleichung f"ur
    die Matrizenmultiplikation berechnet werden:
    \Beq{Berk84Equ4}
        c_{i,j}= \sum_{k=1}^p a_{i,k} b_{k,j}
    \Eeq
    \begin{enumerate}
        \item
            Um zu beweisen, da"s $C$ ebenfalls eine untere Dreiecksmatrix
            darstellt, ist zu zeigen
            \[ i < j \Rightarrow c_{i,j} = 0 \]
            Dies erfolgt durch Fallunterscheidung anhand des Index $k$ in
            Gleichung \equref{Berk84Equ4}. Es gibt zwei F"alle:
            \begin{MyDescription}
                \MyItem{ $i < k$ }
                    Da $A$ nach Voraussetzung eine untere Dreiecksmatrix
                    ist und somit
                    \[ i < j \Rightarrow a_{i,j} = 0 \]
                    gilt, folgt 
                    \[ a_{i,k} = 0 \MyKomma\]
                    wodurch der entsprechende Summand in Gleichung
                    \equref{Berk84Equ4} zu $0$ wird.
                \MyItem{ $i \geq k$ }
                    Nach Voraussetzung gilt \[ i < j \MyKomma \] 
                    da f"ur die
                    Elemente oberhalb der Hauptdiagonalen von $C$ zu zeigen
                    ist, da"s sie gleich $0$ sind. Daraus folgt aber
                    \[ k < j \MyPunkt \]
                    Nach Voraussetzung ist $B$ ebenfalls eine untere
                    Dreiecksmatrix und es gilt somit
                    \[ i < j \Rightarrow b_{i,j} = 0 \]
                    Daraus folgt \[ b_{k,j} = 0 \MyKomma \]
                    wodurch wiederum der
                    entsprechende Summand in Gleichung \equref{Berk84Equ4}
                    zu $0$ wird.
            \end{MyDescription}
            In beiden F"allen sind die betrachteten Summanden von Gleichung
            \equref{Berk84Equ4} gleich $0$. Also ist dann auch
            \[ c_{i,j} = 0 \MyKomma \]
            was zu zeigen war.
        \item
            Damit $C$ eine Toeplitz-Matrix ist, mu"s gelten
            \[ c_{i,j} = c_{i+1,j+1} \MyPunkt \]
            Mit Hilfe von Gleichung \equref{Berk84Equ4} ausgedr"uckt
            bedeutet dies
            \Beq{Berk84Equ5}
                 \sum_{k=1}^p a_{i,k} b_{k,j}
               = \sum_{l=1}^p a_{i+1,l} b_{l,j+1} \MyPunkt
            \Eeq
            Da $C$ eine untere Dreiecksmatrix ist, wie oben bewiesen wurde,
            m"ussen nur \[ c_{i,j} \] betrachtet werden, f"ur die gilt
            \[ i \geq j \MyPunkt \]
            Man kann Fallunterscheidungen anhand der Indizes $k$ und $l$
            durchf"uhren. Es gibt f"ur jeden Index drei F"alle, also
            insgesamt sechs:
            \begin{MyDescription}
                \MyItem{ $k>i$ }
                    Da $A$ nach Voraussetzung eine untere Dreiecksmatrix
                    ist, gilt in diesem Fall \[ a_{i,k}= 0 \MyKomma \]
                    und der
                    entsprechende Summand wird zu $0$.
                \MyItem{ $j>k$ }
                    Da $B$ nach Voraussetzung ebenfalls eine untere
                    Dreiecksmatrix ist, gilt in diesem Fall
                    \[ b_{k,j} = 0 \MyKomma \]
                    und der entsprechende Summand wird zu $0$.
                \MyItem{ $i \geq k \geq j$ }
                    Nur in diesem Fall ergibt sich auf der linken Seite
                    von Gleichung \equref{Berk84Equ5} f"ur den jeweiligen
                    Summand ein von $0$ verschiedener Wert. Deshalb kann man
                    die linke Seite dieser Gleichung auch schreiben als
                    \[ \sum_{k=j}^i a_{i,k} b_{k,j} \MyPunkt \]
                \MyItem{ $l>i+1$ }
                    In diesem Fall gilt, da $A$ eine obere Dreiecksmatrix
                    ist, \[ a_{i+1,l} = 0 \MyPunkt \]
                    Der entsprechende Summand der
                    Summe in Gleichung \equref{Berk84Equ5} wird somit zu
                    $0$ und mu"s nicht l"anger betrachtet werden.
                \MyItem{ $j+1>l$ }
                    In diesem Fall gilt \[ b_{l,j+1} = 0 \MyKomma \]
                    da $B$ eine
                    obere Dreiecksmatrix ist und der entsprechende Summand
                    in Gleichung \equref{Berk84Equ5} mu"s nicht l"anger
                    betrachtet werden.
                \MyItem{ $i+1 \geq l \geq j+1$ }
                    Nur in diesem Fall ergibt sich auf der rechten Seite
                    von Gleichung \equref{Berk84Equ5} ein von $0$
                    verschiedener Wert f"ur den entsprechenden Summanden.
                    Man kann also die rechte Seite dieser Gleichung auch
                    schreiben als
                    \[ \sum_{l=j+1}^{i+1} a_{i+1,l} b_{l,j+1} \]
            \end{MyDescription}
            Nach der Betrachtung dieser sechs F"alle reduziert sich
            Gleichung \equref{Berk84Equ5} also, falls man nur die von
            $0$ verschiedenen Summanden betrachtet, auf die Form
            \[ \sum_{k=j}^i a_{i,k} b_{k,j} =
               \sum_{l=j+1}^{i+1} a_{i+1,l} b_{l,j+1}
            \]
            Anders geschrieben hat diese Gleichung die Form
            \begin{eqnarray*}
            &   a_{i,j} b_{j,j} + a_{i,j+1} b_{j+1,j} + a_{i,j+2} b_{j+2,j}
                + \ldots + a_{i,i} b_{i,j} =
            & \\
            &   a_{i+1,j+1} b_{j+1,j+1} + a_{i+1,j+2} b_{j+2,j+1} +
                a_{i+1,j+3} b_{j+3,j+1} + \ldots + a_{i+1,i+1} b_{i+1,j+1}
            &
            \end{eqnarray*}
            Da $A$ und $B$ Toeplitz-Matrizen sind, haben die beiden
            Seiten dieser Gleichung den gleichen Wert, was zu beweisen war.
        \item
            Da $C$ wiederum eine Toeplitz-Matrix ist,
            m"ussen nur die $c_{i,j}$ mit $j=1$ neu berechnet
            werden. Alle anderen Elemente sind entweder gleich Null oder
            gleich einem $c_{i,1}$. L"a"st man zus"atzlich alle
            Multiplikationen mit Null weg, kommt man
            zur Berechnung von $C$ insgesamt mit
            \[ \lc \log(p) \rc + 1 \] Schritten und
            \begin{eqnarray*}
                & & \sum_{k=1}^{\min(p,\,m)} k + p * \max(n-p,\,0) \\
                & = & \frac{ \min(p,\,m)* (\min(p,\,m) +1) }{2}
                      + p * \max(n-p,0)
            \end{eqnarray*} Prozessoren aus. Einschlie"slich der 
            Multiplikationen mit Null erh"alt man
            \[ \lc \log(p) \rc + 1 \] Schritte und
            \[ n * p \] Prozessoren.
    \end{enumerate}
\end{beweis}

% **************************************************************************

\MySection{Der Satz von Samuelson}
\label{SecSamuelson}

In diesem Unterkapitel wird der theoretische Hintergrund des
darzustellenden Algorithmus behandelt.

Zur Beschreibung des Satzes von Samuelson \cite{Samu42} wird folgende
Schreibweise eingef"uhrt ($A$ ist eine $n \times n$-Matrix):
\label{SeiteRMSSchreibweise}
\begin{itemize}
\item
     Den Vektor $S_i$ erh"alt man aus dem $i$-ten Spaltenvektor von $A$
     durch Entfernen der ersten $i$ Elemente. Er hat also folgendes
     Aussehen:
     \[ \left[
        \begin{array}{c}
            a_{i+1,i} \MatStrut \\
            a_{i+2,i} \MatStrut \\
            \vdots    \MatStrut \\
            a_{n,i}
        \end{array}
        \right]
     \]
\item
     Den Vektor $R_i$ erh"alt man aus dem $i$-ten Zeilenvektor von $A$
     durch Entfernen der ersten $i$ Elemente. Er hat also folgendes
     Aussehen:
     \[ [a_{i,i+1}, a_{i,i+2}, \ldots , a_{i,n} ] \]
\item
     Die Matrix $M_i$ erh"alt man aus der Matrix $A$ durch Entfernen
     der ersten $i$ Zeilen und Spalten. Sie hat also folgendes Aussehen:
     \[ \left[
        \begin{array}{cccc}
            a_{i+1,i+1} & a_{i+1,i+2} & \cdots & a_{i+1,n} \MatStrut \\
            a_{i+2,i+1} & a_{i+2,i+2} & \cdots & a_{i+2,n} \MatStrut \\
            \vdots      & \vdots      & \ddots & \vdots    \MatStrut \\
            a_{n,i+1}   & a_{n,i+2}   & \cdots & a_{n,n}   \MatStrut
        \end{array}
        \right]
     \]
\item
     Statt $S_1$, $R_1$ und $M_1$ wird auch $S$, $R$ und $M$ geschrieben.
\end{itemize}

Die Matrix $A$ l"a"st sich also auch in den Formen
\[
   \left[
   \begin{array}{cc}
       a_{11} & R \MatStrut \\
       S      & M \MatStrut
   \end{array}
   \right]
\]
oder
\[
   \left[
   \begin{array}{ccccc}
       a_{11}     & R_1        & \rightarrow &             & \MatStrut \\
       S_1        & a_{22}     & R_2         & \rightarrow & \MatStrut \\
       \downarrow & S_2        & a_{33}      & R_3         & \rightarrow
                                                             \MatStrut \\
                  & \downarrow & S_3         & \ddots      & \ddots
                                                             \MatStrut \\
                  &            & \downarrow  & \ddots      & \MatStrut
   \end{array}
   \right]
\]
darstellen.

Im folgenden Lemma wird das charakteristische Polynom einer Matrix
mit Hilfe der oben definierten $R$, $S$ und $M$ ausgedr"uckt:

\begin{lemma}
\label{Berk84Satz1}
% $$$ Claim 1
    Sei $p(\lambda)$ das charakteristische Polynom der $n \times n$-Matrix
    $A$. Dann gilt:
    \[
        p(\lambda) = (a_{1,1} - \lambda) * \det(M - \lambda * E_{n-1})
                     - R * \adj(M - \lambda * E_{n-1}) * S
    \]
\end{lemma}
\begin{beweis}
    Es gilt \[ p(\lambda) = \det(A - \lambda * E_n) \]
    Durch Entwicklung nach der ersten Zeile erh"alt man:
    \[
        p(\lambda)= (a_{1,1} - \lambda) * \det(M - \lambda * E_{n-1}) +
        \sum_{j=2}^n (-1)^{1+j} a_{1,j}
        \underline{ \det( (A - \lambda * E_n)_{(1|j)} ) }
    \]
    Nun werden die in der obigen Gleichung unterstrichenen Determinanten
    jeweils nach der ersten Spalte entwickelt:
    \begin{eqnarray*}
        & p(\lambda)=
        & (a_{1,1} - \lambda) * \det(M - \lambda * E_{n-1}) +
    \\  & & \sum_{j=2}^n
            \underbrace{ (-1)^{1+j} a_{1,j} }_{ \mbox{(*1)} }
        \sum_{k=2}^n (-1)^{1+(k-1)} a_{k,1}
            \det(
                \underbrace{
                    (A - \lambda * E_n)_{(1,k|1,j)}
                }_{ \mbox{(*2)} }
            )
    \end{eqnarray*}
    Wenn man in dieser Gleichung (*1) mit der inneren Summe multipliziert
    und (*2) mit Hilfe von $M$ ausdr"uckt erh"alt man:
    \[
        p(\lambda)= (a_{1,1} - \lambda) * \det(M - \lambda * E_{n-1}) +
        \sum_{j=2}^n
        \sum_{k=2}^n (-1)^{1+j+k} \underbrace{ a_{1,j} a_{k,1} }_{ \mbox{(*)} }
            \det( (M - \lambda * E_{n-1})_{(k-1|j-1)} )
    \]
    Hier l"a"st sich (*) mit Hilfe von $R$ und $S$ formulieren:
    \[
        p(\lambda)= (a_{1,1} - \lambda) * \det(M - \lambda * E_{n-1}) +
        \sum_{j=2}^n
        \sum_{k=2}^n (-1)^{1+j+k} r_j s_k
            \det( (M - \lambda * E_{n-1})_{(k-1|j-1)} )
    \]
    Dies wiederum ist in Matrizenschreibweise und mit Hilfe der Adjunkten
    einer Matrix ausgedr"uckt nichts anderes als
    \[
        p(\lambda) = (a_{1,1} - \lambda) * \det(M - \lambda * E_{n-1})
                     - R * \adj(M - \lambda * E_{n-1}) * S \MyKomma
    \]
    was zu beweisen war.
\end{beweis}

Vor Lemma \ref{Berk84Satz2} m"ussen wir hier zun"achst einen wichtigen
Satz behandeln (\cite{MM64} S. 50 f):

\begin{satz}[Cayley und Hamilton]
\label{SatzCayleyHamilton}
\index{Cayley und Hamilton!Satz von}
    Sei $p(\lambda)$ das charakteristische Polynom von $A$. Dann gilt:
    \[ p(A)=0_{n,n} \]
\end{satz}
\begin{beweis}
    Aus Satz \ref{SatzAdj} folgt
    \begin{equation}
    \label{Equ1SatzCayleyHamilton}
        (A - \lambda E_n) \adj(A - \lambda E_n) = p(A) E_n
    \end{equation}
    Da die Elemente von \[ \adj(A - \lambda E_n) \] aus Unterdeterminanten
    von $A$ gewonnen werden, bestehen diese Elemente aus Polynomen "uber
    $\lambda$ vom maximalen Grad \[ n - 1 \] Also gilt f"ur geeignete
    $n \times n$-Matrizen \[ B_j, 1 \leq j \leq n-1 \] die folgende 
    Beziehung:
    \Beq{Equ2SatzCayleyHamilton} 
        \adj(A - \lambda E_n) = 
        B_{n-1} \lambda^{n-1} + \ldots + B_1 \lambda + B_0
    \Eeq
    Au"serdem kann man $p(A)$ schreiben als
    \Beq{Equ3SatzCayleyHamilton}
        p(A) = c_n \lambda^n + \ldots + c_1 \lambda + c_0
    \Eeq
    Dr"uckt man \equref{Equ1SatzCayleyHamilton} mit Hilfe von
    \equref{Equ2SatzCayleyHamilton} und \equref{Equ3SatzCayleyHamilton} aus,
    erh"alt man
    \[
        (A - \lambda E_n)(B_{n-1} \lambda^{n-1} + \ldots
        + B_1 \lambda + B_0)
            =
        (c_n \lambda^n + \ldots + c_1 \lambda + c_0) E_n
    \]
    Multipliziert man die Terme auf beiden Seiten aus und vergleicht die
    Koeffizienten miteinander, erh"alt man folgende Gleichungen:
    \[
        \begin{array}{lllcr}
                     & - & B_{n-1} & =      & c_n E_n
        \\  AB_{n-1} & - & B_{n-2} & =      & c_{n-1} E_n
        \\  AB_{n-2} & - & B_{n-3} & =      & c_{n-2} E_n
        \\           &   &         & \vdots &
        \\  AB_1     & - & B_0     & =      & c_1 E_n
        \\  AB_0     &   &         & =      & c_0 E_n
        \end{array}
    \]
    Multipliziert man beide Seiten der
    ersten dieser Gleichungen mit $A^n$, beide Seiten der zweiten
    mit $A^{n-1}$, allgemein beide Seiten der $j$-ten mit $A^{n-j+1}$, und
    addiert sie, erh"alt man
    \[ 0_{n,n} =
       c_n \lambda^n + c_{n-1} \lambda^{n-1} + \ldots + c_1 A = p(A)
    \]
\end{beweis}

Die Adjunkte in Lemma \ref{Berk84Satz1} l"a"st sich mit Hilfe der
Koeffizienten des charakteristischen Polynoms $q(\lambda)$ von $M$
ausdr"ucken. In Koeffizientendarstellung besitzt $q(\lambda)$ die
Form:
\[ q(\lambda) = q_{n-1} \lambda^{n-1} + q_{n-2} \lambda^{n-2} + \ldots
                + q_1 \lambda + q_0
\]

Es gilt folgende Aussage:

\begin{lemma}
\label{Berk84Satz2}
% $$$ Claim 2
    \begin{equation}
    \label{Berk84Equ1}
        \adj(M - \lambda * E_{n-1}) =
            - \sum_{k=0}^{n-2} \lambda^{k} \sum_{l=k+1}^{n-1} M^{l-k-1} q_l
    \end{equation}
\end{lemma}
\begin{beweis}
    Multipliziert man beide Seiten von \equref{Berk84Equ1} mit
    \[ M - \lambda * E_{n-1} \MyKomma \]
    erh"alt man auf der linken Seite
    \[ \adj(M - \lambda * E_{n-1}) * (M - \lambda * E_{n-1}) \MyPunkt \]
    Dies ist nach Satz \ref{SatzAdj} gleich
    \begin{eqnarray*}
        & & E_{n-1} * \det(M - \lambda * E_{n-1}) \\
        & = & q(\lambda) * E_{n-1} \MyPunkt
    \end{eqnarray*}

    Auf der rechten Seite von Gleichung \equref{Berk84Equ1} erh"alt man
    \[ - ( \underbrace{M}_{\mbox{(*1)}}
           \underbrace{- \lambda * E_{n-1})}_{\mbox{(*2)}}
         )
         \sum_{k=0}^{n-2} \lambda^{k} \sum_{l=k+1}^{n-1} M^{l-k-1} q_l
    \]
    Bei der Multiplikation erh"alt man f"ur (*1) und (*2) im obigen je
    eine Doppelsumme:
    \[ - \sum_{k=0}^{n-2} \lambda^{k} \sum_{l=k+1}^{n-1} M^{l-k} q_l
       + \sum_{k=0}^{n-2} \lambda^{k+1} \sum_{l=k+1}^{n-1} M^{l-k-1} q_l
    \]
    Durch Umordnen der Indizes der zweiten Doppelsumme erh"alt man
    \begin{equation}
    \label{Berk84Equ2}
        - \sum_{k=0}^{n-2} \lambda^{k} \sum_{l=k+1}^{n-1} M^{l-k} q_l
        + \sum_{k=1}^{n-1} \lambda^{k} \sum_{l=k+1}^{n-1} M^{l-k} q_l
    \end{equation}
    Nach Satz \ref{SatzCayleyHamilton} gilt
    \begin{equation}
    \label{Berk84Equ3}
        \sum_{l=0}^{n-1} M^l q_l = 0
    \end{equation}
    Somit kann man die linke Seite von Gleichung \equref{Berk84Equ3} zur
    zweiten Doppelsumme von Term \equref{Berk84Equ2} addieren und erh"alt
    \[
        - \sum_{k=0}^{n-2} \lambda^{k} \sum_{l=k+1}^{n-1} M^{l-k} q_l
        + \sum_{k=0}^{n-1} \lambda^{k} \sum_{l=k+1}^{n-1} M^{l-k} q_l
    \]
    Wenn man nun die Vorzeichen der beiden Doppelsummen sowie
    die benutzten Indizes betrachtet, erkennt man, da"s sich der Gesamtterm
    vereinfacht darstellen l"a"st, da gro"se Teile zusammengenommen $0$
    ergeben. Die Teile, die sich nicht auf diese
    Weise gegenseitig aufheben, lassen sich schreiben als
    \[ \sum_{k=0}^{n-1} \lambda^k E_{n-1} q_{k} \MyKomma \]
    was gleichbedeutend ist mit
    \[ q(\lambda) * E_{n-1} \MyPunkt \]

    Also stimmen die beiden Seiten von Gleichung \equref{Berk84Equ1}
    "uberein.
\end{beweis}

Die beiden Lemmata \ref{Berk84Satz1} und \ref{Berk84Satz2} f"uhren zu
folgendem Satz \cite{Samu42}:

\begin{satz}[Samuelson]
\label{SatzSamuelson}
\index{Samuelson!Satz von}
% $$$ Claim 2 into Claim 1
    \begin{equation}
    \label{EquSatzSamuelson}
        p(\lambda) =
            (a_{1,1} - \lambda) * \det(M - \lambda * E_{n-1})
            + R * \left(
              \sum_{k=0}^{n-2} \lambda^{k} \sum_{l=k+1}^{n-1} M^{l-k-1} q_l
            \right) * S
    \end{equation}
\end{satz}
\begin{beweis}
    Lemma \ref{Berk84Satz2} angewendet auf Lemma \ref{Berk84Satz1}
    ergibt die Behauptung.
\end{beweis}

% **************************************************************************

\MySection{Determinantenberechnung mit Hilfe des Satzes von Samuelson}
\label{SecAlgBerk}
\index{Berkowitz!Algorithmus von}
\index{Algorithmus!von Berkowitz}

Um Satz \ref{SatzSamuelson} zur Determinantenberechnung zu benutzen
\cite{Berk84}, sind weitere "Uberlegungen notwendig, die in diesem
Unterkapitel behandelt werden.

Betrachtet man die Methodik des entstehenden Algorithmus, erkennt man
"Ahnlichkeit zu C-Alg. . Auch dort wird ein schon l"anger bekannter Satz
mit Hilfe von zus"atzlichen "Uberlegungen f"ur eine parallelen Algorithmus
verwendet.

Zu beachten ist, da"s in diesem Unterkapitel f"ur die Multiplikation 
zweier $n \times n$-Matrizen $n^{2+\gamma}$ Prozessoren in Rechnung 
gestellt werden (vgl. S. \pageref{PageAlg2MatMult}).

Benutzt man die Koeffizientendarstellung f"ur die charakteristischen
Polynome von $A$ und $M$, l"a"st sich Gleichung
\equref{EquSatzSamuelson} umformulieren in
\[
   \sum_{i=0}^n p_i \lambda^i =
       (a_{1,1} - \lambda) * \sum_{i=0}^{n-1} q_i \lambda^i
       + R * \left(
         \sum_{k=0}^{n-2} \lambda^{k} \sum_{l=k+1}^{n-1} M^{l-k-1} q_l
       \right) * S  
   \MyPunkt
\]
Vergleicht man die Koeffizienten der $\lambda^i$ auf beiden Seiten
der Gleichung und definiert \[ q_{-1} := 0 \MyKomma \] erh"alt man
\begin{eqnarray}
    p_n     & = & -q_{n-1}                 \label{Equ1Berk84KoeffVergl}
\\  p_{n-1} & = & a_{1,1}q_{n-1} - q_{n-2} \label{Equ2Berk84KoeffVergl}
\\  \forall i=n-2 \ldots 0 : \: p_i & = &  \label{Equ3Berk84KoeffVergl}
        a_{1,1}q_i-q_{i-1}+\sum_{j=i+1}^{n-1}RM^{j-i-1}S q_i
\end{eqnarray}
Die Beziehungen zwischen den Koeffizienten, die diese Gleichungen
beschreiben, kann man auch durch eine Matrizengleichung 
ausdr"ucken. Dazu
wird Matrix $C_t$ definiert als untere Dreiecks-Toeplitz-Matrix der
Gr"o"se $(n-t+2) \times (n-t+1)$. Ihre Elemente werden definiert durch
\[ (c_t)_{i,j} :=
       \left\{
           \begin{array}{lcr}
               -1                       & : & i=1
            \\ a_{t,t}                  & : & i=2
            \\ R_t M_t^{i-3} S_t        & : & i>2
           \end{array}
       \right.
\]
Die Matrix hat also das folgende Aussehen:
\[
    \left[ \begin{array}{ccc}
        -1                   & 0         & \cdots \MatStrut
    \\  a_{t,t}              & -1        & \ddots \MatStrut
    \\  R_t S_t              & a_{t,t}   & \ddots \MatStrut
    \\  R_t M_t S_t          & R_t S_t   & \ddots \MatStrut
    \\  \vdots               & \ddots    & \ddots \MatStrut
    \\  R_t M_t^{n-t-1} S_t  &           &        \MatStrut
    \end{array} \right]
\] \MyPunktA{30em} 
Insbesondere hat $C_n$ die Form
\[ 
    \left[ \begin{array}{c}
    -1 \\ a_{n,n}
    \end{array} \right]
\]

Mit Hilfe dieser Definition erh"alt man aus den Gleichungen
\equref{Equ1Berk84KoeffVergl}, \equref{Equ2Berk84KoeffVergl} und
\equref{Equ3Berk84KoeffVergl} die folgende Matrizengleichung:
\Beq{Berk84Equ19}
   \left[ \begin{array}{c}
       p_n \\
       p_{n-1} \\
       \vdots \\
       p_0
   \end{array} \right]
   =
   C_1
   \left[ \begin{array}{c}
       q_{n-1} \\
       q_{n-2} \\
       \vdots \\
       q_0
   \end{array} \right]
\Eeq
Auf die gleiche Weise, wie man 
Satz \ref{SatzSamuelson} auf die Matrizen $A$ und $M$ anwendet, kann man 
diesen Satz auch auf die Matrizen $M$ und $M_2$, $M_2$ und $M_3$, etc.
anwenden und erh"alt so Matrizengleichungen, die in ihrer Form der 
Gleichung \equref{Berk84Equ19} entsprechen. 

Wendet man diese Matrizengleichungen aufeinander an, erh"alt man:
\Beq{EquProdCi}
   \left[ \begin{array}{c}
       p_n \\
       p_{n-1} \\
       \vdots \\
       p_0
   \end{array} \right]
   =
   \prod_{i=1}^{n} C_i
\Eeq
Um die Koeffizienten des charakteristischen Polynoms von $A$ auf die
geschilderte Weise zu berechnen, mu"s man also die Matrizen $C_i$
berechnen und dann miteinander multiplizieren. Nach
\ref{SatzDdurchP} ist damit auch die Determinante der Matrix $A$ berechnet.

F"ur jede  $(n-i+2) \times (n-i+1)$-Matrix $C_i$ bei ist 
der $(n-i)$-elementige Vektor
\Beq{EquRMSVektor}
    T_t := [ R_i S_i, \, R_i M_i S_i, \, R_i M_i^2 S_i, \, 
    \ldots , \, R_i M_i^m S_i ], \: m:= n-i-1
\Eeq
zu berechnen. Da also $T_n$ keine Elemente enth"alt, ist die Berechnung
der Vektoren $T_1$ bis $T_{n-1}$ erforderlich.

Man kann jeden Exponenten $k$ eines Elementes \[ R_i * M_i^k * S_i \] von 
$T_i$ in der Form \[ k = u + v * \left\lceil \sqrt{m} \right\rceil \] mit
\begin{eqnarray*}
    & 0 \leq u < \left\lceil \sqrt{m} \right\rceil &
\\  & 0 \leq v \leq \left\lfloor \sqrt{m} \right\rfloor &
\end{eqnarray*}
eindeutig darstellen. Man k"onnte statt $\sqrt{m}$ auch einen anderen
Wert zwischen $0$ und $m$ nehmen. Jedoch f"uhrt die Wahl von $\sqrt{m}$ 
dazu, da"s sich die Gr"o"se der Mengen aller $u$ und $v$ um h"ochstens $1$
unterscheidet.

Um $T_i$ effizient zu erhalten, kann man zun"achst die den Mengen der $u$
und $v$ entsprechenden Vektoren
\[
   U_i := \left[ R_i,\, R_i M_i,\, R_i M_i^2,\, 
                 \ldots ,\, R_i M_i^{\lc \sqrt{m} \rc - 1} 
          \right]
\]
und
\[
   V_i := \left[ S_i,\, M_i^{\lc\sqrt{m}\rc} S_i,\, 
                    M_i^{2\lc\sqrt{m}\rc} S_i,\,
                \ldots,\, M_i^{\lf\sqrt{m}\rf \lc\sqrt{m}\rc} S_i
          \right]
\]
berechnen und danach jedes Element des einen Vektors mit jedem Element
des anderen multiplizieren.

Genau genommen werden auf diese Weise einige Werte zuviel berechnet, wie
sich bei noch exakterer Analyse des Algorithmus zeigt. Es sind jedoch
vernachl"assigbar wenige. Die Berechnung dieser Werte kann durch
vernachl"assigbar geringen zus"atzlichen Aufwand verhindert werden. Um die
Darstellung des Algorithmus nicht unn"otig un"ubersichtlich zu machen,
werden diese Werte nicht weiter beachtet.

Vor Beginn der Rechnung wird ein \label{PageWahlEpsilon} 
\[ \epsilon \in \Rationals \, , \: 0 < \epsilon \leq 0.5 \]
festgelegt\footnote{ein Wert $\epsilon>0.5$ ist m"oglich, jedoch von
                     seinen Auswirkungen her uninteressant}.
O. B. d. A. sei $\epsilon$ so gew"ahlt, da"s 
gilt\footnote{erf"ullt $\epsilon$ diese Bedingung nicht, wird dadurch
              die Analyse des Algorithmus unn"otig un"ubersichtlich}
\[
   \exists \, p \in \Nat : \: p * \epsilon = 0.5 \MyPunkt
\]

Die Wahl von $\epsilon$ beeinflu"st das Verh"altnis zwischen der Anzahl
der Schritte und der Anzahl der dabei besch"aftigten Prozessoren.
Dies wird weiter unten durch die Analyse deutlich.

F"ur den Rest dieses Unterkapitels gelte die Vereinbarung, da"s mit
\[ a^b \] der Wert \[ \lc a^b \rc \] gemeint ist.

Mit Hinweis auf die Bemerkungen im Anschlu"s an die Behandlung der
Matrizenmultiplikation in Satz \ref{SatzAlgMatMult} wird im folgenden
f"ur die Multiplikation zweier $n \times n$-Matrizen ein Aufwand von
\[ \gamma_S (\lc \log(n) \rc + 1) \] Schritten und
\[ \gamma_P n^{2+\gamma} \] Prozessoren in Rechnung gestellt.

Im folgenden ist mit $T$, $U$ und $V$ jeweils $T_i$, $U_i$ bzw. $V_i$
gemeint, wobei $1 \leq i < n$ gilt.

Um den Vektor $U$ zu berechnen, benutzen wir folgenden iterativen 
Algorithmus\footnote{zur Vereinfachung der Darstellung werden keine 
ganzzahligen Werte zur Indizierung benutzt}: \nopagebreak[3]
\begin{itemize}
\item
      Der Vektor $Z_\alpha$ wird wie folgt definiert:
      \[ 
         Z_\alpha := \left[ R_i,\, R_i M_i,\, R_i M_i^2,\, 
                           \ldots ,\, R_i M_i^{m^\alpha - 1} 
                     \right]
      \]
      Das bedeutet, es gilt
      \[
          Z_0 = [ R_i ]
      \]
      Das Ziel ist es, $Z_{0.5}= U$ zu berechnen. 
      Der Vektor $Z_0$ ist bekannt, da $R_i$ Teil der Eingabe ist.
\item
      Wenn $Z_{\alpha}$ bekannt ist, im ersten Schleifendurchlauf
      also $Z_0$, dann wird daraus $Z_{\alpha+\epsilon}$ wie folgt 
      berechnet:
      \begin{itemize}
      \item
            Berechne
            \[ Y_{\alpha+\epsilon} :=
               \left[
                   M_i^{m^\alpha},\, M_i^{2m^\alpha},\, M_i^{3m^\alpha},
                      \, \ldots,\,
                   M_i^{m^\epsilon m^\alpha}
               \right]
            \]
            Nach \ref{SatzAlgPraefix} in Verbindung mit \ref{SatzAlgMatMult}
            erh"alt man f"ur die Anzahl der Schritte
            \begin{eqnarray*}
               & & \gamma_S \lceil \log(m^\epsilon) \rceil 
                   (\lceil \log(m) \rceil +1)
            \\ & \leq & %  <  ist hier falsch !
                        \gamma_S \lceil (\epsilon \lceil \log(m) \rceil + 1)
                            (\lceil \log(m) \rceil + 1 )
                     \rceil
            \\ & = & \gamma_S \lc \epsilon \lceil \log(m) \rceil^2 +
                         (\epsilon + 1) \lceil \log(m) \rceil + 1
                     \rc
            \end{eqnarray*}
            und f"ur die Anzahl der
            Prozessoren 
            \[ \gamma_P \lf 0.75 m^\epsilon \rf m^{2+\gamma}
                   <
               \gamma_P m^{2+\gamma+\epsilon} \MyPunkt
            \]
            Die
            daf"ur n"otige Startmatrix $M_i^\alpha$ erh"alt man als
            Nebenergebnis aus der Berechnung von $Y_\alpha$. Die Startmatrix
            f"ur die Berechnung von $Y_\epsilon$ ist $M_i$.
      \item
            Der Vektor $X_{\alpha+\epsilon}$ wird folgenderma"sen definiert:
            \[
               X_{\alpha+\epsilon} := 
               \left[
                   R_i M_i^{m^\alpha}, \,R_i M_i^{m^\alpha + 1}, \,
                   R_i M_i^{m^\alpha + 2}, \, \ldots, 
                   \, R_i M_i^{m^{\alpha+\epsilon}-1}
               \right]
            \]
            Es wird nun $X_{\alpha+\epsilon}$ aus $Z_\alpha$ und 
            $Y_{\alpha+\epsilon}$ berechnet.

            Der Vektor
            $Z_\alpha$ besitzt $m^\alpha$ Elemente, die ihrerseits Vektoren
            der L"ange $m$ darstellen. Sie werden mit
            \[
               z_{\alpha,1},\, z_{\alpha,2},\, \ldots,\, z_{\alpha,m^\alpha}
            \] bezeichnet.
            Der Vektor $Y_{\alpha+\epsilon}$ besitzt $m^\epsilon$
            Elemente. Diese Elemente sind $m \times m$-Matrizen und werden
            mit
            \[ y_{\alpha+\epsilon,1},\, y_{\alpha+\epsilon,2},\, \ldots,\,
               y_{\alpha+\epsilon,m^\epsilon}
            \] bezeichnet.
           
            \begin{tabbing}
                Der Vektor $X_{\alpha+\epsilon}$ wird wie folgt
                berechnet: \\
                    \hspace{1.5em} \= \hspace{1.5em} \= \kill 
                \> Parallel f"ur $i:= 1$ bis $m^\epsilon-1$ : \\
                \> \>  Parallel f"ur $j:= 1$ bis $m^\alpha$:
            \end{tabbing}
            \vspace{-4ex}
            \[
               x_{ \alpha+\epsilon,(i-1)*m^\epsilon+j}
                 := z_{\alpha,j} * y_{\alpha+\epsilon,i}
            \]
            Bei dieser Berechnung f"allt auf, da"s 
            $y_{\alpha+\epsilon,m^\epsilon}$ nicht verwendet wird. Diese 
            Matrix bildet die Startmatrix f"ur die Berechnung von
            $Y_{\alpha+2\epsilon}$ im n"achsten Schleifendurchlauf
            (s. o.).

            F"ur die Analyse des Aufwandes der Berechnung von 
            $X_{\alpha+\epsilon}$ wird $Z_\alpha$ als Matrix betrachtet.
            Die $z_{\alpha,j}$ bilden die Zeilenvektoren dieser Matrix.
            So gesehen sind also $m^\epsilon$ Matrizenmultiplikationen 
            durchzuf"uhren. Dies kann von 
            \[ \gamma_P m^{2+\gamma+\epsilon} \]
            Prozessoren in 
            \[ \gamma_S \lceil \log(m) \rceil + 1 \] 
            Schritten durchgef"uhrt werden.
      \item 
            Die ersten $m^\alpha$ Elemente des in diesem 
            Schleifendurchlauf gesuchten Vektors $Z_{\alpha+\epsilon}$
            werden durch die Elemente des Vektors $Z_\alpha$ gebildet und
            alle weiteren durch die Elemente des soeben berechneten 
            Vektors $X_{\alpha+\epsilon}$. 
      \end{itemize}
      Betrachtet man den Aufwand zur Berechnung von $Y_{\alpha+\epsilon}$
      und $X_{\alpha+\epsilon}$ zusammen, erh"alt man f"ur die
      Berechnung von $Z_{\alpha+\epsilon}$ aus $Z_{\alpha}$
      \[ 
         \gamma_S \lc \epsilon \lceil \log(m) \rceil^2 +
            (\epsilon + 2) \lceil \log(m) \rceil + 2                      
         \rc
      \]
      Schritte und \[ \gamma_P m^{2+\gamma+\epsilon} \] Prozessoren.
\item
      Insgesamt erfolgen \[ \frac{1}{2 \epsilon} \]
      Schleifendurchl"aufe. Der Aufwand zur Berechnung von $U$ betr"agt
      deshalb
      \begin{eqnarray*}
         & & \frac{ 0.5 \gamma_S }{ \epsilon }
         \lc \epsilon \lceil \log(m) \rceil^2 +
            (\epsilon + 2) \lceil \log(m) \rceil + 2                      
         \rc
      \\ 
         & \leq & 0.5 \gamma_S
         \lc \lceil \log(m) \rceil^2 +
            \left( 1 + \frac{2}{\epsilon} \right) \lceil \log(m) \rceil + 
            \frac{2}{\epsilon}
         \rc
      \end{eqnarray*} Schritte und
      \[ \gamma_P m^{2+\gamma+\epsilon} \] Prozessoren.
\end{itemize}

Im Anschlu"s an die Berechnung von $U$ erfolgt
die Berechnung von $V$ auf die gleiche Weise.
Der einzige wesentliche Unterschied zwischen den beiden
Berechnungsvorg"angen ist die andere
Startmatrix zur Berechnung des $Y_{\epsilon}$ entsprechenden Vektors. Hier
wird $M_i^{m^{0.5}}$ statt $M_i$ ben"otigt. Man erh"alt $M_i^{m^{0.5}}$
aus $M_i$ mit Hilfe der Bin"arbaummethode
nach \ref{SatzAlgBinaerbaum}. Dies kann in
\begin{eqnarray*}
    &      & \gamma_S \lc \log(m^{0.5}) \rc (\lc \log(m) \rc + 1)
\\  & \leq & \gamma_S \lc 0.5 \lc \log(m) \rc (\lc \log(m) \rc + 1) \rc
\\  & =    & \gamma_S \lc 0.5 (\lc \log^2(m) \rc + \lc \log(m) \rc + 1)  \rc
\end{eqnarray*}
Schritten von
\begin{eqnarray*}
    &      & \gamma_P \lc 0.5 m^{0.5} m^{2+\gamma} \rc
\\  & \leq & \gamma_P \lc 0.5 m^{2.5+\gamma} \rc
\end{eqnarray*}
Prozessoren geleistet werden.
Ist die Startmatrix berechnet, ist der weitere Aufwand zur Berechnung von
$V$ gleich dem Aufwand zur Berechnung von $U$. Also kann $V$ insgesamt 
in\footnote{Da die Terme, die die Anzahl der Schritte und Prozessoren
    beschreiben, bereits nach oben abgesch"atzt sind, wird bei der
    Zusammenfassung von Termen, die durch Gau"sklammern eingefa"st sind,
    auf eine weitere Absch"atzung verzichtet.}
\begin{eqnarray*}
   & & \gamma_S \lc 0.5 (\lc \log(m) \rc^2 + \lc \log(m) \rc + 1) +
           0.5  \left( \lceil \log(m) \rceil^2 +
           \left( 1 + \frac{2}{\epsilon} \right) \lceil \log(m) \rceil +
           \frac{2}{\epsilon} \right)
       \rc
\\ & = & \gamma_S
     \lc
         \lc \log(m) \rc^2 + 
         \left( 1 + \frac{1}{\epsilon} \right) \lc \log(m) \rc + 
         \frac{1}{\epsilon} + 0.5
     \rc
\end{eqnarray*}
Schritten erledigt werden. Die Anzahl der Prozessoren betr"agt
\begin{eqnarray*}
    \max \left( 
              \underbrace{ \gamma_P m^{2+\gamma+\epsilon} }_{\mbox{Term 1}}
         \, ,
              \underbrace{ \gamma_P \lc 0.5 m^{2.5+\gamma} \rc 
                         }_{\mbox{Term 2}}
         \right) \MyPunkt
\end{eqnarray*}
Da mit steigendem $m$ Term 2 st"arker w"achst als Term 1, wird die 
Analyse mit Term 2 f"ur die Anzahl der Prozessoren fortgesetzt.

Parallel zur Berechnung von $U$ wird zuerst
$M_i^{m^{0.5}}$ und mit Hilfe dieser Matrix dann $V$ berechnet. Der Aufwand
daf"ur betr"agt insgesamt
\[   \gamma_S
     \lc
         \lc \log(m) \rc^2 + 
         \left( 1 + \frac{1}{\epsilon} \right) \lc \log(m) \rc + 
         \frac{1}{\epsilon} + 0.5
     \rc
\] 
Schritte und
\[ 
   \gamma_P \lb m^{2+\gamma+\epsilon} + \lc 0.5 m^{2.5+\gamma} \rc \rb
\] Prozessoren.

Um den nach \equref{EquRMSVektor} gesuchten Vektor $T$ zu erhalten, 
m"ussen 
noch die Elemente der Vektoren $U$ und $V$, die ja ihrerseits wiederum 
Vektoren darstellen, miteinander multipliziert werden.
Die Vektoren $U$ und $V$ besitzen eine L"ange von $m^{0.5}$.
Die Multiplikation zweier Elemente dieser Vektoren k"onnen
analog zur Matrizenmultiplikation in \ref{SatzAlgMatMult} in 
\[ \lc \log(m) \rc + 1 \] Schritten von \[ m \] Prozessoren erledigt 
werden. Insgesamt sind \[ m^{0.5} * m^{0.5} = m \] solcher Multiplikationen
durchzuf"uhren. Die Berechnung von $T$ aus $U$ und $V$ kann
also in 
\[ \lc \log(m) \rc + 1 \] Schritten von \[ m^2 \] Prozessoren 
durchgef"uhrt werden.

Betrachtet man den Gesamtaufwand zur Berechnung von $U$, $V$ und $T$,
kommt man auf
\[ \gamma_S
   \lc
       \lc \log(m) \rc^2
       + \left( 1 + \frac{1}{\gamma_S} 
       + \frac{1}{\epsilon} \right) \lc \log(m) \rc
       + \frac{1}{\epsilon} + \frac{1}{\gamma_S} + 0.5
   \rc
\] 
Schritte und  
\Beq{EquBerkProzT}
   \gamma_P \lb m^{2+\gamma+\epsilon} + \lc 0.5 m^{2.5+\gamma} \rc \rb
      \leq
   \gamma_P \lb m^{2+\gamma+\epsilon} + 0.5 m^{2.5+\gamma} + 1 \rb
\Eeq
Prozessoren. 

Nach der
obigen Analyse der Berechnung einer der Vektoren kann die parallele
Berechnung aller Vektoren $T_1$ bis $T_{n-1}$ in
\Beq{TermBerkSchritte}
   \gamma_S
   \lc
       \lc \log(n-2) \rc^2 
       + \left( 1 + \frac{1}{\gamma_S}
       + \frac{1}{\epsilon} \right) \lc \log(n-2) \rc + 
       \frac{1}{\epsilon} + \frac{1}{\gamma_S} + 0.5
   \rc
\Eeq
Schritten durchgef"uhrt werden. Da die Berechnung eines Vektors $T_i$ f"ur 
$i>1$ bei gleichem $\epsilon$ schneller ist als die Berechnung von 
$T_1$, ist es m"oglich, dadurch Prozessoren zu sparen, da"s man 
$\epsilon$ f"ur jeden Vektor $T_i$ verschieden w"ahlt, und zwar als Funktion
von 
\begin{itemize}
\item
      der Gr"o"se $n$ der Eingabematrix $A$,
\item 
      der L"ange $m+1$ des jeweiligen Vektors $T_i$ und
\item
      dem $\epsilon$, da"s zur Berechnung des Vektors $T_1$
      verwendet wird.
\end{itemize}
Das separat f"ur jeden Vektor $T_i$ zu w"ahlende $\epsilon$ wird 
mit\footnote{ Es wurde bereits definiert: $m:= n-i-1$.}
$\epsilon_m$ bezeichnet.

Da die Vektoren $T$ f"ur $m \leq n-2$ berechnet werden sollen, mu"s f"ur
jedes $\epsilon_m$ mit $\epsilon_m \neq \epsilon$ die 
Bedingung $m \leq n-3$ erf"ullt sein.
Da gleichzeitig $m \geq 1$ erf"ullt sein mu"s, wird f"ur die folgenden
Analysen $n \geq 4$ angenommen. Andernfalls ist die Anwendung der Idee
zur Wahl der $\epsilon_m$ nicht sinnvoll.

Wie $\epsilon_m$ zu w"ahlen ist, ergibt sich aus
Term \equref{TermBerkSchritte}. Es mu"s gelten:
\[
   \lc \log(m) \rc^2 
   + \left( 1 + \frac{1}{\gamma_S} 
   + \frac{1}{\epsilon_m} \right) \lc \log(m) \rc +
   \frac{1}{\epsilon_m}
       \leq
   \lc \log(n-2) \rc^2 
   + \left( 1 + \frac{1}{\gamma_S} 
   + \frac{1}{\epsilon} \right) \lc \log(n-2) \rc +
   \frac{1}{\epsilon}
\]
L"ost man diese Ungleichung nach $\epsilon_m$ auf erh"alt man:
\Beq{EquWaehleEpsilonM}
   \epsilon_m
       \geq
   \frac{
       \lceil \log(m) \rceil + 1
   }{
       \lceil \log(n-2) \rceil^2 +
       \lb 1 + \frac{1}{\gamma_S} 
              + \frac{1}{\epsilon} 
       \rb \lceil \log(n-2) \rceil +
       \frac{1}{\epsilon} -
       \lceil \log(m) \rceil^2 -
       \lb 1 + \frac{1}{\gamma_S} \rb \lceil \log(m) \rceil
   }
\Eeq
Die Gau"sklammern in dieser Ungleichung f"uhren zu einigen wichtigen
Konsequenzen f"ur $n$, $m$ und $\epsilon_m$. Es gelte dazu 
\begin{eqnarray*}
    & k \in \Nat & \\
    & 1 \leq 2^k < m_1 \leq 2^{k+1} \leq n-2 & \\
    & 1 \leq 2^k < m_2 \leq 2^{k+1} \leq n-2 & \MyPunkt
\end{eqnarray*}
Aus \equref{EquWaehleEpsilonM} folgt dann
\[
    \epsilon_{m_1} = \epsilon_{m_2} \MyPunkt
\]
Das bedeutet insbesondere, da"s es u. U. einige $m$ mit 
$m \leq n-3$ gibt, f"ur die gilt 
\[ \epsilon_m = \epsilon \MyPunkt \] Der ung"unstigste Fall tritt f"ur
\[ n-2 = 2^{k+1} \] ein. Bei diesem Fall ist nur f"ur 
\[ m \leq \frac{n-2}{2} \]
die Bedingung \[ \epsilon_m < \epsilon \] erf"ullt.

F"ur die weitere Analyse ist es an dieser Stelle sinnvoll, die Gau"sklammern
im Term auf der rechten Seite von \equref{EquWaehleEpsilonM} zu beseitigen. 
Dazu wird
der Term nach oben abgesch"atzt.

Terme, die in Gau"sklammern eingefa"st sind, kann man mit Hilfe der
Beziehung
\Beq{EquSchaetzeGauss}
    a \leq \lc a \rc \leq a + 1
\Eeq
absch"atzen. Es gilt jedoch
\begin{eqnarray*}
   &      & \lc \log(x) \rc \\
   & \leq & \log(x) + 1     \\
   & =    & \log(2x) \MyPunkt
\end{eqnarray*} 
Zu beachten sind hier die Konsequenzen, wenn die Absch"atzung mit Hilfe
von \equref{EquSchaetzeGauss} vorgenommen werden. Falls $n-2$ eine 
Zweierpotenz ist, ergibt die auf diese Weise abgesch"atzte Ungleichung 
\equref{EquWaehleEpsilonM} nur
f"ur \[ m \leq \frac{n-2}{4} \] Werte f"ur $\epsilon_m$, so da"s 
\[ \epsilon_m < \epsilon \MyPunkt \]
Eine Verbesserung dieser Absch"atzung der Gau"sklammerfunktion ist 
w"unschenswert.

Dazu wird definiert, da"s eine Funktion $f(x)$
\index{konkav} {\em konkav auf einem Intervall I} ist, falls f"ur ihre
zweite Ableitung $f''(x)$ gilt:
\[ \forall x \in I: \: f''(x) \leq 0 \MyPunkt \]

Soll die Gau"sklammer einer konkaven Funktion $h(x)$ gebildet und die
Fl"ache unter der resultierenden Kurve berechnet werden, so l"a"st sich
der Ausdruck auch durch
\[ \int \lc h(x) dx \rc \leq \int (h(x) + 0.5) dx \]
nach oben absch"atzen. In Abbildung \ref{PicKonkav} ist dies verdeutlicht.
Dort ist Fl"ache 1 gr"o"ser als Fl"ache 2.
\begin{figure}[htb]
\begin{center}
    %
% You need 'bezier.sty'
% You need 'epic.sty'
%
\setlength{\unitlength}{1mm}
\makeatletter
\def\Thicklines{\let\@linefnt\tenlnw \let\@circlefnt\tencircw
\@wholewidth4\fontdimen8\tenln \@halfwidth .5\@wholewidth}
\makeatother
\begin{picture}(100.00,75.00)
\drawline(95.00,69.75)(70.25,69.75)(70.25,39.75)(15.00,39.75)(15.00,9.75)(5.00,9.75)
\bezier{100}(15.00,24.75)(40.00,54.75)(70.25,54.75)
\put(50.25,45.00){Fl"ache 1}
\bezier{100}(15.00,9.75)(40.00,39.75)(70.25,39.75)
\put(19.50,35.50){\vector(1,-4){0}}
\drawline(19.50,35.50)(15.50,52.00)
\put(10.75,52.75){Fl"ache 2}
\put(70.75,54.75){\vector(-4,3){0}}
\drawline(70.75,54.75)(81.25,46.75)
\put(70.50,39.75){\vector(-2,-1){0}}
\drawline(70.50,39.75)(81.25,45.50)
\put(82.50,44.75){$+0.5$}
\end{picture}
%

    \caption{Integration der Gau"sklammer einer konkaven Funktion}
    \label{PicKonkav}
\end{center}
\end{figure}
Somit kommen wir auf
\begin{eqnarray}
    & & \int \lc \log(x) \rc dx \nonumber \\
    & \leq & \int (\log(x) + 0.5 ) dx \label{EquLogNullFuenf} \\
    & = & \int \log(\sqrt{2}x) dx \nonumber \MyPunkt
\end{eqnarray}
Ist der abzusch"atzende Gau"sklammerterm Teil einer Funktion 
\[ h_2(\lc h(x) \rc,\, x ) \MyKomma \] so l"a"st sich die beschriebene
Absch"atzung durchf"uhren, falls $h_2$ monoton ist. Diese Bedingung ist bei 
den folgenden Anwendungen erf"ullt.

F"ur die folgenden Untersuchungen wird die Funktion
\[ g : \, \Rationals \rightarrow \Rationals \] eingef"uhrt. Sie sei
{ \em im Riemann'schen Sinne integrierbar } \cite{BS87} (S. 289)
auf dem Intervall \[ ( - \infty, \infty ) \] und
wird als Platzhalter f"ur die Funktion verwendet, die schlie"slich zur 
Absch"atzung der Gau"sklammerfunktion benutzt wird.
Je genauer sie die Gau"sklammerfunktion absch"atzt, umso besser werden
die Analyseergebnisse.

Mit Hilfe von \equref{EquWaehleEpsilonM} wird folgende Funktion zur 
Berechnung von $\epsilon_m$ bei gegebenen $n$ und $\epsilon$ definiert:
\[ f(m):=
   \frac{
       g(\log(m))+ 1
   }{
       g^2(\log(n-2))
       + \lb 1 + \frac{1}{\gamma_S} 
               + \frac{1}{\epsilon} 
         \rb g(\log(n-2))
       + \frac{1}{\epsilon} -
       g^2(\log(m))-
       \lb 1 + \frac{1}{\gamma_S} \rb g(\log(m))
   }
\]
Mit Hilfe dieser Funktion kommt man anhand von \equref{EquBerkProzT}
f"ur die Anzahl der Prozessoren zur Berechnung aller Vektoren $T$ auf:
\begin{eqnarray}
\nonumber
   & & 
   4 + (n-2)^{2+\gamma+\epsilon} + \frac{ (n-2)^{2.5+\gamma} }{ 2 } +
   \sum_{m=2}^{n-3}
       \left( m^{2+\gamma+f(m)} + \frac{ m^{2.5+\gamma} }{ 2 } + 1
       \right)
\\ 
\nonumber
   & \leq & 
   \overbrace{
       4 + (n-2)^{2+\gamma+\epsilon} + \frac{ (n-2)^{2.5+\gamma} }{ 2 }
   }^{t_1:=}
       +
   \int\limits_{2}^{n-2}
       \left( m^{2+\gamma+f(m)} + \frac{ m^{2.5+\gamma} }{ 2 } + 1
       \right) dm
\\ 
%%\label{EquBerkProzInt1} wird nicht ben"otigt
   & = &
       \overbrace{ 
           t_1 + \left. m \right|_2^{n-2} +
           \left. \left(
               \frac{1}{7+2\gamma} m^{3.5+\gamma}
           \right) \right|_2^{n-2} 
       }^{t_2:=}
       +
       \int\limits_2^{n-2}
           m^{2+\gamma+f(m)} dm
\\
\label{EquBerkProzInt2}
   & = & t_2 +
       \int\limits_2^{n-2}
       \overbrace{
           \MathE^{(2+\gamma+f(m))\ln(m)}
       }^{t_3:=} \, dm
\end{eqnarray}
F"ur die Integration von $t_3$ sind 3 Methoden von Bedeutung:
\begin{MyDescription}
\MyItem{Numerische Berechnung}
    Diese Methode ist f"ur gegebene $n$ und $\epsilon$ eine gangbare
    M"oglichkeit \cite{EM88} (Kap. 9). F"ur eine allgemeine 
    Analyse ist sie jedoch nicht geeignet.
\MyItem{Analytische Berechnung}
    Bezeichne $t_4$ die erste Ableitung von 
    \[ (2+\gamma+f(m))\ln(m) \MyPunkt \]
    Bezeichne $t_5'$ die erste Ableitung des zu bestimmenden Terms $t_5$.
    Da $t_3$ eine Exponentialfunktion ist, mu"s die gesuchte Stammfunktion,
    die Form $t_3 t_5$ besitzen.
    Die Ableitung dieses Ausdrucks ergibt 
    \[ (t_3 t_5)'= t_3 t_4 t_5+t_3 t_5' = t_3 \,
       \underline{ (t_4 t_5 + t_5') } \MyPunkt 
    \]
    Da der unterstrichene Teil den Wert 1 besitzen mu"s, ist die 
    Differentialgleichung \[ t_5' = 1 - t_4 t_5 \] zu l"osen. 
    Dies ist eine {\em explizite gew"ohnliche Differentialgleichung 
    erster Ordnung} \cite{BS87} (S. 414 ff.).

    Man gelangt zu der Vermutung, da"s zu $t_3$ keine Stammfunktion 
    existiert, da sowohl die Integration der Differentialgleichung, 
    als auch die Integration von $t_3$ mit Hilfe der Eigenschaften
    unbestimmter Integrale \cite{BS87} (S. 295) nicht zu 
    einem Ergebnis zu f"uhren scheinen. Unterst"utzt wird diese Vermutung
    durch die Tatsache, da"s f"ur \[ \MathE^{m^2} \] keine Stammfunktion 
    existiert.
\MyItem{Absch"atzung der Stammfunktion nach oben}
    Diese Methode ist am besten geeignet und wird im folgenden benutzt.
    Dazu wird der oben erw"ahnte Term $t_5$ so bestimmt, da"s gilt
    \Beq{EquBerkUngleichung}
        t_4 t_5 + t_5' \geq 1 \MyPunkt
    \Eeq 
\end{MyDescription}

Der Nenner von $f(m)$ wird mit $t_6$ bezeichnet, der Z"ahler mit $t_7$.
Die Ableitung von $t_3$ ergibt:
\begin{eqnarray*}
   \lefteqn{ 
       t_3' = t_3 
       \lb
           \frac{2}{m} + \frac{\gamma}{m}
       \rb
   } \\
   & + &
       \frac{
           \lb 
               g'(\log(m)) \frac{\log(m)}{m} + \frac{ g(\log(m)) }{m} 
               + \frac{1}{m} 
           \rb t_6
       }{ t_6^2 
       } \\
   & + &
       \frac{
           t_7 \ln(m) 
           \lb \frac{ 2 g(\log(m)) g'(\log(m)) }{m \ln(2) }
               + \frac{ \lb 1 + \frac{1}{\gamma_S} \rb
                        g'(\log(m)) 
                 }{ m \ln(2) }
           \rb
       }{ t_6^2
       } 
\end{eqnarray*}
Wie bereits beschrieben wurde, ist es an dieser Stelle m"oglich
\Beq{EquDefGaussNullFuenf}
    g(x) := x + 0.5
\Eeq
als obere Absch"atzung der Gau"sklammerfunktion zu verwenden. Es ist zu
beachten, da"s diese Absch"atzung nicht f"ur die Berechnung von $\epsilon_m$
bei der Anwendung des Algorithmus in einer konkreten Situation
benutzt werden darf. Die Benutzung von \equref{EquDefGaussNullFuenf} an
dieser Stelle ist nur zul"assig, weil die Absch"atzung des gesamten 
Integrals in \equref{EquBerkProzInt2} das Ziel ist.

Wie ebenfalls bereits beschrieben wurde, ist darauf zu achten, da"s die
auftretenden Werte f"ur $\epsilon_m$ kleiner oder gleich $\epsilon$ sind.
Damit dies der Fall ist muss wegen der mit \equref{EquDefGaussNullFuenf} 
gew"ahlten Absch"atzung gelten:
\begin{MyEqnArray}
    \MT  \log( \sqrt{2}m ) \MT \leq \MT \lc \log( n - 2 ) \rc 
\MNl
    \Rightarrow \MT \log(m) \MT \leq \MT
        \lc \log \lb \frac{ n - 2 }{ \sqrt{2} } \rb \rc
\MNl
    \Rightarrow \MT m \MT \leq \MT 
        2^{ \lc \log \lb \frac{ n - 2 }{ \sqrt{2} } \rb \rc }
\end{MyEqnArray}

Ungleichung
\equref{EquBerkUngleichung} wird erf"ullt, wenn man 
\[ t_5 = m \] w"ahlt. Die G"ultigkeit dieser Behauptung ergibt
sich insbesondere aus
der Betrachtung der Gr"o"senordnungen der Z"ahler und Nenner in der
Ableitung von $t_3$.

So erh"alt man durch Absch"atzung von \equref{EquBerkProzInt2} nach oben
f"ur die Anzahl der Prozessoren:
\begin{eqnarray*}
    & & t_2 + 
        \int\limits_{ 2^{\lf \log \lb \frac{n-2}{\sqrt{2}} \rb \rf} }^{n-2} 
            m^{2+\gamma+0.5} dm
        + \int\limits_2^{ 2^{\lf \log \lb \frac{n-2}{\sqrt{2}} \rb \rf} }
            \MathE^{ (2+\gamma+f(m)) \ln(m) } dm \\
    & \leq & t_2 + 
        \left. \frac{1}{3.5+\gamma} m^{3.5+\gamma} 
        \right|_{ 2^{\lf \log \lb \frac{n-2}{\sqrt{2}} \rb \rf} }^{n-2}
        + \left. \lb m^{ 2+\gamma+f(m) } m \rb
          \right|_2^{ 2^{\lf \log \lb \frac{n-2}{\sqrt{2}} \rb \rf} } \\
    & = &  4 + (n-2)^{2+\gamma+\epsilon} + 
               \frac{ (n-2)^{2.5+\gamma} }{ 2 } \\
    & & 
         + \left. m \right|_2^{n-2} +
           \left. \left(
               \frac{1}{7+2\gamma} m^{3.5+\gamma}
           \right) \right|_2^{n-2} 
        +  
        \left. \frac{1}{3.5+\gamma} m^{3.5+\gamma} 
        \right|_{ 2^{\lf \log \lb \frac{n-2}{\sqrt{2}} \rb \rf} }^{n-2} \\
    & & +
         \left. m^{3+\gamma+f(m)}
         \right|_2^{ 2^{\lf \log \lb \frac{n-2}{\sqrt{2}} \rb \rf} }
\end{eqnarray*}
Dieser Term wird mit $t_8$ bezeichnet. An ihm erkennt man, da"s die
Anzahl der Prozessoren mit wachsendem $n$ asymptotisch
\[
    \frac{3}{7+2\gamma} (n-2)^{3.5+\gamma}
\]
betr"agt.

Schlie"slich m"ussen noch die Matrizen $C_1$ bis $C_n$ miteinander
multipliziert werden. Dies geschieht mit Hilfe der Bin"arbaummethode
nach \ref{SatzAlgBinaerbaum}. Es wird definiert
\[ n':= \lf \frac{n}{2} \rf \MyPunkt \]
Da nach \ref{SatzToeplizMult}
bei allen Multiplikationen Dreiecks-Toeplitz-Matrizen verkn"upft werden
und $C_i$ eine $(n-i+2) \times (n-i+1)$-Matrix handelt, k"onnen diese
Multiplikationen in
\Beq{TermBerkSchritteB}
   (\lc \log(n+1) \rc + 1) \lc \log(n) \rc
\Eeq Schritten von weniger als
\begin{eqnarray*}
    & & \sum_{k=1}^{n'} (n-2k+2) * (n-2k+1) \\
    & = & \sum_{k=1}^{n'} (2k(2k-1)) \\
    & = & 4 \sum_{k=1}^{n'} k^2 - 2 \sum_{k=1}^{n'} k \\
    & \MyStack{\ref{SatzSumK}, \, \ref{SatzSumK2} }{ = } & 
        4 \frac{n'(n'+1)(2n'+1)}{6} - 2\frac{n'(n'+1)}{2}
\end{eqnarray*}
Prozessoren durchgef"uhrt werden. F"ur ein gegebenes $n$ ist
der Wert dieses Terms ist kleiner als
der Wert von $t_8$.

In Verbindung mit \ref{SatzDdurchP}
ergibt sich als Endergebnis der Analyse, da"s mit Hilfe des in diesem 
Kapitel vorgestellen Algorithmus die Determinante einer $n \times n$-Matrix
in weniger als\footnote{Summe von \equref{TermBerkSchritte} und
 \equref{TermBerkSchritteB}}
\[
   \gamma_S
   \lc
       \lc \log(n-2) \rc^2 
       + \left( 1 + \frac{1}{\gamma_S}
       + \frac{1}{\epsilon} \right) \lc \log(n-2) \rc + 
       \frac{1}{\epsilon} + \frac{1}{\gamma_S} + 0.5
   \rc +
   (\lc \log(n+1) \rc + 1) \lc \log(n) \rc
\]
Schritten von weniger als $t_8$ Prozessoren berechnet werden kann.

Da in der Praxis f"ur die Matrizenmultiplikation Satz 
\ref{SatzAlgMatMult} statt der in \cite{CW90} angegebenen Methode 
benutzt wird, ist f"ur diesen Fall in allen obigen Termen
\[ \gamma = \gamma_S = \gamma_P = 1 \] zu setzen.

Vergleicht man B-Alg. mit C-Alg., BGH-Alg. und P-Alg., f"allt wiederum
das Fehlen von Fallunterscheidungen auf. Weiterhin werden wie bei BGH-Alg.
keine Divisionen verwendet, so da"s B-Alg. auch in Ringen anwendbar ist.

Betrachtet man die Aufwandsanalyse, so erkennt man, da"s B-Alg. 
leicht schlechter ist als C-Alg. und deutlich besser als BGH-Alg. .
In Kapitel \ref{ChapPan} wird P-Alg. in diese Rangfolge eingereiht.

