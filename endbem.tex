%
% Datei: endbem.tex
%
\MyChapter{Nachbetrachtungen}
\label{ChapEndbem}

In diesem Kapitel werden die Ergebnisse dieser Arbeit abschlie"send 
betrachtet und bewertet.

\MySection{Vergleich der Algorithmen}
% Csanky: schnellster, am einfachsten zu implementieren, jedoch Divisionen
% BGH: relativ gesehen sehr langsam, Implementierung sehr aufw"andig,
%      nur f"ur positive ganze Zahlen (mit Tricks: ...)
% Berk: etwas langsamer als 'Csanky', etwas schwieriger zu implementieren,
%       keine Divisionen!
% Pan: nicht f"ur alle Matrizen verwendbar; nur ganze Zahlen

In diesem Unterkapitel werden die vier haupts"achlich in dieser Arbeit 
behandelten Algorithmen zusammenfassend miteinander verglichen.

Auf die Algorithmen wird, wie in Unterkapitel \ref{SecBez} definiert 
wird, mit Hilfe der Anfangsbuchstaben ihrer Autoren
Bezug genommen: C-Alg., BGH-Alg., B-Alg. und P-Alg..

Der Vergleich wird nach folgenden Kriterien durchgef"uhrt:
\begin{itemize}
\item
      Werden Divisionen benutzt?
\item
      Werden Fallunterscheidungen durchgef"uhrt?
\item
      Wie gro"s ist die Anzahl der Schritte?
\item
      Wie gro"s ist die Anzahl der Prozessoren?
\item
      Welche Einschr"ankungen f"ur die Anwendbarkeit gibt es?
\item
      Wie aufwendig ist die Implementierung?
\item
      Welche "Ahnlichkeiten in der Methodik gibt es? 
\end{itemize}
Im Anschlu"s an diesen Vergleich wird eine Bewertung der Algorithmen 
anhand der genannten Gesichtspunkte vorgenommen.

Zun"achst sei angemerkt, da"s keiner der Algorithmen Fallunterscheidungen
verwendet. Dies vereinfacht eine Betrachtung aus der Sicht des 
Schaltkreisentwurfs.

Es kommen lediglich B-Alg. und BGH-Alg. ohne Divisionen aus und k"onnen 
somit in beliebigen Ringen angewendet werden. C-Alg. und P-Alg. k"onnen nur
in K"orpern verwendet werden, falls man exakte Ergebnisse verlangt

P-Alg. darf nur auf ganzzahlige Matrizen
angewendet werden. Da P-Alg. Divisionen verwendet, ist es jedoch auch dann
nicht gew"ahrleistet, da"s er die Determinante exakt liefert. Hinzu kommt
bei P-Alg. als erheblicher Nachteil die eingeschr"ankte Verwendbarkeit, auch 
wenn sich dies in der praktischen Anwendung nicht sehr stark auswirkt. 
Diese Eigenschaft von P-Alg. ist mit dem Laufzeitverhalten von `Quicksort'
vergleichbar, das im {\em Average Case} sehr gut, jedoch im {\em Worst 
Case} sehr schlecht ist. P-Alg. ist theoretisch nur eingeschr"ankt 
verwendbar, praktisch jedoch nahezu uneingeschr"ankt, da zuf"allig
zusammengestellte Matrizen mit sehr hoher Wahrscheinlichkeit invertierbar
sind und die Bedingungen f"ur P-Alg. erf"ullen.

Betrachtet man die Anzahl der Schritte, stellt man fest, da"s C-Alg. 
am schnellsten ist, gefolgt von B-Alg. und P-Alg.. Das Schlu"slicht 
bildet mit einigem Abstand BGH-Alg. 

Bez"uglich der Anzahl der Prozessoren hat P-Alg. die Nase vorn, gefolgt
von C-Alg. und B-Alg.. Das Schlu"slicht bildet BGH-Alg. wiederum mit
Abstand. Zu beachten ist der sehr gute Wert f"ur die Prozessoren bei P-Alg.,
der sich vermutlich kaum weiter verbessern l"a"st.

Auf einem Rechner mit einem Prozessor ist P-Alg. der effizienteste, da
er insgesamt die wenigsten Operationen ben"otigt. Er wird gefolgt von 
C-Alg. B-Alg. liegt hier an dritter Stelle gefolgt von BGH-Alg..

Betrachtet man den Aufwand f"ur die Implementierung gemessen in Anzahl der
Quelltextzeilen\footnote{Dies besitzt in der Praxis in Verbindung mit
den anderen Eigenschaften der Algorithmen eine nicht zu untersch"atzende 
Bedeutung.}, stellt man fest, da"s C-Alg. bei weitem am einfachsten 
zu implementieren ist. Ihm folgt P-Alg. dichtauf. Die Implementierung
von B-Alg. ist bereits etwas aufwendiger, aber noch zumutbar. BGH-Alg.
bildet auch in diesem Punkt mit relativ gro"sem Abstand das Schlu"slicht.

Bei der Betrachtung der Verfahren, die die einzelnen Algorithmen verwenden,
erkennt man Parallelen zwischen C-Alg. und B-Alg. sowie zwischen 
BGH-Alg. und P-Alg.. Sowohl in C-Alg. als auch in B-Alg. wird jeweils
ein Satz verwendet, der zum Zeitpunkt der Ver"offentlichung der Algorithmen
bereits seit mehreren Jahrzehnte bekannt war. Die beiden S"atze sind 
durch diverse Umformungen f"ur die parallele Determinanteberechnung nutzbar
gemacht worden. In BGH-Alg. und P-Alg. hingegen werden jeweils mehrere
auch separat bedeutsame Verfahren zu einem Algorithmus in Verbindung 
gebracht.

Unter dem Gesichtspunkt, da"s BGH-Alg. nicht der einzige divisionsfreie
Algorithmus mit polynomiellem Aufwand ist, besitzt er wegen des
erforderlichen Implementierungsaufwandes und seiner relativen Langsamkeit
nur theoretisches Interesse. 

C-Alg. ist am brauchbarsten f"ur schnelle Erstellung einer Implementierung,
falls das Vorhandensein von Divisionen nicht weiter st"ort.

P-Alg. sollte verwendet werden, falls es im wesentlichen auf Geschwindigkeit
ankommt und die Einschr"ankungen f"ur die Anwendbarkeit sowie die
Existenz von Divisionen nicht st"oren.

B-Alg. ist der Algorithmus unter den vieren mit den ausgewogensten
Leistungsmerkmalen und geht insgesamt als Sieger aus dem Vergleich hervor.
Er besitzt eine ausreichende Effizienz, kommt ohne Divisionen aus und
unterliegt keinen sonstigen Einschr"ankungen, wie z. B. P-Alg.. Im
Zweifelsfall sollte immer B-Alg. verwendet werden.

Um zum Abschlu"s einen Eindruck von der Effizienz der
Algorithmen im direkten Vergleich
zu liefern, sind f"ur Tabelle \ref{TabVergleich} die aus den
Aufwandsanalysen hervorgegangene Terme beispielhaft ausgewertet worden.
In der Tabelle werden bei den Anzahlen der Prozessoren die guten
Werte f"ur P-Alg. und die auff"allig schlechten Werte f"ur BGH-Alg.
besonders deutlich.

\begin{table}[htb]
    \begin{center}
    \begin{tabular}{|c||r|r||r|r||r|r||r|r|}
        \hline
        n & \multicolumn{2}{|c||}{C-Alg.}   & 
            \multicolumn{2}{|c||}{BGH-Alg.} &
            \multicolumn{2}{|c||}{B-Alg.}   &
            \multicolumn{2}{|c|}{P-Alg.} \\
        \cline{2-9}
          & Schr. & Proz. & Schr. & Proz. &
            Schr. & Proz. & Schr. & Proz. \\
        \hline
        2 & 9 & 8  & 16 & 24  & 12 & 4  & 15 & 8 \\ \hline
        4  & 16 & 128  & 55 & 1486  & 19 & 27  & 31 & 64 \\ \hline
        6  & 25 & 648  & 99 & 15343  & 27 & 361  & 53 & 216 \\ \hline
        8  & 25 & 2048  & 118 & 81284  & 36 & 2234  & 53 & 512 \\ \hline
        10  & 36 & 5000  & 172 & 298460  & 41 & 6417  & 81 & 1000 \\ \hline
        12  & 36 & 10368  & 180 & 867715  & 49 & 15670  & 81 & 1728 \\ \hline
        14  & 36 & 19208  & 196 & 2145346  & 49 & 41922  & 81 & 2744 \\ \hline
        16  & 36 & 32768  & 205 & 4708392  & 53 & 72317  & 81 & 4096 \\ \hline 
        18  & 49 & 52488  & 265 & 9431425  & 59 & 120961  & 115 & 5832 \\ \hline
        20  & 49 & 80000  & 275 & 17574870  & 69 & 194580  & 115 & 8000 \\ \hline
    \end{tabular}
    \end{center}
    \caption{Vergleich der Algorithmen}
    \label{TabVergleich}
\end{table}

% $$$ Reihenentwicklung Grad >n fuer BGH-Alg. sch"adlich
%     Algorithmen auch ohne Parallelisierung interessant

\MySection{Ausblick}

Zum Schlu"s folgt noch eine kurze Liste weiterer Themen, auf die man bei der 
Bearbeitung der vorliegenden Arbeit st"o"st:

\begin{itemize}
\item Analyse von Schaltkreisen f"ur die beschriebenen Algorithmen
      % \cite{ ... Ingos Buch}
\item Betrachtung von Matrizen mit Elementen aus $\Complex$
      % Diagonalisierbarkeit in $\Complex$ !? --> Methode von Krylov
\item Analyse verschiedener Varianten der beschriebenen Algorithmen
      % Pan: 3 Verfahren zur iterativen Invertierung
      %      etliche Verfahren zur Bestimmung der N"aherungsinversen
      % Berk: Wahl von \epsilon; Wahl der Parallelisierung von
      %       Algorithmusteilen
      % BGH: etliche M"oglichkeiten, die Konvergenz der Potenzreihen
      %      sicherzustellen
\item Analyse des Aufwandes f"ur die Aufgabenverteilung zwischen mehreren
      Prozessoren
\item Betrachtung anderer Rechnermodelle (insbesondere Rechnermodelle
      ohne gemeinsamen Speicher f"ur die Prozessoren)
      % CREW, Arten der Zugriffsregelung
      % kein gemeinsamer Speicher (Analyse des Kommunikationsaufwandes)
% \item Suche weiterer Algorithmen  % selbstverst"andlich ...
\item Analyse des Speicherplatzverbrauchs
%\item Implementierung von Flie"skommazahlen mit einer beliebigen Anzahl von
%      Stellen
% Implementierung:  Zahlen beliebiger L"ange 
% $$$$$            (--> Satz "uber maximale Gr"o"se der Eigenwerte)
\end{itemize}

Es bleibt also einiges zu tun... . F"ur dieses Mal soll das jedoch alles 
sein.
\vspace{5ex}

Wenn die Gedanken wieder leichter flie"sen...
\begin{verse}
    Himmlische Stille rauscht durch die Nacht \\
    Samtenes Schweigen str"omt durch die Luft \\
    glitzernd breitet sich das Tal in der Ferne \\[2ex]
    
    kein Laut \\
    ohne Hast l"a"st die Ruhe verbreiten ihr Gl"uck
\end{verse}

