%
% Datei: bgh.tex (Textteile nach 'BGH82')
%

\MyChapter{Der Algorithmus von {Borodin,} Von zur Gathen und Hopcroft}
\label{ChapBGH}

Der Algorithmus \cite{BGH82}, der in diesem Kapitel dargestellt wird,
verbindet die Vermeidung von Divisionen \cite{Stra73}, das Gau"s'sche
Eliminationsverfahren (z. B. \cite{BS87} S. 735)
und die parallele Berechnung von Termen \cite{VSBR83} miteinander, um
die Determinante einer Matrix zu berechnen. Auf diesen Algorithmus wird
mit {\em BGH-Alg.} Bezug genommen (vgl. Unterkapitel \ref{SecBez}).

Er unterschiedet sich in
seiner Methodik von den anderen Algorithemen (C-Alg., B-Alg. und P-Alg.)
vor allem dadurch, da"s er die Koeffizienten des charakteristischen
Polynoms in keiner Weise beachtet (vgl. \ref{SatzDdurchP}), sondern die
Determinante durch miteinander verkn"upfte Transformationen, nicht zuletzt
auch durch Ausnutzung von Satz \ref{SatzDetPermut}, direkt
berechnet.

Wie sich in diesem Kapitel zeigen wird, besitzt der Algorithmus durch
die Verbindung der drei o. g. Verfahren eine gewisse Eleganz, besonders,
was die Handhabung der Konvergenz von Potenzreihen angeht. 

Ein Nachteil des Algorithmus ist die vergleichsweise schlechte Effizienz.

% **************************************************************************

\MySection{Das Gau"s'sche Eliminationsverfahren}
\label{SecGauss}

\index{Gau{\Mys}s'sches Eliminationsverfahren}
Das Gau"s'sche Eliminationsverfahren wird dazu benutzt, lineare
Gleichungssysteme der Form \[ Ax=b \] zu l"osen. Dazu wird die sogenannte
{\em erweiterte Koeffizientenmatrix} betrachtet. Sie ist eine
$n \times (n+1)$-Matrix, deren erste $n$ Spalten aus den Spalten der
Koeffizientenmatrix $A$ bestehen und deren $(n+1)$-te Spalte aus dem
Vektor $b$ besteht.

Die Idee des Gau"s'schen Eliminationsverfahrens ist es, die erweiterte
Koeffizientenmatrix so zu transformieren,
da"s die darin enthaltene Matrix $A$ die Form einer {\em oberen
Dreiecksmatrix} \index{Dreiecksmatrix} bekommt. F"ur eine solche
$n \times n$-Matrix gilt:
\[ \forall 1 \leq j < i \leq n: a_{i,j} = 0 \]
Falls f"ur die Matrix
\[ \forall 1 \leq i < j \leq n: a_{i,j} = 0 \]
erf"ullt ist, nennt man sie {\em untere Dreiecksmatrix}.
Zur Transformation werden \index{Zeilenoperationen!elementare}
{\em elementare Zeilenoperationen} verwendet.
Sie werden in Definition \ref{DefDet} der Determinanten einer Matrix
unter D1 und D3 beschrieben. Sie haben nicht nur die dort genannten
Beziehungen zur Determinanten einer Matrix, sondern noch zus"atzlich die
Eigenschaft, da"s sie, angewandt auf die erweiterte Koeffizientenmatrix,
die L"osungsmenge des linearen Gleichungssystems unver"andert lassen.

F"ur die Determinantenberechnung wird die erweiterte Koeffizientenmatrix
nicht weiter beachtet. Alle Operationen beziehen sich nur auf die Matrix
$A$. Die Matrizenelemente unterhalb der
Hauptdiagonalen\footnote{ Die Hauptdiagonale bilden $a_{1,1}$ bis
$a_{n,n}$.} werden spaltenweise durch Nullen ersetzt, beginnend mit der
ersten Spalte. Die Transformationen werden durch folgende Gleichungen
beschrieben\footnote{ Das Gau"s'sche Eliminationsverfahren wird im
weiteren Text so modifiziert, da"s Divisionen durch Null nicht
vorkommen k"onnen. Dieser Fall wird deshalb schon hier au"ser Acht
gelassen.}:
\begin{eqnarray}
    \label{Equ1GaussDef}
    a_{i,j}^{(0)} & := & a_{i,j}
\\  \label{Equ2GaussDef}
    a_{i,j}^{(k)} & := & \left\{
                            \begin{array}{lcr}
                                a_{i,j}^{(k-1)} & : & i \leq k
                            \\  a_{i,j}^{(k-1)}-a_{k,j}^{(k-1)}
                                    \frac{ a_{i,k}^{(k-1)} }{
                                           a_{k,k}^{(k-1)}  }
                                               & : & i > k
                            \end{array}
                        \right.
\end{eqnarray}
Die so gewonnene Matrix $A^{(n)}$ ist die gesuchte obere Dreiecksmatrix.
Betrachtet man Satz \ref{SatzDetPermut}, erkennt man, da"s sich die
Determinante dieser Dreiecksmatrix dadurch berechnen l"a"st, da"s man
die Elemente der Hauptdiagonalen miteinander multipliziert. Da man nur
die in \ref{DefDet} erw"ahnten Operationen verwendet hat, erh"alt man so
auch die Determinante der Matrix $A$.

% **************************************************************************

\MySection{Potenzreihenringe}
\label{SecPotRing}

Im darzustellenden Algorithmus spielen Potenzreihenringe eine wichtige
Rolle. Deshalb werden in diesem Unterkapitel die f"ur uns interessanten
Eigenschaften dieser Ringe behandelt. F"ur unsere Betrachtungen sind Ringe
mit einer zus"atzlichen Eigenschaft von besonderem Interesse:

\MyBeginDef
\label{DefEinheit}
    Sei $R$ ein Ring. Ein $x \in R$ wird als \index{Einheit!in einem Ring}
    {\em Einheit}\footnote{nicht zu verwechseln mit Einselement}
    bezeichnet, wenn es ein $y \in R$ gibt, so da"s
    \[ x * y = 1 \MyPunkt \] Gibt es in $R$ solche Elemente, so wird $R$ 
    als { \em Ring mit Division durch Einheiten } bezeichnet. 
\MyEndDef

Falls in diesem
Kapitel von Ringen die Rede ist, sind immer Ringe mit Division durch
Einheiten gemeint, falls nicht ausdr"ucklich etwas anderes angegeben wird.

Sei $M$ eine Menge von Unbestimmten:
\[ M:= \{ x_1,\,x_2,\, \ldots,\, x_u \} \MyPunkt \]
Dann hei"st $ R[M] $ \index{Ring!{\Myu}ber {\mit M}}
{\em Ring "uber $M$}. F"ur $R[M]$ schreiben wir auch abk"urzend $R[]$.
Die Elemente von $R[]$ sind Terme, in denen neben den
Elementen von $R$ zus"atzlich Elemente von $M$ als Unbestimmte auftreten
d"urfen. 

Analog zur Definition von $R[]$ wird $R[[M]]$ definiert als {\em
Potenzreihenring "uber $M$}. \index{Potenzreihenring!{\Myu}ber {\mit M}}
F"ur $R[[M]]$ wird auch $R[[]]$ geschrieben.
Die Elemente von $R[[]]$
besitzen folgendes Aussehen:
\begin{itemize}
\item
      Sei $T$ eine Teilmenge\footnote{$T$ kann unendlich gro"s sein}
      von $\Nat^{n^2}$.
\item
      F"ur ein $e\in T$ bezeichne $e_i$ das $i$-te Element.
\item
      F"ur $e \in T$ sei \[ k_{e_1,e_2,\ldots,e_{n^2}} \in R \]
\item
      Jedes $u \in R[[]]$ hat f"ur geeignete $k_i$ und eine geeignete Menge
      $T$ die Form:
      \Beq{EquAllgemeinePotenzreihe}
         \sum_{e: \{e_1,e_2,\ldots,e_u \} \in T}
                                                 k_{e_1,e_2,\cdots,e_u}
             \prod_{i=1}^u x_i^{e_i}
      \Eeq
\end{itemize}
Die Summe der Glieder von $u$, f"ur die gilt
   \[ \sum_{i=1}^u e_i = p \]
wird {\em homogene Komponente vom Grad
$p$} \index{homogene Komponente} genannt. Die homogene Komponente
vom Grad $0$ wird auch {\em konstanter Term} \index{konstanter Term}
genannt.

Der Ring $R[]$ enth"alt $R[[]]$ als Unterring und dieser wiederum als
Unterring den Ring der Polynome "uber den Unbestimmten $M$.

\sloppy
Der Potenzreihenring $R[[]]$ besitzt eine f"ur uns besonders interessante
Eigenschaft. Dazu zu\-n"achst der folgende Satz: \fussy
\begin{satz}[Taylor]
\index{Taylor!Satz von}
\label{SatzTaylor}
    Eine Funktion $f$ sei in \[ (x_0-\alpha,x_0+\alpha) \] mit
    \[ \alpha > 0 \] $(n+1)$-mal differenzierbar. Dann gilt f"ur
    \[ x \in (x_0-\alpha,x_0+\alpha) \] die {\em Taylorentwicklung}
    \[
        f(x)= \sum_{\nu = 0}^n \frac{ f^{(\nu)}(x_0) }{ \nu! }
                  (x-x_0)^{\nu} + R_n(x)
    \]
    mit
    \[
        R_n(x):= \frac{ f^{(n+1)}(x_0+ \vartheta(x-x_0)) }{ (n+1)! }
                 (x-x_0)^{n+1}
    \]
    wobei \[ \vartheta \in (0,1) \] und $x_0$ der sogenannte
    {\em Entwicklungspunkt} ist.
\end{satz}
\begin{beweis}
    \cite{Hild74} S. 33-35
\end{beweis}
Weitere Literatur zum Thema 'Taylorreihen' ist z. B. \cite{BS87} (S. 31 und
269). Ein Beispiel f"ur die Anwendung von Satz \ref{SatzTaylor} ist die
Funktion
\Beq{Equ1TylorBeispiel}
    f_1(x) := \frac{1}{1-x} \MyPunkt
\Eeq
Sie ist unendlich oft
differenzierbar mit dem Entwicklungspunkt $x_0=0$ erh"alt man die
Potenzreihe
\Beq{Equ2TaylorBeispiel}
    f_2(x) = \sum_{i=0}^{\infty} x^i \MyPunkt
\Eeq
Der {\em Konvergenzradius} \index{Konvergenzradius} (\cite{BS87} S. 366)
betr"agt $1$, d. h. nur f"ur
\[ |x| < 1 \] gilt \[ f_1(x)=f_2(x) \MyPunkt \]
F"ur den Konvergenzradius $k$ wird das Intervall 
\[ (k,-k) \] als {\em Konvergenzbereich} \index{Konvergenzbereich} 
bezeichnet.

Satz \ref{SatzTaylor} l"a"st sich auch auf mehrere Unbestimmte
verallgemeinern. F"ur uns ist dabei nur folgendes interessant:
\begin{quote}
     Seien \[ f,g \in R[[]] \MyPunkt \]
     Der konstante Term von $g$ sei gleich Null.
     F"ur die Unbestimmten gelte 
     \Beq{Equ2Konvergenz}
         x_1,\ldots, x_u \in (-1,1) \MyPunkt
     \Eeq
     Sei $g$ konvergent.
     Dann folgt aus Satz \ref{SatzTaylor}, da"s sich
     in $R[[]]$ Divisionen der Form
     \Beq{Equ1ZuErsetzen}
         \frac{f}{1-g}
     \Eeq
     ersetzen lassen durch
     \Beq{Equ1StattDivision}
        f*(
              \underbrace{1+g+g^2+\ldots}_{ (*) }
          ) \MyPunkt
     \Eeq
\end{quote}
Die Potenzreihe $g$ wird als {\em innere} Reihe bezeichnet.
Die Terme $(*)$
sind ebenfalls Potenzreihen und werden als
{\em "au"sere} Reihen bezeichnet. Setzt man die {\em innere} Reihe in eine
der {\em "au"seren} ein, erh"alt man wiederum eine Potenzreihe. Diese wird
als {\em Gesamtreihe} bezeichnet.

Im obigen Beispiel konvergiert die Gesamtreihe, falls die innere
Reihe konvergiert und ihre Unbestimmten innerhalb des Konvergenzradius
der "au"seren liegen. Da diese Bedingungen erf"ullt sind, folgt die
Konvergenz der Reihe \equref{Equ1StattDivision}. Um die Konvergenz
in praktisch nutzbarem Ma"se sicherzustellen, sollte der Betrag der Werte,
die f"ur die Unbestimmten eingesetzt werden, nicht beliebig nahe bei $1$
liegen.

Konvergenz ist beim Umgang mit Potenzreihen ein wichtiges Thema. 
Besonders beim Ver\-kn"u\-pfen von Potenzreihen mit mehreren 
Unbestimmten, wie
im vorliegenden Fall, sind Konvergenzbetrachtungen u. U. komplex.
Allgemeine Betrachtungen der Konvergenz von Potenzreihen mit mehreren
Unbestimmten f"uhren an dieser Stelle zu weit und sind z. B. in
\begin{itemize}
\item
      \cite{BT70} ab S. 1 sowie ab S. 49 \hspace{2em} und
\item
      \cite{Hoer73} ab S. 34 
\end{itemize}
zu finden.

Bei praktischen Berechnungen k"onnen Potenzreihen nicht beliebig weit
entwickelt werden, da die Rechenleistung beschr"ankt ist. Deshalb mu"s
ein Grad festgelegt werden, bis zu dem die Potenzreihen entwickelt werden.
Dieser Grad ist i. A. besonders von der St"arke der Konvergenz der Reihe
abh"angig, die entwickelt werden soll. Die Festsetzung eines solchen
Grades erfordert eine Analyse des jeweiligen Problems, das mit Hilfe der
Entwicklung in Potenzreihen gel"ost werden soll. So kann eine Potenzreihe
als Endergebnis mehrerer hintereinander durchgef"uhrter Verkn"upfungen von
Potenzreihen u. U. auch dann konvergieren, wenn als Zwischenergebnis
auftretende Reihen divergieren\footnote{In dem Algorithmus zur
Determinantenberechnung, der in diesem Kapitel vorgestellt wird,
tritt diese Besonderheit auf. In \cite{BGH82} wird darauf in keiner Weise
eingegangen, was sich bei der Bearbeitung als st"orend herausgestellt
hat. }

Da f"ur uns an dieser Stelle weitere allgemeine Betrachtungen uninteressant 
sind, erfolgt die Konvergenzanalyse im Zusammenhang mit der Anwendung 
der Potenzreihenentwicklung auf unser Problem der Determinantenberechnung.

% $$$ hier behandeln, wie Potenzreihen verkn"upft werden ?
%     (vgl. \ref{SecAlgBGH}  (-> letztes Unterkapitel) )

% **************************************************************************

\MySection{Das Gau"s'sche Eliminationsverfahren ohne Divisionen}
\label{SecGaussOhneDiv}

Die M"oglichkeiten zur Vermeidung von Divisionen wurden von V. Strassen
\cite{Stra73} allgemein untersucht. In diesem Unterkapitel wird dargestellt,
wie sich Strassens Ergebnisse auf das Gau"s'sche Eliminationsverfahren
anwenden lassen.

Die Hauptidee zur Vermeidung von Divisionen ist es, alle Berechnungen nicht
in einem Ring $R$ mit Division durch Einheiten
durchzuf"uhren, sondern im zugeh"origen Potenzreihenring $R[[]]$, wobei
Matrizenelemente als Unbestimmte auftreten. Um die
Berechnungen in diesen Ring zu "ubertragen, wird das
Kroneckersymbol \index{Kroneckersymbol} definiert als
\[ \delta_{i,j} :=
       \left\{
           \begin{array}{rcl}
               1 & : & i = j \\
               0 & : & i \neq j
           \end{array}
       \right.
\]
Es sei die Determinante der $n \times n$-Matrix $A$ zu berechnen. Ihre
Elemente werden mit Hilfe der Definition
\Beq{EquDefBGHErsetzung}
    a_{i,j}' := \delta_{i,j} - a_{i,j}
\Eeq
ersetzt. Das bedeutet, jedes
Matrizenelement $a_{i,j}$ wird ersetzt durch
\[ \delta_{i,j} - a_{i,j}' \MyPunkt \]
Wendet man nun das Gau"s'sche Eliminationsverfahren an, bekommt jede
Division die Form \equref{Equ1ZuErsetzen} und kann somit ersetzt werden
durch \equref{Equ1StattDivision}, wie durch das Beispiel in
Unterkapitel \ref{SecBeispielOhneDiv} deutlich wird.

Berechnet man mit Hilfe des Eliminationsverfahrens die Determinante von
$A$, wie in \ref{SecGauss} beschrieben ist, und rechnet dabei in 
$R[[]]$ statt in $R[]$, erh"alt man als Endergebnis eine Potenzreihe $d'$
"uber den Unbestimmten $a_{i,j}'$,
die die Determinante von $A$ beschreibt.

In der praktischen Berechnung ersetzt man die $a_{i,j}'$ mit Hilfe von 
\equref{EquDefBGHErsetzung} durch konkrete Werte und wertet die 
Potenzreihe $d'$ aus, um die Determinante als Element von $R$ zu erhalten.

Ein bis hierhin ungel"ostes Problem ist die Sicherstellung der Konvergenz
von $d'$. Dazu ist die Frage zu beantworten:
\begin{quote}
    Wie gro"s ist der Konvergenzradius von $d'$?
\end{quote}

Hierf"ur m"ussen wir zun"achst eine Eigenschaft der Determinante n"aher
betrachten\footnote{Literatur zu diesem Lemma ist die in Kapitel 
\ref{ChapBase} aufgelistete Grundlagenliteratur.}:
\begin{lemma}
\label{SatzDetEindeutig}
    Es gibt genau eine Abbildung, die jeder Matrix ihre Determinante
    zuordnet.
\end{lemma}
\begin{beweis}
    Der Beweis wird anhand der Matrix $A$ gef"uhrt.

    Seien $f$ und $\hat(f)$ zwei voneinander verschiedene Abbildungen, mit
    den in der Definition \ref{DefDet} der Determinante beschriebenen
    Eigenschaften.

    Es werden zwei F"alle unterschieden:
    \begin{itemize}
    \item
          Bei \[ \rg(A) < n \] gilt nach Satz \ref{SatzRgDetInv}
          \[ f(A) = \hat{f}(A) = 0 \MyPunkt \]
    \item
          Sei \Beq{Equ1SatzDetEindeutig}  \rg(A) = n \MyPunkt \Eeq
          Entsteht die Matrix $B$ aus $A$ durch Zeilenumformungen 
          entsprechend D1 in Definition \ref{DefDet}, dann gibt es 
          ein $c \neq 0$, so da"s gilt: 
          \[ f(B) = c * f(A) \MyPunkt \]
          Das gleiche gilt auch f"ur $\hat{f}$:
          \[ \hat(B) = c * \hat(A) \MyPunkt \]
          Wegen \equref{Equ1SatzDetEindeutig} ist es m"oglich, durch
          Zeilenumformungen \[ B = E_n \] zu erreichen. Aus D4 in 
          Definition \ref{DefDet} folgt dann:
          \[ f(A) = \frac{1}{c} f(E_n) = \frac{1}{c} 
                  = \frac{1}{c}\hat{f}(E_n) = \hat{f}(A) \MyPunkt
          \]
    \end{itemize}
    In beiden F"allen gilt also $ f = \hat(f) $ im Widerspruch zur 
    Voraussetzung, da"s $f$ und $\hat(f)$ verschieden sind.
\end{beweis}

Mit der Unterst"utzung durch dieses Lemma gelangt man zu einer
wichtigen Aussage:
\begin{satz}
\label{SatzBGHKonvergenz}
    Bezeichne $d$ die Potenzreihe "uber den Unbestimmten $a_{i,j}$, die
    man aus $d'$ (s. o.) dadurch erh"alt, da"s man alle 
    Unbestimmten $a_{i,j}'$ mit Hilfe von \equref{EquDefBGHErsetzung} 
    ersetzt.

    F"ur $d$ gilt:
    \begin{quote}
         Alle homogenen Komponenten mit einem Grad ungleich $n$ sind
         gleich Null.
    \end{quote}
\end{satz}
\begin{beweis}
    Aus der Richtigkeit der im vorliegenden Kapitel beschriebenen Verfahren
    folgt, da"s $d$ eine Determinante von $A$ entsprechend der 
    Definition \ref{DefDet} beschreibt.

    Bezeichne $f$ die nach Satz \ref{SatzDetPermut} berechnete Determinante
    von $A$ als Summe, deren Summanden jeweils aus einem Produkt von $n$
    Matrizenelementen bestehen.

    Nach Lemma \ref{SatzDetEindeutig} gilt:
    \[
        d = f \MyPunkt
    \]
    Betrachtet man die Termstruktur von $d$ und beachtet, da"s f"ur die 
    Matrizenelemente keine zus"atzlichen Eigenschaften vorausgesetzt werden,
    folgt aus dieser Gleichung die Behauptung.
\end{beweis}
Der Satz wird in Unterkapitel \ref{SecBeispielOhneDiv} an einer
$3 \times 3$-Matrix demonstriert.

Sowohl $d$ als auch $d'$ beschreiben die Determinante von $A$. 
Der Konvergenzradius von beiden Reihen ist also {\em Unendlich}.

Aus Satz \ref{SatzBGHKonvergenz} folgt insbesondere, da"s sich alle 
homogenen Komponenten mit einem Grad gr"o"ser als $n$ gegenseitig aufheben.
Da alle Divisionen durch Additionen und Multiplikationen ersetzt worden 
sind, gehen diese Komponenten nicht in den Wert von Komponenten geringeren
Grades ein. Komponenten mit einem bestimmten Grad beeinflussen
im Verlauf der Rechnungen lediglich die Werte von Komponenten gleichen oder
h"oheren Grades. 

Also ist es unn"otig, die homogenen Komponenten mit einem Grad gr"o"ser
als $n$ "uberhaupt zu berechnen. Dies ist ein wichtiges Ergebnis f"ur die
Analyse der Effizienz des Algorithmus.

% **************************************************************************

\MySection{Beispiel zur Vermeidung von Divisionen}
\label{SecBeispielOhneDiv}

F"ur eine $3 \times 3$-Matrix wird in diesem Unterkapitel
gezeigt, wie die Determinante mit Hilfe des Gau"s'schen
Eliminationsverfahrens ohne Divisionen berechnet
wird\footnote{Das Beispiel wurde mit Hilfe eines Programms zur 
symbolischen Manipulation von Termen berechnet. Wegen der vielen Indizes
ist das Nachrechnen ohne Computer nicht ratsam.}. Wie im
vorangegangenen Unterkapitel \ref{SecGaussOhneDiv} begr"undet ist, werden
bei allen Potenzreihen nur die homogenenen Komponenten bis maximal zum
Grad $3$ betrachtet.

Es ist die Determinante von
\Beq{Equ1BGHBeispiel}
    \left[
        \begin{array}{ccc}
            a_{1,1} & a_{1,2} & a_{1,3} \MatStrut \\
            a_{2,1} & a_{2,2} & a_{2,3} \MatStrut \\
            a_{3,1} & a_{3,2} & a_{3,3} \MatStrut
        \end{array}
    \right]
\Eeq 
zu berechnen.
Die Ersetzung mit Hilfe von Gleichung \equref{EquDefBGHErsetzung} ergibt:
\[
    \left[
        \begin{array}{ccc}
            1 - a_{1,1}' & 0 - a_{1,2}' & 0 - a_{1,3}' \MatStrut \\
            0 - a_{2,1}' & 1 - a_{2,2}' & 0 - a_{2,3}' \MatStrut \\
            0 - a_{3,1}' & 0 - a_{3,2}' & 1 - a_{3,3}' \MatStrut
        \end{array}
    \right]
    \begin{array}{c}
        \MatStrut \\ \MatStrut \\ \MyPunkt \MatStrut
    \end{array}
\]
Nun werden Vielfache der ersten Zeile von
den folgenden Zeilen subtrahiert, und man erh"alt:
\[
    \left[
        \begin{array}{ccc}
            1 - a_{1,1}'
        &   0 - a_{1,2}'
        &   0 - a_{1,3}' \MatStrut
        \\     0
        &   (1 - a_{2,2}') - (0 - a_{1,2}')
            \frac{ (0 - a_{2,1}') }{ (1 - a_{1,1}') }
        &   (0 - a_{2,3}')  - (0 - a_{1,3}')
            \frac{ (0 - a_{2,1}') }{ (1 - a_{1,1}') } \MatStrut
        \\     0
        &   (0 - a_{3,2}') - (0 - a_{1,2}')
            \frac{ (0 - a_{3,1}') }{ (1 - a_{1,1}') }
        &   (1 - a_{3,3}') - (0 - a_{1,3}')
            \frac{ (0 - a_{3,1}') }{ (1 - a_{1,1}') } \MatStrut
        \end{array}
    \right]
    \begin{array}{c}
        \MatStrut \\ \MatStrut \\ \MyPunkt \MatStrut
    \end{array}
\]
Durch Ersetzung der Divisionen und Vereinfachung der Terme erh"alt man:
\[
    \left[
        \begin{array}{ccc}
            1 - a_{1,1}'
        &   0 - a_{1,2}'
        &   0 - a_{1,3}' \MatStrut
        \\     0
        &   \begin{array}{c}
                (1 - a_{2,2}') - a_{1,2}'a_{2,1}' *
            \\  (1 + a_{1,1}' + (a_{1,1}')^2 + (a_{1,1}')^3)
            \end{array}
        &   \begin{array}{c}
                (0 - a_{2,3}')  - a_{1,3}'a_{2,1}' *
            \\  (1 + a_{1,1}' + (a_{1,1}')^2 + (a_{1,1}')^3)
            \end{array} \LMatStrut
        \\     0
        &   \begin{array}{c}
                (0 - a_{3,2}') - a_{1,2}'a_{3,1}' *
            \\  (1 + a_{1,1}' + (a_{1,1}')^2 + (a_{1,1}')^3)
            \end{array}
        &   \begin{array}{c}
                (1 - a_{3,3}') - a_{1,3}'a_{3,1}' *
            \\  (1 + a_{1,1}' + (a_{1,1}')^2 + (a_{1,1}')^3)
            \end{array}
        \end{array}
    \right]
    \begin{array}{c}
        \MatStrut \\ \MatStrut \\ \MyPunkt \MatStrut
    \end{array}
\]
Da nur die homogenen Komponenten bis maximal zum Grad $3$ ber"ucksichtigt
werden, erh"alt man durch weitere Vereinfachung der Terme:
\[
    \left[
        \begin{array}{ccc}
            1 - a_{1,1}'
        &   0 - a_{1,2}'
        &   0 - a_{1,3}' \MatStrut
        \\     0
        &   \begin{array}{c}
                1 - (a_{2,2}' + a_{1,2}'a_{2,1}' +
            \\   a_{1,2}'a_{2,1}'a_{1,1}')
            \end{array}
        &   \begin{array}{c}
                0 - (a_{2,3}'  + a_{1,3}'a_{2,1}' +
            \\  a_{1,3}'a_{2,1}'a_{1,1}')
            \end{array} \LMatStrut
        \\     0
        &   \begin{array}{c}
                0 - (a_{3,2}' + a_{1,2}'a_{3,1}' +
            \\  a_{1,2}'a_{3,1}'a_{1,1}')
            \end{array}
        &   \begin{array}{c}
                1 - (a_{3,3}' + a_{1,3}'a_{3,1}' +
            \\  a_{1,3}'a_{3,1}'a_{1,1}')
            \end{array} \LMatStrut
        \end{array}
    \right]
    \begin{array}{c}
        \MatStrut \\ \LMatStrut \\ \MyPunkt \LMatStrut
    \end{array}
\]
Man erkennt, da"s alle Elemente der Hauptdiagonalen wieder die Form
\[ 1 - g_{i,j} \] und alle anderen Elemente wieder die Form
\[ 0 - g_{i,j} \] besitzen, wobei der konstante Term der $g_{i,j}$ jeweils
gleich $0$ ist. Bei der Fortsetzung des Eliminationsverfahrens k"onnen 
auftretende Divisionen also wiederum auf die gleiche Weise ersetzt werden.

Als n"achstes wird ein Vielfaches der zweiten Zeile von der dritten
subtrahiert. Dazu sei
\begin{eqnarray*}
    a_{2,2}'' & := &
        1 - (a_{2,2}' + a_{1,2}'a_{2,1}' + a_{1,2}'a_{2,1}'a_{1,1}') \\
    a_{2,3}'' & := &   
        0 - (a_{2,3}'  + a_{1,3}'a_{2,1}' + a_{1,3}'a_{2,1}'a_{1,1}') \\
    a_{3,2}'' & := &
        0 - (a_{3,2}' + a_{1,2}'a_{3,1}' + a_{1,2}'a_{3,1}'a_{1,1}') \\
    a_{3,3}'' & := &
        1 - (a_{3,3}' + a_{1,3}'a_{3,1}' + a_{1,3}'a_{3,1}'a_{1,1}') \\
    a_{3,3}''' & := &
        a_{3,3}'' - \frac{ a_{3,2}'' }{ a_{2,2}'' } a_{2,3}'' 
        \MyPunkt
\end{eqnarray*}
Man erh"alt die Matrix:
\[
    \left[
        \begin{array}{ccc}
            1 - a_{1,1}'
        &   0 - a_{1,2}'
        &   0 - a_{1,3}' \MatStrut
        \\     0
        &   a_{2,2}''
        &   a_{2,3}'' \MatStrut
        \\     0
        &      0
        &   a_{3,3}''' \MatStrut
        \end{array}
    \right]
    \begin{array}{c}
        \MatStrut \\ \MatStrut \\ \MyPunkt \MatStrut
    \end{array}
\]
Da nur die homogenen Komponenten bis zum Grad $3$ betrachtet werden sollen,
erh"alt man durch Ersetzung der Division in der beschriebenen Weise und
Vereinfachung der Terme\footnote{Da alle Prokukte aus mehr als $3$ 
Unbestimmten sofort weggelassen werden, d"urfen die Rechenschritte nicht
durch $=$ verbunden werden.} f"ur $a_{3,3}''$:
\begin{eqnarray*} % mit 'form' geprueft (Dateien: bgh2.for bgh2.log)
    & & a_{3,3}'' - \frac{ a_{3,2}'' }{ a_{2,2}'' } a_{2,3}'' \\
    & \rightarrow &
        a_{3,3}'' - a_{3,2}'' \\
    & & * (1 + a_{2,2}' + a_{1,2}'a_{2,1}' + a_{2,2}'^2 + a_{1,2}'a_{2,1}'a_{1,1}' \\
    & & \: \: + 2 a_{2,2}'a_{1,2}'a_{2,1}' + a_{2,2}'^3) \\
    & & * a_{2,3}'' \\
    & \rightarrow &
        1 - a_{1,1}'a_{1,3}'a_{3,1}' - a_{1,2}'a_{2,3}'a_{3,1}'
        - a_{1,3}'a_{2,1}'a_{3,2}' \\
    & & - a_{1,3}'a_{3,1}' - a_{2,2}'a_{2,3}'a_{3,2}'
        - a_{2,3}'a_{3,2}' - a_{3,3}' \MyPunkt
% Form:
%        1 - a[11]*[a13]*[a31] - [a12]*[a23]*[a31] - [a13]*[a21]*[a32]
%          - [a13]*[a31] - [a22]*[a23]*[a32] - [a23]*[a32] - [a33];
\end{eqnarray*}
Um die Determinante zu berechnen, werden die Elemente der Hauptdiagonalen
$a_{1,1}'$, $a_{2,2}''$ und $a_{3,3}'''$ miteinander multipliziert.
Wiederum werden die Komponenten mit zu gro"sem Grad weggelassen. Man
erh"alt:
\begin{eqnarray*}
   & & a_{1,1}' a_{2,2}'' a_{3,3}''' \\
   & \rightarrow &
   1-a_{1,1}'a_{2,2}'a_{3,3}'+ a_{1,1}'a_{2,2}' + a_{1,1}'a_{2,3}'a_{3,2}'
   + a_{1,1}'a_{3,3}' - a_{1,1}' + a_{1,2}'a_{2,1}'a_{3,3}' \\
   & &
   - a_{1,2}'a_{2,1}' - a_{1,2}'a_{2,3}'a_{3,1}' - a_{1,3}'a_{2,1}'a_{3,2}'
   + a_{1,3}'a_{2,2}'a_{3,1}' - a_{1,3}'a_{3,1}' \\
   & &
   + a_{2,2}'a_{3,3}' - a_{2,2}'  - a_{2,3}'a_{3,2}' - a_{3,3}' \MyPunkt
%Form:
%  det =
%     1 - [a11]*[a22]*[a33] + [a11]*[a22] + [a11]*[a23]*[a32] + [a11]*[a33] -
%       [a11] + [a12]*[a21]*[a33] - [a12]*[a21] - [a12]*[a23]*[a31] - [a13]*
%       [a21]*[a32] + [a13]*[a22]*[a31] - [a13]*[a31] + [a22]*[a33] - [a22] -
%       [a23]*[a32] - [a33];
\end{eqnarray*}
Um die gesuchte Determinante zu erhalten, setzt man die mit Hilfe von
Gleichung \equref{EquDefBGHErsetzung} aus den $a_{i,j}$ erhaltenen
Werte f"ur die $a_{i,j}'$ ein.

Zum Beweis, da"s wir richtig gerechnet haben, machen wir nun die
durch \equref{EquDefBGHErsetzung} definierte Substitution im obigen Term
wieder r"uckg"angig. Nach der Vereinfachung des Terms lautet das Ergebnis,
ohne da"s zus"atzlich irgendwelche Teilterme weggelassen worden sind:
\begin{eqnarray*}
  \lefteqn{ a_{1,1}' a_{2,2}'' a_{3,3}''' = } \\
 & & a_{1,1}a_{2,2}a_{3,3} + a_{1,2}a_{2,3}a_{3,1} + a_{1,3}a_{2,1}a_{3,2} \\
 & & - a_{1,1}a_{2,3}a_{3,2} - a_{1,2}a_{2,1}a_{3,3} - a_{1,3}a_{2,2}a_{3,1}
    \MyPunkt
\end{eqnarray*}
Die Richtigkeit dieses Ergebnisses wird beim Vergleich mit Satz
\ref{SatzDetPermut} deutlich.

% **************************************************************************

\MySection{Parallele Berechnung von Termen}
\label{SecVSBR}

Durch die Methode von Strassen zur Vermeidung von Divisionen entstehen
Terme, die es parallel auszuwerten gilt. In diesem Unterkapitel wird
ein Verfahren \cite{VSBR83} beschrieben, welches diese Auswertung
erm"oglicht. Die
Beschreibung des Verfahrens ist auf die Verwendung im Rahmen des
Kapitels angepa"st. Eine ausf"uhrliche Beschreibung ist auch in
\cite{Wald87} ab S. 22 zu finden.

Zun"achst wird die Berechnung von Termen formalisiert.
Dazu wird die Menge \[  \{v_i \MySetProperty 1 \leq i \leq c \} \]
mit $V$ bezeichnet. Die Menge der
Elemente $a_{i,j}$ der $n \times n$-Matrix $A$, die hier als
Unbestimmte auftreten, wird mit $X$ bezeichnet. Sei $R$ der Ring, in dem
alle Rechnungen durchgef"uhrt werden. Es wird definiert
\[ \bar{V} := V \cup X \cup R \MyPunkt \]

\MyBeginDef
\label{DefProgramm}
    Sei $R[]$ der bereits erw"ahnte Ring "uber den Elementen von $X$.
    Sei $c \in \Nat$ gegeben. Seien $+$ und $*$ die
    beiden Ringoperatoren f"ur Addition bzw. Multiplikation.
    Es gelte \[ \circ \in \{+,*\} \MyPunkt \] Weiterhin gelte
    \[ \forall 1\leq i\leq c: \: v_i', v_i'' \in \bar{V}
           \backslash \{ v_i, \, v_{i+1}, \, \ldots, \, v_c \} \MyPunkt
    \]
    Jede Folge der Form
    \[ v_i := v_i' \circ v_i'', \: i = 1, \, \ldots, \, c \]
    hei"st dann { \em Programm "uber R[] } . \index{Programm "uber R[]}
    Ein Element einer solchen Folge wird {\em Anweisung} genannt. Abh"angig
    davon, ob $\circ$ die Addition oder die Multiplikation bezeichnet, wird
    das $v_i$ auch als {\em Additions-} bzw.
    {\em Multiplikationsknoten} bezeichnet. Falls das genaue Aussehen von
    Anweisungen von untergeordnetem Interesse ist, werden diese zur
    Abk"urzung durch ihren Additions- bzw. Multiplikationsknoten
    repr"asentiert.
\MyEndDef

Falls in diesem Unterkapitel im Einzelfall nichts anderes festgelegt wird,
ist mit $v_i'$ bzw. $v_i''$ jeweils der erste bzw. zweite Operand der 
Anweisung $v_i$ gemeint. Dies gilt auch dann, wenn andere Buchstaben
benutzt werden oder keine Indizierung erfolgt.

Jeder Term "uber $R[]$ l"a"st sich durch ein Programm "uber $R[]$
berechnen. Um Aussagen "uber solche Programme machen zu k"onnen, sind
eine Reihe weiterer Vereinbarungen erforderlich, die im folgenden
aufgef"uhrt sind.

Der durch eine Anweisung $v_i$ berechnet Term wird mit $f(v_i)$
bezeichnet.

Sei $x \in \bar{V}$. Seien $x',x'' \in V$.
Dann wird der {\em Grad von $x$} mit $g(x)$
bezeichnet und folgenderma"sen definiert:
\[ g(x) := \left\{
           \begin{array}{rcl}
               0 & : & x \in R \\
               1 & : & x \in X \\
               g(x') + g(x'') & : & (x \in V) \und (x := x' * x'') \\
               \max(g(x'),\, g(x'')) & : &
                                    (x \in V) \und (x := x' + x'')
           \end{array}
           \right.
\]
Der Grad von $x$ stimmt nicht mit dem Grad des Polynoms "uberein,
da"s dem Term $f(x)$ entspricht. Dazu ein Beispiel: \nopagebreak[3]
\[
    \begin{array}{ccc}
        v_1:= y * (-1) & f(v_1) = -y & g(v_1)=1 \\  
        v_2:= y + v_1  & f(v_2) = 0  & g(v_2)=1
    \end{array}
\]

Es wird o. B. d. A. angenommen, da"s f"ur jede Anweisung 
\[ x := x' \circ x'' \] die Bedingung \[ g(x') \geq g(x'') \]
erf"ullt ist.

F"ur alle $a \in \Nat$ wird definiert:
\begin{eqnarray*}
   V_a & := & \{ u \in V \MySetProperty
                 g(u) > a, \, u:= u' * u'', \, g(u') \leq a \}  \\
   V_a'& := & \{ u \in V \MySetProperty
                 g(u) > a, \, u:= u' + u'', \, g(u'') \leq a \}
\end{eqnarray*}

\MyBeginDef
\label{DefTiefe}
    Sei $v\in V$. Sei $v_1, \, \ldots , \, v_k$ die l"angste Folge von 
    Elementen von $\bar{V}$, so da"s gilt
    \begin{eqnarray*}
        & & v_1 = v \\
        \forall 1 \leq i \leq k-1 & : & (v_{i+1} = v_i') \oder 
                                        (v_{i+1} = v_i'') \\
        & & v_k \in F \cup X
        \MyPunkt
    \end{eqnarray*}
    Dann bezeichnet $d(v)=k$ die {\em Tiefe von $v$}.
\MyEndDef

\MyBeginDef
\label{Deffvw}
    Sei $v,w \in \bar{V}$.
    Dann wird $f(v;w) \in R[]$ wie folgt definiert:

    Bezeichnen $v$ und $w$ denselben Knoten, so gilt:
    \[ f(v;w) := 1 \MyPunkt \]
    Falls dies nicht erf"ullt ist und $w \in R \cup X$, dann gilt:
    \[ f(v;w) := 0 \MyPunkt \]
    Falls dies ebenfalls nicht erf"ullt ist und
    \[ w:= w' + w'' \MyKomma \]
    dann gilt \[ f(v;w) := f(v;w') + f(v;w'') \MyPunkt \]
    Falls auch dies nicht erf"ullt ist, bleibt nur noch der Fall "ubrig
    da"s gilt
    \[ w:= w' * w'' \MyPunkt \] Daf"ur wird definiert
    \[ f(v;w) := f(v;w') * f(w'') \MyPunkt \]
\MyEndDef
Durch die Art und Weise, wie $f(v;w)$ definiert ist, ergibt sich eine
besondere Eigenschaft f"ur den Fall, da"s $g(w) < 2g(v)$ erf"ullt ist.
Falls n"amlich in dem Programm, zu dem $v$ und $w$ geh"oren,
der Knoten $v$ durch eine neue 
Unbestimmte $v'$ ersetzt wird, dann ist $f(v;w)$ der Koeffizient von 
$v'$ in $f(w)$. 

In Verbindung mit $f(v;w)$ besitzen die Funktionen $g()$ und $d()$ 
eine Eigenschaft, die weiter unten von Bedeutung ist:
\begin{lemma}
\label{SatzGrad}
    \[ g(v) > g(w) \Rightarrow f(v;w) = 0 \]
\end{lemma}
\begin{beweis}
    Der Beweis erfolgt durch Induktion nach $d(w)$.
    \begin{MyDescription}
    \MyItem{$ d(w) = 0 $ }
        Es gilt:
        \begin{eqnarray*}
            g(w) = 0 & \Rightarrow & w \in R \\
            g(v) > g(w) & \Rightarrow & v \in V \cup X 
        \end{eqnarray*}
        Also ist $f(v;w) = 0$ .
    \MyItem{$ d(w) > 0 $ }
        Das Lemma gelte f"ur alle $ u\in\bar{V}, \, d(u)<d(w)$.
        Aus \ref{DefTiefe} folgt direkt, da"s
        f"ur jede Anweisung \[ w:= w' \circ w'' \] gilt 
        \[ d(w) > d(w'), \: d(w) > d(w'') \MyPunkt \]
        Mit Hilfe von \ref{Deffvw} folgt daraus die G"ultigkeit
        des Lemmas.
    \end{MyDescription}
\end{beweis}

\begin{lemma}
\label{SatzTiefe}
    \[ d(v) > d(w) \Rightarrow f(v;w) = 0 \]
\end{lemma}
\begin{beweis}
    analog zu \ref{SatzGrad}
\end{beweis}

Es lassen sich nun zwei Aussagen formulieren.
Dazu gelte jeweils $v,w \in V$ und $0 < g(v) \leq a < g(w)$.

\begin{lemma}
\label{Satz1VSBR}
    \[
       f(v;w) =
           \sum_{u\in V_a} (f(v;u) * f(u;w)) +
           \sum_{u\in V_a'} (f(v;u'') * f(u;w))
    \]
\end{lemma}
\begin{beweis}
    Der Beweis erfolgt durch Induktion nach $d(w)$. Aufgrund der
    Struktur der zu beweisenden Aussage sind die Beweise von
    Induktionsanfang und Induktionsschlu"s nicht voneinander 
    getrennt.
    
    Wegen der Voraussetzung \[ 0 < a < d(w) \] folgt aus 
    \[ d(w) \leq 1 \Rightarrow w \in R \cup X \MyKomma \]
    da"s $d(w)= 1$ nicht auftreten kann. Sei im folgenden also $d(w)>1$.

    Vier F"alle sind zu unterscheiden:
    \begin{MyDescription}
    \MyItem{ $ w:= w' + w'', \: g(w'') \leq a $ }
        Das Lemma gelte f"ur $w'$.
        Aus der Voraussetzung folgt:
        \[ w \in V'_a \MyPunkt \]
        Aus \ref{SatzTiefe} folgt:
        \[ f(w;w') = 0 \MyPunkt \]
        Au"serdem gilt:
        \[ g(w'') \leq a \Rightarrow
           \forall u \in V'_a: \: f(u;w'') = 0 \MyPunkt
        \]
        Nach \ref{Deffvw} gilt: \[ f(w,w) = 1 \MyPunkt \]
        So ergibt sich:
        \begin{eqnarray*}
            f(v;w') & = &
                \sum_{u \in V_a} (f(v;u) * f(u,w')) \\
              & & + \sum_{u \in V'_a} (f(v;u'') * f(u;w')) \\
            & = &
                \sum_{u \in V_a} (f(v;u) * (f(u,w') + f(u;w''))) \\
              & & + \sum_{u \in V'_a} (f(v;u'') * (f(u;w') + f(u;w''))) \\
            & = &
                \sum_{u \in V_a} (f(v;u) * f(u,w)) \\
              & & + \sum_{u \in V'_a \backslash \{w\} } (f(v;u'') * f(u;w))
        \end{eqnarray*}
        Es folgt mit Hilfe von \ref{Deffvw}:
        \begin{eqnarray*}
            f(v;w) & = & f(v;w') + f(v;w'') \\
            & = & 
                f(v;w') + f(v;w'') * f(w;w) \\
            & = & 
                \sum_{u \in V_a} (f(v;u) * f(u,w)) \\
              & & + \sum_{u \in V'_a \backslash \{w\} }
                     (f(v;u'') * f(u;w)) + f(v;w'') * f(w;w) \\
            & = &
                \sum_{u \in V_a} (f(v;u) * f(u,w)) \\
              & & + \sum_{u \in V'_a} (f(v;u'') * f(u;w))
        \end{eqnarray*}
    \MyItem{ $ w:= w' + w'', \: g(w'') > a $ }
        Das Lemma gelte f"ur $w'$ und $w''$.
        \begin{eqnarray*}
            f(v;w) & = & f(v;w') + f(v;w'') \\
            & = &
                \sum_{u \in V_a} (f(v;u)*f(u;w')) +
                \sum_{u \in \bar{V_a}} (f(v;u'')*f(u;w')) \\
            & & + \sum_{u \in V_a} (f(v;u)*f(u;w'')) +
                  \sum_{u \in \bar{V_a}} f(v;u'')*f(u;w'')) \\
            & = &
                \sum_{u \in V_a} (f(v;u) * (f(u;w') + f(u;w''))) \\
            & & + \sum_{u \in \bar{V_a}} (f(v;u'')*(f(u;w') + f(u;w''))) \\
            & = &
                \sum_{u \in V_a} (f(v;u) * f(u;w)) +
                \sum_{u \in \bar{V_a}} (f(v;u'') * f(u;w))
        \end{eqnarray*}
    \MyItem{ $ w:= w' * w'', \: g(w') \leq a $ }
        Es gilt: 
        \begin{eqnarray*}
            w & \in & V_a \\
            f(v;w) & = & f(v;w) * f(w;w) \MyPunkt
        \end{eqnarray*}
        Andererseits gilt:
        \begin{eqnarray*}
            \forall u \in V_a \backslash \{w\} & : & f(u;w') = 0 \\
            \Rightarrow
            \forall u \in V_a \backslash \{w\} & : &
                f(u;w) = f(w'') * f(u;w') = f(w'') * 0 = 0
        \end{eqnarray*}
        Also folgt:
        \[ \sum_{u \in V_a} (f(v;u) * f(u;w)) = f(v;w) * f(w;w) = f(v;w)
           \MyPunkt
        \]
        Weiterhin folgt aus \ref{SatzGrad}:
        \begin{eqnarray*}
            \lefteqn{ \forall u \in V'_a : \: f(u;w') = 0 } \\
            & \Rightarrow &
               \sum_{u \in V'_a} (f(u'') * f(u;w)) \\
            & & = \sum_{u \in V'_a} (f(u'') * f(w'') * f(u;w')) = 0
        \end{eqnarray*}
        Also ist das Lemma f"ur diesen Fall richtig.    
    \MyItem{ $ w:= w' * w'', \: g(w') > a $ }
         Das Lemma gelte f"ur $w'$.
         \begin{eqnarray*}
            f(v;w) & = & f(w'') * f(v;w') \\
            & = & f(w'') *
                \left(
                    \sum_{u\in V_a} (f(v;u) * f(u;w')) +
                    \sum_{u\in V_a'} (f(v;u'') * f(u;w'))
                \right) \\
            & = &
                \sum_{u\in V_a} (f(v;u) * f(w'') * f(u;w')) +
                \sum_{u\in V_a'} (f(v;u'') * f(w'') * f(u;w')) \\
            & = &
                \sum_{u\in V_a} (f(v;u) * f(u;w)) +
                \sum_{u\in V_a'} (f(v;u'') * f(u;w))
        \end{eqnarray*}
   \end{MyDescription}
\end{beweis}
  
\begin{lemma}
\label{Satz2VSBR}
    \[
       f(w) = 
           \sum_{u\in V_a} (f(u) * f(u;w)) +
           \sum_{u\in V_a'} (f(u'') * f(u;w))
    \]
\end{lemma}
\begin{beweis}
    Bis auf den Unterschied, da"s die auftretenden Terme entsprechend
    unterschiedlich sind, ist der Beweis identisch zum Beweis von
    \ref{Satz1VSBR}.
\end{beweis}

Mit Hilfe von \ref{Satz1VSBR} und \ref{Satz2VSBR} l"a"st sich ein
Verfahren zur parallelen Berechnung von Termen angeben, das im
folgenden beschrieben wird.

Gegeben sei ein Programm der L"ange $c$,
d. h. \[ V = \{v_1, \, v_2, \, \ldots, v_c\} \MyPunkt \]
Es ist $f(v_c)$ zu berechnen. Die Berechnung erfolgt stufenweise.
Seien $v,w \in V$.
In Stufe $0$ werden alle $f(w)$ mit \[ g(w)=1 \]  und alle
$f(v;w)$ mit \[ g(w) - g(v) = 1 \] berechnet.

In Stufe $i$ werden alle $f(w)$ mit
\[ 2^{i-1} < g(w) \leq 2^i \] und alle $f(v;w)$ mit
\[ 2^{i-1} < g(w) - g(v) \leq 2^i \] berechnet. Dabei werden die Ergebnisse
der vorangegangenen Stufen benutzt.

Auf diese Weise ist $f(v_c)$ nach
\[ \lc \log(g(v_c)) \rc \] Stufen berechnet.

In Stufe $i$ werden zun"achst die $f(w)$ mit Hilfe von \ref{Satz2VSBR}
berechnet. Dazu wird $a=2^{i-1}$ gew"ahlt:
\begin{eqnarray}
    f(w) \nonumber
    & = & \nonumber
        \sum_{u\in V_a} (f(u) * f(u;w)) +
        \sum_{u\in V_a'} (f(u'') * f(u;w)) \\
    & = & \label{EquStepIfw}
        \sum_{u\in V_a} (f(u')*f(u'')* f(u;w)) +
        \sum_{u\in V_a'} (f(u'') * f(u;w))
\end{eqnarray}
Anhand der Definitionen erkennt man, da"s f"ur alle auftretenden 
$f(\ldots)$ gilt: \[ g(f(\ldots)) \leq 2^{i-1} \MyPunkt \]
Also wurden alle zu benutzenden Terme bereits in einer der vorangegangenen
Stufen berechnet. 

Man erkennt anhand der bisher angestellten Betrachtungen "uber Programme
zur Berechnung von Termen, da"s der Aufwand f"ur alle Programme der
L"ange $c$ gleich ist. Da eine Aufgabe, die in $a$ Schritten von 
$b$ Prozessoren erledigt wird, auch in $2a$ Schritten von $b/2$ 
Prozessoren erledigt werden kann, erfolgt die Analyse des Aufwandes 
f"ur eine bestimmte Stufe $i$ zun"achst mit Hilfe der durchschnittlich f"ur
eine Stufe zu erwartenden erforderlichen Operationen\footnote{Diese 
Betrachtungsweise kennt man in der Literatur unter dem Begriff 
{\em Rescheduling}.}.

F"ur eine bestimmte Stufe l"a"st sich die Gr"o"se der Mengen $V_a$ und
$V_a'$ nicht genau vorherbestimmen. Falls die Berechnung in $z$ Stufen 
durchgef"uhrt wird, dann gilt jedoch
\begin{eqnarray*}
    & & 0 \leq i,j \leq z \\
    & & i \neq j \\
    & & a_k := 2^{k-1} \\
    & & V_{a_i} \cap V_{a_k} = V'_{a_i} \cap V'_{a_k} = \emptyset \\
    & & \sum_{0 \leq i \leq z} |V_{a_i}| \leq c \\
    & & \sum_{0 \leq i \leq z} |V'_{a_i}| \leq c 
\end{eqnarray*}
Die Mengen $V_a$ und $V'_a$ besitzen also durchschnittlich h"ochstens
\[ \frac{c}{z} = \frac{c}{ \lc \log(g(v_c)) \rc } \]
Elemente. Dieser Wert wird mit $m$ bezeichnet.

Aus den vorangegangenen "Uberlegungen folgt, da"s \equref{EquStepIfw} in
\[ \lc \log(m) \rc + 3 =
   \lc \log \lb \frac{c}{ \lc \log(g(v_c)) \rc } \rb \rc + 3
\]
Schritten von
\[ 2m = 2 \frac{c}{ \lc \log(g(v_c)) \rc } \] 
Prozessoren berechnet werden kann.

Nachdem in Stufe $i$ die $f(w)$ berechnet worden sind, werden die $f(v;w)$
mit Hilfe von \ref{Satz1VSBR} ausgerechnet. Dazu wird $a= g(v) + 2^{i-1}$ 
gew"ahlt:
\begin{eqnarray*}
    f(v;w)
    & = &
        \sum_{u\in V_a} (f(v;u) * f(u;w)) +
        \sum_{u\in V_a'} (f(v;u'') * f(u;w)) \\
    & = &
        \sum_{u\in V_a} (f(u'') * f(v;u') * f(u;w)) +
        \sum_{u\in V_a'} (f(v;u'') * f(u;w)) \\
\end{eqnarray*}
Anhand der Definitionen erkennt man, da"s alle $f(v;u')$, $f(u;w)$ und
$f(v;u'')$ bereits berechnet wurden. F"ur $f(u'')$ gibt es kritische
F"alle, die separat untersucht werden m"ussen:
\begin{MyDescription}
\MyItem{ $g(u') \geq g(u'') > 2^i, \: f(v;u') = 0$ }
    Der Fall ist kein Problem, da der Wert des jeweiligen gesamten Terms
    gleich Null ist.
\MyItem{ $g(u') \geq g(u'') > 2^i, \: f(v;u') \neq 0$ }
    Es mu"s gelten:
    \[ g(u') \geq g(v) \MyPunkt \]
    Daraus ergibt sich:
    \begin{eqnarray*}
        g(u) & = & g(u') + g(u'') \\
             & > & g(v) + 2^i \\
             & \geq & g(w) \\
             & \Rightarrow & f(u;w) = 0
    \end{eqnarray*}
    Der Wert des Terms ist also wiederum gleich Null.
\end{MyDescription}

F"ur die Analyse des Aufwandes gelten die gleichen Bemerkungen wie f"ur
die Berechnung der $f(w)$.

Insgesamt kann $f(v_c)$ in
\Beq{Equ1VSBRAnalyse}
        2\lc \log(g(v_c)) \rc (\lc \log(m) \rc + 3) =
        2\lc \log(g(v_c)) \rc
        \lb \lc \log \lb \frac{c}{ \lc \log(g(v_c)) \rc } 
                     \rb 
             \rc + 3
        \rb
\Eeq
Schritten von
\Beq{Equ2VSBRAnalyse}
   2m = 2 \frac{c}{ \lc \log(g(v_c)) \rc } 
\Eeq
Prozessoren berechnet werden.

F"ur das bis hierhin beschriebene und analysierte Verfahren gibt es einen
Sonderfall, der mit geringerem Aufwand gel"ost werden kann.
\MyBeginDef
\label{DefHomogen}
    Sei $v_1,\, \ldots ,\, v_c$ ein Programm im Sinne von \ref{DefProgramm}.
    Falls f"ur alle Additionsknoten $v_i:= v_i'+ v_i''$ dieses Programms 
    gilt:
    \[ g(v_i') = g(v_i'') \MyKomma \]
    so wird das Programm als {\em homogen} bezeichnet.
\MyEndDef
F"ur homogene Programme sind alle Mengen $V_a'$ leer. Mit dieser 
Feststellung ergeben sich aus \ref{Satz1VSBR} und \ref{Satz2VSBR} zwei
Folgerungen f"ur homogene Programme:
\begin{korollar}
\label{Satz3VSBR}
    \[
       f(v;w) =
           \sum_{u\in V_a} (f(v;u) * f(u;w))
    \]
\end{korollar}

\begin{korollar}
\label{Satz4VSBR}
    \[
       f(w) = 
           \sum_{u\in V_a} (f(u) * f(u;w))
    \]
\end{korollar}

Werden im angegebenen Verfahren zur parallelen Berechnung von Termen 
die Folgerungen
\ref{Satz3VSBR} und \ref{Satz4VSBR} statt der Lemmata \ref{Satz1VSBR} und
\ref{Satz2VSBR} benutzt, so f"uhrt das zu leicht verringertem
Berechnungsaufwand. Dann kann $f(v_c)$ analog zur obigen Analyse 
f"ur $f(v_c)$ bei nicht homogenen Programmen in
\Beq{Equ3VSBRAnalyse}
        2\lc \log(g(v_c)) \rc
        \lb \lc \log \lb \frac{c}{ \lc \log(g(v_c)) \rc } 
                     \rb 
             \rc + 2
        \rb
\Eeq Schritten von 
\Beq{Equ4VSBRAnalyse}
    \frac{c}{ \lc \log(g(v_c)) \rc } 
\Eeq Prozessoren berechnet werden.

% **************************************************************************

\MySection{Das Gau"s'sche Eliminationsverfahren parallelisiert}
\label{SecAlgBGH}

Das Thema dieses Unterkapitels ist es, wie das in \ref{SecVSBR}
beschriebene Verfahren benutzt werden kann, um mit Hilfe des in
\ref{SecGaussOhneDiv} angegebenen
Gau"s'schen Eliminationsverfahrens ohne Divisionen parallel die
Determinante einer $n \times n$-Matrix zu berechnen. Auf den so
entstehenden Algorithmus wird mit BGH-Alg. Bezug genommen
(vgl. Unterkapitel \ref{SecBez}).

Da keine Divisionen durchgef"uhrt werden, ist BGH-Alg. ebenso wie
B-Alg. auch in Ringen anwendbar. In dieser Hinsicht besitzen die
beiden Algorithmen gegen"uber C-Alg. und P-Alg. einen Vorteil.

Im folgenden wird beschrieben, wie ein Programm im Sinne von
\ref{SecVSBR} anhand der Ergebnisse von \ref{SecGaussOhneDiv}
aufgebaut wird. Um die Auswirkungen des Grades, bis zu dem Potenzreihen
entwickelt werden, auf die Effizienz der Rechnung besser demonstrieren 
zu k"onnen, werden die folgenden Betrachtungen zun"achst unabh"angig von
einem konkreten Grad durchgef"uhrt. F"ur alle Potenzreihen werden
die homogenen Komponenten bis maximal zum Grad $s$ betrachtet.

Es gibt drei wesentliche Elementaroperationen f"ur Potenzreihen, die 
zun"achst auf ihren Aufwand hin untersucht werden. Da sich nach
\ref{SecVSBR} die Anzahl der Prozessoren und der Schritte aus der
Programml"ange ergibt, wird im folgenden nur die Anzahl der Anweisungen
im Sinne von \ref{DefProgramm} betrachtet:
\begin{MyDescription}
\MyItem{Addition}
    Dieser Fall gilt f"ur {\em Subtraktion} analog. Es werden die
    homogenen Komponenten gleichen Grades addiert. Da die homogenen 
    Komponenten bis zum Grad $s$ betrachtet werden, sind hierf"ur 
    $s+1$ Anweisungen erforderlich.
\MyItem{Multiplikation}
    Seien $a$ und $b$ die zu multiplizierenden Potenzreihen. F"ur eine
    Potenzreihe $x$ bezeichne $x_i$ deren homogene Komponente vom
    Grad $i$. Das Ergebnis der zu Multiplikation von $a$ und $b$ sei $c$.
    Man erh"alt $c$ mit:
    \[ c_i := \sum_{j=0}^{i} a_j * b_{i-j} \MyPunkt \]
    Da f"ur $c$ auch nur die homogenen Komponenten bis zum
    Grad $s$ berechnet werden m"ussen, folgt mit Hilfe der Gleichung
    \[ 2^i = \sum_{j=0}^{i-1} 2^j + 1 \] f"ur die Anzahl der Anweisungen
    bei Benutzung der Bin"arbaummethode nach \ref{SatzAlgBinaerbaum}:
    \begin{eqnarray*}
        & & \sum_{i=0}^s 
            \lb 
                i + \sum_{j=0}^{\lc \log(i) \rc-1} 2^j 
            \rb \\
        & = &
            \sum_{i=0}^s \lb i + 2^{\lc \log(i) \rc} - 1 \rb \\
        & \leq &
            \sum_{i=0}^s \lb i + 2^{\log(i) + 1} - 1 \rb \\
        & = &
            \sum_{i=0}^s \lb 3i - 1 \rb \\
        & = & 3 \sum_{i=0}^s i - (s+1) \\
        & = & \frac{3}{2} s (s + 1) - (s+1) \\
        & = & \frac{3s^2 + s - 2}{2} \MyPunkt
    \end{eqnarray*}
\MyItem{Division}
    Die Divisionen werden entsprechend der Ausf"uhrungen in
    \ref{SecPotRing} und \ref{SecGaussOhneDiv} durch Additionen und
     Multiplikationen ersetzt (vgl. S. \pageref{Equ1ZuErsetzen}).
    Da nur die homogenen Komponenten bis zum Grad $s$ betrachtet werden,
    erfolgt die Potenzreihenentwicklung wie in \equref{Equ1StattDivision}
    nur bis zum $s$-ten Glied.

    Somit sind
    $s$ Multiplikationen und $s-1$ Additionen von Potenzreihen sowie
    die Addition des konstanten Terms durchzuf"uhren. In Verbindung mit
    den vorangegangenen Analysen von Addition und Multiplikation ergibt
    sich f"ur die Anzahl der Anweisungen:
    \begin{eqnarray*}
        &   & s * \lb \frac{3s^2 + s - 2}{2} \rb + (s-1) * (s+1) + 1 \\
        & = & \frac{3s^3 + 3s^2 - 2s}{2} \MyPunkt
    \end{eqnarray*}
\end{MyDescription}

Als n"achstes wird untersucht, wieviele der einzelnen Elementaroperationen
zur Berechnung der Determinante benutzt werden. Dazu werden zwei 
Gleichungen benutzt:

\begin{bemerkung}
\label{SatzSumK}
    Sei $n \in \Nat_0$. Dann gilt:
    \[ \sum_{k=1}^n k = \frac{ n(n+1) }{ 2 } \]
\end{bemerkung}

\begin{bemerkung}
\label{SatzSumK2}
    Sei $n \in \Nat_0$. Dann gilt:
    \[ \sum_{k=1}^n k^2 = \frac{ n(n+1)(2n+1) }{ 6 } \]
\end{bemerkung}

Das in \ref{SecGauss} beschriebene Verfahren verwendet die Gleichungen
\equref{Equ1GaussDef} und \equref{Equ2GaussDef}. Werden mit Hilfe dieser 
Gleichungen zun"achst alle Matrizenelemente transformiert, betr"agt 
die Anzahl der Berechnungen neuer Elemente:
\begin{eqnarray*}
   &   & \sum_{i=1}^{n-1} \sum_{j=i+1}^n (n-(j-1)) \\
   & = & \sum_{i=2}^n \sum_{j=i}^n (n-(j-1)) \\
   & = & \sum_{i=2}^n \sum_{j=1}^{n-(i-1)} j \\
   & \MyStack{nach \ref{SatzSumK}}{=} &
         \sum_{i=2}^n \frac{ (n-(i-1))*((n-(i-1))+1) }{2} \\
   & = & \frac{1}{2} \sum_{i=2}^n ((n-i+1)*(n-i+2)) \\
   & = & \frac{1}{2} \sum_{i=2}^n (n^2+3n-2ni+i^2-3i+2) \\
   & = & \frac{1}{2}
         \lb (n-1)(n^2+3n+2)
             + \sum_{i=2}^n i^2
             - \sum_{i=2}^n 2ni
             - \sum_{i=2}^n 3i
         \rb \\
   & \MyStack{nach \ref{SatzSumK},\ref{SatzSumK2}}{=} &
         \frac{1}{2}
         \lb (n^3+2n^2-n-2)
             + \frac{1}{6} n(n+1)(2n+1) - 1 \right. \\
   & &   \left.
             - 2n \lb\frac{ n(n+1) }{ 2 } - 1 \rb
             - 3 \lb \frac{ n(n+1) }{ 2 } - 1 \rb
         \rb \\
   & = &
          \frac{1}{2}
          \lb (n^3+2n^2-n-2)
              + \frac{ 2n^3+3n^2+n }{ 6 } - 1 \right. \\
   & &   \left.
              - ( n^3+n^2 - 2n)
              - \lb \frac{ 3(n^2+n) }{ 2 } - 3 \rb
          \rb \\
   & = & \frac{1}{6}( n^3- n)
\end{eqnarray*}
F"ur jede einzelne Transformation eines Matrixelements
werden nach \equref{Equ2GaussDef}
eine Subtraktion, eine Multiplikation und eine Division durchgef"uhrt.
Da alle Rechnungen in $R[[]]$ erfolgen, werden dabei Potenzreihen 
miteinander verkn"upft, wof"ur der Aufwand
gemessen in durchzuf"uhrenden Anweisungen bereits analysiert worden 
ist (s. o.).
F"ur die identische Abbildung nach \equref{Equ1GaussDef} wird kein Aufwand
in Rechnung gestellt. So kommt man auf
\begin{eqnarray}
   & & \nonumber
       \frac{1}{6}( n^3 - n) *
       \lb
           s + 1
           + \frac{3s^2 + s - 2}{2}
           + \frac{3s^3 + 3s^2 - 2s}{2}
       \rb \\
  & = & \label{AnwNeueElem}
% Form:
%    1/4*n^3*s^3 + 1/2*n^3*s^2 + 1/12*n^3*s - 1/4*n*s^3 - 1/2*n*s^2
%    - 1/12*n*s;
         \frac{1}{4}
       \lb n^3 s^3 + 2n^3 s^2 + \frac{n^3 s}{3} - n s^3 - 2 n s^2
       - \frac{n s}{3} \rb
\end{eqnarray}
Anweisungen, um eine gegebene Matrix mit Hilfe des Gau"s'schen Verfahrens
in eine obere Dreiecksmatrix zu "uberf"uhren. Zu Berechnung der
Determinante sind im Anschlu"s daran noch die Elemente der Hauptdiagonalen
miteinander zu multiplizieren. Dies kann mit $n-1$ Multiplikationen
geleistet werden, denen
\begin{eqnarray}
   & & \nonumber
     (n-1) * 
     \frac{3s^2 + s - 2}{2} \\
   & = & \label{AnwDiagMult}
     \frac{1}{2} ( 3 n s^2 + n s - 2n - 3 s^2 - s + 2)
\end{eqnarray}
Anweisungen entsprechen. So hat man bereits das Ergebnis als Element
von $R[[]]$. Um die Determinante als Element von $R$ zu erhalten, m"ussen
nun noch die homogenen Komponenenten bis zum Grad $s$ addiert werden.
Dies kann mit Hilfe von $s$ Anweisungen erfolgen. Abgesehen von diesen 
Additionen ist das Programm homogen im Sinne von \ref{DefHomogen}.
Deshalb ist es von Vorteil, die in \ref{SecVSBR} beschriebene Methode
auf das Programm ohne die letzten Additionen anzuwenden und diese
Additionen mit Hilfe der Bin"arbaummethode nach \ref{SatzAlgBinaerbaum}
durchzuf"uhren. Die Addition der homogenen Komponenten kann so in 
\[ \lc \log(s+1) \rc \] Schritten von \[ \lf \frac{s+1}{2} \rf \]
Prozessoren geleistet werden.

Man erh"alt das Gesamtergebnis
f"ur die Programml"ange ohne die letzten Additionen als
Summe von \equref{AnwNeueElem} und \equref{AnwDiagMult}:
\[ % \label{Gesamt}
%  1 + 1/4*n^3*s^3 + 1/2*n^3*s^2 + 1/12*n^3*s - 1/4*n*s^3 + n*s^2 + 5/12*n*
%  s - n - 3/2*s^2 - 1/2*s;
   \frac{1}{4} 
   \lb
       n^3 s^3 + 2 n^3 s^2 + \frac{1}{3}n^3 s - n s^3 + n s^2 + 
       \frac{5}{3}n s - 4 n - 6 s^2 - 2 s + 4
   \rb
\]
Anweisungen. Dieser Wert wird entsprechend der Terminologie in
\ref{SecVSBR} mit $c$ bezeichnet. Da bei allen Rechnungen nur die
homogenen Komponenten bis zum Grad $s$ beachtet werden, gilt
\[ g(v_c) = s \MyPunkt \]
Aus $c$ und $g(v_c)$ erh"alt man mit Hilfe der Analyseergebnisse
\equref{Equ3VSBRAnalyse} und \equref{Equ4VSBRAnalyse} aus
\ref{SecVSBR} f"ur die in diesem Kapitel beschriebene Methode zur
parallelen Determinantenberechnung einen Aufwand 
von\footnote{Genau genommen mu"s der Wert noch um $1$ erh"oht werden 
f"ur die Berechnung der $a_{i,j}'$ aus den urspr"unglichen 
Matrizenelementen $a_{i,j}$ entsprechend Gleichung 
\equref{EquDefBGHErsetzung}.}
\begin{eqnarray*}
    & & 2\lc \log(s) \rc * (\lc \log(c) \rc + 2) + \lc \log(s+1) \rc \\
    & = &
        2\lc \log(s) \rc \\
    & &
        * \lb
            \lc
            \log\lb
   \frac{1}{4} 
   \lb
       n^3 s^3 + 2 n^3 s^2 + \frac{1}{3}n^3 s - n s^3 + n s^2 + 
       \frac{5}{3}n s - 4 n - 6 s^2 - 2 s + 4
   \rb
            \rb
            \rc + 2
        \rb \\
    & & + \lc \log(s+1) \rc
\end{eqnarray*}
Schritten und
\begin{eqnarray*}
  \lefteqn{ \max \lb c \: , \lf \frac{s}{2} \rf \rb } \\
  & = & c \\
  & = &
   \frac{1}{4}
   \lb
       n^3 s^3 + 2 n^3 s^2 + \frac{1}{3}n^3 s - n s^3 + n s^2 +
       \frac{5}{3}n s - 4 n - 6 s^2 - 2 s + 4
   \rb
\end{eqnarray*}
Prozessoren. Man erkennt an diesen Werte die Bedeutung des Parameters
$s$, dem maximal ber"ucksichtigten Grad der homogenen Komponenten der
Potenzreihen. Betrachtet man $s$ als Konstante, so kann der Algorithmus in
\[ O(\log(n)) \] Schritten von \[ O(n^3) \] Prozessoren bearbeitet werden.

Die Analyse in Unterkapitel \ref{SecGaussOhneDiv} ergibt, da"s $s=n$ zu
setzen ist, so da"s der Algorithmus in
\[ O(\log^2(n)) \] Schritten von \[ O(n^6) \] Prozessoren bearbeitet
werden kann.

Die Aufwandanalyse ergibt, da"s BGH-Alg. insbesondere bei der 
Gr"o"senordnung der Anzahl der Prozessoren deutlich hinter C-Alg., B-Alg.
und P-Alg. zur"uckliegt. Die Konstanten bei der Anzahl der Schritte sind
ebenfalls vergleichsweise schlecht.

In BGH-Alg. werden, wie bei den anderen drei Algorithmen, keine 
Fallunterscheidungen verwendet, was aus den bereits in Unterkapitel
\ref{SecAlgFrame} erw"ahnten Gr"unden beim Entwurf von Schaltkreisen
vorteilhaft ist. 

Betrachtet man die Methodik von BGH-Alg., so ist er P-Alg. am "ahnlichsten.
Beide fassen mehrere auch unabh"angig voneinander bedeutsame Verfahren
zu einem Algorithmus zur Determiantenberechnung zusammen. C-Alg. und B-Alg.
hingegen st"utzen sich jeweils auf bestimmte schon seit 40 bis
50 Jahren bekannte S"atze, die nach einigen Umformungen f"ur einen
parallelen Algorithmus verwendet werden.

