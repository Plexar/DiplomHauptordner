%
%
% Datei: main.tex
%
% Haupt-Tex-Datei der Diplomarbeit
%
%%
% noch zu bearbeitende Stellen sind mit $$$$ markiert;
% Anmerkungen sind mit $$$ markiert;
% die aktuell in Berarbeitung befindliche Stelle ist mit $$$$$ markiert;
%%
% - Invertierung von Dreiecksmatrizen aus Csan74
% 
% - Warum laufen die Algorithmen von Csanky nur in K"orper
%       der Charakteristik 0 ?
%   Antwort: Weil Divisionen benutzt werden !!!!!!!!!!!!!!
% - Csan76 : K"orper mit Charakteristik 0
%   BGH82  : beliebige K"orper
%   Berk84 : beliebige K"orper
%   Pan85  : Iterationsverfahren f"ur det(A)
% - Schaltkreise???
%%
% ********************************
% Lesen von zusaetzlichen Dateien:
% ********************************
%%\input amssym.def
%%\input amssym
%%
%==============================================================
\documentstyle[german,ifthen,din_a4,makeidx,bezier,epic]{book}
%==============================================================
%%
% *************************
% Stiloptionen im Vorspann:
% *************************
\pagestyle{myheadings} % vgl. \newcommand{\MySection}{ ... } (s. u.)
\makeindex
\frenchspacing % europ"aische Behandlung der Satzenden
% Nummerierung der Textuntergliederung bis einschlie"slich 'section':
\setcounter{secnumdepth}{1}
\setcounter{tocdepth}{1}
%%
% ******************************
% Ausnahmen von Trennungsregeln:
% ******************************
\hyphenation
    { CRCW Pa-ral-lel-rech-ner Mo-dell De-ter-mi-nan-ten-be-rech-nung
      Ar-beits-spei-cher Fourier-trans-for-ma-tion PRAM
    }
%%
% ***********************
% Auswahl von Textteilen:
% ***********************
\typeout{}
\typein[\eingabe]{Textteile auswaehlen (j/n)?}
\ifthenelse{\equal{\eingabe}{j}
}{
    \typeout{}
    \typeout{Textteile:}
    \typeout{ inhalt, vorbem, csanky, bgh, berk}
    \typeout{ pan, implemen, endbem, index, anhang}
    \typeout{}
    % Auswahl der Textteile eingeben:
    \typein[\auswahl]{Welche Textteile?}

    \includeonly{\auswahl}
% ****Datei-Ein/Ausgabe funktioniert nicht****
%    \newwrite\AuswahlAusgabeDatei
%    \immediate\openout\AuswahlAusgabeDatei=\jobname.aus
%    \ifthenelse{\equal{\auswahl}{a}
%    }{ 
%        \typeout{ ...in then}
%        \write\AuswahlAusgabeDatei{
%            \typeout{Alle Textteile sind ausgewaehlt.}
%        }
%    }{
%        \typeout{ ...in else}
%        \write\AuswahlAusgabeDatei{
%            \typeout{Ausgewaehlte Textteile:}
%            \typeout{\auswahl}
%            \includeonly{\auswahl}
%        }
%    }
%    \closeout\AuswahlAusgabeDatei
}{}

%\newread\AuswahlEingabeDatei
%\openin\AuswahlEingabeDatei=\jobname.aus
%\read\AuswahlEingabeDatei to \ZeileI
%\ZeileI
%\closein\AuswahlEingabeDatei

%%
%===============
\begin{document}
%===============
%%
% *************************
% Stiloptionen im Textteil:
% *************************
\bibliographystyle{mygalpha}
\parindent0pt  % Absatzanf"ange nicht einr"ucken
\parskip2ex plus0.4ex minus0.4ex % Abst"ande zwischen Abs"atzen 2ex +-0.4ex
%%
% *******************************
% eigene Dokumentuntergliederung:
% *******************************
% bei "Anderung der Gliederung sind evtl. die Aufrufe von 
% \addcontentsline in 'tail.tex' anzupassen
\newcommand{\MyMark}[1]{ \thesection \hspace{0.5em} \sc #1 }
\newcommand{\MyChapter}[1]{\chapter{#1}}
\newcommand{\MySection}[1]{ 
                            \section{#1}
                            \markboth{ \MyMark{#1} }{ \MyMark{#1} }
                          }
\newcommand{\MySectionA}[2]{
                            \section[#1]{#2}
                            \markboth{ \MyMark{#1} }{ \MyMark{#1} }
                           }
\newcommand{\MySubSection}[1]{\subsection{#1}}
\newcommand{\MySubSectionA}[2]{\subsection[#1]{#2}}
\newcommand{\MySubSubSection}[1]{\subsubsection{#1}}
\newcommand{\MyParagraph}[1]{\paragraph{#1}}
%%
% ************************
% diverse neue Umgebungen:
% ************************
% Auswahl von 'subsection' in folgender Zeile ggf. anzupassen:
\newtheorem{satz}{Satz}[section]
\newtheorem{lemma}[satz]{Lemma}
\newtheorem{korollar}[satz]{Folgerung}
\newtheorem{definition}[satz]{Definition}
\newcommand{\MyBeginDef}{\begin{definition} \rm}
\newcommand{\MyEndDef}{\end{definition} \vspace{2ex}}
\newtheorem{algorithmus}[satz]{Algorithmus}
\newtheorem{bemerkung}[satz]{Bemerkung}
\newenvironment{beweis}{\medbreak {\bf Beweis} \quad
                       }{ \hfill $ \Box $ \bigbreak }
% f"ur den Anhang (Listings):
\newenvironment{MyListing}{ \small % '\normalsize' ist zu gross
                          }{ }
\newenvironment{DefModul}[1]{ \MySection{Definitionsmodul '#1`}
                              \begin{MyListing}
                            }{ \end{MyListing} }
\newenvironment{ImpModul}[1]{ \MySection{Implementierungsmodul '#1`}
                              \begin{MyListing}
                            }{ \end{MyListing} }
\newenvironment{ProgModul}[1]{ \MySection{Programmodul '#1`}
                               \begin{MyListing}
                             }{ \end{MyListing} }
% ***********************
% eigene Listen-Umgebung:
% ***********************
\newenvironment{MyDescription}{ \begin{list}{ $\bullet$
                                      }{ \leftmargin3.51em \labelsep0.5em
                                         \labelwidth3em \listparindent0em
                                         \rightmargin0em \itemsep3ex
                                         \parsep2ex
                                      }
                              }{ \end{list} }
\newcommand{\MyItem}[1]{\item[#1] \hspace{1em} \\} % Item f"ur MyDescription
% ***********************
% eigene Gleichungsliste:
% ***********************
\newcommand{\DS}{\displaystyle}
%               Abk"urzung f"ur die Verwendung in 'array'-Umgebung f"ur
%               mehrzeilige Formeln
% Umgebung:
\newenvironment{MyEqnArray}{   \[ \begin{array}{lrcl} \DS \MatStrut
                           }{  \end{array} \]
                           }
% Tabulator f"ur Umgebung:
\newcommand{\MT}{ & \DS } %MyTab
% Zeilenende f"ur Umgebung:
\newcommand{\MNl}{ \\ \DS \MatStrut} %MyNewline
%%
% ************************
% Schreibweisen (Symbole):
% ************************
% Zahlenmengenzeichen aus 'lsii_la.tex':
\font\sanss=cmss10
\newcommand{\Integers}{ \! \hbox{\sanss { Z\kern-.4em Z}} } %\IZ
\newcommand{\Nat}{ \hbox{\sanss {I\kern-.14em N}} }   %\IN
\newcommand{\Rationals}{ \hbox{\vrule width 0.6pt height 6pt depth 0pt
                         \hskip -3.0pt{\sanss Q}}
                       } % \IR
\newcommand{\Complex}{ \hbox{\vrule width 0.6pt height 6pt depth 0pt 
                       \hskip -3.0pt{\sanss C}}
                     } % \IC
% Zahlenmengenzeichen aus den Euler-Fonts:
%%\newcommand{\Integers}{  \Bbb{Z} }
%%\newcommand{\Nat}{       \Bbb{N} }
%%\newcommand{\Rationals}{ \Bbb{R} }
%%\newcommand{\Complex}{   \Bbb{C} }
% eigenes:
\newcommand{\proc}{\cal P \mit \,}  % Anzahl zu besch"aftigender Prozessoren
\newcommand{\permut}{\cal S \mit \! } % Menge aller Permutationen
\newcommand{\base}{\cal B \mit \,}   % Basis der logarithmischen
%                                       Zahlendarstellung
\newcommand{\accuracy}{\cal A \mit \,}
%                                   Schreibweise f"ur Anzahl der Stellen,
%                                       mit denen gerechnet wird
%                                       (accuracy <-> Genauigkeit)
\newcommand{\ExpBound}{\cal E \mit \,}
%                                   Schranke f"ur Exponenten in der
%                                       logarithmischen Darstellung
\newcommand{\LogRep}{\cal L \mit \,}
%                                   logarithmische Darstellung
%                                       (logarithmic representation)
\newcommand{\round}{\cal R \mit \,}  % Symbol f"ur Rundungsfunktion
\newcommand{\RepErr}{\cal F \mit \,} % Symbol f"ur Darstellungs Fehler
\newcommand{\necess}{\cal N \mit \,} % Symbol f"ur Anzahl n"otiger Stellen
\newcommand{\PRing}{R \, [[]]}
%           Potenzreihenring R (Liste der Unbestimmten in [[]] weggelassen)
\newcommand{\MathE}{\mbox{\rm e}} % Konstante 2.718...
%%
% ***************
% Funktionsnamen:
% ***************
\newcommand{\adj}{ \mbox{\rm adj} \,}   % Funktionsname 'adj'
\newcommand{\tr}{ \mbox{\rm tr} \,}     % Funktionsname 'tr'
\newcommand{\sgn}{ \mbox{\rm sgn} \,}   % Funktionsname 'sgn'
\newcommand{\sig}{ \mbox{\rm sig} }     % Signatur einer Permutation
\newcommand{\rg}{ \mbox{\rm rg} \,}     % Rang einer Matrix
\newcommand{\MyKer}{ \mbox{\rm ker} \,} % Kern einer Matrix
\newcommand{\MyDim}{ \mbox{\rm dim} \,} % Dimension eines Vektorraumes
\newcommand{\cond}{ \mbox{\rm cond} \,} % -> 'Pan' ...
%%
%****************
% eigene Befehle:
%****************
\newcommand{\MatStrut}{\mbox{\rule[-2ex]{0ex}{5ex}}}
\newcommand{\LMatStrut}{\mbox{\rule[-4ex]{0ex}{7ex}}}
%               St"utzen f"ur Matrizen
\newcommand{\equref}[1]{\mbox{(\ref{#1})}}
%               Verweis auf Gleichungen: Nummer in Klammern
%
\newcommand{\Mya}{"a} % ... zur Benutzung von Umlauten in Index-Begriffen
\newcommand{\Myo}{"o}
\newcommand{\Myu}{"u}
\newcommand{\Mys}{"s}
\newcommand{\MyPunkt}{ \mbox{\hspace{0.5em}.} }
\newcommand{\MyPunktA}[1]{ \nopagebreak \mbox{\hspace{#1}.} \\ }
\newcommand{\MyKomma}{ \mbox{\hspace{0.5em},} }
\newcommand{\MyKommaA}[1]{ \nopagebreak \mbox{\hspace{#1},} \\ }
%               falls ein Punkt oder ein Komma als Satzzeichen 
%               direkt hinter einer abgesetzten Gleichung stehen soll
\newcommand{\MyChoose}[2]{ \left( { #1 \atop #2 } \right) }
%               statt TeX-Befehl \choose (sieht besser aus)
\newcommand{\MySetProperty}{ \: | \: }
%               f"ur Mengen: Trennsymbol zwischen Mengenelement und
%                            Eigenschaftschaftsangabe f"ur Element
\newcommand{\lc}{\left\lceil}
\newcommand{\rc}{\right\rceil}
\newcommand{\lf}{\left\lfloor}
\newcommand{\rf}{\right\rfloor}
\newcommand{\lb}{\left(}
\newcommand{\rb}{\right)}
\newcommand{\Beq}[1]{\begin{equation} \label{#1}}
\newcommand{\Eeq}{\end{equation}}
%               Abk"urzungen
\newcommand{\und}{\wedge}
\newcommand{\oder}{\vee}
%               Verbesserung der Lesbarkeit
\newcommand{\MyStack}[2]{ \stackrel{ \mbox{\scriptsize\rm #1} }{ #2 } }
%               fuer Hinweise ueber Relationszeichen in Gleichungen
% **********************
% Text der Diplomarbeit:
% *********************
% 
% Datei: inhalt.tex (Titelseite, Referenzen und Inhaltsverzeichnis)
%
% Titelseite:
\begin{titlepage}
    \begin{center}
        \vspace*{6cm}
        \LARGE Algorithmen zur parallelen Determinantenberechnung 
                                                             \\[1.5cm]
        \Large    Holger Burbach \\[1cm]
        Oktober 1992 \\[4cm]
        Diplomarbeit, geschrieben am Lehrstuhl Informatik II \\
        der Universit"at Dortmund \\[1cm]
        Betreut von Prof. Ingo Wegener
    \end{center}
\end{titlepage}
\setcounter{page}{2}
%
% Inhaltsverzeichnis:
{  \parskip0ex plus 0.5ex
   \tableofcontents
}


%
% Datei: vorbem.tex
%
\MyChapter{Vorbemerkungen}
\label{ChapIntro}

% **************************************************************************

\MySection{Einf"uhrung}

Die Berechnung der Determinante einer quadratischen Matrix ist ein
Problem,
dessen effiziente L"osung in vielen Bereichen von Interesse ist, in der
Informatik z. B. in der Computergrafik und der Kodierungstheorie.

Ein Problem in der analytischen Geometrie ist es, die 
Lage von geometrischen Objekten zueinander festzustellen und 
Schnittpunkte oder -ebenen zu berechnen. Ein Teilproblem dabei ist die
Pr"ufung der linearen Unabh"angigkeit von Vektoren. Es
l"a"st sich auf die Berechnung einer Determinante zur"uckf"uhren.

Ein weiterer Bereich, in dem die Determiantenberechnung angewendet wird,
ist
die theoretische Physik. Dort wird in vielen Theorien auf Matrizen zu
Beschreibung der verschiedenen Sachverhalte zur"uckgegriffen. In der
Einstein'sche Relativit"atstheorie z. B. wird die {\em Tersorrechnung},
die die Eigenschaften sich von Koordinatensystem zu Koordinatensystem
"andernder Ma"szahlen untersucht, ausgibig verwendet. Die Determinante ist
eine solche Ma"szahl. Beim Studium von Literatur, die diese Thematiken
behandelt (z. B. \cite{BS86} ab S. 70), st"o"st man immer wieder auf
Matrizen und ihre Determinanten.

Seitdem
sich die Forschung im Bereich der Informatik zunehmend mit 
Parallelrechnern
be\-sch"af\-tigt, werden f"ur alle bekannten Probleme Algorithmen gesucht, 
die die Tatsache, da"s auf einem Parallelrechner mehrere Prozessoren 
gleichzeitig an der L"osung desselben Problems arbeiten, besonders 
effizient ausnutzen.

Betrachtet man eine Matrix aus der Sicht der Informatik als Datenstruktur,
so dr"angt sich die Benutzung dieser Datenstruktur in Parallelrechnern 
geradezu auf, denn intuitiv, ohne zun"achst alle Probleme ausgearbeitet 
zu haben, kann man auf die Idee kommen, die Matrizenelemente jeweils
einzelnen Prozessoren oder Gruppen von Prozessoren zuzuordnen, die das 
zugrundeliegende Problem f"ur dieses Matrizenelement bearbeiten. 
Selbstverst"andlich ist die praktische Verwendung dieser Idee nicht in 
jedem Fall ganz so einfach.

So war die effiziente Parallelisierung der Determinantenberechnung lange
Zeit ein ungel"ostes Problem, bis 1976, als Laszlo Csanky einen in jenen 
Tagen "uberraschenden Algorithmus ver"offentlichte \cite{Csan76}. Es 
folgten eine Reihe weiterer Algorithmen verschiedener Autoren mit 
vergleichbaren Leistungsmerkmalen.

In all diesen Ver"offentlichungen wird vorrangig die Gr"o"senordnung 
der Laufzeiten und Anzahlen der Prozessoren betrachtet. Es werden in
einzelnen Ver"offentlichungen auch bereits einige Vergleiche mit den 
anderen Algorithmen durchgef"uhrt. So ist es w"unschenswert einen 
"Uberblick "uber die existierenden Algorithmen zu bekommen und sie 
insgesamt miteinander zu vergleichen.

Die vorliegende Diplomarbeit behandelt vier Algorithmen zur 
parallelen De\-ter\-mi\-nan\-ten\-be\-rech\-nung\footnote{ \cite{Csan76}, 
\cite{BGH82}, \cite{Berk84} und \cite{Pan85}}. Dabei wird auf die 
Verwendung von Gr"o"senordnungen ( O-Notation ) in Aufwandsanalysen 
weitgehend verzichtet.
Um den Einsteig in das Thema zu 
erleichtern, wird zus"atzlich noch der Entwicklungssatz von
Laplace zur Berechnung der Determinante erw"ahnt.

Die Darstellung der vier Algorithmen in den zugeh"origen Kapiteln
\ref{ChapCsanky} bis \ref{ChapPan} umfa"st neben den Grundlagen und der 
Algorithmen selbst, jeweils eine Analyse der Rechenzeit und des Grades der
Parallelisierung\footnote{Diese Begriffe sind in Kapitel \ref{SecBez}
definiert.}. Da in der Praxis die Gr"o"se des ben"otigen 
Speicherplatzes kein vorrangiges Problem mehr darstellt, wird dieser 
Wert nicht analysiert. Matrizen mit Elementen aus $\Complex$ werden nicht
betrachtet. In diesen vier Kapitel wird versucht, auf Unterschiede
und Gemeinsamkeiten der Algorithmen einzugehen.

Im Anschlu"s an die Darstellung der Algorithmen wird in Kapitel 
\ref{ChapImplemen} ihre Implementierung beschrieben. Die Quelltexte 
sind im Anhang zu finden.
Schlie"slich erfolgt in Kapitel \ref{ChapEndbem}
ein zusammenfassender Vergleich der Algorithmen.

Der Text soll es erm"oglichen, die Algorithmen ohne weitere Literatur 
anhand
von Grundkenntnissen aus der Mathematik und Informatik zu verstehen. 
Aus diesem Grund und um einheitliche Bezeichnungen zu vereinbaren sind 
Grundlagen, insbesondere aus der Linearen Algebra, an den ben"otigten 
Stellen aufgef"uhrt. F"ur den Fall, da"s die im Text enthaltenen
Informationen nicht ausreichen, sind die benutzten Quellen an den
jeweiligen Stellen angegeben.

Alle Betrachtungen abstrahieren von technischen Problemen bei der 
Konstruktion von Parallelrechnern. Dazu wird das Rechnermodell der
PRAM benutzt. Die Beschreibung dieses Modells erfolgt in 
Kapitel \ref{SecModell}. 

Vor anderen Teilen dieser Diplomarbeit sollten zun"achst die Kapitel 
\ref{SecModell} und \ref{SecBez} gelesen werden. Alle 
weiteren Teile von Kapitel \ref{ChapIntro} sowie Kapitel \ref{ChapBase}
sind als Sammlung von Grundlagen zu verstehen, auf die bei Bedarf 
zur"uckgegriffen werden kann\footnote{Im Text kommen h"aufig 
Punkte und Kommata als Satzzeichen direkt im Anschlu"s an
abgesetzte Gleichungen vor. An einigen Stellen, besonders hinter
Vektoren und Matrizen, fehlen diese Satzzeichen aus technischen 
Gr"unden.}.

% **************************************************************************

\MySection{Das Berechnungsmodell}
\label{SecModell}
\index{Berechnungsmodell} \index{PRAM} \index{CRCW}
\index{Modellrechner}
In diesem Kapitel wird der f"ur Komplexit"atsbetrachtungen verwendete
Modellrechner beschrieben. Es ist die 
{\em Arbitrary Concurrent Read Concurrent Write Parallel Random Access 
  Machine ( arbitrary CRCW PRAM) }.
Sie besteht aus
gleichen Prozessoren, die alle auf denselben Arbeitsspeicher zugreifen.
Innerhalb einer Zeiteinheit k"onnen diese
Prozessoren, und zwar alle gleichzeitig,
zwei Operanden aus dem Speicher lesen, eine der in Tabelle \ref{Csan76Tab2}
aufgef"uhrten Operationen ausf"uhren und
das Ergebnis wieder im Speicher ablegen. Falls beim
Schreiben mehrere Prozessoren auf eine Speicherzelle zugreifen, mu"s der
Algorithmus unabh"angig davon korrekt sein, welcher Prozessor seinen
Schreibzugriff tats"achlich ausf"uhrt.
\begin{table}[htb]
    \begin{center}
    \begin{tabular}{|p{4cm}|c|}
        \hline
        Operation & Symbol \\
        \hline
        \hline
        Addition & $+$ \\
        \hline
        Subtraktion & $-$ \\
        \hline
        Multiplikation & $*$ \\
        \hline
        Division (Ergebnis in $\Rationals$) & $/$ \\
        \hline
        Division (Ergebnis in $\Integers$) & div \\
        \hline
        $ x - (x \; \mbox{div} \; y) * y $ & $x$ mod $y$ \\
        \hline
    \end{tabular}
    \end{center}
    \caption{Operationen des Modellrechners}
    \label{Csan76Tab2}
\end{table}
Da die
Komplexit"at der vier haupts"achlich interessierenden 
Algorithmen nicht nur auf ihre Gr"o"senordnung hin
untersucht wird, sondern die Ausdr"ucke zur Beschreibung der Komplexit"at
genau angegeben werden sollen, ist es erforderlich, von Details
der Implementierung, die die Konstanten
beeinflussen, zu abstrahieren, so da"s die Aussagen allgemeing"ultig sind.
Aus diesem Grund wird
\begin{itemize}
    \item f"ur die Verarbeitung von Schleifenbedingungen,
    \item f"ur die Verarbeitung von Verzweigungsbedingungen,
    \item f"ur die Ein- und Ausgabe von Daten,
    \item f"ur die Initialisierung von Speicherbereichen,
    \item und f"ur komplexe Adressierungsarten
          (z. B. indirekte Adressierung) beim Zugriff auf Speicherbereiche
\end{itemize}
kein zus"atzlicher Aufwand in Rechnung gestellt. Es werden also nur die
arithmetischen Operationen gez"ahlt.

Zu beachten ist, da"s bei einer PRAM jeder Aufwand zur Verteilung von
Aufgaben auf verschiedene Prozessoren vernachl"assigt wird. Diese 
Eigenschaft bietet die M"oglichkeit zur Kritik, da so jedes
Problem deutlich vereinfacht wird, jedoch in der Praxis die Organisation der 
Aufgabenverteilung nicht unerheblichen Aufwand erfordert.
Eine genaue Analyse der 
Auswirkungen dieser Vernachl"assigung ist umfangreich und nicht 
Thema des vorliegenden Textes.

% **************************************************************************

\MySection{Bezeichnungen}
\label{SecBez}

In diesem Kapitel werden die verwendeten Begriffe und Symbole 
definiert.

Um auf die in der Arbeit haupts"achlich behandelten Algorithmen 
einfach Bezug
nehmen zu k"onnen, werden mit Hilfe der Namen ihrer Autoren die
folgenden Abk"urzungen vereinbart:
\begin{itemize}
\item
      C-Alg. steht f"ur den Algorithmus von Csanky 
      (Unterkapitel \ref{SecAlgFrame} ab S. \pageref{SecAlgFrame}).
\item
      BGH-Alg. steht f"ur den Algorithmus von Borodin, von zur Gathen 
      und Hopcroft (Unterkapitel \ref{SecAlgBGH} ab 
      S. \pageref{SecAlgBGH}).
\item
      B-Alg. steht f"ur den Algorithmus von Berkowitz 
      (Unterkapitel \ref{SecAlgBerk} ab S. \pageref{SecAlgBerk}).
\item
      P-Alg. steht f"ur den Algorithmus von Pan
      (Unterkapitel \ref{SecAlgPan} ab S. \pageref{SecAlgPan}).
\end{itemize}

\index{Bezeichnungen!nat{\Myu}rliche Zahlen}
Die Menge der positiven ganzen Zahl {\em ohne Null} wird mit \[ \Nat \]
bezeichnet. Die Menge der
positiven ganzen Zahl einschlie"slich der Null wird mit \[ \Nat_0 \]
bezeichnet. Falls nicht im Einzelfall anders festgelegt erfolgen alle
Darstellungen von Zahlen zur Basis $10$.

\index{Bezeichnungen!Schritt}
Der Vorgang, in dem beliebig viele Prozessoren gleichzeitig je
zwei Operanden aus dem Arbeitsspeicher lesen, aus diesen Operanden
ein Ergebnis berechnen und dieses Ergebnis wieder im
Arbeitsspeicher ablegen, wird als ein {\em Schritt} bezeichnet.

\index{Zeitkomplexit{\Mya}t}
\index{Parallelisierungsgrad}
Die {\em parallele Zeitkomplexit"at eines Algorithmus} bezeichnet die
Anzahl der Schritte, die dieser ben"otigt, um die L"osung\footnote{Die
von uns betrachteten Probleme besitzen nur eine L"osung.}
f"ur das
zugrunde liegende Problem zu berechnen.
Die maximale Anzahl der Prozessoren, die dabei gleichzeitig
besch"aftigt werden, wird mit {\em Parallelisierungsgrad des
Algorithmus} bezeichnet.

Falls nicht im Einzelfall anders festgelegt, gilt folgende 
Regelung: Gro"sbuchstaben bezeichnen
Matrizen und Kleinbuchstaben Zahlen oder Vektoren, $A$ bezeichnet eine
$n \times n$-Matrix, indizierte
Kleinbuchstaben beziehen sich auf die Elemente der mit dem zugeh"origen 
Gro"sbuchstaben bezeichneten
Matrix.

Die in Tabelle \ref{Csan76Tab1} aufgelisteten Schreibweisen werden
benutzt.
\index{Bezeichnungen!Indizierung}
\index{Einheitsmatrix} \index{Nullmatrix}
\index{Permutation}
\begin{table}[htb]
    \begin{center}
    \begin{tabular}{|p{10cm}|c|}
        \hline
            Begriff & Schreibweise \\
        \hline\hline
            Element in Zeile $i$, Spalte $j$ von $A$ & $ a_{i,j} $ \\
        \hline
            $i$-tes Element des Vektors $v$ & $v_i$ \\
        \hline
            Matrix, die aus A durch Streichen der Zeilen $v$ und der
            Spalten $w$ entsteht (dabei seien $v$ und $w$
            echte Teilmengen der Menge der Zahlen von $1$ bis $n$; diese
            Mengen werden hier als durch Kommata getrennt Zahlenfolge
            geschrieben) & $ {A}_{(v|w)} $ \\
        \hline
            Einheitsmatrix (Elemente der Haupt\-di\-ago\-na\-len gleich $1$;
            alle anderen Elemente gleich $0$)
            mit $n$ Zeilen und Spalten & $E_n$ \\
        \hline
            Einheitsmatrix (Anzahl der
            Zeilen und Spalten aus dem Zusammenhang klar) & $E$ \\
        \hline
            Nullmatrix (alle Elemente sind gleich 0) mit $m$ Zeilen und
            $n$ Spalten & $0_{m,n}$ \\
        \hline
            Nullvektor (alle Elemente sind gleich 0) der L"ange $m$ &
            $0_m$ \\
        \hline
            Logarithmus von $x$ zur Basis $2$      & $\log(x)$ \\
        \hline
            Menge aller $n$-stelligen Permutationen & $\permut_n$ \\
        \hline
            \begin{minipage}{10em}
                \begin{math} \displaystyle
                     \lim_{n\rightarrow \infty}
                     \left( 1 + \frac{1}{n} \right)^n
                \end{math}
            \end{minipage} \LMatStrut
            $(\: = 2.718281\ldots)$ &  \MathE  \\
        \hline
            Logarithmus von $x$ zur Basis $\MathE$ & $\ln(x)$ \\
        \hline
            Anzahl der Elemente der Menge $M$ & $|M|$ \\
        \hline
    \end{tabular}
    \end{center}
    \caption{Bezeichnungen}
    \label{Csan76Tab1}
\end{table}

% **************************************************************************

\MySection{Das Pr"afixproblem}
Von einer effizienten L"osung des Pr"afixproblems wird an verschiedenen
Stellen Gebrauch gemacht. Es ist also von "ubergreifendem Interesse und
wird deshalb hier behandelt (\cite{LF80},
\cite{Wege89} S. 83 ff.). Es l"a"st sich folgenderma"sen formulieren:
\begin{quote}
\index{Pr{\Mya}fixproblem}
\label{PagePraefixproblem}
    Gegeben sei die Halbgruppe \[ (M,\circ) \MyPunkt \] 
    D. h. die Verkn"upfung 
    $\circ$ ist assoziativ auf $M$. Weiterhin seien 
    \[ x_1,x_2,x_3,\ldots,x_{n} \]
    Elemente aus $M$. Es wird definiert
    \[ p_i := x_1 \circ x_2 \circ x_3 \circ \ldots \circ x_i \MyPunkt \]
    Das Pr"afixproblem besteht darin, alle Elemente der Menge
    \[ \{ p_i | 1 \leq i \leq n \} \] zu berechnen.
\end{quote}
Es sind u. a. zwei M"oglichkeiten\footnote{die sich zu einer dritten 
zusammenfassen lassen (Satz \ref{SatzAlgPraefix})} denkbar, dies mit 
parallelen Algorithmen zu erreichen.
\begin{itemize}
    \item Die erste M"oglichkeit:
        \begin{enumerate}
            \item L"ose das Pr"afixproblem parallel f"ur
                  \[ x_1,\ldots,x_{\lceil n/2 \rceil} \]
                  und
                  \[ x_{\lceil n/2+1 \rceil},\ldots,x_n \MyKomma \]
                  so da"s nach diesem Schritt
                  \[ p_1,\ldots,p_{\lceil n/2 \rceil} \]
                  bereits berechnet sind.
            \item
                  Berechne aus \[ p_{\lceil n/2 \rceil} \] und
                  der L"osung des Problems f"ur
                  \[ x_{\lceil n/2+1 \rceil},\ldots,x_n \]
                  parallel in einem weiteren Schritt
                  \[ p_{\lceil n/2+1 \rceil},\ldots,p_n \]
        \end{enumerate}
    \item
        Die zweite M"oglichkeit, die hier kurz dargestellt werden soll,
        sieht folgenderma"sen aus (o. B. d. A. sei $n$ eine Zweierpotenz):
        \begin{enumerate}
            \item
                Berechne parallel in einem Schritt
                \[ x_1 \circ x_2, x_3 \circ x_4, \ldots,
                   x_{n-1} \circ x_n \]
            \item
                L"ose das Pr"afixproblem f"ur diese
                $n/2$ Werte. Damit werden alle $p_i$ mit geradem $i$
                berechnet.
            \item
                Die noch fehlenden $p_i$ f"ur ungerade $i$ k"onnen nun
                parallel in einem weiteren Schritt aus der L"osung f"ur die
                $n/2$ Werte und den $x_i$ mit ungeradem $i$ berechnet
                werden.
        \end{enumerate}
\end{itemize}
Diese beiden M"oglichkeiten k"onnen zu einem Algorithmus zusammengefa"st
werden:
\begin{satz}
\label{SatzAlgPraefix}
\index{Algorithmus!Pr{\Mya}fixproblem}
    Gegeben sei die Halbgruppe \[ (M,\circ) \MyPunkt \] 
    Das Pr"afixproblem
    f"ur $n$ Elemente \[ x_1,x_2,\ldots,x_n \] von $M$ l"a"st sich
    von \[ \lf \frac{3}{4}n \rf \] Prozessoren in
    \[ \lceil \log(n) \rceil \] Schritten l"osen.
\end{satz}
\begin{beweis}
    O. B. d. A. sei $n$ eine Zweierpotenz. In dem Fall, da"s $n$ keine
    Zweierpotenz ist, wird $n$ durch die n"achst h"ohere Zweierpotenz $n'$
    ersetzt und alle Verkn"upfungen mit Elementen $x_i$ von $M$ f"ur
    \[ i>n \] werden nicht durchgef"uhrt. 

    Benutze folgenden Algorithmus:
    \begin{enumerate}
    \item
          Wenn \[ n=1 \] dann ist $x_1$ das Ergebnis.
    \item
          Wenn \[ n=2 \] dann ist \[ x_1, x_1 \circ x_2 \]
          das Ergebnis.
    \item
          Schritte \ref{StepPraefix3a} und \ref{StepPraefix3b} parallel:
          \begin{enumerate}
          \item \label{StepPraefix3a}
                \begin{enumerate}
                \item \label{StepPraefix3a1}
                      Berechne parallel in einem Schritt
                      \[ x_1 \circ x_2,\ldots,
                         x_{n/2-1} \circ x_{n/2}
                      \]
                \item \label{StepPraefix3a2}
                      Benutze den Algorithmus rekursiv zur L"osung des
                      Problems f"ur die in Schritt \ref{StepPraefix3a1}
                      erhaltenen $n/4$ Werte. Auf diese Weise sind die
                      $p_i$ f"ur
                      \[ 1 \leq i \leq n/2 \] mit geraden $i$, u. a.
                      auch $p_{n/2}$, bereits berechnet.
                \end{enumerate}
          \item \label{StepPraefix3b}
                Benutze den Algorithmus rekursiv zur L"osung des Problems
                f"ur \[ x_{n/2+1},\ldots,x_n \]
          \end{enumerate}
    \item Schritte \ref{StepPraefix4a} und \ref{StepPraefix4b} parallel:
          \begin{enumerate}
          \item \label{StepPraefix4a}
               F"ur $i$ gelte \[ 1 \leq i \leq n/2 \MyPunkt \]
               Wenn $n/2 > 2$, dann
               berechne parallel in einem Schritt mit Hilfe der $p_i$
               aus \ref{StepPraefix3a2} und der $x_i$ mit ungeradem $i$
               die fehlenden $p_i$ mit ungeradem $i$.
          \item \label{StepPraefix4b}
                Berechne aus $p_{n/2}$ und den Ergebnissen von 
                \ref{StepPraefix3b} die $p_i$ mit
                \[ n/2+1 \leq i \leq n \MyPunkt \]
          \end{enumerate}
    \end{enumerate}
    Zur Analyse des Algorithmus bezeichnet $s(n)$ die Anzahl der Schritte,
    die er ben"otigt, um das Pr"afixproblem f"ur $n$ Eingabewerte zu
    l"osen, und $p(n)$ die Anzahl der Prozessoren, die dabei besch"aftigt
    werden k"onnen.
    \begin{itemize}
    \item Hier wird zun"achst die Anzahl der Schritte betrachtet.
          Es gilt \[ s(1) = 0,\,s(2) = 1,\, s(4) = 2 \MyPunkt \]
          Bei der Betrachtung des Algorithmus erkennt man, da"s folgende
          Rekursionsgleichung G"ultigkeit besitzt:
          \Beq{Berk84Equ6}
              \forall n>4: \: s(n) = \max(s(n/4)+1,\,s(n/2)) + 1 \MyPunkt
          \Eeq
          Wenn man diese Formel auf $s(n/2)$ anwendet und das Ergebnis in
          die obige Formel einsetzt, erh"alt man
          \[ \forall n>4: \: s(n) = 
                 \max(\: s(n/4)+1,\, \max(s(n/8)+1,\, s(n/4))+1 \:) \:+ 1 
          \]
          Aufgrund der Assoziativit"at der $\max$-Funktion ist dies
          gleichbedeutend mit
          \begin{MyEqnArray}
             \MT \forall n>4: \: s(n) \MT = \MT
                 \max(s(n/4)+1,\, s(n/8)+2,\, s(n/4)+ 1) + 1 \MNl
             \Rightarrow 
             \MT \forall n>4: \: s(n) \MT = \MT s(n/4)+2 
          \end{MyEqnArray}
          Es gilt also f"ur jedes $i$:
          \[ \forall n>4: \: s(n) = s(n/2^{2i}) + 2i \]
          Mit \[ i = \frac{\log(n)}{2} \] erh"alt man als Endergebnis 
          \[ s(n) = \log(n) \MyPunkt \]
    \item F"ur die Anzahl der besch"aftigten Prozessoren $p(n)$ gilt:
          \[ p(1) = 0, p(2) = 1, p(4)= 2 \]
          Ferner gilt offensichtlich folgende Rekursionsgleichung:
          \Beq{Berk84Equ8}
              \forall n>4:\: p(n)= 
              \max(n/2+n/4,\, p(n/4)+p(n/2),\, n/4+p(n/2))
          \Eeq
          Beim Ausrechnen der Werte von $p(8)$, $p(16)$ und $p(32)$
          mit Hilfe dieser Rekursionsgleichung gelangt man zu der
          Vermutung, da"s gilt:
          \Beq{Berk84Equ7}
              \forall n>4: \: p(n) = \frac{3}{4} n 
          \Eeq
          Dies wird durch Induktion bewiesen. Zu beachten
          ist, da"s nach Voraussetzung nur die Potenzen von $2$ als Werte
          f"ur $n$ in Frage kommen. 
          
          Sei also nun \[ n>4 \] und es gelte
          \begin{eqnarray*}
              p(n) & = & \frac{3}{4}n \\
              p(n/2) & = & \frac{3}{8}n \MyPunkt
          \end{eqnarray*}
          Es ist zu zeigen, da"s dann auch
          \[ p(2n)= \frac{3}{2}n \]
          richtig ist.
          Nach der Rekursionsgleichung \equref{Berk84Equ8} gilt
          \[ p(2n) = \max(3/2*n, p(n/2)+p(n), n/2 + p(n)) \]
          Die Anwendung der Induktionsvoraussetzung f"uhrt zu
          \[ p(2n) = \max(3/2*n, 3/8*n + 3/4*n, n/2 + 3/4*n) \] und somit zu
          \[ p(2n) = \frac{3}{2}n \] was zu zeigen war. F"ur die Anzahl der 
          ben"otigen Prozessoren gilt also
          \[ \forall n > 4: \: p(n) = \frac{3}{4}n \]
          Da die L"osung des Problems f"ur \[ n \leq 4 \] einfach ist, wird
          die Quantifizierung nicht weiter beachtet.
    \end{itemize}
    Damit die Aussagen nicht nur f"ur Zweierpotenzen, werden die Werte
    f"ur $s(n)$ und $p(n)$ mit Gau"sklammern versehen. F"ur $p(n)$ ist
    dies ohne weitere Begr"undung problematisch. Betrachtet man den 
    Algorithmus jedoch genauer, stellt man fest, da"s beim ersten 
    ausgef"uhrten Schritt die meisten Prozessoren besch"aftigt werden. 
    Die Anzahl dieser Prozessoren gibt der Term in Gau"sklammern an.
\end{beweis}

% **************************************************************************

\MySection{L"osungen grundlegender Probleme}
In diesem Kapitel werden Algorithmen zur L"osung einiger
grundlegender Probleme angegeben und auf ihre Komplexit"at hin untersucht.

\begin{satz}[Bin"arbaummethode]  % $$$ wird benutzt (nicht loeschen)
\label{SatzAlgBinaerbaum}
    Wird das Pr"afixproblem (siehe Beschreibung Seite
    \pageref{PagePraefixproblem}) dahingehend vereinfacht, da"s nur
    $p_n$ zu berechnen ist, so l"a"st sich dieses vereinfachte Problem in
    \[ \lc \log(n) \rc \] Schritten von \[ \lf \frac{n}{2} \rf \] 
    Prozessoren l"osen.
\end{satz}
\begin{beweis}
    Verkn"upfe die $x_i$ nach dem Schema in Abbildung \ref{PicBinBaum}.
    \begin{figure}[htb]
    \begin{center}
        %
% You need 'epic.sty'
%
\setlength{\unitlength}{1mm}
\makeatletter
\def\Thicklines{\let\@linefnt\tenlnw \let\@circlefnt\tencircw
\@wholewidth4\fontdimen8\tenln \@halfwidth .5\@wholewidth}
\makeatother
\begin{picture}(110.00,54.00)
\put(1.00,49.00){$x_1$}
\put(1.00,45.00){$x_2$}
\put(1.00,41.00){$x_3$}
\put(1.00,37.00){$x_4$}
\put(1.00,33.00){$x_5$}
\put(1.00,29.00){$x_6$}
\put(1.00,25.00){$x_7$}
\put(1.00,21.00){$x_8$}
\put(2.25,16.50){$\vdots$}
\put(1.00,13.00){$x_{n-3}$}
\put(1.00,9.00){$x_{n-2}$}
\put(1.00,5.00){$x_{n-1}$}
\put(1.00,1.00){$x_n$}
\drawline(7.25,50.00)(15.25,48.00)
\drawline(7.25,46.00)(15.25,48.00)
\drawline(7.25,42.00)(15.25,40.00)
\drawline(7.25,38.00)(15.25,40.00)
\drawline(7.25,34.00)(15.25,32.00)
\drawline(7.25,30.00)(15.25,32.00)
\drawline(7.25,26.00)(15.25,24.00)
\drawline(7.25,22.00)(15.25,24.00)
\drawline(7.00,15.00)(15.00,13.00)
\drawline(7.00,11.00)(15.00,13.00)
\drawline(7.00,7.00)(15.00,5.00)
\drawline(7.00,3.00)(15.00,5.00)
\put(17.25,47.25){$\circ$}
\put(17.25,39.25){$\circ$}
\put(17.25,31.25){$\circ$}
\put(17.25,23.25){$\circ$}
\put(17.25,11.25){$\circ$}
\put(17.25,3.25){$\circ$}
\drawline(21.00,13.00)(33.00,9.00)
\drawline(21.00,5.00)(33.00,9.00)
\drawline(21.00,49.00)(33.00,45.00)
\drawline(21.00,41.00)(33.00,45.00)
\drawline(21.00,33.00)(33.00,29.00)
\drawline(21.00,25.00)(33.00,29.00)
\put(34.00,8.00){$\cdots$}
\put(35.25,27.50){$\circ$}
\put(35.25,43.50){$\circ$}
\drawline(41.00,45.00)(55.00,37.00)
\drawline(41.00,29.00)(55.00,37.00)
\put(99.00,21.00){$p_n$}
\put(79.25,7.25){$\circ$}
\drawline(77.00,9.00)(71.00,15.00)
\drawline(77.00,9.00)(71.00,3.00)
\put(79.25,35.25){$\circ$}
\drawline(77.00,37.00)(71.00,43.00)
\drawline(77.00,37.00)(71.00,31.00)
\drawline(97.00,23.00)(83.00,37.00)
\drawline(97.00,23.00)(83.00,9.00)
\put(65.00,41.50){$\cdots$}
\put(65.00,29.50){$\cdots$}
\put(65.00,13.50){$\cdots$}
\put(65.00,1.50){$\cdots$}
\put(56.00,36.00){$\cdots$}
\end{picture}
%

        \caption{Bin"arbaummethode}
        \label{PicBinBaum}
    \end{center}
    \end{figure}
    Falls $n$ keine Zweierpotenz ist, werden die Verkn"upfungen mit den 
    $x_j$, f"ur die gilt \[ n < j \leq 2^{\lc \log(n) \rc} \MyKomma \]
    nicht durchgef"uhrt. Die L"osung des Problems erfordert offensichtlich
    den angegebenen Aufwand.
    \mbox{ \hspace{4em} \hfill }
\end{beweis}

\begin{korollar}[Parallele Grundrechenarten]
\label{SatzAlgRechnen}             % $$$ wird benutzt (nicht loeschen)
\index{Algorithmus!parallele Grundrechenarten}
    Seien $n$ Zahlen durch die gleiche Rechenoperation miteinander zu
    verkn"upfen. Diese Rechenoperation sei eine der Grundrechenarten
    Addition oder Multiplikation. Die Verkn"upfung kann in
    \[ \lc \log(n) \rc \] Schritten von
    \[ \lf \frac{n}{2} \rf \] Prozessoren durchgef"uhrt
    werden.
\end{korollar}
\begin{beweis}
    Aufgrund der Assoziativit"at der Rechenoperationen folgt
    dies direkt aus Satz \ref{SatzAlgBinaerbaum}. 
    \mbox{ \hspace{4em} \hfill }
\end{beweis}

\begin{satz}[Parallele Matrizenmultiplikation]
\label{SatzAlgMatMult}            % $$$ wird benutzt (nicht loeschen)
\index{Algorithmus!parallele Matrizenmultiplikation}
    Sei $A$ eine $m \times p$-Matrix und $B$ eine $p \times n$-Matrix.
    Sie lassen sich in
    \[ \lceil \log(p) \rceil + 1 \] Schritten von
    \[ m * n * p \] Prozessoren miteinander
    multiplizieren.
\end{satz}
\begin{beweis}
    Sei $C$ die $m \times n$-Ergebnismatrix. 
    Sie wird mit Hilfe der Gleichung
    \[ c_{i,j} = \sum_{k=1}^p a_{i,k} b_{k,j} \]
    berechnet. Dazu werden zuerst parallel in einem Schritt
    \[ d_{i,k,j} := a_{i,k} b_{k,j} \] mit
    \begin{eqnarray*}
         1 \leq & i & \leq m \\
         1 \leq & j & \leq n \\
         1 \leq & k & \leq p 
    \end{eqnarray*} von
    \[ m * n * p \] Prozessoren berechnet.
    Die Ergebnismatrix erh"alt man dann nach der Gleichung
    \[ c_{i,j} = \sum_{k=1}^p d_{i,k,j} \]
    Die Berechnung der Matrix $C$ aus den $d_{i,k,j}$ kann nach
    \ref{SatzAlgRechnen} f"ur ein Matrizenelement in
    \[ \lceil \log(p) \rceil \] Schritten von
    \[ \lf \frac{p}{2} \rf \] Prozessoren durchgef"uhrt
    werden, also f"ur
    die gesamte Matrix in genauso vielen Schritten von
    \[ m * n * \lf \frac{p}{2} \rf \] Prozessoren.
    Die Werte
    f"ur Schritte und Prozessoren zusammengenommen ergeben die Behauptung.
\end{beweis}

Zwei $n \times n$-Matrizen lassen sich also in 
\[ \lceil \log(n) \rceil + 1 \] Schritten von
\[ n^3 \] Prozessoren miteinander multiplizieren.

Die \label{PageAlg2MatMult}
Matrizenmultiplikation l"a"st sich asymptotisch, d. h. f"ur $n \to \infty$
auch mit
\[ O(n^{2+\gamma}), \: \gamma = 0.376 \]
Prozessoren durchf"uhren \cite{CW90}. Gegen"uber \ref{SatzAlgMatMult} 
ergibt sich wegen des erheblichen konstanten Aufwandes nur f"ur gro"se $n$ 
eine Verbesserung. Es wird jeweils gesondert darauf hingewiesen, falls
auf diese M"oglichkeit zur"uckgegriffen wird.

% **************************************************************************
% **************************************************************************
% **************************************************************************

\MyChapter{Grundlagen aus der Linearen Algebra}
\label{ChapBase}

In diesem Kapitel werden die f"ur den gesamten weiteren Text wichtigen 
Begriffe und S"atze aus der Linearen Algebra behandelt. Falls in sp"ateren
Kapiteln an einzelnen Stellen weitergehende Grundlagen insbesondere aus 
anderen Bereichen n"otig sind, werden diese an den jeweiligen Stellen 
behandelt.

Da es sich bei dem Inhalt dieses Kapitels um
Grundlagen handelt, sind einige Beweise etwas oberfl"achlicher bzw. 
fehlen ganz.

Literatur:
\begin{itemize}
\item
      \cite{MM64} Kapitel 1 und 2
\item
      \cite{Doer77} Kapitel 6, 9 und 12
\item
      \cite{BS87} ab Seite 148
\end{itemize}

Im folgenden sind $A$ und $B$ $n \times n$-Matrizen. F"ur uns reichen
Betrachtungen im K"orper der rationalen Zahlen aus.

% **************************************************************************

\MySection{Matrizen und Determinanten}
\label{SecMatUndDet}

In diesem Kapitel werden die grundlegendsten Begriffe "uber Matrizen
und Determinanten aufgef"uhrt, um eine Grundlage f"ur den weiteren 
Text zu vereinbaren. 

\MyBeginDef
\index{invertierbar}
\label{DefInvertierbar}
    $A$ hei"st {\em invertierbar}, wenn es eine
    Matrix $B$ gibt, so da"s \[ AB = BA = E_n \]
    In diesem Fall hei"st $B$ {\em Inverse von $A$} und wird auch mit
    \[ A^{-1} \] bezeichnet.
\MyEndDef

\MyBeginDef
\index{Transponierte}
    Falls f"ur die Matrizen $A$ und $B$ gilt
    \[ b_{i,j} = a_{j,i} \] so hei"st $B$ { \em Transponierte von $A$}.
    F"ur die Transponierte von $A$ wird auch $A^T$ geschrieben.
\MyEndDef

\MyBeginDef
\label{DefTr}
\index{Spur}
    \[ \tr(A) := \sum_{i=1}^n a_{i,i} \]
    hei"st {\em Spur der Matrix $A$}.
\MyEndDef

\MyBeginDef
\index{Permutation}
\index{Inversion}
    Eine bijektive Abbildung
    \[ f : \{1,\ldots,n \} \rightarrow \{1, \ldots, n \} \]
    hei"st {\em $n$-Permutation}.
    Sei \[ 1 \leq i < j \leq n \] Falls gilt
    \[ f(i) > f(j) \MyKomma \] so hei"st diese Bedingung {\em Inversion der
    $n$-Permutation}.
\MyEndDef

\[ \permut_n \] bezeichnet die Menge aller $n$-Permutationen. Zusammen mit
der Konkatenation von Abbildungen bildet sie eine Gruppe, die
{\em symmetrische Gruppe $\permut_n$}.

\MyBeginDef
\index{Signatur einer Permutation}
\label{DefSig}
    Sei $f$ eine $n$-Permutation. Dann hei"st
    \begin{equation}
    \label{EquDefSig}
        \sig(f) := \prod_{1 \leq i < j \leq n} \frac{f(i)-f(j)}{i - j}
    \end{equation}
    {\em Signatur von $f$}.
\MyEndDef

Die so definierte Signatur besitzt folgende Eigenschaften:
\begin{itemize}
\item 
      Es gilt
      \[ \forall f \in \permut_n: \sig(f) \in \{ 1,-1 \} \]
      Aus der Permutationseigenschaft ergibt sich, da"s es f"ur jede 
      Differenz, die als Faktor im Z"ahler von \equref{EquDefSig} auftaucht,
      eine Differenz im Nenner mit dem gleichen Betrag existiert, so da"s 
      der Wert des gesamten Produktes den Betrag $1$ besitzt. Das Vorzeichen
      wird durch die Anzahl der Inversionen beeinflu"st.
\item
      Falls die Anzahl der Inversionen der $n$-Permutation $f$ gerade ist,
      so gilt \[ \sig(f)= 1 \MyKomma \] andernfalls gilt
      \[ \sig(f) = -1 \]
\item

      Die Anzahl der $n$-Permutationen $f$ mit \[ \sig(f)=1 \MyKomma \]
      ist gleich
      der Anzahl der $n$-Permutationen $g$ mit \[ \sig(g)=-1 \MyPunkt \]
      (\cite{Doer77} Seite 196)
\end{itemize}

Die Permutationen mit \[ \sig(f)=1 \] nennt man {\em gerade}, die anderen 
{\em ungerade}.

\MyBeginDef
\index{Determinante!Definition}
\label{DefDet}
    Seien \[ A,B \in \Rationals^{n^2} \]
    Sei \[ \det : \: \Rationals^{n^2} \rightarrow \Rationals \]
    eine Abbildung mit folgenden Eigenschaften:
    \begin{MyDescription}
    \MyItem{D1}
        Entsteht $B$ aus $A$ durch Multiplikation einer Zeile mit
        \[ r \in \Rationals \] so gilt:
        \[ \det(B) = r \det(A) \]
    \MyItem{D2}
        Enth"alt $A$ zwei gleiche Zeilen, so gilt:
        \[ \det(A) = 0 \]
    \MyItem{D3}
        Entsteht $B$ aus $A$ durch Addition des $r$-fachen einer Zeile zu
        einer anderen, so gilt:
        \[ \det(B) = \det(A) \]
    \MyItem{D4} F"ur die Einheitsmatrix gilt:
        \[ \det(E_n) = 1 \]
    \end{MyDescription}
    Dann hei"st
    \[ \det(A) \] {\em Determinante der Matrix $A$}.
\MyEndDef

\MyBeginDef
\label{DefZeilenOp}
\index{Zeilenoperationen!elementare}
\index{Spaltenoperationen!elementare}
    Die auf einer Matrix definierten Operationen
    \begin{itemize}
    \item 
          Vertauschung zweier Zeilen 
    \item 
          Multiplikation einer Zeile mit einem 
          Faktor\footnote{vgl. D1 in \ref{DefDet}}
    \item 
          Addition des Vielfachen einer Zeile zu einer 
          anderen\footnote{vgl. D3 in \ref{DefDet}}
    \end{itemize}
    werden {\em elementare Zeilenoperationen} genannt. Die entsprechenden
    Operationen auf Matrizenspalten werden 
    {\em elementare Spaltenoperationen} genannt.
\MyEndDef

\begin{satz}
\label{SatzDetPermut}
    \begin{equation}
    \label{EquDet}
       g(A) =
       \sum_{f \in \permut_n} 
           \sig(f) a_{1,f(1)} a_{2,f(2)} \ldots a_{n,f(n)}
    \end{equation}
    besitzt die Eigenschaften aus \ref{DefDet}.
\end{satz}
\begin{beweis}
    Da es sich hier um Grundlagen handelt, wird der Beweis weniger
    ausf"uhrlich angegeben:
    \begin{MyDescription}
    \MyItem{D1}
         F"ur jede Permutation kommt im entsprechenden Summanden in
         \equref{EquDet} aus jeder Zeile der Matrix genau ein Element als
         Faktor vor. Falls eine Zeile mit $r$ multipliziert wurde, kann
         man also aus jedem Summanden $r$ ausklammern und erh"alt die
         Behauptung.
    \MyItem{D2}
         Seien Zeile $i$ und Zeile $j$ gleich $(i \neq j)$.
         Berechne die
         Summe in \equref{EquDet} getrennt f"ur die ungeraden und die
         geraden Permutationen.

         Die
         $n$-Permutation $g$ vertausche $i$ mit $j$ und lasse alles andere
         gleich. Aus den Grundlagen der Theorie der
         Halbgruppen und Gruppen ergibt sich, da"s man die ungeraden
         $n$-Permutationen erh"alt, indem man die geraden
         $n$-Permutationen jeweils einzeln mit der Permutation $g$
         zusammen ausf"uhrt.

         Deshalb entsprechen sich die Summanden der beiden Teilsummen
         paarweise und unterscheiden sich nur durch das Vorzeichen. Der
         Gesamtausdruck besitzt also den Wert $0$.
    \MyItem{D3}
         Es werde das $r$-fache von Zeile $i$ zu Zeile $j$ addiert. Dadurch
         enth"alt jeder Summand in \equref{EquDet} genau einen Faktor,
         der seinerseits wieder die Summe zweier Matrizenelemente ist.
         Deshalb kann man die gesamte Summe in zwei Summen aufteilen.
         Die eine entspricht genau der Summe in \equref{EquDet}, die andere
         enth"alt in jedem Summanden zwei gleiche
         Faktoren, sowie den Faktor $r$. 
         Diesen Faktor kann man ausklammern (mit Hilfe
         von \mbox{D1}). Nach \mbox{D2} ist der Wert dieser Summe dann
         gleich Null, und nur die andere bleibt "ubrig.
    \MyItem{D4}
         Au"ser f"ur die identische Abbildung enth"alt jeder Summand
         der entsprechenden Permutation in \equref{EquDet} mindestens zwei
         Nullen als Faktoren und ist deshalb gleich Null. Der Summand, der
         der identischen Abbildung entspricht, hat den Wert $1$.
    \end{MyDescription}
    \nopagebreak
\end{beweis}

% **************************************************************************

\MySection{Der Rang einer Matrix}
\label{SecRang}
\index{Rang}

Zum Verst"andnis des weiteren Textes wird in diesem Kapitel der
Begriff des {\em Rangs} einer Matrix eingef"uhrt. Da dieser Begriff
f"ur die Untersuchung linearer Gleichungssysteme wichtig ist, werden
teilweise auch nichtquadratische Matrizen betrachtet\footnote{ Literatur: 
siehe \ref{SecMatUndDet} } . 

Da in diesem Kapitel 
wiederum Grundlagen aus der Linearen Algebra behandelt werden, ist
die Darstellung auf die f"ur uns wichtigen Aspekte beschr"ankt.

% $$$ Kenntnis der Begriffe 'linear unabh"angig' und 'Linearkombination'
%     wird hier vorausgesetzt
\MyBeginDef
    Die maximale Anzahl linear unab"angiger Spaltenvektoren einer Matrix
    wird mit {\em Spaltenrang} \index{Spaltenrang} bezeichet.
    
    Die maximale Anzahl linear unabh"angiger Zeilenvektoren einer Matrix
    wird mit {\em Zeilenrang} \index{Zeilenrang} bezeichnet.
\MyEndDef

Die folgenden Betrachtungen gelten f"ur den Zeilenrang analog.

Die $m \times n$-Matrix $A$ habe den Spaltenrang $r$. Es gilt also
\[ 0 \leq r \leq n \MyPunkt \] Die Spaltenvektoren von $A$ werden mit
\[ a_1, \, \ldots , \, a_n \] bezeichnet. Seien die Spaltenvektoren
\[a_{i_1}, \, \ldots, \, a_{i_r}\] linear unabh"angig. Das bedeutet, aus
\begin{eqnarray}
  & & d_1 a_{i_1} + \cdots + d_r a_{i_r} \nonumber \\
  & = & \left[
            \begin{array}{c}
                d_1 a_{1,{i_1}} \\ \vdots \\ d_1 a_{n,{i_1}}
            \end{array}
        \right]
        + \cdots +
        \left[
            \begin{array}{c}
                d_r a_{1,{i_r}} \\ \vdots \\ d_r a_{n,{i_r}}
            \end{array}
        \right]
        \nonumber \\
  & = & \left[
             \begin{array}{c}
                 d_1 a_{1,{i_1}} + \cdots + d_r a_{1,{i_r}} \\
                 \vdots \\
                 d_1 a_{n,{i_1}} + \cdots + d_r a_{n,{i_r}}
             \end{array}
        \right] \label{EquRangZeilentausch} \\
  & = & 0_n \nonumber
\end{eqnarray}
folgt
\[ d_1 = \ldots = d_r = 0 \MyPunkt \]
Vertauscht man zwei Elemente des Vektors \equref{EquRangZeilentausch},
bleibt die G"ultigkeit dieser Bedingung davon unber"uhrt.
Daraus folgt, da"s die Vertauschung zweier Matrixzeilen den Spaltenrang
der Matrix unber"uhrt l"a"st. 

Ebenso verh"alt es sich mit der Multiplikation einer Zeile der Matrix
mit einem Faktor $c \neq 0$.
Analog zur Argumentation bei der Vertauschung zweier Zeilen erh"alt man
\[
        \left[
             \begin{array}{c}
                 c (d_1 a_{1,{i_1}} + \cdots + d_r a_{1,{i_r}}) \\
                 \vdots \\
                 (d_1 a_{n,{i_1}} + \cdots + d_r a_{n,{i_r}})
             \end{array}
        \right]
\]
\MyPunktA{35em}
Die Bedingung $d_1 = \ldots = d_r = 0$ bleibt durch den Faktor unber"uhrt.

Das gleiche Ergebnis erh"alt man f"ur die Addition des Vielfachen einer
Zeile zu einer anderen.

Die elementaren Zeilenoperationen\footnote{siehe \ref{DefZeilenOp}}
lassen den Spaltenrang also unver"andert.
Analog verh"alt es sich mit den elementare Spaltenoperationen und
dem Zeilenrang.

Durch die Zeilen- und Spaltenoperationen l"a"st sich jede Matrix in
Diagonalform "uberf"uhren\footnote{nicht zu verwechseln mit
Diagonalisierbarkeit!}, ohne da"s dadurch der Rang ver"andert wird.

F"ur eine Matrix, bei der nur die Elemente der Hauptdiagonalen von Null
verschieden sein k"onnen, stimmen Zeilen- und Spaltenrang offensichtlich
"uberein. Da die Zeilen und Spaltenoperationen den Rang unver"andert
lassen, erh"alt man:
\begin{korollar}
\label{SatzRang}
    F"ur jede Matrix stimmen Zeilen- und Spaltenrang "uberein.
\end{korollar}
Aus diesem Grund ist die folgende Definition sinnvoll:

\MyBeginDef
\label{DefRang} \index{Rang}
    Die Anzahl linear unabh"angiger Spalten einer Matrix $A$ wird als
    {\em Rang von $A$} bezeichnet, kurz
    \[ \rg(A) \MyPunkt \]
\MyEndDef

Es gilt offensichtlich:
\[ rg(A) \leq \min(m,n) \MyPunkt \]

An dieser Stelle k"onnen wir drei wichtige Begriffe zueinander in
Beziehung setzen:

\begin{satz}
\label{SatzRgDetInv}
    F"ur eine $n \times n$-Matrix $A$ sind folgende Aussagen "aquivalent:
    \begin{itemize}
    \item
         Matrix $A$ ist invertierbar.
    \item
         F"ur die Determinante gilt:
         \[ \det(A) \neq 0 \]
    \item
         F"ur den Rang gilt:
         \[ \rg(A) = n \]
    \end{itemize}
\end{satz}
\begin{beweis}
    Es ist zu beachten, da"s jede Matrix durch elementare Zeilen-
    und Spaltenoperationen in eine Diagonalmatrix "uberf"uhrt werden kann,
    ohne da"s sich die Invertierbarkeitseigenschaft, der Betrag der 
    Determinante oder der Rang dadurch ver"anderen. Man kann also 
    o. B. d. A. davon ausgehen, da"s $A$ Diagonalform besitzt.
    
    Durch diese "Uberlegung wird die G"ultigkeit der Aussage offensichtlich.
\end{beweis}

% **************************************************************************

\MySection{L"osbarkeit linearer Gleichungssysteme}
\label{SecLinEqu}
\index{Gleichungssystem}

Da lineare Gleichungssysteme eine wichtige Rolle beim Verst"andnis
noch folgender Ausf"uhrungen spielen, werden sie in diesem Kapitel
n"aher betrachtet. Die nun folgenden Grundlagen aus der Linearen 
Algebra sind in der in \ref{SecMatUndDet} aufgelisteten Literatur
ausf"uhrlich behandelt. Wir beschr"anken uns hier auf die f"ur
den nachfolgenden Text wichtigen Sachverhalte.

F"ur die weiteren Beschreibungen drei grundlegende Begriffe aus der 
Linearen Algebra von Bedeutung, deren Definitionen deshalb hier
angegeben sind:

Sei $K$ ein K"orper. Eine Menge $V$ zusammen mit zwei Verkn"upfungen
\begin{eqnarray}
    + & : & V \times V \rightarrow V \\
    * & : & K \times V \rightarrow V \MyKomma
\end{eqnarray}
die die folgenden Bedingungen erf"ullt:
\begin{itemize}
\item Die Menge $V$ in Verbindung mit $+$ ist eine Gruppe.
\item F"ur alle $v,w \in V$ und alle $r,s \in K$ gelten die 
      Gleichungen
      \begin{eqnarray*}
          (r+s)v = rv +sv \\
          r(v+w) = rv + rw \\
          (rs)v = r(sv) \\
          1v = v \MyPunkt 
      \end{eqnarray*}
\end{itemize}
wird als $K$-Vektorraum bezeichnet.

\MyBeginDef
\label{DefUnterraum}
    Eine nichtleere Teilmenge $U$ eines $K$-Vektorraumes $V$ wird als 
    {\em Unterraum} \index{Unterraum} von $V$ bezeichnet, falls gilt
    \[
        \forall u,v \in U, \, r \in K : \: 
        \left\{
            \begin{array}{rcl}
                u + v & \in & U \\
                r u & \in & U \MyPunkt 
            \end{array}
        \right.
    \]
\MyEndDef

\MyBeginDef
\label{DefKern}
\index{Kern!einer linearen Abbildung}
    Die Menge aller Vektoren $x$,
    f"ur die bei einer gegebenen $m \times n$-Matrix $A$ gilt
    \Beq{EquKern}
        A x = 0_n
    \Eeq
    wird als {\em Kern der Matrix A} , kurz $\MyKer(A)$,
    bezeichnet\footnote{ In der Literatur wird h"aufig an dieser
    Stelle der Kern einer linearen Abbildung betrachtet. Da die 
    theoretischen Hintergr"unde hier nicht von Interesse sind, ist in
    unser Betrachtung von Matrizen die Rede.} .
\MyEndDef

F"ur unsere Zwecke ben"otigen wir noch zwei weitere Feststellungen:

\begin{bemerkung}
\label{SatzKernUnterraum}
    Betrachtet man \ref{DefUnterraum}, erkennt man,
    da"s alle $x$, die Gleichung \equref{EquKern} er\-f"ul\-len,
    einen Unterraum bilden.
\end{bemerkung}

Der Rang einer Matrix und die Dimension von $\MyKer(A)$ h"angen auf
eine f"ur uns wichtige Weise zusammen:

\begin{lemma}
\label{SatzDimKer}
    Sei $V$ ein $K$-Vektorraum der Dimension $n$ und
    $W$ ein $K$-Vektorraum der Dimension $m$. Sei
    \[ f: V \rightarrow W \] eine lineare Abbildung mit der
    entsprechenden $m \times n$-Matrix $A$. Dann gilt:
    \[ \rg(A) + \MyDim(\MyKer(A)) = \MyDim(V) \MyPunkt \]
\end{lemma}
\begin{beweis}
    Zum Beweis der obigen Gleichung wird die Dimension von 
    $\MyKer(A)$ in Abh"angigkeit vom Rang von $A$ betrachtet.

    Die Matrix $A$ habe den Rang $r$. Die maximale Anzahl linear
    unabh"angiger Spaltenvektoren betr"agt also $r$. O. B. d. A. seien
    die ersten $r$ Spaltenvektoren linear unabh"angig.
    
    Es wird in zwei Schritten gezeigt, da"s in $\MyKer(A)$ die maximale 
    Anzahl der linear unabh"angigen Vektoren, also die Dimension von $A$,
    $n-r$ betr"agt.
    
    \begin{itemize}
    \item Bezeichne $a_i$ den $i$-ten Spaltenvektor von $A$. Der Kern 
          von $A$ ist die Menge
          \begin{eqnarray*}
              & & \{ x \in V \MySetProperty A x = 0_m \} \\
              & = & \{ x \in V \MySetProperty
               a_1 x_1 + a_2 x_2 + \cdots + a_n x_n = 0_m \} \MyPunkt
          \end{eqnarray*}
          Die Dimension dieser Menge ist die maximale Anzahl linear
          unabh"angiger Vektoren $x$, die in ihr enthalten sind. Die
          Elemente eines Vektors $x$ kann man, wie an der obigen
          Darstellung ersichtlich ist, als Faktoren in einer
          Linearkombination von Spaltenvektoren von $A$ betrachten.
          Anhand der Voraussetzungen, die f"ur die Spaltenvektoren
          gelten, lassen sich Aussagen f"ur die Vektoren $x$ machen.
          
          Da $r+1$
          Spaltenvektoren von $A$ immer linear abh"angig sind, ist es
          m"oglich, durch Linearkombination von jeweils $r+1$ 
          Spaltenvektoren den Nullvektor zu erhalten. Sind die "ubrigen 
          $n-(r+1)$ Spaltenvektoren nicht an einer Linearkombination 
          beteiligt, entspricht das einer Null als entsprechendes Element 
          von $x$.
          
          Es gibt $n-r$ M"oglichkeiten,
          zu den ersten $r$ linear unabh"angigen Spaltenvektoren von $A$
          genau einen weiteren
          auszusuchen. Aufgrund der Verteilung der Nullen in den 
          entsprechenden Vektoren $x$ mu"s es mindestens $n-r$ linear 
          unabh"angige Vektoren im Kern von $A$ geben.
    \item 
          Angenommen, die Anzahl der linear unabh"angigen Vektoren 
          in $\MyKer(A)$ ist gr"o"ser als $n-r$. O. B. d. A. seien die 
          ersten $n-r$ dieser Vektoren
          so gew"ahlt, da"s bei jedem dieser Vektoren unter den
          Vektorelementen $x_{r+1}$ bis $x_n$ genau eines ungleich Null ist.
          Aus den vorangegangenen Ausf"uhrungen folgt, da"s solche
          Vektoren existieren m"ussen. Die Vektoren werden mit 
          $v_{r+1}, \, \ldots, \, v_n$ bezeichnet. F"ur den Vektor $v_i$
          sei das $i$-te Vektorelement ungleich Null.
          
          Da die ersten $r$ Spaltenvektoren von $A$ linear unabh"angig sind,
          gibt es f"ur die ersten $r$ Elemente jedes Vektors $v_i$
          keine Wahlm"oglichkeit, sobald das $i$-te Element im
          Rahmen der Randbedingungen dem Wert nach festliegt.

          Sei $w$ ein weiterer Vektor aus $\MyKer(A)$, der zu den Vektoren
          $v$ linear unabh"angig ist. Wird das $j$-te Element eines Vektors
          $v_i$ mit $v_{i,j}$ bezeichnet, und das $k$-te Element von $w$
          mit $w_k$, dann gibt es Faktoren $d_{r+1}, \, \ldots , \, d_n$,
          so da"s
          \[ \forall r+1 \leq k \leq n : \: w_k = d_k * v_{k,k} \MyPunkt \]
          
          Ebenso wie f"ur die Vektoren $v$ liegen damit auch die
          ersten $r$ Elemente des Vektors $w$ fest. Sie k"onnen anhand
          der obigen Beziehung aus den Elementen $r+1$ bis $n$ der
          Vektoren $v$ und $w$ berechnet werden, sofern $w$ "uberhaupt die
          Rahmenbedinungen erf"ullt und in $\MyKer(A)$ ist.

          Damit $w$ dennoch zu den Vektoren $v$ linear unabh"angig ist,
          mu"s es einen $r+1$-ten Spaltenvektor von $A$ geben, der zu den
          ersten $r$ Spaltenvektoren linear unabh"angig ist und der
          bei der Zusammenstellung der Vektoren $v$ mit Hilfe der
          Betrachtung von Linearkombinationen (s. o.) noch nicht benutzt
          wurde. Dies ist jedoch im Widerspruch zu den Voraussetzungen.
    \end{itemize}
    Somit folgt:
    \[ \rg(A) + \dim(\MyKer(A)) = r + (n-r) = n = \dim(V) \MyPunkt \]
\end{beweis}
Mit Hilfe von \ref{SatzKernUnterraum} und \ref{SatzDimKer} k"onnen wir
die f"ur uns wichtigen Eigenschaften linearer Gleichungssysteme
betrachten.

Ein lineares Gleichungssystem besteht aus $n$ Gleichungen mit $m$
Unbekannten:
\begin{eqnarray*}
    a_{1,1} x_1 + a_{1,2} x_2 + \cdots + a_{1,n} x_n & = & b_1 \\
    \vdots & \vdots & \vdots \\
    a_{n,1} x_1 + a_{n,2} x_2 + \cdots + a_{m,n} x_n & = & b_m
\end{eqnarray*}
Die $x_i$ sind die zu bestimmenden Unbekannten. Betrachtet man die
gegebenen Konstanten $a_{i,j}$ und $b_i$ als Elemente einer Matrix
bzw. eines Vektors, kann man das Gleichungssystem auch kompakter
darstellen:
\[ A x = b \MyPunkt \]
Dabei ist $A$ ein $n \times m$-Matrix und $b$ ein Vektor der L"ange $m$.
Ist $b= 0_m$, so bezeichnet man das Gleichungssystem als
homogen \index{Gleichungssystem!homogenes}.
Ist $b \neq 0_m$, so bezeichnet man das Gleichungssystem als
inhomogen \index{Gleichungssystem!inhomogenes}.

Uns interessieren die L"osungsmengen eines solchen Gleichungssystems.

\begin{satz}
\label{SatzLoesungsraum}
    Sei $V$ der zugrundeliegende $K$-Vektorraum der Dimension $n$.
    Die L"osungsmenge $L(A,0_m)$ des homogenen Gleichungssystems
    \[ A x = 0_m \] ist ein Unterraum von $V$ der Dimension
    \[ n - \rg(A) \MyPunkt \]
\end{satz}
\begin{beweis}
    Die L"osungsmenge $L(A,0_m)$ ist der Kern von $A$. Damit folgt
    die Behauptung aus \ref{SatzKernUnterraum} und
    \ref{SatzDimKer}.
\end{beweis}

Ist $\rg(A)=n$, gilt also $L(A,0_m) = \{ 0_n \}$.

Uns interessieren auch die L"osungen inhomogener Gleichungssysteme.
Dazu betrachtet man die erweiterte Matrix $[A,b]$, die aus der Matrix
$A$ und dem Vektor $b$ als $n+1$-te Spalte besteht:

\begin{satz}
\label{SatzRangGleich}
    Die Gleichung
    \Beq{EquInhomogen}
        A x = b \MyPunkt
    \Eeq
    ist genau dann l"osbar, wenn gilt
    \Beq{EquRangGleich}
        \rg(A) = \rg([A,b]) \MyPunkt
    \Eeq
\end{satz}
\begin{beweis}
    Man kann die linke Seite von \equref{EquInhomogen} als 
    Linearkombination von Spaltenvektoren von $A$ betrachten. Die Faktoren 
    der Linearkombination bilden die Elemente des L"osungsvektors $x$.

    Der Vektor $b$ ist genau dann als Linearkombination der
    Spaltenvektoren von $A$ darstellbar, wenn die maximale Anzahl linear
    unabh"angiger Vektoren in $A$ und $[A,b]$ gleich ist. Dies ist
    gleichbedeutend mit der G"ultigkeit von \equref{EquRangGleich}.
\end{beweis}

Dieser Satz f"uhrt zu einer f"ur uns bedeutsamen Charakterisierung
der L"osbarkeit:

\begin{korollar}
\label{SatzGenauEine}
    Ist \equref{EquInhomogen} l"osbar und $\rg(A) = n$,
    also $n \leq m$, dann sind die Spaltenvektoren von $A$
    linear unabh"angig
    und werden alle f"ur die Linearkombination zur Darstellung von $b$
    ben"otigt. Es gibt in diesem Fall genau einen Vektor $x$, der
    \equref{EquInhomogen} l"ost.
\end{korollar}

Ist $\rg(A) = m$, also $m \leq n$, dann ist \equref{EquInhomogen} f"ur
jedes $b$ l"osbar, denn es gilt allgemein
\[ \rg(A) \leq \rg([A,b]) \leq m \MyPunkt \] F"ur $\rg(A) = m$ folgt
daraus mit Hilfe der vorangegangenen Bemerkungen die L"osbarkeit.

% $$$ hier evtl. noch Wegener Satz 7.4 und Hilfssatz 7.7

% **************************************************************************

\MySection{Das charakteristische Polynom}
\label{ChapCharPoly}

Betrachtet \label{PageEigenMotiv}
man $A$ als lineare Abbildung im $n$-dimensionalen Vektorraum,
so lautet eine interessante Fragestellung:
\begin{quote}
    Gibt es einen Vektor $x$ ungleich dem Nullvektor sowie einen Skalar 
    $\lambda$, so da"s gilt
    \Beq{EquEigenMotiv}
        Ax=\lambda x \mbox{\hspace{1em}?}
    \Eeq
\end{quote}
Man kann \equref{EquEigenMotiv} umformen in
\[ (A - \lambda E_n)x = 0 \MyPunkt \] Dies ist eine homogenes lineares
Gleichungssystem mit $n$ Gleichungen in $n$ Unbekannten. 
Aus \ref{SatzLoesungsraum} und \ref{SatzRgDetInv} folgt, da"s es genau
dann eine nichttriviale L"osung besitzt, wenn gilt
\[ \det(A - \lambda E_n) = 0 \MyPunkt \]
Dies motiviert die folgende Definition:

\MyBeginDef
\label{DefCharPoly}
\index{Polynom!charakteristisches}
\index{Gleichung!charakteristische}
    \[ \det(A-\lambda E_n) \] hei"st\footnote{Die Form
    $\det(\lambda E_n - A)$ ist in der Literatur auch gebr"auchlich. Welche
    Form gew"ahlt wird, h"angt von den Vorlieben im Umgang mit den
    Vorzeichen ab. Bis auf die Vorzeichen bleibt die G"ultigkeit von
    Aussagen unber"uhrt. F"ur uns ist die in der Definition angegebene
    Form bequemer.} {\em charakteristisches Polynom der Matrix $A$}. Die
    obige Darstellung als Determinante einer Matrix 
    hei"st \index{Matrizendarstellung!des charakteristischen Polynoms} 
    {\em Matrizendarstellung} des charakteristischen Polynoms.
    \[ \det(A-\lambda E_n) = 0 \] hei"st {\em charakteristische Gleichung
    der Matrix $A$}.
    Die Matrix \[ A - \lambda E_n \] wird als 
    {\em charakteristische Matrix} \index{Matrix!charakteristische}
    bezeichnet.
\MyEndDef

Das charakteristische Polynom einer Matrix l"a"st sich auch in der
\index{Koeffizientendarstellung} {\em Koeffizientendarstellung} 
\Beq{EquKoeffDarst}
    c_n \lambda^n + c_{n-1} \lambda^{n-1} + \cdots + c_1 \lambda + c_0
\Eeq
angeben.

In der Literatur sind als
Vorzeichen von $c_i$ nicht nur $+$, sondern auch $-$ oder $(-1)^i$
gebr"auchlich. Mancherorts herrscht auch die Gewohnheit,
die Indizierung der Koeffizienten in der Form
$c_0 \lambda^n + c_1 \lambda^{n-1} + \cdots + c_{n-1} \lambda + c_n$
durchzuf"uhren. Da diese Vielfalt der Gebr"auche in der Literatur nicht 
nur auf die Koeffizientendarstellung beschr"ankt ist, kann es u. U. sein, 
da"s $c_n$ sowohl das Vorzeichen $+$ als auch immer den Wert $1$ besitzt.
In diesen F"allen wird $c_n$ auch h"aufig weggelassen.
F"ur uns ist die gew"ahlte Form \equref{EquKoeffDarst} der Darstellung
am sinnvollsten.
% $$$ in Verbindung mit meinem anderen Krempel: c_n = (-1)^n

\MyBeginDef 
\label{DefEigenwert}
\index{Eigenwert} \index{Eigenvektor}
    Die Nullstellen des charakteristischen Polynoms einer Matrix hei"sen
    {\em Eigenwerte} der Matrix. Ein Vektor $x$ der
    Gleichung \equref{EquEigenMotiv} zusammen mit einem Eigenwert
    $\lambda$ erf"ullt, hei"st {\em Eigenvektor}. Der Nullvektor ist
    als Eigenvektor nicht zugelassen.
\MyEndDef

\MyBeginDef
    Sei $\lambda_i$ ein Eigenwert der $n \times n$-Matrix $A$. Dann wird
    \[ \rg(A) - \rg(A - \lambda_i E_n) \] als
    {\em Rangabfall} \index{Rangabfall} des Eigenwertes $\lambda_i$ 
    bezeichnet.
\MyEndDef

Die Determinante und die Spur findet man unter den Koeffizienten des
charakteristischen Polynoms wieder. Besonders f"ur die Berechnung der
Determinante ist dies eine wichtige Beobachtung:
\begin{bemerkung}
\label{SatzDdurchP}
    \begin{eqnarray}
       \det(A) & = & c_0               \nonumber
    \\ \tr(A) & = & (-1)^{n-1} c_{n-1} \label{EquTrCoefficient}
    \end{eqnarray}
\end{bemerkung}
Wenn $p(\lambda)$ das charakteristische Polynom der Matrix $A$ ist, gilt
also: \[ \det(A)= p(0) \MyPunkt \]

Rechnet man die Matrizendarstellung des charakteristischen Polynoms um in
die Koeffizientendarstellung, wird dabei die G"ultigkeit von
\equref{EquTrCoefficient} deutlich. Der Wert von $c_{n-1}$ wird nur durch
die Matrixelemente in der Hauptdiagonalen beeinflu"st. Das Produkt dieser
Elemente besteht aus $n$ Linearfaktoren der Form
\[ a_{i,i} - \lambda \MyPunkt \] Berechnet man den Wert dieses Produktes,
so wird $c_n$ durch alle Summanden der Form
\[ (-1)^{n-1} a_{i,i}\lambda^{n-1} \] beeinflu"st. Insgesamt gilt also
\[ c_{n-1} = (-1)^{n-1} \sum_{i=1}^n a_{i,i} \MyKomma \] so da"s man mit
\equref{EquTrCoefficient} die Spur erh"alt.

Ein Algorithmus zur Berechnung der Koeffizienten des charakteristischen
Polynoms einer Matrix berechnet somit u. a. auch deren Determinante und
deren Spur.



%
% Datei: csanky.tex (Textteile nach 'Csan74' und 'Csan76')
%

\MyChapter{Die Algorithmen von Csanky}
\label{ChapCsanky}

In diesem Kapitel werden zwei von L. Csanky\footnote{ \cite{Csan74} und 
\cite{Csan76} } vorgeschlagene Algorithmen 
behandelt. Der erste davon verwendet eine relativ bekannte Methode zur
Determinantenberechnung\footnote{Entwicklungssatz von Laplace}. 
Die Effizienz dieser Methode ist jedoch nicht befriedigend. Ihre 
Darstellung ist als Einstieg gedacht.

\MySection{Die Stirling'schen Ungleichungen}

Als Vorarbeit f"ur den ersten Algorithmus werden in diesem Unterkapitel 
weitere Grundlagen behandelt.

\begin{satz}[Stirling'sche Ungleichungen]
\label{SatzStirling}
\index{Stirling'sche Ungleichungen}
    Sei \[ n \in \Nat \] Dann gilt
    \begin{eqnarray}
        \label{Equ1SatzStirling}
        n! & \geq & \sqrt{2\pi n}
                    \left( \frac{n}{\MathE} \right)^n
                    \MathE^{\frac{1}{12n+1}}
    \\  
        \label{Equ2SatzStirling}
        n! & \leq & \sqrt{2\pi n}
                    \left( \frac{n}{\MathE} \right)^n
                    \MathE^{\frac{1}{12n}}
    \end{eqnarray}
\end{satz}
\begin{beweis}
    Die Ungleichungen folgen aus \cite{Mang67} S. 154.
\end{beweis}

Die Ungleichungen werden in den folgenden Lemmata angewendet. Die Richtung
der Ab\-sch"a\-tzung der jeweiligen Werte wird durch deren Verwendung
im sp"ateren Text bestimmt.

\begin{lemma}
\label{SatzStirling2Anwendung}
    \[ \MyChoose{ n }{ \frac{n}{2} }
           \geq
       \frac{2^n}{ \sqrt{\pi \frac{n}{2}} \: \MathE }
    \]
\end{lemma}
\begin{beweis}
    Es soll nach unten abgesch"atzt werden. Deshalb wird der Z"ahler mit
    Hilfe von \equref{Equ1SatzStirling} und der Nenner mit Hilfe von
    \equref{Equ2SatzStirling} abgesch"atzt. Man erh"alt:
    \begin{eqnarray}
       \MyChoose{ n }{ \frac{n}{2} }
           & = &
       \frac{n!}{ \left( \frac{n}{2} \right)^2 } \nonumber
    \\ \label{Equ1LemmaStirling2}
           & \geq &
       \frac{
            \sqrt{2\pi n}
            \left( \frac{n}{\MathE} \right)^n
            \MathE^{\frac{1}{12n+1}}
       }{
            \left(
                \sqrt{\pi n}
                \left( \frac{ \frac{n}{2} }{\MathE} \right)^{\frac{n}{2}}
                \MathE^{\frac{1}{6n}}
            \right)^2
       }
    \\     & \geq &
       \frac{
           \sqrt{2}
       }{
           \sqrt{\pi n} \left( \frac{1}{2} \right)^n
       }
       \, \MathE^{ \frac{ -18n-2 }{ (12n + 1)6n } }
    \\     & \geq & 
       \frac{2^n}{ \sqrt{\pi \frac{n}{2}} \: \MathE 
       } \nonumber
    \end{eqnarray}
\end{beweis}

Auf die gleiche Weise, wie im vorangegangenen Lemma
\ref{SatzStirlingAnwendung} eine Absch"atzung nach unten erfolgt,
erh"alt man eine Absch"atzung von
\[ \log \MyChoose{ n }{ \frac{n}{2} } \] nach oben:

\begin{lemma}
\label{SatzStirlingAnwendung}
    \[ \log{ n \choose \frac{n}{2} }
           \leq
       n - \frac{\log \left( \frac{n}{2} \right) + 1}{2}
    \]
\end{lemma}
\begin{beweis}
    Aus \equref{Equ1SatzStirling} folgt:
    \[ \left( \frac{n}{2} \right) !
           \geq
       \sqrt{\pi n}
       \left( \frac{ \frac{n}{2} }{\MathE} \right)^{\frac{n}{2}}
       \MathE^{\frac{1}{6n+1}}
    \]
    Also gilt:
    \begin{eqnarray}
    \label{EquStirling1Schaetzung}
       \frac{n!}{\left( \frac{n}{2}! \right)^2 }
           & \leq &
       \frac{
            \sqrt{2\pi n}
            \left( \frac{n}{\MathE} \right)^n
            \MathE^{\frac{1}{12n}}
       }{
            \left(
                \sqrt{\pi n}
                \left( \frac{ \frac{n}{2} }{\MathE} \right)^{\frac{n}{2}}
                \MathE^{\frac{1}{6n+1}}
            \right)^2
       }
    \\ \label{EquStirling2Schaetzung}
           & = &
       \frac{
           \sqrt{2}
       }{
           \sqrt{\pi n} \left( \frac{1}{2} \right)^n
       }
       \MathE^{ \frac{ 1-18n }{ 12n(6n+1) } }
    \\ \nonumber
        & \leq & \frac{2^n}{ \sqrt{ \pi \frac{n}{2} }  }
    \end{eqnarray}
    Bildet man nun auf beiden Seiten der Ungleichungskette den
    Logarithmus, erh"alt man mit Hilfe einiger Logarithmengesetze
    \[ \log{ n \choose \frac{n}{2} }
           \leq
       n - \frac{1}{2} \log\left( \frac{n}{2} \right) -
                                                   \frac{1}{2} \log(\pi)
           \leq
       n - \frac{ \log \left( \frac{n}{2} \right) + 1 }{2}
    \]
\end{beweis}

% **************************************************************************

\MySection{Der Entwicklungssatz von Laplace}
\label{SecLaplace}

F"ur den ersten im Kapitel \ref{ChapCsanky} darzustellenden Algorithmus
wird ein bekannter Satz verwendet. Er wird in diesem Unterkapitel zusammen 
mit einer h"aufig benutzten Folgerung dargestellt.

\begin{satz}[Entwicklungssatz von Laplace]
\label{SatzLaplace}
\index{Laplace!Entwicklungssatz von}
    Sei $k$ eine nat"urliche Zahl mit
    \[ 1 \leq k \leq n-1 \] Sei $D_n$ die Determinante der Matrix $A$.

    F"ur die nat"urliche Zahl $i$ gelte \[ 1 \leq i \leq {n \choose k} \]
    Sei $D_k^{(2i-1)}$ die Determinante einer Untermatrix $A_k^{(2i-1)}$ von
    $A$, die aus $k$ Spalten der ersten $k$ Zeilen gebildet werde.
    von
    $A$, die aus den f"ur $A_k^{(2i-1)}$ nicht gew"ahlten $n-k$ Zeilen und
    Spalten gebildet werde.

    F"ur jedes $i$ werde eine andere der \[ {n \choose k} \] m"oglichen
    Auswahlen f"ur die $k$ Spalten f"ur $A^{(2i-1)}$ getroffen.

    F"ur eine Untermatrix $A_k^{(2i-1)}$ bezeichne \[ f(A_k^{(2i-1)}) \] die
    $n$-Permutation, die die Spalten von $A$ so vertauscht, da"s
    $A_k^{(2i-1)}$ aus den ersten $k$ und $A_{n-k}^{(2i)}$
    aus den weiteren $n-k$ Zeilen und Spalten von $A$ besteht.

    Dann gilt:
    \begin{equation}
    \label{EquSatzLaplace}
    D_n = \sig(f(A_k^1)) D_k^1 D_{n-k}^2 + \sig(f(A_k^3)) D_k^3 D_{n-k}^4
          + \cdots
          + \sig \left( f \left( A_k^{2{n \choose k}-1} \right) \right)
            D_k^{2{n \choose k}-1} D_k^{2{n \choose k}}
    \end{equation}
\end{satz}
\begin{beweis}
    (vergl. \cite{Csan74} Seite 21)
    Es ist zu zeigen, da"s die Berechnung der Determinante nach
    \equref{EquSatzLaplace} mit der Berechnung nach \equref{EquDet}
    "ubereinstimmt.

    Die einzelnen Determinanten in \equref{EquSatzLaplace} werden auf
    die in \equref{EquDet} angegebene Weise berechnet.
    Dazu ist zu beachten, da"s $\permut_n$ genau $n!$
    Elemente besitzt.
    Es gen"ugt zu zeigen, da"s in \equref{EquSatzLaplace}
    ebenso viele Permutationen auf die Indizes von $1$ bis $n$ angewendet
    werden und alle voneinander verschieden sind.

    Berechnet man einen Summanden
    \[ \sig\lb f\lb A_k^{(2i-1)}\rb\rb D_k^{(2i-1)} D_{n-k}^{(2i)} \]
    in \equref{EquSatzLaplace} anhand von \equref{EquDet},
    kann man die zwei Summen durch Ausmultiplizieren zu einer
    zusammenfassen. Den $k!$ Permutationen zur Berechnung von $D_k^{(2i-1)}$
    und den $(n-k)!$ Permutationen zur Berechnung von $D_{n-k}^{(2i)}$
    entsprechen \[ k! \, (n-k)! \] $n$-Permutationen zur Berechnung der
    zusammengefa"sten Summen f"ur diesen einen Summanden. Die Menge dieser
    $n$-Permutationen wird mit $M_i$ bezeichnet.

    Die $k$-Permutationen bzw. $(n-k)$-Permutationen zur Berechnung aller
    Determinanten $D_k$ und $D_{n-k}$ werden jeweils auf
    verschiedene Mengen von Indizes angewendet, d. h. f"ur zwei beliebige
    dieser Mengen von Indizes gibt es in der einen mindestens einen Index,
    der in der anderen nicht enthalten ist. Das bedeutet, da"s alle Mengen
    $M_i$ voneinander verschieden sind.

    Die Anzahl der Mengen $M_i$, die man auf diese Weise f"ur
    \equref{EquSatzLaplace} erh"alt, betr"agt \[ {n \choose k} \]
    Ihre Vereinigung ergibt eine Menge mit 
    \[ {n \choose k} k! \, (n-k)! \]
    also \[ n! \] Elementen, was zu zeigen war.
\end{beweis}

Aus diesem Satz erh"alt man mit $k=1$:

\begin{korollar}[Zeilen- und Spaltenentwicklung]
\label{SatzEntw}
\index{Zeilenentwicklung}
\index{Spaltenentwicklung}
\index{Entwicklung! nach einer Zeile oder Spalte}
     Seien \[ 1 \leq p \leq n \] und \[ 1 \leq q \leq n \] beliebig.
     Dann gilt die {\em Entwicklung nach Zeile $p$}
     \[ \det(A)= \sum_{j=1}^n (-1)^{p+j} a_{p,j} \det(A_{(p|j)}) \]
     und die {\em Entwicklung nach Spalte $q$}
     \[ \det(A)= \sum_{i=1}^n (-1)^{i+q} a_{i,q} \det(A_{(i|q)}) \]
\end{korollar}

% **************************************************************************

\MySection{Determinantenberechnung durch 'Divide and Conquer'}
\label{SecDivCon}
\index{Divide and Conquer}
\index{Csanky!Algorithmus von}
\index{Algorithmus!von Csanky}

Der Algorithmus (\cite{Csan74} S. 21 ff.), der hier betrachtet wird,
berechnet die Determinante
rekursiv mit Hilfe der Methode {\em Divide and Conquer}, d. h. durch
die Berechnung der Determinanten von Untermatrizen der gegebenen Matrix.

Da es sich hierbei nur um ein einleitendes Beispiel handelt, wird zur
Vereinfachung angenommen,
da"s die Anzahl der Zeilen und Spalten $n$ der
Matrix eine Zweierpotenz ist. Falls dies nicht der Fall ist, wird sie um
entsprechend viele Zeilen und Spalten erweitert, so da"s die neuen
Elemente der Hauptdiagonalen jeweils gleich $1$ und alle weiteren
neuen Elemente jeweils gleich $0$ sind.

Zur rekursiven Berechnung der Determinate wird Satz \ref{SatzLaplace}
benutzt. Man w"ahlt
\[ k := \frac{1}{2}n \MyPunkt \]
Somit gilt auch \[ n-k = \frac{1}{2}n \MyPunkt \]

Die Anzahl der Schritte, die der Algorithmus ben"otigt, um mit Hilfe 
dieses Satzes die Determinante einer $n \times n$-Matrix zu berechnen, 
wird mit \[ s(n) \] bezeichnet. Die Anzahl der Prozessoren, die dabei
besch"aftigt werden, wird mit \[ p(n) \] bezeichnet.

Bei der Berechnung wird Gleichung
\equref{EquSatzLaplace} rekursiv ausgewertet. Das f"uhrt dazu, da"s
\[ \MyChoose{n}{ \frac{n}{2} } \] Determinanten von Untermatrizen zu
berechnen sind. Die Berechnung einer Determinanten erfordert
\[ s\left( \frac{n}{2} \right) \] Schritte und
\[ 2 \MyChoose{n}{ \frac{n}{2} } p\left( \frac{n}{2} \right) \] Prozessoren.

Aus \ref{SatzAlgRechnen} folgt, da"s die in
\equref{EquSatzLaplace} auftretenden Additionen in
\[ \log \MyChoose{ n }{ \frac{n}{2} } \] Schritten von
\[ \frac{ \MyChoose{ n }{ \frac{n}{2}} }{2} \] Prozessoren erledigt werden
k"onnen. Die Multiplikationen k"onnen in einem Schritt von ebenfalls
\[ \frac{ \MyChoose{ n }{ \frac{n}{2}} }{2} \] Prozessoren durchgef"uhrt
werden. Also gilt f"ur die Anzahl der Schritte, die der Algorithmus 
ben"otigt, folgende Rekursionsgleichung:
\[
   s(n) = \left\{
              \begin{array}{lcr}
                  0 & : & n = 1
              \\ s\left( \frac{n}{2} \right) + 1 +
                 \log \MyChoose{ n }{ \frac{n}{2} }
                   & : & n > 1
              \end{array}
          \right.
\]
Mit \ref{SatzStirlingAnwendung} folgt
\[ s(n)
       \leq
   s\left( \frac{n}{2} \right) + 1 +
   n - \frac{\log \left( \frac{n}{2} \right) + 1}{2}
\]
Das ist "aquivalent zu
\[
   s(n)
       \leq
   s\left( \frac{n}{2} \right) + 1 +
   n - \frac{\log(n)}{2}
\]
Die Aufl"osung der Rekursion ergibt:
\[
   s(n)
       \leq
   \log(n) + \sum_{j=1}^{\log(n)} \frac{n}{j} -
   \frac{1}{2} \sum_{k=1}^{\log(n)} \log\left( \frac{n}{k} \right)
\]
Man kann nun die folgenden Absch"atzungen vornehmen:
\begin{eqnarray*}
   \sum_{j=1}^{\log(n)} \frac{1}{j}
       & = &
   1 + \frac{1}{2} + \frac{1}{3} + \cdots + \frac{1}{\log(n)}
\\
   & \leq & 1 + \int\limits_1^{\log(n)} \frac{1}{x} \, dx
\\
   & = & [ \ln(x) ]_1^{\log(n)} = 1 + \ln(\log(n))
\\ 
   & \leq & 1 + \log(\log(n))
\end{eqnarray*}
und
\[
   \sum_{k=1}^{\log(n)} \log(k)
       \leq
   \sum_{k=1}^{\log(n)} \log(\log(n))
       =
   \log(n)\log(\log(n))
\]
und kommt so in Verbindung mit einigen Logarithmengesetzen auf
\[
   s(n)
       \leq
   \log(n) + n (1 + \lc \log(\log(n)) \rc ) 
   - \frac{1}{2} \log^2(n) + \log(n) \lc \log(\log(n)) \rc
\] % Probe: s(2) \leq 2.5
Also gilt \[ s(n) = O(n\log(\log(n))) \]

F"ur die Anzahl der besch"aftigten Prozessoren gilt
\[ p(n) = \left\{
              \begin{array}{lcr}
                  0 & : & n = 1
              \\  2 & : & n = 2
              \\  \max\left(
                         2 \MyChoose{ n }{ \frac{n}{2} }
                         p\left( \frac{n}{2} \right)
                      ,  \frac{ \MyChoose{ \scriptstyle n }{ \frac{n}{2} } 
                              }{2}
                      \right)
                    & : & \mbox{sonst}
              \end{array}
          \right.
\]
Bei diesem Beispielalgorithmus interessiert uns nur die Gr"o"senordnung
der Anzahl besch"aftigter Prozessoren. Deshalb wird diese Anzahl grob
nach unten abgesch"atzt durch
\[ p(n) > \MyChoose{ n }{ \frac{n}{2} } \]
Mit \ref{SatzStirling2Anwendung} folgt
\[ p(n) > \frac{ 2^n }{ \sqrt{\pi \frac{n}{2}} \: \MathE } \]
Da nach unten abgesch"atzt wurde, folgt aus dieser Ungleichung
\[ p(n) = \Omega\left( \frac{2^n}{\sqrt{n}} \right) \]
Die Anzahl der Prozessoren ist also von exponentieller Gr"o"senordnung.

Es wird sich bei den noch zu betrachtenden Algorithmen zeigen, da"s 
sowohl f"ur die Anzahl der Schritte als auch f"ur die
Anzahl der besch"aftigten Prozessoren deutlich bessere Werte m"oglich sind.

% **************************************************************************

\MySection{Die Linearfaktorendarstellung}

In diesem Unterkapitel werden einige Aussagen behandelt, die sich in
Verbindung mit einer weiteren Darstellungsm"oglichkeit f"ur das 
charakteristische Polynom ergeben. Diese Aussagen werden f"ur den 
Beweis des Satzes von Frame (siehe \ref{SecFrame}) ben"otigt. Literatur
dazu ist bereits in Kapitel \ref{ChapBase} aufgelistet.

Neben der Matrizendarstellung und der Koeffizientendarstellung gibt es 
noch eine dritte f"ur uns wichtige Darstellung f"ur das charakteristische
Polynom, die 
\index{Linearfaktorendarstellung} {\em Linearfaktorendarstellung}. 
Ein Polynom $n$-ten Grades
kann man auch als Produkt von $n$ Linearfaktoren darstellen, so da"s das
charakteristische Polynom folgendes Aussehen bekommt:
\Beq{TermLinearfaktoren}
    (\lambda_1 - \lambda) (\lambda_2 - \lambda) \ldots
    (\lambda_n - \lambda)
\Eeq
Dabei sind die $\lambda_i$ die Eigenwerte der zugrunde liegenden
$n \times n$-Matrix.
Sie sind {\em nicht} paarweise verschieden.
Diese Tatsache f"uhrt uns zum Begriff der
{\em Vielfachheit}. Man kann in der obigen Linearfaktorendarstellung
gleiche Faktoren mit Hilfe von Potenzen beschreiben. Dazu gelte
\[ k \in \Nat, \: 1 \leq k \leq n \MyPunkt \]
Die Eigenwerte seien
\[ \lambda'_1, \lambda'_2, \ldots, \lambda'_k \MyKomma \]
jedoch alle paarweise voneinander verschieden. So bekommt
\equref{TermLinearfaktoren} folgendes Aussehen:
\[ (\lambda'_1 - \lambda)^{m(\lambda'_1)}
   (\lambda'_2 - \lambda)^{m(\lambda'_2)} \ldots
   (\lambda'_k - \lambda)^{m(\lambda'_k)}
\]
\index{Vielfachheit}
In dieser Darstellung wird $m(\lambda'_i)$ mit {\em Vielfachheit} des
Eigenwertes $\lambda'_i$ bezeichnet.

Die $n$ Eigenwerte einer $n \times n$-Matrix besitzen zur Determinante 
und zur Spur jeweils eine wichtige Beziehung, welche in den n"achsten 
beiden S"atzen dargestellt wird:
\begin{satz}
    \[ \det(A)= \prod_{i=1}^{n} \lambda_i \]
\end{satz}
\begin{beweis}
    Die G"ultigkeit der Behauptung wird bei Betrachtung der Berechnung der
    Koeffizientendarstellung des charakteristischen Polynoms aus dessen
    Linearfaktorendarstellung deutlich. Beim Ausmultiplizieren der 
    Linearfaktoren wird der Wert von $c_0$ nur durch das Produkt der 
    Eigenwerte beeinflu"st.
\end{beweis}

\begin{satz}
\label{SatzTrEigenwerte}
    \[ \tr(A)= \sum_{i=1}^{n} \lambda_i \]
\end{satz}
\begin{beweis}
    Betrachtet man die Berechnung der Koeffizientendarstellung des
    charakteristischen Polynoms aus dessen Linearfaktorendarstellung durch
    Ausmultiplizieren, erkennt man, da"s gilt:
    \[ c_{n-1} = (-1)^{n-1} \sum_{i=1}^n \lambda_i \MyPunkt \]
    Mit \ref{SatzDdurchP} folgt daraus die Behauptung.
\end{beweis}

Die Eigenwerte besitzen weiterhin folgende interessante Eigenschaft:
\begin{satz}
\label{SatzEigenPotenz}
    Seien $\lambda_1, \lambda_2, \ldots, \lambda_n$ die Eigenwerte der
    $n \times n$-Matrix $A$. Sei $k \in \Nat$. Dann gilt:
    \[ \lambda_1^k, \lambda_2^k, \ldots, \lambda_n^k \] sind die
    Eigenwerte von $A^k$.
\end{satz}
\begin{beweis}
    Ein Skalar $\lambda$ ist genau dann Eigenwert von $A$, wenn es einen
    Vektor $x \neq 0_n$ gibt, so da"s gilt\footnote{vgl. 
    Erl"auterungen auf S. \pageref{PageEigenMotiv}}
    \Beq{Equ1EigenPotenz}
        Ax = \lambda x \MyPunkt
    \Eeq
    Mit Hilfe dieser Beziehung wird die Behauptung per Induktion
    bewiesen.

    Nach Voraussetzung ist die Behauptung f"ur $k=1$ erf"ullt. Es gelte
    also nun
    \[ \forall \lambda= \lambda_1, \ldots, \lambda_n \exists x : \:
       A^k x = \lambda^k x \MyPunkt
    \]
    Zu zeigen ist, da"s dann auch gilt
    \[ \forall \lambda= \lambda_1, \ldots, \lambda_n \exists x : \:
       A^{k+1} x = \lambda^{k+1} x
    \]
    Diese Gleichung kann man umformen in
    \[ \forall \lambda= \lambda_1, \ldots, \lambda_n \exists x : \:
       A^k \underbrace{A x}_{(*)}= \lambda^k \underbrace{\lambda x}_{(*)}
    \]
    Nach Voraussetzung sind die Terme $(*)$ gleich. Sie werden mit $y$ 
    bezeichnet. Die Gleichung bekommt dann folgendes Aussehen:
    \[ \forall \lambda= \lambda_1, \ldots, \lambda_n \exists y: \:
       A^k y= \lambda^k y
    \]
    Dies ist wiederum nach Voraussetzung richtig.
\end{beweis}

Aus diesem Satz ergibt sich mit Hilfe von \ref{SatzTrEigenwerte}
eine f"ur uns wichtige Beziehung:
\begin{korollar}
\label{SatzTraceLambda}
    \[ \tr(A^k) = \sum_{i=1}^n \lambda_i^k \]
\end{korollar}

% **************************************************************************

\MySection{Die Newton'schen Gleichungen f"ur Potenzsummen}
\label{SecNewtonPotenz}

Mit den {\em Newton'schen Gleichungen f"ur Potenzsummen} werden in diesem
Unterkapitel weitere Grundlagen f"ur den Beweis der S"atze von Frame
(siehe \ref{SecFrame}) behandelt. Die gesamten Hintergr"unde f"ur diese
Gleichungen werden z. B. in \cite{Haup52} (Kapitel 7 und 8) behandelt.

Eine {\em Potenzsumme} ist eine Summe von Potenzen einer oder mehrerer
Unbestimmter. Auf Seite \ref{SatzSumK} sind weitere Beispiele f"ur 
einfachere Potenzsummengleichungen zu finden.

Da uns diese Gleichungen im Zusammenhang mit dem charakteristischen
Polynom einer Matrix interessieren, werden sie anhand dieses Polynoms
entwickelt. Dazu werden folgende Vereinbarungen getroffen:
\begin{itemize}
\item
      Das charakteristische Polynom der $n \times n$-Matrix $A$ sei
      \[
         p(\lambda) := c_n \lambda^n + c_{n-1} \lambda^{n-1} + \cdots
                       + c_1 \lambda + c_0 \MyPunkt 
      \]
\item
      Die erste Ableitung von $p(\lambda)$ wird mit \[ p'(\lambda) \]
      bezeichnet.
\item Die Eigenwerte von $A$ seien
      \[ \lambda_1, \lambda_2, \ldots, \lambda_n \MyPunkt \]
      Es wird definiert
      \[ s_k := \sum_{i=1}^n \lambda_i^k \MyPunkt \]
\end{itemize}

Zun"achst besch"aftigen wir uns mit der 
Polynomdivision. \index{Polynomdivsion}
Sei dazu ein $i$
mit $ 1 \leq i \leq n $ gegeben. Da $\lambda_i$ Eigenwert von $A$ und somit
Nullstelle von $p(\lambda)$ ist, l"a"st sich $p(\lambda)$ ohne Rest durch
\[ \lambda_i - \lambda \] teilen. Das Ergebnis ist ein Polynom vom Grad
$n-1$. Das folgende Lemma liefert eine Aussage "uber dessen
Koeffizienten:

\begin{lemma}
\label{SatzPolynomDiv}
    Gegeben sei die Gleichung:
    \Beq{Equ1SatzPolynomDiv}
        \frac{ p(\lambda) }{ (\lambda_i - \lambda) }
            =
        \hat{c}_{n-1}\lambda^{n-1} +
        \hat{c}_{n-2}\lambda^{n-2} + \ldots +
        \hat{c}_1 \lambda + \hat{c}_0
    \Eeq
    Dann gilt f"ur $1 \leq k \leq n-1$:
    \Beq{Equ2SatzPolynomDiv}
        \hat{c}_k =
        - \sum_{j=1}^{n-k} \lambda_i^{j-1} c_{k+j}
%        = - ( \lambda_i^{n-k-1} c_n + \lambda_i^{n-k-2} c_{n-2} +
%              \cdots + \lambda c_{k+2} + c_{k+1} )
    \Eeq
\end{lemma}
\begin{beweis}
     Multipliziert man beide Seiten von Gleichung
     \equref{Equ1SatzPolynomDiv} mit \[ (\lambda_i - \lambda) \] und
     multipliziert die rechte Seite der so gewonnenen Gleichung aus,
     ergibt sich:
     \begin{eqnarray*}
         \lefteqn{c_n \lambda^n + c_{n-1} \lambda^{n-1} + \cdots
                       + c_1 \lambda + c_0 }
     \\ & = &
         - \hat{c}_{n-1} \lambda^n +
         (\lambda_i \hat{c}_{n-1} - \hat{c}_{n-2}) \lambda^{n-1} +
         (\lambda_i \hat{c}_{n-2} - \hat{c}_{n-3}) \lambda^{n-2} + \cdots +
         (\lambda_i \hat{c}_1 - \hat{c}_0) \lambda +
         \lambda_i \hat{c}_0
     \end{eqnarray*}
     Setzt man \[ \hat{c}_n = \hat{c}_{-1} = 0 \MyKomma \]
     erh"alt man durch Koeffizientenvergleich:
     \begin{eqnarray}
        \nonumber
            & \forall 1 \leq l \leq n :
            c_k = \lambda_i \hat{c}_k - \hat{c}_{k-1}
     \\ \label{Equ3SatzPolynomDiv}
            \Leftrightarrow &
                \forall 1 \leq l \leq n :
                \hat{c}_{l-1} = \lambda_i \hat{c}_l - c_l
     \end{eqnarray}
     Gleichung \equref{Equ2SatzPolynomDiv} wird durch
     Induktion\footnote{Um die Art und Weise, wie die Koeffizienten der
     Polynome indiziert sind, einheitlich zu halten, verl"auft die
     Induktion etwas ungewohnt.} bewiesen.
     F"ur $k = n-1$ folgt die G"ultigkeit von \equref{Equ2SatzPolynomDiv} 
     aus \equref{Equ3SatzPolynomDiv}. 
     
     Gelte \equref{Equ2SatzPolynomDiv} also nun f"ur $k = l$.
     Es ist zu zeigen, da"s die Gleichung dann auch f"ur $k = l-1$ gilt:
     \[
        \hat{c}_{l-1} = - \sum_{j=1}^{n-(l-1)} \lambda_i^{j-1} c_{l-1+j}
     \]
     Zieht man $c_l$ aus der Summe heraus und indiziert neu, erh"alt man:
     \[
        \hat{c}_{l-1} = - c_l - \sum_{j=1}^{n-l} \lambda_i^{j} c_{l+j}
     \]
     Zieht man nun noch $\lambda_i$ aus der Summe heraus, erh"alt man:
     \[
        \hat{c}_{l-1} = - c_l - \lambda_i 
                                   \sum_{j=1}^{n-l} \lambda_i^{j-1} c_{l+j}
     \]
     Nach Induktionsvoraussetzung ist dies gleichbedeutend mit:
     \[
        \hat{c}_{l-1} = - c_l + \lambda_i \hat{c}_l
     \]
     Die G"ultigkeit dieser Gleichung folgt aus \equref{Equ3SatzPolynomDiv}.
\end{beweis}

Eine an dieser Stelle wichtige Regel f"ur das Differenzieren 
lautet (\cite{Haup52} S. 160):

\begin{bemerkung}
\label{SatzProdDiff}
\index{Differenzierung!von Produkten}
    Seien \[ g_1(x), g_2(x), \ldots, g_n(x) \]
    auf einem Intervall $I$ differenzierbare Funktionen. Es gelte 
    \[ f(x) = \prod_{i=1}^n g_i(x) \MyPunkt \]
    Dann ist auch $f(x)$ auf $I$ differenzierbar mit
    \[ f'(x) = \sum_{i=1}^n g_1(x) g_2(x) \cdots 
                            g_{i-1}(x) g_i'(x) g_{i+1}(x) \cdots
                            g_{n-1}(x) g_n(x) \MyPunkt
    \]
\end{bemerkung}

Falls gilt \[ \forall x \in I, 1 \leq i \leq n: \: g_i(x) \neq 0 \MyKomma \]
l"a"st sich diese Regel auch einfacher formulieren:
\[ f'(x) = \sum_{i=1}^n \frac{f(x)}{g_i(x)} g_i'(x) \]

Betrachtet man das charakteristische Polynom $p(\lambda)$ in 
Linearfaktorendarstellung und beachtet, da"s die Ableitung von
\[ (\lambda_i - \lambda) \] $-1$ ergibt, dann erh"alt man mit Hilfe von 
\ref{SatzProdDiff}:

\begin{korollar}
\label{SatzCharPolyDiff}
    \Beq{EquCharPolyDiff} 
        p'(\lambda)= - \sum_{i=1}^n 
                               \frac{ p(\lambda) }{ (\lambda_i - \lambda) }
    \Eeq
\end{korollar}

Mit Hilfe von \ref{SatzPolynomDiv} und \ref{SatzCharPolyDiff} erh"alt man
nun die gesuchten Gleichungen (\cite{Haup52} S. 181):

\begin{satz}[Newton'sche Gleichungen f"ur Potenzsummen]
\label{SatzNewtonPotenz}
\index{Newton!Gleichungen von}
    \hfill \mbox{\hspace{1cm}} \\ 
    F"ur die Koeffizienten des charakteristischen Polynoms 
    gilt\footnote{Die Terme $s_i$ sind am Beginn des Unterkapitels 
    definiert.}:
    \[
       \forall 0 \leq k \leq n-1 : \:
       - (n - (k+1)
         ) c_{k+1}
       - \sum_{j=2}^{n-k} s_{j-1} c_{k+j}
            = 0
    \]
\end{satz}
\begin{beweis}
    Die $\hat{c}_k$ aus Lemma \ref{SatzPolynomDiv} sind abh"angig vom
    gew"ahlten $\lambda_i$. Deshalb definieren wir diese Koeffizienten
    als Funktionen von $\lambda_i$:
    \[ \hat{c}_k(\lambda_i) :=
           - \sum_{j=1}^{n-k} \lambda_i^{j-1} c_{k+j} \MyPunkt
    \]

    Dann folgt aus Lemma \ref{SatzPolynomDiv}:
    \Beq{Equ4SatzNewtonPotenz}
       \sum_{i=1}^n \hat{c}_k(\lambda_i)
           = \overbrace{- \sum_{j=1}^{n-k} s_{j-1} c_{k+j}
                       }^{ \dot{c}_k := } \MyPunkt
    \Eeq
    "Ubertr"agt man diese Beziehung zwischen den Koeffizienten auf die
    entsprechenden Polynome erh"alt man:
    \Beq{Equ1SatzNewtonPotenz}
       \sum_{i=1}^n \frac{ p(\lambda) }{ \lambda_i - \lambda }
           =
       \sum_{k=0}^{n-1} \dot{c}_k \lambda^k
    \Eeq
    Die erste Ableitung des charakteristischen Polynoms in der
    Koeffizientendarstellung sieht folgenderma"sen aus:
    \Beq{Equ2SatzNewtonPotenz}
       p'(\lambda) = n c_n \lambda^{n-1} + (n-1) c_{n-1} \lambda^{n-2}
                     + \cdots +
                     2 c_2 \lambda + c_1
    \Eeq
    Aus den drei Gleichungen \equref{EquCharPolyDiff},
    \equref{Equ1SatzNewtonPotenz} und \equref{Equ2SatzNewtonPotenz} folgt:
    \[ - \sum_{k=0}^{n-1} \dot{c}_k \lambda^k
           =
       n c_n \lambda^{n-1} + (n-1) c_{n-1} \lambda^{n-2}
           + \cdots +
       2 c_2 \lambda + c_1
    \]
    Durch Koeffizientenvergleich erh"alt man aus dieser Gleichung:
    \[ 
        \forall 0 \leq k \leq n-1 : \: - \dot{c}_k = (k+1) c_{k+1} 
    \]
    Setzt man f"ur $\dot{c}_j$ den entsprechenden Term aus 
    Gleichung \equref{Equ4SatzNewtonPotenz} ein, ergibt sich:
    \begin{MyEqnArray}
       \MT
       \forall 0 \leq k \leq n-1 : \: 
         \sum_{j=1}^{n-k} s_{j-1} c_{k+j}
           \MT = \MT 
       (k+1) c_{k+1}
    \MNl \Leftrightarrow \MT
       \forall 0 \leq k \leq n-1 : \: 
         n c_{k+1} + \sum_{j=2}^{n-k} s_{j-1} c_{k+j}
           & \DS = & \DS
       (k+1) c_{k+1}
    \MNl \Leftrightarrow \MT 
       \forall 0 \leq k \leq n-1 : \: 
         (n-(k+1)) c_{k+1} + \sum_{j=2}^{n-k} s_{j-1} c_{k+j}
           \MT = \MT 0 
    \end{MyEqnArray}
\end{beweis}

% **************************************************************************

\MySection{Die Adjunkte einer Matrix}
\label{SecAdj}

Bei den Beweisen des Satzes von Frame in Unterkapitel \ref{SecFrame} 
spielt die Adjunkte einer Matrix eine bedeutende Rolle und wird
deshalb hier behandelt.

\MyBeginDef
\label{DefAdj}
\index{Adjunkte}
    Sei $A$ eine $n \times n$-Matrix.
    Erh"alt man die Matrix $B$ aus der Matrix $A$ nach
    \[
        b_{i,j} := (-1)^{i+j} \det(A_{(j|i)}) \MyKomma
    \] so hei"st $B$ {\em Adjunkte der Matrix $A$}. Die Adjunkte wird mit
    \[ \adj(A) \] bezeichnet.
\MyEndDef

Zum Beweis einer uns besonders interessierenden Eigenschaft der Adjunkten
ben"otigen wir zun"achst noch ein Lemma. Es
behandelt den Fall einer Zeilen- bzw. Spaltenentwicklung\footnote{ vgl. 
\ref{SatzEntw} }, bei der jedoch
als Faktoren f"ur die Unterdeterminanten die Matrizenelemente nicht aus
der Zeile bzw. Spalte entnommen werden, nach der die Determinante 
entwickelt wird:

\begin{lemma}
\label{SatzFalscheEntw}
    Seien \[ 1 \leq p,p' \leq n \] und \[ 1 \leq q,q' \leq n \] mit
    \[ p \neq p' \] und \[ q \neq q' \] Dann gilt:
    \[ \sum_{j=1}^n (-1)^{p+j} a_{p',j} \det(A_{(p|j)}) = 0 \] und
    \[ \sum_{i=1}^n (-1)^{i+q} a_{i,q'} \det(A_{(i|q)}) = 0 \]
\end{lemma}
\begin{beweis}
    Betrachtet man die Berechnung der Unterdeterminanten in den obigen
    Gleichungen nach \equref{EquDet}, erkennt man beim
    Vergleich mit der Berechnung der Determinante einer Matrix, die
    zwei gleiche Zeilen enth"alt, da"s die Terme beider Berechnungen 
    nach einigen Vereinfachungen "ubereinstimmen. Nach Satz
    \ref{SatzDetPermut} ist die Determinante in diesem Fall gleich $0$.
\end{beweis}

Mit Hilfe von \ref{SatzFalscheEntw} erhalten wir nun:

\begin{satz}
\label{SatzAdj}
    \begin{eqnarray}
        \label{EquSatzAdj1}
        A * \adj(A) = E_n * \det(A) 
     \\ \label{EquSatzAdj2}
        \adj(A) * A = E_n * \det(A)
    \end{eqnarray}
\end{satz}
\begin{beweis}
    Spalte $k$ der Matrix $\adj(A)$ sieht so aus:
    \[ 
        \left[
        \begin{array}{c}
            (-1)^{1+k} \det(A_{(k|1)}) 
         \\ (-1)^{2+k} \det(A_{(k|2)})
         \\ \vdots
         \\ (-1)^{n+k} \det(A_{(k|n)})
        \end{array} 
        \right]
    \]
    Das Element an der Stelle $(i,k)$ der Produktmatrix \[ A * \adj(A) \]
    ist also gleich \[ \sum_{j=1}^n a_{i,j} (-1)^{j+k} A_{(k|j)} \]
    Dies ist nach \ref{SatzEntw} gleich $\det(A)$ f"ur \[ i = k \] und
    nach \ref{SatzFalscheEntw} gleich $0$ f"ur \[ i \neq k \]
    Daraus folgt die G"ultigkeit von \equref{EquSatzAdj1}. 
    Die Argumentation
    f"ur \equref{EquSatzAdj2} verl"auft analog.
\end{beweis}

% **************************************************************************

\MySection{Der Satz von Frame}
\label{SecFrame}

In diesem Unterkapitel wird eine Methode von J. S. Frame (\cite{Fram49};
\cite{Dwye51} S. 225-235) vorgestellt\footnote{ Die
Originalver"offentlichung \cite{Fram49} enth"alt keinen Beweis. Dieser
Beweis ist schwer zu bekommen. Er wird hier deshalb
frei nachvollzogen und d"urfte sich vom Original nicht wesentlich 
unterscheiden. In diesem Zusammenhang m"ochte ich mich bei R. T. Bumby 
f"ur seinen Hinweis auf die Newton'schen Gleichungen f"ur Potenzsummen
bedanken.} , die es u. a. erlaubt, die
Determinante einer Matrix zu berechnen. Diese Methode ist im wesentlichen
eine Neuentdeckung der 
Methode von Leverrier \index{Leverrier!Methode von}
aus dem 19. Jahrhundert zur 
Bestimmung der Koeffizienten des charakteristischen Polynoms 
(z. B. \cite{Hous64} S. 166 ff. ). Die Darstellung wird hier auf die Teile 
beschr"ankt, die f"ur die Determinantenberechnung wichtig sind.

Die Adjunkte von
\Beq{TermFrameAminusLambda} 
    A - \lambda E_n
\Eeq 
besteht aus lauter Determinanten von $(n-1) \times (n-1)$-Matrizen. Sie
kann deshalb durch ein Polynom vom Grad $n-1$ dargestellt werden
(siehe dazu auch \ref{DefCharPoly} und \ref{DefAdj}).
Dies motiviert die folgende Vereinbarung zus"atzlich zu den in
\ref{SecNewtonPotenz} aufgef"uhrten Bezeichnungen:
\begin{quote}
    Seien \[ B_i \: , \: 0 \leq i \leq n-1 \] geeignet gew"ahlte 
    $n \times n$-Matrizen.
    Dann bezeichnet
    \[
       c(\lambda) := B_{n-1} \lambda^{n-1} + B_{n-2} \lambda^{n-2} + 
                     \cdots + B_2 \lambda^2 + B_1 \lambda + B_0 \MyPunkt
    \] die Adjunkte von \equref{TermFrameAminusLambda}.
\end{quote}

\begin{lemma}
\label{SatzFrame1}
    \begin{eqnarray*}
       B_{n-1} & = & (-1) E_n
    \\ \forall n-2 \geq i \geq 0 : \:
       B_{i} & = & A B_{i+1} - c_{i+1} E_n
    \end{eqnarray*}
\end{lemma}
\begin{beweis}
    Aus Satz \ref{SatzAdj} in Verbindung mit Definition \ref{DefCharPoly}
    folgt:
    \[ (A - \lambda E_n) c(\lambda) = p(\lambda) E_n \]
    Setzt man die Koeffizientendarstellungen von $c(\lambda)$ und
    $p(\lambda)$ in diese Gleichung ein, erh"alt man
    \[
    \begin{array}{p{1.8em}p{3.6em}l}
    & \multicolumn{2}{l}{
          \DS (A - \lambda E_n) (B_{n-1} \lambda^{n-1} + 
               B_{n-2} \lambda^{n-2} + \cdots + 
               B_2 \lambda^2 + B_1 \lambda + B_0) \MatStrut
      } 
    \\ & & \DS \MatStrut
        = (c_n \lambda^n + c_{n-1} \lambda^{n-1} + \cdots
                       + c_2 \lambda^2 + c_1 \lambda + c_0 ) E_n
    \\ \mbox{ $\DS \Leftrightarrow$ } & 
       \multicolumn{2}{l}{ \MatStrut
           \DS AB_{n-1} \lambda^{n-1} + AB_{n-2} \lambda^{n-2} +
                       \cdots + AB_2 \lambda^2 + AB_1 \lambda + AB_0
       }
    \\ &
       \multicolumn{2}{l}{ \MatStrut
           \DS - B_{n-1} \lambda^{n} - B_{n-2} \lambda^{n-1} -
                       \cdots - B_2 \lambda^3 - B_1 \lambda^2 - B_0 \lambda
       }
    \\ & & \DS \MatStrut
       = (c_n \lambda^n + c_{n-1} \lambda^{n-1} + \cdots
                   + c_2 \lambda^2 + c_1 \lambda + c_0 ) E_n
    \\ \mbox{ $\DS \Leftrightarrow$ } & 
       \multicolumn{2}{l}{ \MatStrut
           \DS - B_{n-1} \lambda^{n} + (AB_{n-1} - B_{n-2}) \lambda^{n-1} +
               (AB_{n-2} - B_{n-3}) \lambda^{n-2} + \cdots 
       }
    \\ &
       \multicolumn{2}{l}{ \MatStrut
           \DS + (AB_2 - B_1) \lambda^2 + (AB_1 - B_0) \lambda
       }
    \\ & & \DS \MatStrut
       = (c_n \lambda^n + c_{n-1} \lambda^{n-1} + \cdots
                         + c_2 \lambda^2 + c_1 \lambda + c_0 ) E_n
    \end{array}
    \]
    Koeffizientenvergleich ergibt die Behauptung.
\end{beweis}

\begin{lemma}
\label{SatzFrame2}
    Es gilt\footnote{Da in dieser Arbeit verschiedene Arten von hoch und
    tiefgestellten Indizes und Markierungen verwendet werden, sei hiermit
    exiplizit darauf hingewiesen, da"s mit $p'(\lambda)$ ,wie allgemein
    "ublich, die erste Ableitung von $p(\lambda)$ gemeint ist.}
    \[ p'(\lambda) = - \tr(c(\lambda)) \MyPunkt \]
\end{lemma}
\begin{beweis}
    Zu zeigen ist:
    \[ \sum_{j=1}^n j c_j \lambda^{j-1} = 
       - \tr\left( \sum_{j=1}^{n} B_{j-1} \lambda^{j-1} \right)
    \]
    Durch Koeffizientenvergleich erh"alt man:
    \[ 
       \forall 1 \leq j \leq n: \: j c_j = - \tr(B_{j-1}) 
    \]
    Durch wiederholte Anwendung von \ref{SatzFrame1} ergibt sich daraus:
    \begin{eqnarray*}
       j c_j & = & - \tr(A B_{j} - c_{j} E_n)
    \\ \Leftrightarrow
       (n-j) c_j & = & \tr(A B_j)
    \\ \Leftrightarrow
       (n-j) c_j & = & \tr(A (A B_{j+1} - c_{j+1} E_n))
    \\ \Leftrightarrow
       (n-j) c_j & = & \tr(A^{n-j}B_{n-1}) 
                       - \sum_{k=1}^{n-j-1} \tr(A^k) c_{j+k}
    \\ \Leftrightarrow
       (n-j) c_j & = & - \tr(A^{n-j}) 
                       - \sum_{k=1}^{n-j-1} \tr(A^k) c_{j+k}
    \\ \Leftrightarrow
       (n-j) c_j & = & - \tr(A^{n-j}) 
                       - \sum_{k=1}^{n-j-1} \tr(A^k) c_{j+k}
    \end{eqnarray*}
    Nach \ref{SatzTraceLambda} ist dies gleichbedeutend mit
    \[
       (n-j) c_j + s_{n-j}
                   + \sum_{k=1}^{n-j-1} s_k c_{j+k} = 0
    \]
    Da f"ur das charakteristische Polynom $c_n=1$ gilt, ist diese 
    Gleichung nach Satz \ref{SatzNewtonPotenz} richtig.
    \\ \hspace{10em} % damit das Viereck auf der rechten Seite steht
\end{beweis}

\begin{lemma}
\label{SatzFrame3}
    \[ \forall 1 \leq i \leq n: \: c_i = \frac{1}{n-i} \tr(A B_i) \]
\end{lemma}
\begin{beweis}
    Wie in \ref{SatzFrame1} folgt zun"achst aus
    \ref{SatzAdj} in Verbindung mit Definition \ref{DefCharPoly}:
    \begin{MyEqnArray}
       \MT (A - \lambda E_n) c(\lambda) \MT = \MT p(\lambda) E_n 
    \MNl \Leftrightarrow \MT
        A c(\lambda) - \lambda c(\lambda) \MT = \MT p(\lambda) E_n
    \MNl \Leftrightarrow \MT
       \tr(A c(\lambda)) \MT = \MT 
       \tr(\lambda c(\lambda)) + \tr(p(\lambda) E_n) \MatStrut
    \end{MyEqnArray}
    Mit Hilfe von \ref{SatzFrame2} folgt:
    \[
    \begin{array}{p{1.8em}rcl}
       & \multicolumn{3}{l}{ 
             \DS \tr(A c(\lambda)) \mbox{ $\DS =$ }
             \DS n p(\lambda) - \lambda p'(\lambda) 
         }
    \\ \mbox{ $\DS \Leftrightarrow$ } &
       \multicolumn{3}{l}{ \MatStrut
          \DS \tr(AB_{n-1} \lambda^{n-1} + AB_{n-2} \lambda^{n-2} + 
                       \cdots + AB_2 \lambda^2 + AB_1 \lambda + AB_0)
       }
    \\ & \mbox{ $\DS = $ } & \MatStrut
        \DS n ( \lambda^n c_n + c_{n-1} \lambda^{n-1} + \cdots
                       + c_2 \lambda^2 + c_1 \lambda + c_0)
    \\ & & \MatStrut
        \DS - (n \lambda^n c_n + (n-1) c_{n-1} \lambda^{n-1} + \cdots
                       + 2 c_2 \lambda^2 + c_1 \lambda )
    \end{array}
    \]
    Koeffizientenvergleich ergibt
    \[ \forall 1 \leq i \leq n: \: \tr(AB_i) = n c_i - i c_i \MyPunkt \]
\end{beweis}

\begin{satz}[Frame]
\label{SatzFrame}
\index{Frame!Satz von}
    \Beq{Equ3SatzFrame}
        \det(A)= \frac{ \tr(A B_0) }{ n }
    \Eeq
\end{satz}
\begin{beweis}
    Man erh"alt die Behauptung aus \ref{SatzFrame1}, \ref{SatzFrame3} und
    \ref{SatzDdurchP}.
\end{beweis}

% **************************************************************************

\MySection{Determinantenberechnung mit Hilfe des Satzes von Frame}
\label{SecAlgFrame}
\index{Algorithmus!von Csanky}
\index{Csanky!Algorithmus von}

In diesem Unterkapitel wird eine effiziente Methode zur parallelen
Determinantenberechnung \cite{Csan76} vorgestellt (abgek"urzt mit 
{\em C-Alg.};
vgl. Unterkapitel \ref{SecBez}). Sie benutzt Divisionen und kann deshalb
nur angewendet werden, wenn die Berechnungen in einem K"orper stattfinden.
Dies ist problematisch,
weil in realen Rechnern nur mit begrenzter Genauigkeit gearbeitet werden 
kann und somit immer auf die eine oder andere Weise modulo gerechnet wird. 
Z. B. besitzt 6 im Ring $ \Integers_8 $ kein multiplikatives Inverses.

Dies motiviert den Entwurf von Algorithmen, die ohne Divisionen
auskommen, und somit auch in Ringen anwendbar sind, wie BGH-Alg. und
B-Alg. . P-Alg. benutzt wie C-Alg. ebenfalls Divisionen.

Die Determinantenberechnung erfolgt in dem Algorithmus, der in diesem
Unterkapitel vorgestellt wird, mit Hilfe des Satzes von Frame
(Satz \ref{SatzFrame}). Der Satz nutzt die bereits in \ref{SatzDdurchP}
erw"ahnte Tatsache aus, da"s sich unter den Koeffizienten des
charakteristischen Polynoms auch die Determiante befindet. Diese Eigenschaft
des charakteristischen Polynoms wird auch in B-Alg. und P-Alg. in
Verbindung mit anderen Verfahren zur Bestimmung der Koeffizienten
verwendet.

Die Berechnung der Determinante nach Satz \ref{SatzFrame} erfolgt mit Hilfe
einer Rekursionsgleichung. F"ur eine effiziente parallele Berechnung ist
dies nicht befriedigend. Deshalb ist sind einige Umformungen \cite{Csan76}
erforderlich. F"ur diese Umformungen wird ein Operator ben"otigt. Dazu seien
$M$ und $N$ jeweils $n \times n$-Matrizen:

\MyBeginDef
\index{Spuroperator}
\label{DefSpurOp}
    Der Operator $T$ wird definiert durch: \[ {T}{N} := \tr(N) \]
    Er wird {\em Spuroperator} genannt.
\MyEndDef

Es gilt also \[ (E + MT)N = N + {M}{T}{N} = N + M \tr(N) \MyPunkt \]

F"ur die Determinantenberechnung nach Satz \ref{SatzFrame} ist die dort
auftretende Matrix $B_0$ zu berechnen. Mit Hilfe der Lemmata
\ref{SatzFrame1} und \ref{SatzFrame2} sowie des soeben definierten
Operators $T$ erh"alt diese Berechnung folgendes Aussehen:
\begin{eqnarray*}
   B_0 & = & A B_1 - c_1 E_n
\\     & = & A B_1 - \frac{E_n}{n-1} \tr(A B_1)
\\     & = & \left( E_n - \frac{E_n}{n-1} T \right) A B_1
\\     & \vdots & 
\\     & = & \left( E_n - \frac{E_n}{n-1}T \right)
             \left\{A
                 \left[
                     \left(E_n - \frac{E_n}{n-2}T \right)
                     \left\{A
                         \left[
                             \cdots
                             (E_n - {E_n}{T})
                             \{A[E_n]\}
                             \cdots
                         \right]
                     \right\}
                 \right]
             \right\}
\end{eqnarray*}
Mit Hilfe der Assoziativit"at der Matrizenmultiplikation erh"alt man:
\Beq{Equ1SatzCsanky}
    B_0 = \left(
              \underbrace{
                  A - \overbrace{ \frac{E}{n-1} }^{\mbox{Term 1}}
                  \overbrace{ {T}{A} }^{\mbox{Term 2}}
              }_{\mbox{Term 3}}
          \right)
          \left(A - \frac{E}{n-2} {T}{A} \right)
          \cdots
          \left(A - \frac{E}{2} {T}{A} \right)
          (A - {E}{T}{A})
\end{equation}
Da $E_n$ nur in der Hauptdiagonalen von $0$ verschiedene Elemente
besitzt,
l"a"st sich Term 1 in einem Schritt von \[ n \] Prozessoren berechnen.
Parallel dazu l"a"st sich Term 2 nach Satz \ref{SatzAlgRechnen} in
\[ \left\lceil \log(n) \right\rceil \] Schritten von
\[ \left\lfloor \frac{n}{2} \right\rfloor \] Prozessoren berechnen.

Anschlie"send ist die Ergebnismatrix von Term 1 mit dem Ergebnis von
Term 2 zu multiplizieren. Dies kann, wie bei der Berechnung von Term 1
in einem Schritt von \[ n \] Prozessoren durchgef"uhrt werden. Die
darauf folgende Matrizensubtraktion zur Berechnung von Term 3 kann in
einem Schritt von \[ n^2 \] Prozessoren erledigt werden.

Insgesamt kann Term 3 also in
\[ \left\lceil \log(n) \right\rceil + 2 \] Schritten von
\[ n^2 \] Prozessoren berechnet werden.

Zur Berechnung von $B_0$ sind $n$ Terme auf die gleiche Weise wie
Term 3 zu berechnen. Term 2 braucht f"ur all diese Terme nur einmal
berechnet zu werden. Insgesamt kann die Berechnung der $n$ Terme in
\[ \left\lceil \log(n) \right\rceil + 2 \] Schritten von
\[ n^3 \] Prozessoren erledigt werden.

Um das Endergebnis $B_0$ zu erhalten sind schlie"slich noch die
Ergebnismatrizen der $n$ Terme miteinander zu multiplizieren. Die
Anzahl der Schritte und Prozessoren daf"ur folgt aus den S"atzen
\ref{SatzAlgBinaerbaum} und \ref{SatzAlgMatMult}.
Zu beachten ist dabei,
da"s eine Verkn"upfung nicht in einem Schritt von
einem Prozessor durchgef"uhrt wird, sondern nach \ref{SatzAlgMatMult} in
\[ \lceil \log(n) \rceil + 1 \] Schritten von \[ n^3 \] Prozessoren.
Deshalb werden diese Matrizenmultiplikationen in
\[ (\lceil \log(n) \rceil + 1) \lceil \log(n) \rceil \] Schritten von
\[ n^3 \left\lfloor \frac{n}{2} \right\rfloor \] Prozessoren
durchgef"uhrt.

Insgesamt wird die Berechnung von $B_0$ also in
\[ \lceil \log(n) \rceil^2 + 2 \lceil \log(n) \rceil + 2 \] Schritten
von \[ n^3 \left\lfloor \frac{n}{2} \right\rfloor \] Prozessoren
durchgef"uhrt.

Um die Determinante zu berechnen, sind noch nacheinander durchzuf"uhren:
\begin{enumerate}
\item
      eine Matrizenmultiplikation nach Satz \ref{SatzAlgMatMult} in
      \[ \lceil \log(n) \rceil + 1 \] Schritten von \[ n^3 \] 
      Prozessoren,
\item 
      die Berechnung der Spur\footnote{siehe Definition
      \ref{DefTr} } einer Matrix nach Satz \ref{SatzAlgBinaerbaum} in
      \[ \lc \log(n) \rc \] Schritten von
      \[ \lf \frac{n}{2} \rf \]
      Prozessoren und
\item
      eine Division in einem Schritt von einem Prozessor.
\end{enumerate}
Diese drei Berechnungsstufen werden insgesamt in
\[ 2 \lceil \log(n) \rceil + 2 \] Schritten von \[ n^3 \] Prozessoren
durchgef"uhrt.

Die Berechnung der Determinanten mit Hilfe von Satz \ref{SatzFrame}
kann also in
\[ \lc \log(n) \rc^2 + 4 \lc \log(n) \rc + 4 \] Schritten
durchgef"uhrt werden. Die Anzahl der Prozessoren betr"agt
\begin{eqnarray*}
   &      & n^3 \lf \frac{n}{2} \rf
\\ & \leq & \lc \frac{n^4}{2} \rc
\end{eqnarray*}

Man erkennt, da"s C-Alg. keine Fallunterscheidungen verwendet. Dies ist
ein Vorteil bei der Konstruktion von Schaltkreisen, da somit keine
Teilschaltkreise f"ur einzelne Zweige entworfen werden m"ussen. Dadurch
wird ein Schaltkreis zur Determinantenberechnung mit Hilfe von C-Alg.
nicht unn"otig vergr"o"sert. Es wird sich zeigen, da"s die anderen
Algorithmen (BGH-Alg., B-Alg. und P-Alg.) die Eigenschaft fehlender 
Fallunterscheidungen ebenfalls besitzen. In dieser Hinsicht besitzt keiner
der Algorithmen einen Vorteil gegen"uber den anderen.

Vergleicht man den Aufwand an Schritten und Prozessoren mit dem der anderen
Algorithmen, zeigt sich, da"s C-Alg. bereits sehr effizient ist.


%
% Datei: bgh.tex (Textteile nach 'BGH82')
%

\MyChapter{Der Algorithmus von {Borodin,} Von zur Gathen und Hopcroft}
\label{ChapBGH}

Der Algorithmus \cite{BGH82}, der in diesem Kapitel dargestellt wird,
verbindet die Vermeidung von Divisionen \cite{Stra73}, das Gau"s'sche
Eliminationsverfahren (z. B. \cite{BS87} S. 735)
und die parallele Berechnung von Termen \cite{VSBR83} miteinander, um
die Determinante einer Matrix zu berechnen. Auf diesen Algorithmus wird
mit {\em BGH-Alg.} Bezug genommen (vgl. Unterkapitel \ref{SecBez}).

Er unterschiedet sich in
seiner Methodik von den anderen Algorithemen (C-Alg., B-Alg. und P-Alg.)
vor allem dadurch, da"s er die Koeffizienten des charakteristischen
Polynoms in keiner Weise beachtet (vgl. \ref{SatzDdurchP}), sondern die
Determinante durch miteinander verkn"upfte Transformationen, nicht zuletzt
auch durch Ausnutzung von Satz \ref{SatzDetPermut}, direkt
berechnet.

Wie sich in diesem Kapitel zeigen wird, besitzt der Algorithmus durch
die Verbindung der drei o. g. Verfahren eine gewisse Eleganz, besonders,
was die Handhabung der Konvergenz von Potenzreihen angeht. 

Ein Nachteil des Algorithmus ist die vergleichsweise schlechte Effizienz.

% **************************************************************************

\MySection{Das Gau"s'sche Eliminationsverfahren}
\label{SecGauss}

\index{Gau{\Mys}s'sches Eliminationsverfahren}
Das Gau"s'sche Eliminationsverfahren wird dazu benutzt, lineare
Gleichungssysteme der Form \[ Ax=b \] zu l"osen. Dazu wird die sogenannte
{\em erweiterte Koeffizientenmatrix} betrachtet. Sie ist eine
$n \times (n+1)$-Matrix, deren erste $n$ Spalten aus den Spalten der
Koeffizientenmatrix $A$ bestehen und deren $(n+1)$-te Spalte aus dem
Vektor $b$ besteht.

Die Idee des Gau"s'schen Eliminationsverfahrens ist es, die erweiterte
Koeffizientenmatrix so zu transformieren,
da"s die darin enthaltene Matrix $A$ die Form einer {\em oberen
Dreiecksmatrix} \index{Dreiecksmatrix} bekommt. F"ur eine solche
$n \times n$-Matrix gilt:
\[ \forall 1 \leq j < i \leq n: a_{i,j} = 0 \]
Falls f"ur die Matrix
\[ \forall 1 \leq i < j \leq n: a_{i,j} = 0 \]
erf"ullt ist, nennt man sie {\em untere Dreiecksmatrix}.
Zur Transformation werden \index{Zeilenoperationen!elementare}
{\em elementare Zeilenoperationen} verwendet.
Sie werden in Definition \ref{DefDet} der Determinanten einer Matrix
unter D1 und D3 beschrieben. Sie haben nicht nur die dort genannten
Beziehungen zur Determinanten einer Matrix, sondern noch zus"atzlich die
Eigenschaft, da"s sie, angewandt auf die erweiterte Koeffizientenmatrix,
die L"osungsmenge des linearen Gleichungssystems unver"andert lassen.

F"ur die Determinantenberechnung wird die erweiterte Koeffizientenmatrix
nicht weiter beachtet. Alle Operationen beziehen sich nur auf die Matrix
$A$. Die Matrizenelemente unterhalb der
Hauptdiagonalen\footnote{ Die Hauptdiagonale bilden $a_{1,1}$ bis
$a_{n,n}$.} werden spaltenweise durch Nullen ersetzt, beginnend mit der
ersten Spalte. Die Transformationen werden durch folgende Gleichungen
beschrieben\footnote{ Das Gau"s'sche Eliminationsverfahren wird im
weiteren Text so modifiziert, da"s Divisionen durch Null nicht
vorkommen k"onnen. Dieser Fall wird deshalb schon hier au"ser Acht
gelassen.}:
\begin{eqnarray}
    \label{Equ1GaussDef}
    a_{i,j}^{(0)} & := & a_{i,j}
\\  \label{Equ2GaussDef}
    a_{i,j}^{(k)} & := & \left\{
                            \begin{array}{lcr}
                                a_{i,j}^{(k-1)} & : & i \leq k
                            \\  a_{i,j}^{(k-1)}-a_{k,j}^{(k-1)}
                                    \frac{ a_{i,k}^{(k-1)} }{
                                           a_{k,k}^{(k-1)}  }
                                               & : & i > k
                            \end{array}
                        \right.
\end{eqnarray}
Die so gewonnene Matrix $A^{(n)}$ ist die gesuchte obere Dreiecksmatrix.
Betrachtet man Satz \ref{SatzDetPermut}, erkennt man, da"s sich die
Determinante dieser Dreiecksmatrix dadurch berechnen l"a"st, da"s man
die Elemente der Hauptdiagonalen miteinander multipliziert. Da man nur
die in \ref{DefDet} erw"ahnten Operationen verwendet hat, erh"alt man so
auch die Determinante der Matrix $A$.

% **************************************************************************

\MySection{Potenzreihenringe}
\label{SecPotRing}

Im darzustellenden Algorithmus spielen Potenzreihenringe eine wichtige
Rolle. Deshalb werden in diesem Unterkapitel die f"ur uns interessanten
Eigenschaften dieser Ringe behandelt. F"ur unsere Betrachtungen sind Ringe
mit einer zus"atzlichen Eigenschaft von besonderem Interesse:

\MyBeginDef
\label{DefEinheit}
    Sei $R$ ein Ring. Ein $x \in R$ wird als \index{Einheit!in einem Ring}
    {\em Einheit}\footnote{nicht zu verwechseln mit Einselement}
    bezeichnet, wenn es ein $y \in R$ gibt, so da"s
    \[ x * y = 1 \MyPunkt \] Gibt es in $R$ solche Elemente, so wird $R$ 
    als { \em Ring mit Division durch Einheiten } bezeichnet. 
\MyEndDef

Falls in diesem
Kapitel von Ringen die Rede ist, sind immer Ringe mit Division durch
Einheiten gemeint, falls nicht ausdr"ucklich etwas anderes angegeben wird.

Sei $M$ eine Menge von Unbestimmten:
\[ M:= \{ x_1,\,x_2,\, \ldots,\, x_u \} \MyPunkt \]
Dann hei"st $ R[M] $ \index{Ring!{\Myu}ber {\mit M}}
{\em Ring "uber $M$}. F"ur $R[M]$ schreiben wir auch abk"urzend $R[]$.
Die Elemente von $R[]$ sind Terme, in denen neben den
Elementen von $R$ zus"atzlich Elemente von $M$ als Unbestimmte auftreten
d"urfen. 

Analog zur Definition von $R[]$ wird $R[[M]]$ definiert als {\em
Potenzreihenring "uber $M$}. \index{Potenzreihenring!{\Myu}ber {\mit M}}
F"ur $R[[M]]$ wird auch $R[[]]$ geschrieben.
Die Elemente von $R[[]]$
besitzen folgendes Aussehen:
\begin{itemize}
\item
      Sei $T$ eine Teilmenge\footnote{$T$ kann unendlich gro"s sein}
      von $\Nat^{n^2}$.
\item
      F"ur ein $e\in T$ bezeichne $e_i$ das $i$-te Element.
\item
      F"ur $e \in T$ sei \[ k_{e_1,e_2,\ldots,e_{n^2}} \in R \]
\item
      Jedes $u \in R[[]]$ hat f"ur geeignete $k_i$ und eine geeignete Menge
      $T$ die Form:
      \Beq{EquAllgemeinePotenzreihe}
         \sum_{e: \{e_1,e_2,\ldots,e_u \} \in T}
                                                 k_{e_1,e_2,\cdots,e_u}
             \prod_{i=1}^u x_i^{e_i}
      \Eeq
\end{itemize}
Die Summe der Glieder von $u$, f"ur die gilt
   \[ \sum_{i=1}^u e_i = p \]
wird {\em homogene Komponente vom Grad
$p$} \index{homogene Komponente} genannt. Die homogene Komponente
vom Grad $0$ wird auch {\em konstanter Term} \index{konstanter Term}
genannt.

Der Ring $R[]$ enth"alt $R[[]]$ als Unterring und dieser wiederum als
Unterring den Ring der Polynome "uber den Unbestimmten $M$.

\sloppy
Der Potenzreihenring $R[[]]$ besitzt eine f"ur uns besonders interessante
Eigenschaft. Dazu zu\-n"achst der folgende Satz: \fussy
\begin{satz}[Taylor]
\index{Taylor!Satz von}
\label{SatzTaylor}
    Eine Funktion $f$ sei in \[ (x_0-\alpha,x_0+\alpha) \] mit
    \[ \alpha > 0 \] $(n+1)$-mal differenzierbar. Dann gilt f"ur
    \[ x \in (x_0-\alpha,x_0+\alpha) \] die {\em Taylorentwicklung}
    \[
        f(x)= \sum_{\nu = 0}^n \frac{ f^{(\nu)}(x_0) }{ \nu! }
                  (x-x_0)^{\nu} + R_n(x)
    \]
    mit
    \[
        R_n(x):= \frac{ f^{(n+1)}(x_0+ \vartheta(x-x_0)) }{ (n+1)! }
                 (x-x_0)^{n+1}
    \]
    wobei \[ \vartheta \in (0,1) \] und $x_0$ der sogenannte
    {\em Entwicklungspunkt} ist.
\end{satz}
\begin{beweis}
    \cite{Hild74} S. 33-35
\end{beweis}
Weitere Literatur zum Thema 'Taylorreihen' ist z. B. \cite{BS87} (S. 31 und
269). Ein Beispiel f"ur die Anwendung von Satz \ref{SatzTaylor} ist die
Funktion
\Beq{Equ1TylorBeispiel}
    f_1(x) := \frac{1}{1-x} \MyPunkt
\Eeq
Sie ist unendlich oft
differenzierbar mit dem Entwicklungspunkt $x_0=0$ erh"alt man die
Potenzreihe
\Beq{Equ2TaylorBeispiel}
    f_2(x) = \sum_{i=0}^{\infty} x^i \MyPunkt
\Eeq
Der {\em Konvergenzradius} \index{Konvergenzradius} (\cite{BS87} S. 366)
betr"agt $1$, d. h. nur f"ur
\[ |x| < 1 \] gilt \[ f_1(x)=f_2(x) \MyPunkt \]
F"ur den Konvergenzradius $k$ wird das Intervall 
\[ (k,-k) \] als {\em Konvergenzbereich} \index{Konvergenzbereich} 
bezeichnet.

Satz \ref{SatzTaylor} l"a"st sich auch auf mehrere Unbestimmte
verallgemeinern. F"ur uns ist dabei nur folgendes interessant:
\begin{quote}
     Seien \[ f,g \in R[[]] \MyPunkt \]
     Der konstante Term von $g$ sei gleich Null.
     F"ur die Unbestimmten gelte 
     \Beq{Equ2Konvergenz}
         x_1,\ldots, x_u \in (-1,1) \MyPunkt
     \Eeq
     Sei $g$ konvergent.
     Dann folgt aus Satz \ref{SatzTaylor}, da"s sich
     in $R[[]]$ Divisionen der Form
     \Beq{Equ1ZuErsetzen}
         \frac{f}{1-g}
     \Eeq
     ersetzen lassen durch
     \Beq{Equ1StattDivision}
        f*(
              \underbrace{1+g+g^2+\ldots}_{ (*) }
          ) \MyPunkt
     \Eeq
\end{quote}
Die Potenzreihe $g$ wird als {\em innere} Reihe bezeichnet.
Die Terme $(*)$
sind ebenfalls Potenzreihen und werden als
{\em "au"sere} Reihen bezeichnet. Setzt man die {\em innere} Reihe in eine
der {\em "au"seren} ein, erh"alt man wiederum eine Potenzreihe. Diese wird
als {\em Gesamtreihe} bezeichnet.

Im obigen Beispiel konvergiert die Gesamtreihe, falls die innere
Reihe konvergiert und ihre Unbestimmten innerhalb des Konvergenzradius
der "au"seren liegen. Da diese Bedingungen erf"ullt sind, folgt die
Konvergenz der Reihe \equref{Equ1StattDivision}. Um die Konvergenz
in praktisch nutzbarem Ma"se sicherzustellen, sollte der Betrag der Werte,
die f"ur die Unbestimmten eingesetzt werden, nicht beliebig nahe bei $1$
liegen.

Konvergenz ist beim Umgang mit Potenzreihen ein wichtiges Thema. 
Besonders beim Ver\-kn"u\-pfen von Potenzreihen mit mehreren 
Unbestimmten, wie
im vorliegenden Fall, sind Konvergenzbetrachtungen u. U. komplex.
Allgemeine Betrachtungen der Konvergenz von Potenzreihen mit mehreren
Unbestimmten f"uhren an dieser Stelle zu weit und sind z. B. in
\begin{itemize}
\item
      \cite{BT70} ab S. 1 sowie ab S. 49 \hspace{2em} und
\item
      \cite{Hoer73} ab S. 34 
\end{itemize}
zu finden.

Bei praktischen Berechnungen k"onnen Potenzreihen nicht beliebig weit
entwickelt werden, da die Rechenleistung beschr"ankt ist. Deshalb mu"s
ein Grad festgelegt werden, bis zu dem die Potenzreihen entwickelt werden.
Dieser Grad ist i. A. besonders von der St"arke der Konvergenz der Reihe
abh"angig, die entwickelt werden soll. Die Festsetzung eines solchen
Grades erfordert eine Analyse des jeweiligen Problems, das mit Hilfe der
Entwicklung in Potenzreihen gel"ost werden soll. So kann eine Potenzreihe
als Endergebnis mehrerer hintereinander durchgef"uhrter Verkn"upfungen von
Potenzreihen u. U. auch dann konvergieren, wenn als Zwischenergebnis
auftretende Reihen divergieren\footnote{In dem Algorithmus zur
Determinantenberechnung, der in diesem Kapitel vorgestellt wird,
tritt diese Besonderheit auf. In \cite{BGH82} wird darauf in keiner Weise
eingegangen, was sich bei der Bearbeitung als st"orend herausgestellt
hat. }

Da f"ur uns an dieser Stelle weitere allgemeine Betrachtungen uninteressant 
sind, erfolgt die Konvergenzanalyse im Zusammenhang mit der Anwendung 
der Potenzreihenentwicklung auf unser Problem der Determinantenberechnung.

% $$$ hier behandeln, wie Potenzreihen verkn"upft werden ?
%     (vgl. \ref{SecAlgBGH}  (-> letztes Unterkapitel) )

% **************************************************************************

\MySection{Das Gau"s'sche Eliminationsverfahren ohne Divisionen}
\label{SecGaussOhneDiv}

Die M"oglichkeiten zur Vermeidung von Divisionen wurden von V. Strassen
\cite{Stra73} allgemein untersucht. In diesem Unterkapitel wird dargestellt,
wie sich Strassens Ergebnisse auf das Gau"s'sche Eliminationsverfahren
anwenden lassen.

Die Hauptidee zur Vermeidung von Divisionen ist es, alle Berechnungen nicht
in einem Ring $R$ mit Division durch Einheiten
durchzuf"uhren, sondern im zugeh"origen Potenzreihenring $R[[]]$, wobei
Matrizenelemente als Unbestimmte auftreten. Um die
Berechnungen in diesen Ring zu "ubertragen, wird das
Kroneckersymbol \index{Kroneckersymbol} definiert als
\[ \delta_{i,j} :=
       \left\{
           \begin{array}{rcl}
               1 & : & i = j \\
               0 & : & i \neq j
           \end{array}
       \right.
\]
Es sei die Determinante der $n \times n$-Matrix $A$ zu berechnen. Ihre
Elemente werden mit Hilfe der Definition
\Beq{EquDefBGHErsetzung}
    a_{i,j}' := \delta_{i,j} - a_{i,j}
\Eeq
ersetzt. Das bedeutet, jedes
Matrizenelement $a_{i,j}$ wird ersetzt durch
\[ \delta_{i,j} - a_{i,j}' \MyPunkt \]
Wendet man nun das Gau"s'sche Eliminationsverfahren an, bekommt jede
Division die Form \equref{Equ1ZuErsetzen} und kann somit ersetzt werden
durch \equref{Equ1StattDivision}, wie durch das Beispiel in
Unterkapitel \ref{SecBeispielOhneDiv} deutlich wird.

Berechnet man mit Hilfe des Eliminationsverfahrens die Determinante von
$A$, wie in \ref{SecGauss} beschrieben ist, und rechnet dabei in 
$R[[]]$ statt in $R[]$, erh"alt man als Endergebnis eine Potenzreihe $d'$
"uber den Unbestimmten $a_{i,j}'$,
die die Determinante von $A$ beschreibt.

In der praktischen Berechnung ersetzt man die $a_{i,j}'$ mit Hilfe von 
\equref{EquDefBGHErsetzung} durch konkrete Werte und wertet die 
Potenzreihe $d'$ aus, um die Determinante als Element von $R$ zu erhalten.

Ein bis hierhin ungel"ostes Problem ist die Sicherstellung der Konvergenz
von $d'$. Dazu ist die Frage zu beantworten:
\begin{quote}
    Wie gro"s ist der Konvergenzradius von $d'$?
\end{quote}

Hierf"ur m"ussen wir zun"achst eine Eigenschaft der Determinante n"aher
betrachten\footnote{Literatur zu diesem Lemma ist die in Kapitel 
\ref{ChapBase} aufgelistete Grundlagenliteratur.}:
\begin{lemma}
\label{SatzDetEindeutig}
    Es gibt genau eine Abbildung, die jeder Matrix ihre Determinante
    zuordnet.
\end{lemma}
\begin{beweis}
    Der Beweis wird anhand der Matrix $A$ gef"uhrt.

    Seien $f$ und $\hat(f)$ zwei voneinander verschiedene Abbildungen, mit
    den in der Definition \ref{DefDet} der Determinante beschriebenen
    Eigenschaften.

    Es werden zwei F"alle unterschieden:
    \begin{itemize}
    \item
          Bei \[ \rg(A) < n \] gilt nach Satz \ref{SatzRgDetInv}
          \[ f(A) = \hat{f}(A) = 0 \MyPunkt \]
    \item
          Sei \Beq{Equ1SatzDetEindeutig}  \rg(A) = n \MyPunkt \Eeq
          Entsteht die Matrix $B$ aus $A$ durch Zeilenumformungen 
          entsprechend D1 in Definition \ref{DefDet}, dann gibt es 
          ein $c \neq 0$, so da"s gilt: 
          \[ f(B) = c * f(A) \MyPunkt \]
          Das gleiche gilt auch f"ur $\hat{f}$:
          \[ \hat(B) = c * \hat(A) \MyPunkt \]
          Wegen \equref{Equ1SatzDetEindeutig} ist es m"oglich, durch
          Zeilenumformungen \[ B = E_n \] zu erreichen. Aus D4 in 
          Definition \ref{DefDet} folgt dann:
          \[ f(A) = \frac{1}{c} f(E_n) = \frac{1}{c} 
                  = \frac{1}{c}\hat{f}(E_n) = \hat{f}(A) \MyPunkt
          \]
    \end{itemize}
    In beiden F"allen gilt also $ f = \hat(f) $ im Widerspruch zur 
    Voraussetzung, da"s $f$ und $\hat(f)$ verschieden sind.
\end{beweis}

Mit der Unterst"utzung durch dieses Lemma gelangt man zu einer
wichtigen Aussage:
\begin{satz}
\label{SatzBGHKonvergenz}
    Bezeichne $d$ die Potenzreihe "uber den Unbestimmten $a_{i,j}$, die
    man aus $d'$ (s. o.) dadurch erh"alt, da"s man alle 
    Unbestimmten $a_{i,j}'$ mit Hilfe von \equref{EquDefBGHErsetzung} 
    ersetzt.

    F"ur $d$ gilt:
    \begin{quote}
         Alle homogenen Komponenten mit einem Grad ungleich $n$ sind
         gleich Null.
    \end{quote}
\end{satz}
\begin{beweis}
    Aus der Richtigkeit der im vorliegenden Kapitel beschriebenen Verfahren
    folgt, da"s $d$ eine Determinante von $A$ entsprechend der 
    Definition \ref{DefDet} beschreibt.

    Bezeichne $f$ die nach Satz \ref{SatzDetPermut} berechnete Determinante
    von $A$ als Summe, deren Summanden jeweils aus einem Produkt von $n$
    Matrizenelementen bestehen.

    Nach Lemma \ref{SatzDetEindeutig} gilt:
    \[
        d = f \MyPunkt
    \]
    Betrachtet man die Termstruktur von $d$ und beachtet, da"s f"ur die 
    Matrizenelemente keine zus"atzlichen Eigenschaften vorausgesetzt werden,
    folgt aus dieser Gleichung die Behauptung.
\end{beweis}
Der Satz wird in Unterkapitel \ref{SecBeispielOhneDiv} an einer
$3 \times 3$-Matrix demonstriert.

Sowohl $d$ als auch $d'$ beschreiben die Determinante von $A$. 
Der Konvergenzradius von beiden Reihen ist also {\em Unendlich}.

Aus Satz \ref{SatzBGHKonvergenz} folgt insbesondere, da"s sich alle 
homogenen Komponenten mit einem Grad gr"o"ser als $n$ gegenseitig aufheben.
Da alle Divisionen durch Additionen und Multiplikationen ersetzt worden 
sind, gehen diese Komponenten nicht in den Wert von Komponenten geringeren
Grades ein. Komponenten mit einem bestimmten Grad beeinflussen
im Verlauf der Rechnungen lediglich die Werte von Komponenten gleichen oder
h"oheren Grades. 

Also ist es unn"otig, die homogenen Komponenten mit einem Grad gr"o"ser
als $n$ "uberhaupt zu berechnen. Dies ist ein wichtiges Ergebnis f"ur die
Analyse der Effizienz des Algorithmus.

% **************************************************************************

\MySection{Beispiel zur Vermeidung von Divisionen}
\label{SecBeispielOhneDiv}

F"ur eine $3 \times 3$-Matrix wird in diesem Unterkapitel
gezeigt, wie die Determinante mit Hilfe des Gau"s'schen
Eliminationsverfahrens ohne Divisionen berechnet
wird\footnote{Das Beispiel wurde mit Hilfe eines Programms zur 
symbolischen Manipulation von Termen berechnet. Wegen der vielen Indizes
ist das Nachrechnen ohne Computer nicht ratsam.}. Wie im
vorangegangenen Unterkapitel \ref{SecGaussOhneDiv} begr"undet ist, werden
bei allen Potenzreihen nur die homogenenen Komponenten bis maximal zum
Grad $3$ betrachtet.

Es ist die Determinante von
\Beq{Equ1BGHBeispiel}
    \left[
        \begin{array}{ccc}
            a_{1,1} & a_{1,2} & a_{1,3} \MatStrut \\
            a_{2,1} & a_{2,2} & a_{2,3} \MatStrut \\
            a_{3,1} & a_{3,2} & a_{3,3} \MatStrut
        \end{array}
    \right]
\Eeq 
zu berechnen.
Die Ersetzung mit Hilfe von Gleichung \equref{EquDefBGHErsetzung} ergibt:
\[
    \left[
        \begin{array}{ccc}
            1 - a_{1,1}' & 0 - a_{1,2}' & 0 - a_{1,3}' \MatStrut \\
            0 - a_{2,1}' & 1 - a_{2,2}' & 0 - a_{2,3}' \MatStrut \\
            0 - a_{3,1}' & 0 - a_{3,2}' & 1 - a_{3,3}' \MatStrut
        \end{array}
    \right]
    \begin{array}{c}
        \MatStrut \\ \MatStrut \\ \MyPunkt \MatStrut
    \end{array}
\]
Nun werden Vielfache der ersten Zeile von
den folgenden Zeilen subtrahiert, und man erh"alt:
\[
    \left[
        \begin{array}{ccc}
            1 - a_{1,1}'
        &   0 - a_{1,2}'
        &   0 - a_{1,3}' \MatStrut
        \\     0
        &   (1 - a_{2,2}') - (0 - a_{1,2}')
            \frac{ (0 - a_{2,1}') }{ (1 - a_{1,1}') }
        &   (0 - a_{2,3}')  - (0 - a_{1,3}')
            \frac{ (0 - a_{2,1}') }{ (1 - a_{1,1}') } \MatStrut
        \\     0
        &   (0 - a_{3,2}') - (0 - a_{1,2}')
            \frac{ (0 - a_{3,1}') }{ (1 - a_{1,1}') }
        &   (1 - a_{3,3}') - (0 - a_{1,3}')
            \frac{ (0 - a_{3,1}') }{ (1 - a_{1,1}') } \MatStrut
        \end{array}
    \right]
    \begin{array}{c}
        \MatStrut \\ \MatStrut \\ \MyPunkt \MatStrut
    \end{array}
\]
Durch Ersetzung der Divisionen und Vereinfachung der Terme erh"alt man:
\[
    \left[
        \begin{array}{ccc}
            1 - a_{1,1}'
        &   0 - a_{1,2}'
        &   0 - a_{1,3}' \MatStrut
        \\     0
        &   \begin{array}{c}
                (1 - a_{2,2}') - a_{1,2}'a_{2,1}' *
            \\  (1 + a_{1,1}' + (a_{1,1}')^2 + (a_{1,1}')^3)
            \end{array}
        &   \begin{array}{c}
                (0 - a_{2,3}')  - a_{1,3}'a_{2,1}' *
            \\  (1 + a_{1,1}' + (a_{1,1}')^2 + (a_{1,1}')^3)
            \end{array} \LMatStrut
        \\     0
        &   \begin{array}{c}
                (0 - a_{3,2}') - a_{1,2}'a_{3,1}' *
            \\  (1 + a_{1,1}' + (a_{1,1}')^2 + (a_{1,1}')^3)
            \end{array}
        &   \begin{array}{c}
                (1 - a_{3,3}') - a_{1,3}'a_{3,1}' *
            \\  (1 + a_{1,1}' + (a_{1,1}')^2 + (a_{1,1}')^3)
            \end{array}
        \end{array}
    \right]
    \begin{array}{c}
        \MatStrut \\ \MatStrut \\ \MyPunkt \MatStrut
    \end{array}
\]
Da nur die homogenen Komponenten bis maximal zum Grad $3$ ber"ucksichtigt
werden, erh"alt man durch weitere Vereinfachung der Terme:
\[
    \left[
        \begin{array}{ccc}
            1 - a_{1,1}'
        &   0 - a_{1,2}'
        &   0 - a_{1,3}' \MatStrut
        \\     0
        &   \begin{array}{c}
                1 - (a_{2,2}' + a_{1,2}'a_{2,1}' +
            \\   a_{1,2}'a_{2,1}'a_{1,1}')
            \end{array}
        &   \begin{array}{c}
                0 - (a_{2,3}'  + a_{1,3}'a_{2,1}' +
            \\  a_{1,3}'a_{2,1}'a_{1,1}')
            \end{array} \LMatStrut
        \\     0
        &   \begin{array}{c}
                0 - (a_{3,2}' + a_{1,2}'a_{3,1}' +
            \\  a_{1,2}'a_{3,1}'a_{1,1}')
            \end{array}
        &   \begin{array}{c}
                1 - (a_{3,3}' + a_{1,3}'a_{3,1}' +
            \\  a_{1,3}'a_{3,1}'a_{1,1}')
            \end{array} \LMatStrut
        \end{array}
    \right]
    \begin{array}{c}
        \MatStrut \\ \LMatStrut \\ \MyPunkt \LMatStrut
    \end{array}
\]
Man erkennt, da"s alle Elemente der Hauptdiagonalen wieder die Form
\[ 1 - g_{i,j} \] und alle anderen Elemente wieder die Form
\[ 0 - g_{i,j} \] besitzen, wobei der konstante Term der $g_{i,j}$ jeweils
gleich $0$ ist. Bei der Fortsetzung des Eliminationsverfahrens k"onnen 
auftretende Divisionen also wiederum auf die gleiche Weise ersetzt werden.

Als n"achstes wird ein Vielfaches der zweiten Zeile von der dritten
subtrahiert. Dazu sei
\begin{eqnarray*}
    a_{2,2}'' & := &
        1 - (a_{2,2}' + a_{1,2}'a_{2,1}' + a_{1,2}'a_{2,1}'a_{1,1}') \\
    a_{2,3}'' & := &   
        0 - (a_{2,3}'  + a_{1,3}'a_{2,1}' + a_{1,3}'a_{2,1}'a_{1,1}') \\
    a_{3,2}'' & := &
        0 - (a_{3,2}' + a_{1,2}'a_{3,1}' + a_{1,2}'a_{3,1}'a_{1,1}') \\
    a_{3,3}'' & := &
        1 - (a_{3,3}' + a_{1,3}'a_{3,1}' + a_{1,3}'a_{3,1}'a_{1,1}') \\
    a_{3,3}''' & := &
        a_{3,3}'' - \frac{ a_{3,2}'' }{ a_{2,2}'' } a_{2,3}'' 
        \MyPunkt
\end{eqnarray*}
Man erh"alt die Matrix:
\[
    \left[
        \begin{array}{ccc}
            1 - a_{1,1}'
        &   0 - a_{1,2}'
        &   0 - a_{1,3}' \MatStrut
        \\     0
        &   a_{2,2}''
        &   a_{2,3}'' \MatStrut
        \\     0
        &      0
        &   a_{3,3}''' \MatStrut
        \end{array}
    \right]
    \begin{array}{c}
        \MatStrut \\ \MatStrut \\ \MyPunkt \MatStrut
    \end{array}
\]
Da nur die homogenen Komponenten bis zum Grad $3$ betrachtet werden sollen,
erh"alt man durch Ersetzung der Division in der beschriebenen Weise und
Vereinfachung der Terme\footnote{Da alle Prokukte aus mehr als $3$ 
Unbestimmten sofort weggelassen werden, d"urfen die Rechenschritte nicht
durch $=$ verbunden werden.} f"ur $a_{3,3}''$:
\begin{eqnarray*} % mit 'form' geprueft (Dateien: bgh2.for bgh2.log)
    & & a_{3,3}'' - \frac{ a_{3,2}'' }{ a_{2,2}'' } a_{2,3}'' \\
    & \rightarrow &
        a_{3,3}'' - a_{3,2}'' \\
    & & * (1 + a_{2,2}' + a_{1,2}'a_{2,1}' + a_{2,2}'^2 + a_{1,2}'a_{2,1}'a_{1,1}' \\
    & & \: \: + 2 a_{2,2}'a_{1,2}'a_{2,1}' + a_{2,2}'^3) \\
    & & * a_{2,3}'' \\
    & \rightarrow &
        1 - a_{1,1}'a_{1,3}'a_{3,1}' - a_{1,2}'a_{2,3}'a_{3,1}'
        - a_{1,3}'a_{2,1}'a_{3,2}' \\
    & & - a_{1,3}'a_{3,1}' - a_{2,2}'a_{2,3}'a_{3,2}'
        - a_{2,3}'a_{3,2}' - a_{3,3}' \MyPunkt
% Form:
%        1 - a[11]*[a13]*[a31] - [a12]*[a23]*[a31] - [a13]*[a21]*[a32]
%          - [a13]*[a31] - [a22]*[a23]*[a32] - [a23]*[a32] - [a33];
\end{eqnarray*}
Um die Determinante zu berechnen, werden die Elemente der Hauptdiagonalen
$a_{1,1}'$, $a_{2,2}''$ und $a_{3,3}'''$ miteinander multipliziert.
Wiederum werden die Komponenten mit zu gro"sem Grad weggelassen. Man
erh"alt:
\begin{eqnarray*}
   & & a_{1,1}' a_{2,2}'' a_{3,3}''' \\
   & \rightarrow &
   1-a_{1,1}'a_{2,2}'a_{3,3}'+ a_{1,1}'a_{2,2}' + a_{1,1}'a_{2,3}'a_{3,2}'
   + a_{1,1}'a_{3,3}' - a_{1,1}' + a_{1,2}'a_{2,1}'a_{3,3}' \\
   & &
   - a_{1,2}'a_{2,1}' - a_{1,2}'a_{2,3}'a_{3,1}' - a_{1,3}'a_{2,1}'a_{3,2}'
   + a_{1,3}'a_{2,2}'a_{3,1}' - a_{1,3}'a_{3,1}' \\
   & &
   + a_{2,2}'a_{3,3}' - a_{2,2}'  - a_{2,3}'a_{3,2}' - a_{3,3}' \MyPunkt
%Form:
%  det =
%     1 - [a11]*[a22]*[a33] + [a11]*[a22] + [a11]*[a23]*[a32] + [a11]*[a33] -
%       [a11] + [a12]*[a21]*[a33] - [a12]*[a21] - [a12]*[a23]*[a31] - [a13]*
%       [a21]*[a32] + [a13]*[a22]*[a31] - [a13]*[a31] + [a22]*[a33] - [a22] -
%       [a23]*[a32] - [a33];
\end{eqnarray*}
Um die gesuchte Determinante zu erhalten, setzt man die mit Hilfe von
Gleichung \equref{EquDefBGHErsetzung} aus den $a_{i,j}$ erhaltenen
Werte f"ur die $a_{i,j}'$ ein.

Zum Beweis, da"s wir richtig gerechnet haben, machen wir nun die
durch \equref{EquDefBGHErsetzung} definierte Substitution im obigen Term
wieder r"uckg"angig. Nach der Vereinfachung des Terms lautet das Ergebnis,
ohne da"s zus"atzlich irgendwelche Teilterme weggelassen worden sind:
\begin{eqnarray*}
  \lefteqn{ a_{1,1}' a_{2,2}'' a_{3,3}''' = } \\
 & & a_{1,1}a_{2,2}a_{3,3} + a_{1,2}a_{2,3}a_{3,1} + a_{1,3}a_{2,1}a_{3,2} \\
 & & - a_{1,1}a_{2,3}a_{3,2} - a_{1,2}a_{2,1}a_{3,3} - a_{1,3}a_{2,2}a_{3,1}
    \MyPunkt
\end{eqnarray*}
Die Richtigkeit dieses Ergebnisses wird beim Vergleich mit Satz
\ref{SatzDetPermut} deutlich.

% **************************************************************************

\MySection{Parallele Berechnung von Termen}
\label{SecVSBR}

Durch die Methode von Strassen zur Vermeidung von Divisionen entstehen
Terme, die es parallel auszuwerten gilt. In diesem Unterkapitel wird
ein Verfahren \cite{VSBR83} beschrieben, welches diese Auswertung
erm"oglicht. Die
Beschreibung des Verfahrens ist auf die Verwendung im Rahmen des
Kapitels angepa"st. Eine ausf"uhrliche Beschreibung ist auch in
\cite{Wald87} ab S. 22 zu finden.

Zun"achst wird die Berechnung von Termen formalisiert.
Dazu wird die Menge \[  \{v_i \MySetProperty 1 \leq i \leq c \} \]
mit $V$ bezeichnet. Die Menge der
Elemente $a_{i,j}$ der $n \times n$-Matrix $A$, die hier als
Unbestimmte auftreten, wird mit $X$ bezeichnet. Sei $R$ der Ring, in dem
alle Rechnungen durchgef"uhrt werden. Es wird definiert
\[ \bar{V} := V \cup X \cup R \MyPunkt \]

\MyBeginDef
\label{DefProgramm}
    Sei $R[]$ der bereits erw"ahnte Ring "uber den Elementen von $X$.
    Sei $c \in \Nat$ gegeben. Seien $+$ und $*$ die
    beiden Ringoperatoren f"ur Addition bzw. Multiplikation.
    Es gelte \[ \circ \in \{+,*\} \MyPunkt \] Weiterhin gelte
    \[ \forall 1\leq i\leq c: \: v_i', v_i'' \in \bar{V}
           \backslash \{ v_i, \, v_{i+1}, \, \ldots, \, v_c \} \MyPunkt
    \]
    Jede Folge der Form
    \[ v_i := v_i' \circ v_i'', \: i = 1, \, \ldots, \, c \]
    hei"st dann { \em Programm "uber R[] } . \index{Programm "uber R[]}
    Ein Element einer solchen Folge wird {\em Anweisung} genannt. Abh"angig
    davon, ob $\circ$ die Addition oder die Multiplikation bezeichnet, wird
    das $v_i$ auch als {\em Additions-} bzw.
    {\em Multiplikationsknoten} bezeichnet. Falls das genaue Aussehen von
    Anweisungen von untergeordnetem Interesse ist, werden diese zur
    Abk"urzung durch ihren Additions- bzw. Multiplikationsknoten
    repr"asentiert.
\MyEndDef

Falls in diesem Unterkapitel im Einzelfall nichts anderes festgelegt wird,
ist mit $v_i'$ bzw. $v_i''$ jeweils der erste bzw. zweite Operand der 
Anweisung $v_i$ gemeint. Dies gilt auch dann, wenn andere Buchstaben
benutzt werden oder keine Indizierung erfolgt.

Jeder Term "uber $R[]$ l"a"st sich durch ein Programm "uber $R[]$
berechnen. Um Aussagen "uber solche Programme machen zu k"onnen, sind
eine Reihe weiterer Vereinbarungen erforderlich, die im folgenden
aufgef"uhrt sind.

Der durch eine Anweisung $v_i$ berechnet Term wird mit $f(v_i)$
bezeichnet.

Sei $x \in \bar{V}$. Seien $x',x'' \in V$.
Dann wird der {\em Grad von $x$} mit $g(x)$
bezeichnet und folgenderma"sen definiert:
\[ g(x) := \left\{
           \begin{array}{rcl}
               0 & : & x \in R \\
               1 & : & x \in X \\
               g(x') + g(x'') & : & (x \in V) \und (x := x' * x'') \\
               \max(g(x'),\, g(x'')) & : &
                                    (x \in V) \und (x := x' + x'')
           \end{array}
           \right.
\]
Der Grad von $x$ stimmt nicht mit dem Grad des Polynoms "uberein,
da"s dem Term $f(x)$ entspricht. Dazu ein Beispiel: \nopagebreak[3]
\[
    \begin{array}{ccc}
        v_1:= y * (-1) & f(v_1) = -y & g(v_1)=1 \\  
        v_2:= y + v_1  & f(v_2) = 0  & g(v_2)=1
    \end{array}
\]

Es wird o. B. d. A. angenommen, da"s f"ur jede Anweisung 
\[ x := x' \circ x'' \] die Bedingung \[ g(x') \geq g(x'') \]
erf"ullt ist.

F"ur alle $a \in \Nat$ wird definiert:
\begin{eqnarray*}
   V_a & := & \{ u \in V \MySetProperty
                 g(u) > a, \, u:= u' * u'', \, g(u') \leq a \}  \\
   V_a'& := & \{ u \in V \MySetProperty
                 g(u) > a, \, u:= u' + u'', \, g(u'') \leq a \}
\end{eqnarray*}

\MyBeginDef
\label{DefTiefe}
    Sei $v\in V$. Sei $v_1, \, \ldots , \, v_k$ die l"angste Folge von 
    Elementen von $\bar{V}$, so da"s gilt
    \begin{eqnarray*}
        & & v_1 = v \\
        \forall 1 \leq i \leq k-1 & : & (v_{i+1} = v_i') \oder 
                                        (v_{i+1} = v_i'') \\
        & & v_k \in F \cup X
        \MyPunkt
    \end{eqnarray*}
    Dann bezeichnet $d(v)=k$ die {\em Tiefe von $v$}.
\MyEndDef

\MyBeginDef
\label{Deffvw}
    Sei $v,w \in \bar{V}$.
    Dann wird $f(v;w) \in R[]$ wie folgt definiert:

    Bezeichnen $v$ und $w$ denselben Knoten, so gilt:
    \[ f(v;w) := 1 \MyPunkt \]
    Falls dies nicht erf"ullt ist und $w \in R \cup X$, dann gilt:
    \[ f(v;w) := 0 \MyPunkt \]
    Falls dies ebenfalls nicht erf"ullt ist und
    \[ w:= w' + w'' \MyKomma \]
    dann gilt \[ f(v;w) := f(v;w') + f(v;w'') \MyPunkt \]
    Falls auch dies nicht erf"ullt ist, bleibt nur noch der Fall "ubrig
    da"s gilt
    \[ w:= w' * w'' \MyPunkt \] Daf"ur wird definiert
    \[ f(v;w) := f(v;w') * f(w'') \MyPunkt \]
\MyEndDef
Durch die Art und Weise, wie $f(v;w)$ definiert ist, ergibt sich eine
besondere Eigenschaft f"ur den Fall, da"s $g(w) < 2g(v)$ erf"ullt ist.
Falls n"amlich in dem Programm, zu dem $v$ und $w$ geh"oren,
der Knoten $v$ durch eine neue 
Unbestimmte $v'$ ersetzt wird, dann ist $f(v;w)$ der Koeffizient von 
$v'$ in $f(w)$. 

In Verbindung mit $f(v;w)$ besitzen die Funktionen $g()$ und $d()$ 
eine Eigenschaft, die weiter unten von Bedeutung ist:
\begin{lemma}
\label{SatzGrad}
    \[ g(v) > g(w) \Rightarrow f(v;w) = 0 \]
\end{lemma}
\begin{beweis}
    Der Beweis erfolgt durch Induktion nach $d(w)$.
    \begin{MyDescription}
    \MyItem{$ d(w) = 0 $ }
        Es gilt:
        \begin{eqnarray*}
            g(w) = 0 & \Rightarrow & w \in R \\
            g(v) > g(w) & \Rightarrow & v \in V \cup X 
        \end{eqnarray*}
        Also ist $f(v;w) = 0$ .
    \MyItem{$ d(w) > 0 $ }
        Das Lemma gelte f"ur alle $ u\in\bar{V}, \, d(u)<d(w)$.
        Aus \ref{DefTiefe} folgt direkt, da"s
        f"ur jede Anweisung \[ w:= w' \circ w'' \] gilt 
        \[ d(w) > d(w'), \: d(w) > d(w'') \MyPunkt \]
        Mit Hilfe von \ref{Deffvw} folgt daraus die G"ultigkeit
        des Lemmas.
    \end{MyDescription}
\end{beweis}

\begin{lemma}
\label{SatzTiefe}
    \[ d(v) > d(w) \Rightarrow f(v;w) = 0 \]
\end{lemma}
\begin{beweis}
    analog zu \ref{SatzGrad}
\end{beweis}

Es lassen sich nun zwei Aussagen formulieren.
Dazu gelte jeweils $v,w \in V$ und $0 < g(v) \leq a < g(w)$.

\begin{lemma}
\label{Satz1VSBR}
    \[
       f(v;w) =
           \sum_{u\in V_a} (f(v;u) * f(u;w)) +
           \sum_{u\in V_a'} (f(v;u'') * f(u;w))
    \]
\end{lemma}
\begin{beweis}
    Der Beweis erfolgt durch Induktion nach $d(w)$. Aufgrund der
    Struktur der zu beweisenden Aussage sind die Beweise von
    Induktionsanfang und Induktionsschlu"s nicht voneinander 
    getrennt.
    
    Wegen der Voraussetzung \[ 0 < a < d(w) \] folgt aus 
    \[ d(w) \leq 1 \Rightarrow w \in R \cup X \MyKomma \]
    da"s $d(w)= 1$ nicht auftreten kann. Sei im folgenden also $d(w)>1$.

    Vier F"alle sind zu unterscheiden:
    \begin{MyDescription}
    \MyItem{ $ w:= w' + w'', \: g(w'') \leq a $ }
        Das Lemma gelte f"ur $w'$.
        Aus der Voraussetzung folgt:
        \[ w \in V'_a \MyPunkt \]
        Aus \ref{SatzTiefe} folgt:
        \[ f(w;w') = 0 \MyPunkt \]
        Au"serdem gilt:
        \[ g(w'') \leq a \Rightarrow
           \forall u \in V'_a: \: f(u;w'') = 0 \MyPunkt
        \]
        Nach \ref{Deffvw} gilt: \[ f(w,w) = 1 \MyPunkt \]
        So ergibt sich:
        \begin{eqnarray*}
            f(v;w') & = &
                \sum_{u \in V_a} (f(v;u) * f(u,w')) \\
              & & + \sum_{u \in V'_a} (f(v;u'') * f(u;w')) \\
            & = &
                \sum_{u \in V_a} (f(v;u) * (f(u,w') + f(u;w''))) \\
              & & + \sum_{u \in V'_a} (f(v;u'') * (f(u;w') + f(u;w''))) \\
            & = &
                \sum_{u \in V_a} (f(v;u) * f(u,w)) \\
              & & + \sum_{u \in V'_a \backslash \{w\} } (f(v;u'') * f(u;w))
        \end{eqnarray*}
        Es folgt mit Hilfe von \ref{Deffvw}:
        \begin{eqnarray*}
            f(v;w) & = & f(v;w') + f(v;w'') \\
            & = & 
                f(v;w') + f(v;w'') * f(w;w) \\
            & = & 
                \sum_{u \in V_a} (f(v;u) * f(u,w)) \\
              & & + \sum_{u \in V'_a \backslash \{w\} }
                     (f(v;u'') * f(u;w)) + f(v;w'') * f(w;w) \\
            & = &
                \sum_{u \in V_a} (f(v;u) * f(u,w)) \\
              & & + \sum_{u \in V'_a} (f(v;u'') * f(u;w))
        \end{eqnarray*}
    \MyItem{ $ w:= w' + w'', \: g(w'') > a $ }
        Das Lemma gelte f"ur $w'$ und $w''$.
        \begin{eqnarray*}
            f(v;w) & = & f(v;w') + f(v;w'') \\
            & = &
                \sum_{u \in V_a} (f(v;u)*f(u;w')) +
                \sum_{u \in \bar{V_a}} (f(v;u'')*f(u;w')) \\
            & & + \sum_{u \in V_a} (f(v;u)*f(u;w'')) +
                  \sum_{u \in \bar{V_a}} f(v;u'')*f(u;w'')) \\
            & = &
                \sum_{u \in V_a} (f(v;u) * (f(u;w') + f(u;w''))) \\
            & & + \sum_{u \in \bar{V_a}} (f(v;u'')*(f(u;w') + f(u;w''))) \\
            & = &
                \sum_{u \in V_a} (f(v;u) * f(u;w)) +
                \sum_{u \in \bar{V_a}} (f(v;u'') * f(u;w))
        \end{eqnarray*}
    \MyItem{ $ w:= w' * w'', \: g(w') \leq a $ }
        Es gilt: 
        \begin{eqnarray*}
            w & \in & V_a \\
            f(v;w) & = & f(v;w) * f(w;w) \MyPunkt
        \end{eqnarray*}
        Andererseits gilt:
        \begin{eqnarray*}
            \forall u \in V_a \backslash \{w\} & : & f(u;w') = 0 \\
            \Rightarrow
            \forall u \in V_a \backslash \{w\} & : &
                f(u;w) = f(w'') * f(u;w') = f(w'') * 0 = 0
        \end{eqnarray*}
        Also folgt:
        \[ \sum_{u \in V_a} (f(v;u) * f(u;w)) = f(v;w) * f(w;w) = f(v;w)
           \MyPunkt
        \]
        Weiterhin folgt aus \ref{SatzGrad}:
        \begin{eqnarray*}
            \lefteqn{ \forall u \in V'_a : \: f(u;w') = 0 } \\
            & \Rightarrow &
               \sum_{u \in V'_a} (f(u'') * f(u;w)) \\
            & & = \sum_{u \in V'_a} (f(u'') * f(w'') * f(u;w')) = 0
        \end{eqnarray*}
        Also ist das Lemma f"ur diesen Fall richtig.    
    \MyItem{ $ w:= w' * w'', \: g(w') > a $ }
         Das Lemma gelte f"ur $w'$.
         \begin{eqnarray*}
            f(v;w) & = & f(w'') * f(v;w') \\
            & = & f(w'') *
                \left(
                    \sum_{u\in V_a} (f(v;u) * f(u;w')) +
                    \sum_{u\in V_a'} (f(v;u'') * f(u;w'))
                \right) \\
            & = &
                \sum_{u\in V_a} (f(v;u) * f(w'') * f(u;w')) +
                \sum_{u\in V_a'} (f(v;u'') * f(w'') * f(u;w')) \\
            & = &
                \sum_{u\in V_a} (f(v;u) * f(u;w)) +
                \sum_{u\in V_a'} (f(v;u'') * f(u;w))
        \end{eqnarray*}
   \end{MyDescription}
\end{beweis}
  
\begin{lemma}
\label{Satz2VSBR}
    \[
       f(w) = 
           \sum_{u\in V_a} (f(u) * f(u;w)) +
           \sum_{u\in V_a'} (f(u'') * f(u;w))
    \]
\end{lemma}
\begin{beweis}
    Bis auf den Unterschied, da"s die auftretenden Terme entsprechend
    unterschiedlich sind, ist der Beweis identisch zum Beweis von
    \ref{Satz1VSBR}.
\end{beweis}

Mit Hilfe von \ref{Satz1VSBR} und \ref{Satz2VSBR} l"a"st sich ein
Verfahren zur parallelen Berechnung von Termen angeben, das im
folgenden beschrieben wird.

Gegeben sei ein Programm der L"ange $c$,
d. h. \[ V = \{v_1, \, v_2, \, \ldots, v_c\} \MyPunkt \]
Es ist $f(v_c)$ zu berechnen. Die Berechnung erfolgt stufenweise.
Seien $v,w \in V$.
In Stufe $0$ werden alle $f(w)$ mit \[ g(w)=1 \]  und alle
$f(v;w)$ mit \[ g(w) - g(v) = 1 \] berechnet.

In Stufe $i$ werden alle $f(w)$ mit
\[ 2^{i-1} < g(w) \leq 2^i \] und alle $f(v;w)$ mit
\[ 2^{i-1} < g(w) - g(v) \leq 2^i \] berechnet. Dabei werden die Ergebnisse
der vorangegangenen Stufen benutzt.

Auf diese Weise ist $f(v_c)$ nach
\[ \lc \log(g(v_c)) \rc \] Stufen berechnet.

In Stufe $i$ werden zun"achst die $f(w)$ mit Hilfe von \ref{Satz2VSBR}
berechnet. Dazu wird $a=2^{i-1}$ gew"ahlt:
\begin{eqnarray}
    f(w) \nonumber
    & = & \nonumber
        \sum_{u\in V_a} (f(u) * f(u;w)) +
        \sum_{u\in V_a'} (f(u'') * f(u;w)) \\
    & = & \label{EquStepIfw}
        \sum_{u\in V_a} (f(u')*f(u'')* f(u;w)) +
        \sum_{u\in V_a'} (f(u'') * f(u;w))
\end{eqnarray}
Anhand der Definitionen erkennt man, da"s f"ur alle auftretenden 
$f(\ldots)$ gilt: \[ g(f(\ldots)) \leq 2^{i-1} \MyPunkt \]
Also wurden alle zu benutzenden Terme bereits in einer der vorangegangenen
Stufen berechnet. 

Man erkennt anhand der bisher angestellten Betrachtungen "uber Programme
zur Berechnung von Termen, da"s der Aufwand f"ur alle Programme der
L"ange $c$ gleich ist. Da eine Aufgabe, die in $a$ Schritten von 
$b$ Prozessoren erledigt wird, auch in $2a$ Schritten von $b/2$ 
Prozessoren erledigt werden kann, erfolgt die Analyse des Aufwandes 
f"ur eine bestimmte Stufe $i$ zun"achst mit Hilfe der durchschnittlich f"ur
eine Stufe zu erwartenden erforderlichen Operationen\footnote{Diese 
Betrachtungsweise kennt man in der Literatur unter dem Begriff 
{\em Rescheduling}.}.

F"ur eine bestimmte Stufe l"a"st sich die Gr"o"se der Mengen $V_a$ und
$V_a'$ nicht genau vorherbestimmen. Falls die Berechnung in $z$ Stufen 
durchgef"uhrt wird, dann gilt jedoch
\begin{eqnarray*}
    & & 0 \leq i,j \leq z \\
    & & i \neq j \\
    & & a_k := 2^{k-1} \\
    & & V_{a_i} \cap V_{a_k} = V'_{a_i} \cap V'_{a_k} = \emptyset \\
    & & \sum_{0 \leq i \leq z} |V_{a_i}| \leq c \\
    & & \sum_{0 \leq i \leq z} |V'_{a_i}| \leq c 
\end{eqnarray*}
Die Mengen $V_a$ und $V'_a$ besitzen also durchschnittlich h"ochstens
\[ \frac{c}{z} = \frac{c}{ \lc \log(g(v_c)) \rc } \]
Elemente. Dieser Wert wird mit $m$ bezeichnet.

Aus den vorangegangenen "Uberlegungen folgt, da"s \equref{EquStepIfw} in
\[ \lc \log(m) \rc + 3 =
   \lc \log \lb \frac{c}{ \lc \log(g(v_c)) \rc } \rb \rc + 3
\]
Schritten von
\[ 2m = 2 \frac{c}{ \lc \log(g(v_c)) \rc } \] 
Prozessoren berechnet werden kann.

Nachdem in Stufe $i$ die $f(w)$ berechnet worden sind, werden die $f(v;w)$
mit Hilfe von \ref{Satz1VSBR} ausgerechnet. Dazu wird $a= g(v) + 2^{i-1}$ 
gew"ahlt:
\begin{eqnarray*}
    f(v;w)
    & = &
        \sum_{u\in V_a} (f(v;u) * f(u;w)) +
        \sum_{u\in V_a'} (f(v;u'') * f(u;w)) \\
    & = &
        \sum_{u\in V_a} (f(u'') * f(v;u') * f(u;w)) +
        \sum_{u\in V_a'} (f(v;u'') * f(u;w)) \\
\end{eqnarray*}
Anhand der Definitionen erkennt man, da"s alle $f(v;u')$, $f(u;w)$ und
$f(v;u'')$ bereits berechnet wurden. F"ur $f(u'')$ gibt es kritische
F"alle, die separat untersucht werden m"ussen:
\begin{MyDescription}
\MyItem{ $g(u') \geq g(u'') > 2^i, \: f(v;u') = 0$ }
    Der Fall ist kein Problem, da der Wert des jeweiligen gesamten Terms
    gleich Null ist.
\MyItem{ $g(u') \geq g(u'') > 2^i, \: f(v;u') \neq 0$ }
    Es mu"s gelten:
    \[ g(u') \geq g(v) \MyPunkt \]
    Daraus ergibt sich:
    \begin{eqnarray*}
        g(u) & = & g(u') + g(u'') \\
             & > & g(v) + 2^i \\
             & \geq & g(w) \\
             & \Rightarrow & f(u;w) = 0
    \end{eqnarray*}
    Der Wert des Terms ist also wiederum gleich Null.
\end{MyDescription}

F"ur die Analyse des Aufwandes gelten die gleichen Bemerkungen wie f"ur
die Berechnung der $f(w)$.

Insgesamt kann $f(v_c)$ in
\Beq{Equ1VSBRAnalyse}
        2\lc \log(g(v_c)) \rc (\lc \log(m) \rc + 3) =
        2\lc \log(g(v_c)) \rc
        \lb \lc \log \lb \frac{c}{ \lc \log(g(v_c)) \rc } 
                     \rb 
             \rc + 3
        \rb
\Eeq
Schritten von
\Beq{Equ2VSBRAnalyse}
   2m = 2 \frac{c}{ \lc \log(g(v_c)) \rc } 
\Eeq
Prozessoren berechnet werden.

F"ur das bis hierhin beschriebene und analysierte Verfahren gibt es einen
Sonderfall, der mit geringerem Aufwand gel"ost werden kann.
\MyBeginDef
\label{DefHomogen}
    Sei $v_1,\, \ldots ,\, v_c$ ein Programm im Sinne von \ref{DefProgramm}.
    Falls f"ur alle Additionsknoten $v_i:= v_i'+ v_i''$ dieses Programms 
    gilt:
    \[ g(v_i') = g(v_i'') \MyKomma \]
    so wird das Programm als {\em homogen} bezeichnet.
\MyEndDef
F"ur homogene Programme sind alle Mengen $V_a'$ leer. Mit dieser 
Feststellung ergeben sich aus \ref{Satz1VSBR} und \ref{Satz2VSBR} zwei
Folgerungen f"ur homogene Programme:
\begin{korollar}
\label{Satz3VSBR}
    \[
       f(v;w) =
           \sum_{u\in V_a} (f(v;u) * f(u;w))
    \]
\end{korollar}

\begin{korollar}
\label{Satz4VSBR}
    \[
       f(w) = 
           \sum_{u\in V_a} (f(u) * f(u;w))
    \]
\end{korollar}

Werden im angegebenen Verfahren zur parallelen Berechnung von Termen 
die Folgerungen
\ref{Satz3VSBR} und \ref{Satz4VSBR} statt der Lemmata \ref{Satz1VSBR} und
\ref{Satz2VSBR} benutzt, so f"uhrt das zu leicht verringertem
Berechnungsaufwand. Dann kann $f(v_c)$ analog zur obigen Analyse 
f"ur $f(v_c)$ bei nicht homogenen Programmen in
\Beq{Equ3VSBRAnalyse}
        2\lc \log(g(v_c)) \rc
        \lb \lc \log \lb \frac{c}{ \lc \log(g(v_c)) \rc } 
                     \rb 
             \rc + 2
        \rb
\Eeq Schritten von 
\Beq{Equ4VSBRAnalyse}
    \frac{c}{ \lc \log(g(v_c)) \rc } 
\Eeq Prozessoren berechnet werden.

% **************************************************************************

\MySection{Das Gau"s'sche Eliminationsverfahren parallelisiert}
\label{SecAlgBGH}

Das Thema dieses Unterkapitels ist es, wie das in \ref{SecVSBR}
beschriebene Verfahren benutzt werden kann, um mit Hilfe des in
\ref{SecGaussOhneDiv} angegebenen
Gau"s'schen Eliminationsverfahrens ohne Divisionen parallel die
Determinante einer $n \times n$-Matrix zu berechnen. Auf den so
entstehenden Algorithmus wird mit BGH-Alg. Bezug genommen
(vgl. Unterkapitel \ref{SecBez}).

Da keine Divisionen durchgef"uhrt werden, ist BGH-Alg. ebenso wie
B-Alg. auch in Ringen anwendbar. In dieser Hinsicht besitzen die
beiden Algorithmen gegen"uber C-Alg. und P-Alg. einen Vorteil.

Im folgenden wird beschrieben, wie ein Programm im Sinne von
\ref{SecVSBR} anhand der Ergebnisse von \ref{SecGaussOhneDiv}
aufgebaut wird. Um die Auswirkungen des Grades, bis zu dem Potenzreihen
entwickelt werden, auf die Effizienz der Rechnung besser demonstrieren 
zu k"onnen, werden die folgenden Betrachtungen zun"achst unabh"angig von
einem konkreten Grad durchgef"uhrt. F"ur alle Potenzreihen werden
die homogenen Komponenten bis maximal zum Grad $s$ betrachtet.

Es gibt drei wesentliche Elementaroperationen f"ur Potenzreihen, die 
zun"achst auf ihren Aufwand hin untersucht werden. Da sich nach
\ref{SecVSBR} die Anzahl der Prozessoren und der Schritte aus der
Programml"ange ergibt, wird im folgenden nur die Anzahl der Anweisungen
im Sinne von \ref{DefProgramm} betrachtet:
\begin{MyDescription}
\MyItem{Addition}
    Dieser Fall gilt f"ur {\em Subtraktion} analog. Es werden die
    homogenen Komponenten gleichen Grades addiert. Da die homogenen 
    Komponenten bis zum Grad $s$ betrachtet werden, sind hierf"ur 
    $s+1$ Anweisungen erforderlich.
\MyItem{Multiplikation}
    Seien $a$ und $b$ die zu multiplizierenden Potenzreihen. F"ur eine
    Potenzreihe $x$ bezeichne $x_i$ deren homogene Komponente vom
    Grad $i$. Das Ergebnis der zu Multiplikation von $a$ und $b$ sei $c$.
    Man erh"alt $c$ mit:
    \[ c_i := \sum_{j=0}^{i} a_j * b_{i-j} \MyPunkt \]
    Da f"ur $c$ auch nur die homogenen Komponenten bis zum
    Grad $s$ berechnet werden m"ussen, folgt mit Hilfe der Gleichung
    \[ 2^i = \sum_{j=0}^{i-1} 2^j + 1 \] f"ur die Anzahl der Anweisungen
    bei Benutzung der Bin"arbaummethode nach \ref{SatzAlgBinaerbaum}:
    \begin{eqnarray*}
        & & \sum_{i=0}^s 
            \lb 
                i + \sum_{j=0}^{\lc \log(i) \rc-1} 2^j 
            \rb \\
        & = &
            \sum_{i=0}^s \lb i + 2^{\lc \log(i) \rc} - 1 \rb \\
        & \leq &
            \sum_{i=0}^s \lb i + 2^{\log(i) + 1} - 1 \rb \\
        & = &
            \sum_{i=0}^s \lb 3i - 1 \rb \\
        & = & 3 \sum_{i=0}^s i - (s+1) \\
        & = & \frac{3}{2} s (s + 1) - (s+1) \\
        & = & \frac{3s^2 + s - 2}{2} \MyPunkt
    \end{eqnarray*}
\MyItem{Division}
    Die Divisionen werden entsprechend der Ausf"uhrungen in
    \ref{SecPotRing} und \ref{SecGaussOhneDiv} durch Additionen und
     Multiplikationen ersetzt (vgl. S. \pageref{Equ1ZuErsetzen}).
    Da nur die homogenen Komponenten bis zum Grad $s$ betrachtet werden,
    erfolgt die Potenzreihenentwicklung wie in \equref{Equ1StattDivision}
    nur bis zum $s$-ten Glied.

    Somit sind
    $s$ Multiplikationen und $s-1$ Additionen von Potenzreihen sowie
    die Addition des konstanten Terms durchzuf"uhren. In Verbindung mit
    den vorangegangenen Analysen von Addition und Multiplikation ergibt
    sich f"ur die Anzahl der Anweisungen:
    \begin{eqnarray*}
        &   & s * \lb \frac{3s^2 + s - 2}{2} \rb + (s-1) * (s+1) + 1 \\
        & = & \frac{3s^3 + 3s^2 - 2s}{2} \MyPunkt
    \end{eqnarray*}
\end{MyDescription}

Als n"achstes wird untersucht, wieviele der einzelnen Elementaroperationen
zur Berechnung der Determinante benutzt werden. Dazu werden zwei 
Gleichungen benutzt:

\begin{bemerkung}
\label{SatzSumK}
    Sei $n \in \Nat_0$. Dann gilt:
    \[ \sum_{k=1}^n k = \frac{ n(n+1) }{ 2 } \]
\end{bemerkung}

\begin{bemerkung}
\label{SatzSumK2}
    Sei $n \in \Nat_0$. Dann gilt:
    \[ \sum_{k=1}^n k^2 = \frac{ n(n+1)(2n+1) }{ 6 } \]
\end{bemerkung}

Das in \ref{SecGauss} beschriebene Verfahren verwendet die Gleichungen
\equref{Equ1GaussDef} und \equref{Equ2GaussDef}. Werden mit Hilfe dieser 
Gleichungen zun"achst alle Matrizenelemente transformiert, betr"agt 
die Anzahl der Berechnungen neuer Elemente:
\begin{eqnarray*}
   &   & \sum_{i=1}^{n-1} \sum_{j=i+1}^n (n-(j-1)) \\
   & = & \sum_{i=2}^n \sum_{j=i}^n (n-(j-1)) \\
   & = & \sum_{i=2}^n \sum_{j=1}^{n-(i-1)} j \\
   & \MyStack{nach \ref{SatzSumK}}{=} &
         \sum_{i=2}^n \frac{ (n-(i-1))*((n-(i-1))+1) }{2} \\
   & = & \frac{1}{2} \sum_{i=2}^n ((n-i+1)*(n-i+2)) \\
   & = & \frac{1}{2} \sum_{i=2}^n (n^2+3n-2ni+i^2-3i+2) \\
   & = & \frac{1}{2}
         \lb (n-1)(n^2+3n+2)
             + \sum_{i=2}^n i^2
             - \sum_{i=2}^n 2ni
             - \sum_{i=2}^n 3i
         \rb \\
   & \MyStack{nach \ref{SatzSumK},\ref{SatzSumK2}}{=} &
         \frac{1}{2}
         \lb (n^3+2n^2-n-2)
             + \frac{1}{6} n(n+1)(2n+1) - 1 \right. \\
   & &   \left.
             - 2n \lb\frac{ n(n+1) }{ 2 } - 1 \rb
             - 3 \lb \frac{ n(n+1) }{ 2 } - 1 \rb
         \rb \\
   & = &
          \frac{1}{2}
          \lb (n^3+2n^2-n-2)
              + \frac{ 2n^3+3n^2+n }{ 6 } - 1 \right. \\
   & &   \left.
              - ( n^3+n^2 - 2n)
              - \lb \frac{ 3(n^2+n) }{ 2 } - 3 \rb
          \rb \\
   & = & \frac{1}{6}( n^3- n)
\end{eqnarray*}
F"ur jede einzelne Transformation eines Matrixelements
werden nach \equref{Equ2GaussDef}
eine Subtraktion, eine Multiplikation und eine Division durchgef"uhrt.
Da alle Rechnungen in $R[[]]$ erfolgen, werden dabei Potenzreihen 
miteinander verkn"upft, wof"ur der Aufwand
gemessen in durchzuf"uhrenden Anweisungen bereits analysiert worden 
ist (s. o.).
F"ur die identische Abbildung nach \equref{Equ1GaussDef} wird kein Aufwand
in Rechnung gestellt. So kommt man auf
\begin{eqnarray}
   & & \nonumber
       \frac{1}{6}( n^3 - n) *
       \lb
           s + 1
           + \frac{3s^2 + s - 2}{2}
           + \frac{3s^3 + 3s^2 - 2s}{2}
       \rb \\
  & = & \label{AnwNeueElem}
% Form:
%    1/4*n^3*s^3 + 1/2*n^3*s^2 + 1/12*n^3*s - 1/4*n*s^3 - 1/2*n*s^2
%    - 1/12*n*s;
         \frac{1}{4}
       \lb n^3 s^3 + 2n^3 s^2 + \frac{n^3 s}{3} - n s^3 - 2 n s^2
       - \frac{n s}{3} \rb
\end{eqnarray}
Anweisungen, um eine gegebene Matrix mit Hilfe des Gau"s'schen Verfahrens
in eine obere Dreiecksmatrix zu "uberf"uhren. Zu Berechnung der
Determinante sind im Anschlu"s daran noch die Elemente der Hauptdiagonalen
miteinander zu multiplizieren. Dies kann mit $n-1$ Multiplikationen
geleistet werden, denen
\begin{eqnarray}
   & & \nonumber
     (n-1) * 
     \frac{3s^2 + s - 2}{2} \\
   & = & \label{AnwDiagMult}
     \frac{1}{2} ( 3 n s^2 + n s - 2n - 3 s^2 - s + 2)
\end{eqnarray}
Anweisungen entsprechen. So hat man bereits das Ergebnis als Element
von $R[[]]$. Um die Determinante als Element von $R$ zu erhalten, m"ussen
nun noch die homogenen Komponenenten bis zum Grad $s$ addiert werden.
Dies kann mit Hilfe von $s$ Anweisungen erfolgen. Abgesehen von diesen 
Additionen ist das Programm homogen im Sinne von \ref{DefHomogen}.
Deshalb ist es von Vorteil, die in \ref{SecVSBR} beschriebene Methode
auf das Programm ohne die letzten Additionen anzuwenden und diese
Additionen mit Hilfe der Bin"arbaummethode nach \ref{SatzAlgBinaerbaum}
durchzuf"uhren. Die Addition der homogenen Komponenten kann so in 
\[ \lc \log(s+1) \rc \] Schritten von \[ \lf \frac{s+1}{2} \rf \]
Prozessoren geleistet werden.

Man erh"alt das Gesamtergebnis
f"ur die Programml"ange ohne die letzten Additionen als
Summe von \equref{AnwNeueElem} und \equref{AnwDiagMult}:
\[ % \label{Gesamt}
%  1 + 1/4*n^3*s^3 + 1/2*n^3*s^2 + 1/12*n^3*s - 1/4*n*s^3 + n*s^2 + 5/12*n*
%  s - n - 3/2*s^2 - 1/2*s;
   \frac{1}{4} 
   \lb
       n^3 s^3 + 2 n^3 s^2 + \frac{1}{3}n^3 s - n s^3 + n s^2 + 
       \frac{5}{3}n s - 4 n - 6 s^2 - 2 s + 4
   \rb
\]
Anweisungen. Dieser Wert wird entsprechend der Terminologie in
\ref{SecVSBR} mit $c$ bezeichnet. Da bei allen Rechnungen nur die
homogenen Komponenten bis zum Grad $s$ beachtet werden, gilt
\[ g(v_c) = s \MyPunkt \]
Aus $c$ und $g(v_c)$ erh"alt man mit Hilfe der Analyseergebnisse
\equref{Equ3VSBRAnalyse} und \equref{Equ4VSBRAnalyse} aus
\ref{SecVSBR} f"ur die in diesem Kapitel beschriebene Methode zur
parallelen Determinantenberechnung einen Aufwand 
von\footnote{Genau genommen mu"s der Wert noch um $1$ erh"oht werden 
f"ur die Berechnung der $a_{i,j}'$ aus den urspr"unglichen 
Matrizenelementen $a_{i,j}$ entsprechend Gleichung 
\equref{EquDefBGHErsetzung}.}
\begin{eqnarray*}
    & & 2\lc \log(s) \rc * (\lc \log(c) \rc + 2) + \lc \log(s+1) \rc \\
    & = &
        2\lc \log(s) \rc \\
    & &
        * \lb
            \lc
            \log\lb
   \frac{1}{4} 
   \lb
       n^3 s^3 + 2 n^3 s^2 + \frac{1}{3}n^3 s - n s^3 + n s^2 + 
       \frac{5}{3}n s - 4 n - 6 s^2 - 2 s + 4
   \rb
            \rb
            \rc + 2
        \rb \\
    & & + \lc \log(s+1) \rc
\end{eqnarray*}
Schritten und
\begin{eqnarray*}
  \lefteqn{ \max \lb c \: , \lf \frac{s}{2} \rf \rb } \\
  & = & c \\
  & = &
   \frac{1}{4}
   \lb
       n^3 s^3 + 2 n^3 s^2 + \frac{1}{3}n^3 s - n s^3 + n s^2 +
       \frac{5}{3}n s - 4 n - 6 s^2 - 2 s + 4
   \rb
\end{eqnarray*}
Prozessoren. Man erkennt an diesen Werte die Bedeutung des Parameters
$s$, dem maximal ber"ucksichtigten Grad der homogenen Komponenten der
Potenzreihen. Betrachtet man $s$ als Konstante, so kann der Algorithmus in
\[ O(\log(n)) \] Schritten von \[ O(n^3) \] Prozessoren bearbeitet werden.

Die Analyse in Unterkapitel \ref{SecGaussOhneDiv} ergibt, da"s $s=n$ zu
setzen ist, so da"s der Algorithmus in
\[ O(\log^2(n)) \] Schritten von \[ O(n^6) \] Prozessoren bearbeitet
werden kann.

Die Aufwandanalyse ergibt, da"s BGH-Alg. insbesondere bei der 
Gr"o"senordnung der Anzahl der Prozessoren deutlich hinter C-Alg., B-Alg.
und P-Alg. zur"uckliegt. Die Konstanten bei der Anzahl der Schritte sind
ebenfalls vergleichsweise schlecht.

In BGH-Alg. werden, wie bei den anderen drei Algorithmen, keine 
Fallunterscheidungen verwendet, was aus den bereits in Unterkapitel
\ref{SecAlgFrame} erw"ahnten Gr"unden beim Entwurf von Schaltkreisen
vorteilhaft ist. 

Betrachtet man die Methodik von BGH-Alg., so ist er P-Alg. am "ahnlichsten.
Beide fassen mehrere auch unabh"angig voneinander bedeutsame Verfahren
zu einem Algorithmus zur Determiantenberechnung zusammen. C-Alg. und B-Alg.
hingegen st"utzen sich jeweils auf bestimmte schon seit 40 bis
50 Jahren bekannte S"atze, die nach einigen Umformungen f"ur einen
parallelen Algorithmus verwendet werden.


%
% Datei: berk.tex (Textteile nach 'Berk84')
%
\MyChapter{Der Algorithmus von Berkowitz}
\label{ChapBerk}

Der in diesem Kapitel vorgestellte Algorithmus \cite{Berk84} berechnet
die Determinante mit Hilfe einer rekursiven Beziehung zwischen den
charakteristischen Polynomen einer Matrix und ihren Untermatrizen. Dabei
wird \ref{SatzDdurchP} ausgenutzt. Auf den Algorithmus wird mit {\em B-Alg.} Bezug
genommen\footnote{vgl. Unterkapitel \ref{SecBez}}. 

Wie in BGH-Alg., werden keine Divisionen
verwendet\footnote{vgl. Erl"auterungen in Unterkapitel \ref{SecAlgFrame}}.

%******************************************************************

\MySection{Toepliz-Matrizen}

Im darzustellenden Algorithmus spielen Toepliz-Matrizen (Definition s. u.)
eine wichtige Rolle und werden deshalb in diesem Unterkapitel behandelt.

Eine Matrix $n \times p$-Matrix $A$ hei"st {\em Toepliz-Matrix}, falls
gilt: \index{Toepliz-Matrizen}
\[ a_{i,j} = a_{i-1,j-1} , \: 1 < i \leq n, \: 1 < j \leq p \MyPunkt \]
Sie hat also folgendes Aussehen:
\[
   \left[ \begin{array}{ccccc}
       a_{1,1} & a_{1,2} & a_{1,3} & a_{1,4} & \cdots \MatStrut \\
       a_{2,1} & a_{1,1} & a_{1,2} & a_{1,3} & \ddots \MatStrut \\
       a_{3,1} & a_{2,1} & a_{1,1} & a_{1,2} & \ddots \MatStrut \\
       a_{4,1} & a_{3,1} & a_{2,1} & a_{1,1} & \ddots \MatStrut \\
       \vdots  & \ddots  & \ddots  & \ddots  & \ddots \MatStrut
   \end{array} \right]
\]

Die folgende Eigenschaft von Toepliz-Matrizen ist f"ur uns wichtig:

\begin{satz}
\label{SatzToeplizMult}
\index{Toepliz-Matrizen!Multiplikation von}
    Sei $A$ eine $n \times p$-Matrix und $B$ eine $p \times m$-Matrix.
    Beide seien untere Dreiecks-Toeplitz-Matrizen.
    Falls f"ur die Matrix $C$ gilt
    \[ C = A * B \MyKomma \]
    dann ist $C$ ebenfalls eine untere Dreiecks-Toeplitz-Matrix.
    Sie kann in \[ \lceil \log(p) \rceil + 1 \] Schritten von
    \begin{eqnarray*} 
        & & \frac{ \min(p,\,m)* (\min(p,\,m) +1) }{2}
            + p * \max(n-p,0) \\
        & \leq & n * p
    \end{eqnarray*} Prozessoren berechnet
    werden. 
\end{satz}
\begin{beweis}
    Es sind drei Eigenschaften von $C$ zu zeigen:
    \begin{enumerate}
        \item $C$ ist eine untere Dreiecksmatrix.
        \item $C$ ist eine Toeplitz-Matrix.
        \item $C$ kann mit dem oben angegebenen Aufwand an Schritten und
              Prozessoren berechnet werden.
    \end{enumerate}
    Dies geschieht in drei entsprechenden Beweisschritten. Dazu ist zu
    beachten, da"s die einzelnen Elemente von $C$ nach der Gleichung f"ur
    die Matrizenmultiplikation berechnet werden:
    \Beq{Berk84Equ4}
        c_{i,j}= \sum_{k=1}^p a_{i,k} b_{k,j}
    \Eeq
    \begin{enumerate}
        \item
            Um zu beweisen, da"s $C$ ebenfalls eine untere Dreiecksmatrix
            darstellt, ist zu zeigen
            \[ i < j \Rightarrow c_{i,j} = 0 \]
            Dies erfolgt durch Fallunterscheidung anhand des Index $k$ in
            Gleichung \equref{Berk84Equ4}. Es gibt zwei F"alle:
            \begin{MyDescription}
                \MyItem{ $i < k$ }
                    Da $A$ nach Voraussetzung eine untere Dreiecksmatrix
                    ist und somit
                    \[ i < j \Rightarrow a_{i,j} = 0 \]
                    gilt, folgt 
                    \[ a_{i,k} = 0 \MyKomma\]
                    wodurch der entsprechende Summand in Gleichung
                    \equref{Berk84Equ4} zu $0$ wird.
                \MyItem{ $i \geq k$ }
                    Nach Voraussetzung gilt \[ i < j \MyKomma \] 
                    da f"ur die
                    Elemente oberhalb der Hauptdiagonalen von $C$ zu zeigen
                    ist, da"s sie gleich $0$ sind. Daraus folgt aber
                    \[ k < j \MyPunkt \]
                    Nach Voraussetzung ist $B$ ebenfalls eine untere
                    Dreiecksmatrix und es gilt somit
                    \[ i < j \Rightarrow b_{i,j} = 0 \]
                    Daraus folgt \[ b_{k,j} = 0 \MyKomma \]
                    wodurch wiederum der
                    entsprechende Summand in Gleichung \equref{Berk84Equ4}
                    zu $0$ wird.
            \end{MyDescription}
            In beiden F"allen sind die betrachteten Summanden von Gleichung
            \equref{Berk84Equ4} gleich $0$. Also ist dann auch
            \[ c_{i,j} = 0 \MyKomma \]
            was zu zeigen war.
        \item
            Damit $C$ eine Toeplitz-Matrix ist, mu"s gelten
            \[ c_{i,j} = c_{i+1,j+1} \MyPunkt \]
            Mit Hilfe von Gleichung \equref{Berk84Equ4} ausgedr"uckt
            bedeutet dies
            \Beq{Berk84Equ5}
                 \sum_{k=1}^p a_{i,k} b_{k,j}
               = \sum_{l=1}^p a_{i+1,l} b_{l,j+1} \MyPunkt
            \Eeq
            Da $C$ eine untere Dreiecksmatrix ist, wie oben bewiesen wurde,
            m"ussen nur \[ c_{i,j} \] betrachtet werden, f"ur die gilt
            \[ i \geq j \MyPunkt \]
            Man kann Fallunterscheidungen anhand der Indizes $k$ und $l$
            durchf"uhren. Es gibt f"ur jeden Index drei F"alle, also
            insgesamt sechs:
            \begin{MyDescription}
                \MyItem{ $k>i$ }
                    Da $A$ nach Voraussetzung eine untere Dreiecksmatrix
                    ist, gilt in diesem Fall \[ a_{i,k}= 0 \MyKomma \]
                    und der
                    entsprechende Summand wird zu $0$.
                \MyItem{ $j>k$ }
                    Da $B$ nach Voraussetzung ebenfalls eine untere
                    Dreiecksmatrix ist, gilt in diesem Fall
                    \[ b_{k,j} = 0 \MyKomma \]
                    und der entsprechende Summand wird zu $0$.
                \MyItem{ $i \geq k \geq j$ }
                    Nur in diesem Fall ergibt sich auf der linken Seite
                    von Gleichung \equref{Berk84Equ5} f"ur den jeweiligen
                    Summand ein von $0$ verschiedener Wert. Deshalb kann man
                    die linke Seite dieser Gleichung auch schreiben als
                    \[ \sum_{k=j}^i a_{i,k} b_{k,j} \MyPunkt \]
                \MyItem{ $l>i+1$ }
                    In diesem Fall gilt, da $A$ eine obere Dreiecksmatrix
                    ist, \[ a_{i+1,l} = 0 \MyPunkt \]
                    Der entsprechende Summand der
                    Summe in Gleichung \equref{Berk84Equ5} wird somit zu
                    $0$ und mu"s nicht l"anger betrachtet werden.
                \MyItem{ $j+1>l$ }
                    In diesem Fall gilt \[ b_{l,j+1} = 0 \MyKomma \]
                    da $B$ eine
                    obere Dreiecksmatrix ist und der entsprechende Summand
                    in Gleichung \equref{Berk84Equ5} mu"s nicht l"anger
                    betrachtet werden.
                \MyItem{ $i+1 \geq l \geq j+1$ }
                    Nur in diesem Fall ergibt sich auf der rechten Seite
                    von Gleichung \equref{Berk84Equ5} ein von $0$
                    verschiedener Wert f"ur den entsprechenden Summanden.
                    Man kann also die rechte Seite dieser Gleichung auch
                    schreiben als
                    \[ \sum_{l=j+1}^{i+1} a_{i+1,l} b_{l,j+1} \]
            \end{MyDescription}
            Nach der Betrachtung dieser sechs F"alle reduziert sich
            Gleichung \equref{Berk84Equ5} also, falls man nur die von
            $0$ verschiedenen Summanden betrachtet, auf die Form
            \[ \sum_{k=j}^i a_{i,k} b_{k,j} =
               \sum_{l=j+1}^{i+1} a_{i+1,l} b_{l,j+1}
            \]
            Anders geschrieben hat diese Gleichung die Form
            \begin{eqnarray*}
            &   a_{i,j} b_{j,j} + a_{i,j+1} b_{j+1,j} + a_{i,j+2} b_{j+2,j}
                + \ldots + a_{i,i} b_{i,j} =
            & \\
            &   a_{i+1,j+1} b_{j+1,j+1} + a_{i+1,j+2} b_{j+2,j+1} +
                a_{i+1,j+3} b_{j+3,j+1} + \ldots + a_{i+1,i+1} b_{i+1,j+1}
            &
            \end{eqnarray*}
            Da $A$ und $B$ Toeplitz-Matrizen sind, haben die beiden
            Seiten dieser Gleichung den gleichen Wert, was zu beweisen war.
        \item
            Da $C$ wiederum eine Toeplitz-Matrix ist,
            m"ussen nur die $c_{i,j}$ mit $j=1$ neu berechnet
            werden. Alle anderen Elemente sind entweder gleich Null oder
            gleich einem $c_{i,1}$. L"a"st man zus"atzlich alle
            Multiplikationen mit Null weg, kommt man
            zur Berechnung von $C$ insgesamt mit
            \[ \lc \log(p) \rc + 1 \] Schritten und
            \begin{eqnarray*}
                & & \sum_{k=1}^{\min(p,\,m)} k + p * \max(n-p,\,0) \\
                & = & \frac{ \min(p,\,m)* (\min(p,\,m) +1) }{2}
                      + p * \max(n-p,0)
            \end{eqnarray*} Prozessoren aus. Einschlie"slich der 
            Multiplikationen mit Null erh"alt man
            \[ \lc \log(p) \rc + 1 \] Schritte und
            \[ n * p \] Prozessoren.
    \end{enumerate}
\end{beweis}

% **************************************************************************

\MySection{Der Satz von Samuelson}
\label{SecSamuelson}

In diesem Unterkapitel wird der theoretische Hintergrund des
darzustellenden Algorithmus behandelt.

Zur Beschreibung des Satzes von Samuelson \cite{Samu42} wird folgende
Schreibweise eingef"uhrt ($A$ ist eine $n \times n$-Matrix):
\label{SeiteRMSSchreibweise}
\begin{itemize}
\item
     Den Vektor $S_i$ erh"alt man aus dem $i$-ten Spaltenvektor von $A$
     durch Entfernen der ersten $i$ Elemente. Er hat also folgendes
     Aussehen:
     \[ \left[
        \begin{array}{c}
            a_{i+1,i} \MatStrut \\
            a_{i+2,i} \MatStrut \\
            \vdots    \MatStrut \\
            a_{n,i}
        \end{array}
        \right]
     \]
\item
     Den Vektor $R_i$ erh"alt man aus dem $i$-ten Zeilenvektor von $A$
     durch Entfernen der ersten $i$ Elemente. Er hat also folgendes
     Aussehen:
     \[ [a_{i,i+1}, a_{i,i+2}, \ldots , a_{i,n} ] \]
\item
     Die Matrix $M_i$ erh"alt man aus der Matrix $A$ durch Entfernen
     der ersten $i$ Zeilen und Spalten. Sie hat also folgendes Aussehen:
     \[ \left[
        \begin{array}{cccc}
            a_{i+1,i+1} & a_{i+1,i+2} & \cdots & a_{i+1,n} \MatStrut \\
            a_{i+2,i+1} & a_{i+2,i+2} & \cdots & a_{i+2,n} \MatStrut \\
            \vdots      & \vdots      & \ddots & \vdots    \MatStrut \\
            a_{n,i+1}   & a_{n,i+2}   & \cdots & a_{n,n}   \MatStrut
        \end{array}
        \right]
     \]
\item
     Statt $S_1$, $R_1$ und $M_1$ wird auch $S$, $R$ und $M$ geschrieben.
\end{itemize}

Die Matrix $A$ l"a"st sich also auch in den Formen
\[
   \left[
   \begin{array}{cc}
       a_{11} & R \MatStrut \\
       S      & M \MatStrut
   \end{array}
   \right]
\]
oder
\[
   \left[
   \begin{array}{ccccc}
       a_{11}     & R_1        & \rightarrow &             & \MatStrut \\
       S_1        & a_{22}     & R_2         & \rightarrow & \MatStrut \\
       \downarrow & S_2        & a_{33}      & R_3         & \rightarrow
                                                             \MatStrut \\
                  & \downarrow & S_3         & \ddots      & \ddots
                                                             \MatStrut \\
                  &            & \downarrow  & \ddots      & \MatStrut
   \end{array}
   \right]
\]
darstellen.

Im folgenden Lemma wird das charakteristische Polynom einer Matrix
mit Hilfe der oben definierten $R$, $S$ und $M$ ausgedr"uckt:

\begin{lemma}
\label{Berk84Satz1}
% $$$ Claim 1
    Sei $p(\lambda)$ das charakteristische Polynom der $n \times n$-Matrix
    $A$. Dann gilt:
    \[
        p(\lambda) = (a_{1,1} - \lambda) * \det(M - \lambda * E_{n-1})
                     - R * \adj(M - \lambda * E_{n-1}) * S
    \]
\end{lemma}
\begin{beweis}
    Es gilt \[ p(\lambda) = \det(A - \lambda * E_n) \]
    Durch Entwicklung nach der ersten Zeile erh"alt man:
    \[
        p(\lambda)= (a_{1,1} - \lambda) * \det(M - \lambda * E_{n-1}) +
        \sum_{j=2}^n (-1)^{1+j} a_{1,j}
        \underline{ \det( (A - \lambda * E_n)_{(1|j)} ) }
    \]
    Nun werden die in der obigen Gleichung unterstrichenen Determinanten
    jeweils nach der ersten Spalte entwickelt:
    \begin{eqnarray*}
        & p(\lambda)=
        & (a_{1,1} - \lambda) * \det(M - \lambda * E_{n-1}) +
    \\  & & \sum_{j=2}^n
            \underbrace{ (-1)^{1+j} a_{1,j} }_{ \mbox{(*1)} }
        \sum_{k=2}^n (-1)^{1+(k-1)} a_{k,1}
            \det(
                \underbrace{
                    (A - \lambda * E_n)_{(1,k|1,j)}
                }_{ \mbox{(*2)} }
            )
    \end{eqnarray*}
    Wenn man in dieser Gleichung (*1) mit der inneren Summe multipliziert
    und (*2) mit Hilfe von $M$ ausdr"uckt erh"alt man:
    \[
        p(\lambda)= (a_{1,1} - \lambda) * \det(M - \lambda * E_{n-1}) +
        \sum_{j=2}^n
        \sum_{k=2}^n (-1)^{1+j+k} \underbrace{ a_{1,j} a_{k,1} }_{ \mbox{(*)} }
            \det( (M - \lambda * E_{n-1})_{(k-1|j-1)} )
    \]
    Hier l"a"st sich (*) mit Hilfe von $R$ und $S$ formulieren:
    \[
        p(\lambda)= (a_{1,1} - \lambda) * \det(M - \lambda * E_{n-1}) +
        \sum_{j=2}^n
        \sum_{k=2}^n (-1)^{1+j+k} r_j s_k
            \det( (M - \lambda * E_{n-1})_{(k-1|j-1)} )
    \]
    Dies wiederum ist in Matrizenschreibweise und mit Hilfe der Adjunkten
    einer Matrix ausgedr"uckt nichts anderes als
    \[
        p(\lambda) = (a_{1,1} - \lambda) * \det(M - \lambda * E_{n-1})
                     - R * \adj(M - \lambda * E_{n-1}) * S \MyKomma
    \]
    was zu beweisen war.
\end{beweis}

Vor Lemma \ref{Berk84Satz2} m"ussen wir hier zun"achst einen wichtigen
Satz behandeln (\cite{MM64} S. 50 f):

\begin{satz}[Cayley und Hamilton]
\label{SatzCayleyHamilton}
\index{Cayley und Hamilton!Satz von}
    Sei $p(\lambda)$ das charakteristische Polynom von $A$. Dann gilt:
    \[ p(A)=0_{n,n} \]
\end{satz}
\begin{beweis}
    Aus Satz \ref{SatzAdj} folgt
    \begin{equation}
    \label{Equ1SatzCayleyHamilton}
        (A - \lambda E_n) \adj(A - \lambda E_n) = p(A) E_n
    \end{equation}
    Da die Elemente von \[ \adj(A - \lambda E_n) \] aus Unterdeterminanten
    von $A$ gewonnen werden, bestehen diese Elemente aus Polynomen "uber
    $\lambda$ vom maximalen Grad \[ n - 1 \] Also gilt f"ur geeignete
    $n \times n$-Matrizen \[ B_j, 1 \leq j \leq n-1 \] die folgende 
    Beziehung:
    \Beq{Equ2SatzCayleyHamilton} 
        \adj(A - \lambda E_n) = 
        B_{n-1} \lambda^{n-1} + \ldots + B_1 \lambda + B_0
    \Eeq
    Au"serdem kann man $p(A)$ schreiben als
    \Beq{Equ3SatzCayleyHamilton}
        p(A) = c_n \lambda^n + \ldots + c_1 \lambda + c_0
    \Eeq
    Dr"uckt man \equref{Equ1SatzCayleyHamilton} mit Hilfe von
    \equref{Equ2SatzCayleyHamilton} und \equref{Equ3SatzCayleyHamilton} aus,
    erh"alt man
    \[
        (A - \lambda E_n)(B_{n-1} \lambda^{n-1} + \ldots
        + B_1 \lambda + B_0)
            =
        (c_n \lambda^n + \ldots + c_1 \lambda + c_0) E_n
    \]
    Multipliziert man die Terme auf beiden Seiten aus und vergleicht die
    Koeffizienten miteinander, erh"alt man folgende Gleichungen:
    \[
        \begin{array}{lllcr}
                     & - & B_{n-1} & =      & c_n E_n
        \\  AB_{n-1} & - & B_{n-2} & =      & c_{n-1} E_n
        \\  AB_{n-2} & - & B_{n-3} & =      & c_{n-2} E_n
        \\           &   &         & \vdots &
        \\  AB_1     & - & B_0     & =      & c_1 E_n
        \\  AB_0     &   &         & =      & c_0 E_n
        \end{array}
    \]
    Multipliziert man beide Seiten der
    ersten dieser Gleichungen mit $A^n$, beide Seiten der zweiten
    mit $A^{n-1}$, allgemein beide Seiten der $j$-ten mit $A^{n-j+1}$, und
    addiert sie, erh"alt man
    \[ 0_{n,n} =
       c_n \lambda^n + c_{n-1} \lambda^{n-1} + \ldots + c_1 A = p(A)
    \]
\end{beweis}

Die Adjunkte in Lemma \ref{Berk84Satz1} l"a"st sich mit Hilfe der
Koeffizienten des charakteristischen Polynoms $q(\lambda)$ von $M$
ausdr"ucken. In Koeffizientendarstellung besitzt $q(\lambda)$ die
Form:
\[ q(\lambda) = q_{n-1} \lambda^{n-1} + q_{n-2} \lambda^{n-2} + \ldots
                + q_1 \lambda + q_0
\]

Es gilt folgende Aussage:

\begin{lemma}
\label{Berk84Satz2}
% $$$ Claim 2
    \begin{equation}
    \label{Berk84Equ1}
        \adj(M - \lambda * E_{n-1}) =
            - \sum_{k=0}^{n-2} \lambda^{k} \sum_{l=k+1}^{n-1} M^{l-k-1} q_l
    \end{equation}
\end{lemma}
\begin{beweis}
    Multipliziert man beide Seiten von \equref{Berk84Equ1} mit
    \[ M - \lambda * E_{n-1} \MyKomma \]
    erh"alt man auf der linken Seite
    \[ \adj(M - \lambda * E_{n-1}) * (M - \lambda * E_{n-1}) \MyPunkt \]
    Dies ist nach Satz \ref{SatzAdj} gleich
    \begin{eqnarray*}
        & & E_{n-1} * \det(M - \lambda * E_{n-1}) \\
        & = & q(\lambda) * E_{n-1} \MyPunkt
    \end{eqnarray*}

    Auf der rechten Seite von Gleichung \equref{Berk84Equ1} erh"alt man
    \[ - ( \underbrace{M}_{\mbox{(*1)}}
           \underbrace{- \lambda * E_{n-1})}_{\mbox{(*2)}}
         )
         \sum_{k=0}^{n-2} \lambda^{k} \sum_{l=k+1}^{n-1} M^{l-k-1} q_l
    \]
    Bei der Multiplikation erh"alt man f"ur (*1) und (*2) im obigen je
    eine Doppelsumme:
    \[ - \sum_{k=0}^{n-2} \lambda^{k} \sum_{l=k+1}^{n-1} M^{l-k} q_l
       + \sum_{k=0}^{n-2} \lambda^{k+1} \sum_{l=k+1}^{n-1} M^{l-k-1} q_l
    \]
    Durch Umordnen der Indizes der zweiten Doppelsumme erh"alt man
    \begin{equation}
    \label{Berk84Equ2}
        - \sum_{k=0}^{n-2} \lambda^{k} \sum_{l=k+1}^{n-1} M^{l-k} q_l
        + \sum_{k=1}^{n-1} \lambda^{k} \sum_{l=k+1}^{n-1} M^{l-k} q_l
    \end{equation}
    Nach Satz \ref{SatzCayleyHamilton} gilt
    \begin{equation}
    \label{Berk84Equ3}
        \sum_{l=0}^{n-1} M^l q_l = 0
    \end{equation}
    Somit kann man die linke Seite von Gleichung \equref{Berk84Equ3} zur
    zweiten Doppelsumme von Term \equref{Berk84Equ2} addieren und erh"alt
    \[
        - \sum_{k=0}^{n-2} \lambda^{k} \sum_{l=k+1}^{n-1} M^{l-k} q_l
        + \sum_{k=0}^{n-1} \lambda^{k} \sum_{l=k+1}^{n-1} M^{l-k} q_l
    \]
    Wenn man nun die Vorzeichen der beiden Doppelsummen sowie
    die benutzten Indizes betrachtet, erkennt man, da"s sich der Gesamtterm
    vereinfacht darstellen l"a"st, da gro"se Teile zusammengenommen $0$
    ergeben. Die Teile, die sich nicht auf diese
    Weise gegenseitig aufheben, lassen sich schreiben als
    \[ \sum_{k=0}^{n-1} \lambda^k E_{n-1} q_{k} \MyKomma \]
    was gleichbedeutend ist mit
    \[ q(\lambda) * E_{n-1} \MyPunkt \]

    Also stimmen die beiden Seiten von Gleichung \equref{Berk84Equ1}
    "uberein.
\end{beweis}

Die beiden Lemmata \ref{Berk84Satz1} und \ref{Berk84Satz2} f"uhren zu
folgendem Satz \cite{Samu42}:

\begin{satz}[Samuelson]
\label{SatzSamuelson}
\index{Samuelson!Satz von}
% $$$ Claim 2 into Claim 1
    \begin{equation}
    \label{EquSatzSamuelson}
        p(\lambda) =
            (a_{1,1} - \lambda) * \det(M - \lambda * E_{n-1})
            + R * \left(
              \sum_{k=0}^{n-2} \lambda^{k} \sum_{l=k+1}^{n-1} M^{l-k-1} q_l
            \right) * S
    \end{equation}
\end{satz}
\begin{beweis}
    Lemma \ref{Berk84Satz2} angewendet auf Lemma \ref{Berk84Satz1}
    ergibt die Behauptung.
\end{beweis}

% **************************************************************************

\MySection{Determinantenberechnung mit Hilfe des Satzes von Samuelson}
\label{SecAlgBerk}
\index{Berkowitz!Algorithmus von}
\index{Algorithmus!von Berkowitz}

Um Satz \ref{SatzSamuelson} zur Determinantenberechnung zu benutzen
\cite{Berk84}, sind weitere "Uberlegungen notwendig, die in diesem
Unterkapitel behandelt werden.

Betrachtet man die Methodik des entstehenden Algorithmus, erkennt man
"Ahnlichkeit zu C-Alg. . Auch dort wird ein schon l"anger bekannter Satz
mit Hilfe von zus"atzlichen "Uberlegungen f"ur eine parallelen Algorithmus
verwendet.

Zu beachten ist, da"s in diesem Unterkapitel f"ur die Multiplikation 
zweier $n \times n$-Matrizen $n^{2+\gamma}$ Prozessoren in Rechnung 
gestellt werden (vgl. S. \pageref{PageAlg2MatMult}).

Benutzt man die Koeffizientendarstellung f"ur die charakteristischen
Polynome von $A$ und $M$, l"a"st sich Gleichung
\equref{EquSatzSamuelson} umformulieren in
\[
   \sum_{i=0}^n p_i \lambda^i =
       (a_{1,1} - \lambda) * \sum_{i=0}^{n-1} q_i \lambda^i
       + R * \left(
         \sum_{k=0}^{n-2} \lambda^{k} \sum_{l=k+1}^{n-1} M^{l-k-1} q_l
       \right) * S  
   \MyPunkt
\]
Vergleicht man die Koeffizienten der $\lambda^i$ auf beiden Seiten
der Gleichung und definiert \[ q_{-1} := 0 \MyKomma \] erh"alt man
\begin{eqnarray}
    p_n     & = & -q_{n-1}                 \label{Equ1Berk84KoeffVergl}
\\  p_{n-1} & = & a_{1,1}q_{n-1} - q_{n-2} \label{Equ2Berk84KoeffVergl}
\\  \forall i=n-2 \ldots 0 : \: p_i & = &  \label{Equ3Berk84KoeffVergl}
        a_{1,1}q_i-q_{i-1}+\sum_{j=i+1}^{n-1}RM^{j-i-1}S q_i
\end{eqnarray}
Die Beziehungen zwischen den Koeffizienten, die diese Gleichungen
beschreiben, kann man auch durch eine Matrizengleichung 
ausdr"ucken. Dazu
wird Matrix $C_t$ definiert als untere Dreiecks-Toeplitz-Matrix der
Gr"o"se $(n-t+2) \times (n-t+1)$. Ihre Elemente werden definiert durch
\[ (c_t)_{i,j} :=
       \left\{
           \begin{array}{lcr}
               -1                       & : & i=1
            \\ a_{t,t}                  & : & i=2
            \\ R_t M_t^{i-3} S_t        & : & i>2
           \end{array}
       \right.
\]
Die Matrix hat also das folgende Aussehen:
\[
    \left[ \begin{array}{ccc}
        -1                   & 0         & \cdots \MatStrut
    \\  a_{t,t}              & -1        & \ddots \MatStrut
    \\  R_t S_t              & a_{t,t}   & \ddots \MatStrut
    \\  R_t M_t S_t          & R_t S_t   & \ddots \MatStrut
    \\  \vdots               & \ddots    & \ddots \MatStrut
    \\  R_t M_t^{n-t-1} S_t  &           &        \MatStrut
    \end{array} \right]
\] \MyPunktA{30em} 
Insbesondere hat $C_n$ die Form
\[ 
    \left[ \begin{array}{c}
    -1 \\ a_{n,n}
    \end{array} \right]
\]

Mit Hilfe dieser Definition erh"alt man aus den Gleichungen
\equref{Equ1Berk84KoeffVergl}, \equref{Equ2Berk84KoeffVergl} und
\equref{Equ3Berk84KoeffVergl} die folgende Matrizengleichung:
\Beq{Berk84Equ19}
   \left[ \begin{array}{c}
       p_n \\
       p_{n-1} \\
       \vdots \\
       p_0
   \end{array} \right]
   =
   C_1
   \left[ \begin{array}{c}
       q_{n-1} \\
       q_{n-2} \\
       \vdots \\
       q_0
   \end{array} \right]
\Eeq
Auf die gleiche Weise, wie man 
Satz \ref{SatzSamuelson} auf die Matrizen $A$ und $M$ anwendet, kann man 
diesen Satz auch auf die Matrizen $M$ und $M_2$, $M_2$ und $M_3$, etc.
anwenden und erh"alt so Matrizengleichungen, die in ihrer Form der 
Gleichung \equref{Berk84Equ19} entsprechen. 

Wendet man diese Matrizengleichungen aufeinander an, erh"alt man:
\Beq{EquProdCi}
   \left[ \begin{array}{c}
       p_n \\
       p_{n-1} \\
       \vdots \\
       p_0
   \end{array} \right]
   =
   \prod_{i=1}^{n} C_i
\Eeq
Um die Koeffizienten des charakteristischen Polynoms von $A$ auf die
geschilderte Weise zu berechnen, mu"s man also die Matrizen $C_i$
berechnen und dann miteinander multiplizieren. Nach
\ref{SatzDdurchP} ist damit auch die Determinante der Matrix $A$ berechnet.

F"ur jede  $(n-i+2) \times (n-i+1)$-Matrix $C_i$ bei ist 
der $(n-i)$-elementige Vektor
\Beq{EquRMSVektor}
    T_t := [ R_i S_i, \, R_i M_i S_i, \, R_i M_i^2 S_i, \, 
    \ldots , \, R_i M_i^m S_i ], \: m:= n-i-1
\Eeq
zu berechnen. Da also $T_n$ keine Elemente enth"alt, ist die Berechnung
der Vektoren $T_1$ bis $T_{n-1}$ erforderlich.

Man kann jeden Exponenten $k$ eines Elementes \[ R_i * M_i^k * S_i \] von 
$T_i$ in der Form \[ k = u + v * \left\lceil \sqrt{m} \right\rceil \] mit
\begin{eqnarray*}
    & 0 \leq u < \left\lceil \sqrt{m} \right\rceil &
\\  & 0 \leq v \leq \left\lfloor \sqrt{m} \right\rfloor &
\end{eqnarray*}
eindeutig darstellen. Man k"onnte statt $\sqrt{m}$ auch einen anderen
Wert zwischen $0$ und $m$ nehmen. Jedoch f"uhrt die Wahl von $\sqrt{m}$ 
dazu, da"s sich die Gr"o"se der Mengen aller $u$ und $v$ um h"ochstens $1$
unterscheidet.

Um $T_i$ effizient zu erhalten, kann man zun"achst die den Mengen der $u$
und $v$ entsprechenden Vektoren
\[
   U_i := \left[ R_i,\, R_i M_i,\, R_i M_i^2,\, 
                 \ldots ,\, R_i M_i^{\lc \sqrt{m} \rc - 1} 
          \right]
\]
und
\[
   V_i := \left[ S_i,\, M_i^{\lc\sqrt{m}\rc} S_i,\, 
                    M_i^{2\lc\sqrt{m}\rc} S_i,\,
                \ldots,\, M_i^{\lf\sqrt{m}\rf \lc\sqrt{m}\rc} S_i
          \right]
\]
berechnen und danach jedes Element des einen Vektors mit jedem Element
des anderen multiplizieren.

Genau genommen werden auf diese Weise einige Werte zuviel berechnet, wie
sich bei noch exakterer Analyse des Algorithmus zeigt. Es sind jedoch
vernachl"assigbar wenige. Die Berechnung dieser Werte kann durch
vernachl"assigbar geringen zus"atzlichen Aufwand verhindert werden. Um die
Darstellung des Algorithmus nicht unn"otig un"ubersichtlich zu machen,
werden diese Werte nicht weiter beachtet.

Vor Beginn der Rechnung wird ein \label{PageWahlEpsilon} 
\[ \epsilon \in \Rationals \, , \: 0 < \epsilon \leq 0.5 \]
festgelegt\footnote{ein Wert $\epsilon>0.5$ ist m"oglich, jedoch von
                     seinen Auswirkungen her uninteressant}.
O. B. d. A. sei $\epsilon$ so gew"ahlt, da"s 
gilt\footnote{erf"ullt $\epsilon$ diese Bedingung nicht, wird dadurch
              die Analyse des Algorithmus unn"otig un"ubersichtlich}
\[
   \exists \, p \in \Nat : \: p * \epsilon = 0.5 \MyPunkt
\]

Die Wahl von $\epsilon$ beeinflu"st das Verh"altnis zwischen der Anzahl
der Schritte und der Anzahl der dabei besch"aftigten Prozessoren.
Dies wird weiter unten durch die Analyse deutlich.

F"ur den Rest dieses Unterkapitels gelte die Vereinbarung, da"s mit
\[ a^b \] der Wert \[ \lc a^b \rc \] gemeint ist.

Mit Hinweis auf die Bemerkungen im Anschlu"s an die Behandlung der
Matrizenmultiplikation in Satz \ref{SatzAlgMatMult} wird im folgenden
f"ur die Multiplikation zweier $n \times n$-Matrizen ein Aufwand von
\[ \gamma_S (\lc \log(n) \rc + 1) \] Schritten und
\[ \gamma_P n^{2+\gamma} \] Prozessoren in Rechnung gestellt.

Im folgenden ist mit $T$, $U$ und $V$ jeweils $T_i$, $U_i$ bzw. $V_i$
gemeint, wobei $1 \leq i < n$ gilt.

Um den Vektor $U$ zu berechnen, benutzen wir folgenden iterativen 
Algorithmus\footnote{zur Vereinfachung der Darstellung werden keine 
ganzzahligen Werte zur Indizierung benutzt}: \nopagebreak[3]
\begin{itemize}
\item
      Der Vektor $Z_\alpha$ wird wie folgt definiert:
      \[ 
         Z_\alpha := \left[ R_i,\, R_i M_i,\, R_i M_i^2,\, 
                           \ldots ,\, R_i M_i^{m^\alpha - 1} 
                     \right]
      \]
      Das bedeutet, es gilt
      \[
          Z_0 = [ R_i ]
      \]
      Das Ziel ist es, $Z_{0.5}= U$ zu berechnen. 
      Der Vektor $Z_0$ ist bekannt, da $R_i$ Teil der Eingabe ist.
\item
      Wenn $Z_{\alpha}$ bekannt ist, im ersten Schleifendurchlauf
      also $Z_0$, dann wird daraus $Z_{\alpha+\epsilon}$ wie folgt 
      berechnet:
      \begin{itemize}
      \item
            Berechne
            \[ Y_{\alpha+\epsilon} :=
               \left[
                   M_i^{m^\alpha},\, M_i^{2m^\alpha},\, M_i^{3m^\alpha},
                      \, \ldots,\,
                   M_i^{m^\epsilon m^\alpha}
               \right]
            \]
            Nach \ref{SatzAlgPraefix} in Verbindung mit \ref{SatzAlgMatMult}
            erh"alt man f"ur die Anzahl der Schritte
            \begin{eqnarray*}
               & & \gamma_S \lceil \log(m^\epsilon) \rceil 
                   (\lceil \log(m) \rceil +1)
            \\ & \leq & %  <  ist hier falsch !
                        \gamma_S \lceil (\epsilon \lceil \log(m) \rceil + 1)
                            (\lceil \log(m) \rceil + 1 )
                     \rceil
            \\ & = & \gamma_S \lc \epsilon \lceil \log(m) \rceil^2 +
                         (\epsilon + 1) \lceil \log(m) \rceil + 1
                     \rc
            \end{eqnarray*}
            und f"ur die Anzahl der
            Prozessoren 
            \[ \gamma_P \lf 0.75 m^\epsilon \rf m^{2+\gamma}
                   <
               \gamma_P m^{2+\gamma+\epsilon} \MyPunkt
            \]
            Die
            daf"ur n"otige Startmatrix $M_i^\alpha$ erh"alt man als
            Nebenergebnis aus der Berechnung von $Y_\alpha$. Die Startmatrix
            f"ur die Berechnung von $Y_\epsilon$ ist $M_i$.
      \item
            Der Vektor $X_{\alpha+\epsilon}$ wird folgenderma"sen definiert:
            \[
               X_{\alpha+\epsilon} := 
               \left[
                   R_i M_i^{m^\alpha}, \,R_i M_i^{m^\alpha + 1}, \,
                   R_i M_i^{m^\alpha + 2}, \, \ldots, 
                   \, R_i M_i^{m^{\alpha+\epsilon}-1}
               \right]
            \]
            Es wird nun $X_{\alpha+\epsilon}$ aus $Z_\alpha$ und 
            $Y_{\alpha+\epsilon}$ berechnet.

            Der Vektor
            $Z_\alpha$ besitzt $m^\alpha$ Elemente, die ihrerseits Vektoren
            der L"ange $m$ darstellen. Sie werden mit
            \[
               z_{\alpha,1},\, z_{\alpha,2},\, \ldots,\, z_{\alpha,m^\alpha}
            \] bezeichnet.
            Der Vektor $Y_{\alpha+\epsilon}$ besitzt $m^\epsilon$
            Elemente. Diese Elemente sind $m \times m$-Matrizen und werden
            mit
            \[ y_{\alpha+\epsilon,1},\, y_{\alpha+\epsilon,2},\, \ldots,\,
               y_{\alpha+\epsilon,m^\epsilon}
            \] bezeichnet.
           
            \begin{tabbing}
                Der Vektor $X_{\alpha+\epsilon}$ wird wie folgt
                berechnet: \\
                    \hspace{1.5em} \= \hspace{1.5em} \= \kill 
                \> Parallel f"ur $i:= 1$ bis $m^\epsilon-1$ : \\
                \> \>  Parallel f"ur $j:= 1$ bis $m^\alpha$:
            \end{tabbing}
            \vspace{-4ex}
            \[
               x_{ \alpha+\epsilon,(i-1)*m^\epsilon+j}
                 := z_{\alpha,j} * y_{\alpha+\epsilon,i}
            \]
            Bei dieser Berechnung f"allt auf, da"s 
            $y_{\alpha+\epsilon,m^\epsilon}$ nicht verwendet wird. Diese 
            Matrix bildet die Startmatrix f"ur die Berechnung von
            $Y_{\alpha+2\epsilon}$ im n"achsten Schleifendurchlauf
            (s. o.).

            F"ur die Analyse des Aufwandes der Berechnung von 
            $X_{\alpha+\epsilon}$ wird $Z_\alpha$ als Matrix betrachtet.
            Die $z_{\alpha,j}$ bilden die Zeilenvektoren dieser Matrix.
            So gesehen sind also $m^\epsilon$ Matrizenmultiplikationen 
            durchzuf"uhren. Dies kann von 
            \[ \gamma_P m^{2+\gamma+\epsilon} \]
            Prozessoren in 
            \[ \gamma_S \lceil \log(m) \rceil + 1 \] 
            Schritten durchgef"uhrt werden.
      \item 
            Die ersten $m^\alpha$ Elemente des in diesem 
            Schleifendurchlauf gesuchten Vektors $Z_{\alpha+\epsilon}$
            werden durch die Elemente des Vektors $Z_\alpha$ gebildet und
            alle weiteren durch die Elemente des soeben berechneten 
            Vektors $X_{\alpha+\epsilon}$. 
      \end{itemize}
      Betrachtet man den Aufwand zur Berechnung von $Y_{\alpha+\epsilon}$
      und $X_{\alpha+\epsilon}$ zusammen, erh"alt man f"ur die
      Berechnung von $Z_{\alpha+\epsilon}$ aus $Z_{\alpha}$
      \[ 
         \gamma_S \lc \epsilon \lceil \log(m) \rceil^2 +
            (\epsilon + 2) \lceil \log(m) \rceil + 2                      
         \rc
      \]
      Schritte und \[ \gamma_P m^{2+\gamma+\epsilon} \] Prozessoren.
\item
      Insgesamt erfolgen \[ \frac{1}{2 \epsilon} \]
      Schleifendurchl"aufe. Der Aufwand zur Berechnung von $U$ betr"agt
      deshalb
      \begin{eqnarray*}
         & & \frac{ 0.5 \gamma_S }{ \epsilon }
         \lc \epsilon \lceil \log(m) \rceil^2 +
            (\epsilon + 2) \lceil \log(m) \rceil + 2                      
         \rc
      \\ 
         & \leq & 0.5 \gamma_S
         \lc \lceil \log(m) \rceil^2 +
            \left( 1 + \frac{2}{\epsilon} \right) \lceil \log(m) \rceil + 
            \frac{2}{\epsilon}
         \rc
      \end{eqnarray*} Schritte und
      \[ \gamma_P m^{2+\gamma+\epsilon} \] Prozessoren.
\end{itemize}

Im Anschlu"s an die Berechnung von $U$ erfolgt
die Berechnung von $V$ auf die gleiche Weise.
Der einzige wesentliche Unterschied zwischen den beiden
Berechnungsvorg"angen ist die andere
Startmatrix zur Berechnung des $Y_{\epsilon}$ entsprechenden Vektors. Hier
wird $M_i^{m^{0.5}}$ statt $M_i$ ben"otigt. Man erh"alt $M_i^{m^{0.5}}$
aus $M_i$ mit Hilfe der Bin"arbaummethode
nach \ref{SatzAlgBinaerbaum}. Dies kann in
\begin{eqnarray*}
    &      & \gamma_S \lc \log(m^{0.5}) \rc (\lc \log(m) \rc + 1)
\\  & \leq & \gamma_S \lc 0.5 \lc \log(m) \rc (\lc \log(m) \rc + 1) \rc
\\  & =    & \gamma_S \lc 0.5 (\lc \log^2(m) \rc + \lc \log(m) \rc + 1)  \rc
\end{eqnarray*}
Schritten von
\begin{eqnarray*}
    &      & \gamma_P \lc 0.5 m^{0.5} m^{2+\gamma} \rc
\\  & \leq & \gamma_P \lc 0.5 m^{2.5+\gamma} \rc
\end{eqnarray*}
Prozessoren geleistet werden.
Ist die Startmatrix berechnet, ist der weitere Aufwand zur Berechnung von
$V$ gleich dem Aufwand zur Berechnung von $U$. Also kann $V$ insgesamt 
in\footnote{Da die Terme, die die Anzahl der Schritte und Prozessoren
    beschreiben, bereits nach oben abgesch"atzt sind, wird bei der
    Zusammenfassung von Termen, die durch Gau"sklammern eingefa"st sind,
    auf eine weitere Absch"atzung verzichtet.}
\begin{eqnarray*}
   & & \gamma_S \lc 0.5 (\lc \log(m) \rc^2 + \lc \log(m) \rc + 1) +
           0.5  \left( \lceil \log(m) \rceil^2 +
           \left( 1 + \frac{2}{\epsilon} \right) \lceil \log(m) \rceil +
           \frac{2}{\epsilon} \right)
       \rc
\\ & = & \gamma_S
     \lc
         \lc \log(m) \rc^2 + 
         \left( 1 + \frac{1}{\epsilon} \right) \lc \log(m) \rc + 
         \frac{1}{\epsilon} + 0.5
     \rc
\end{eqnarray*}
Schritten erledigt werden. Die Anzahl der Prozessoren betr"agt
\begin{eqnarray*}
    \max \left( 
              \underbrace{ \gamma_P m^{2+\gamma+\epsilon} }_{\mbox{Term 1}}
         \, ,
              \underbrace{ \gamma_P \lc 0.5 m^{2.5+\gamma} \rc 
                         }_{\mbox{Term 2}}
         \right) \MyPunkt
\end{eqnarray*}
Da mit steigendem $m$ Term 2 st"arker w"achst als Term 1, wird die 
Analyse mit Term 2 f"ur die Anzahl der Prozessoren fortgesetzt.

Parallel zur Berechnung von $U$ wird zuerst
$M_i^{m^{0.5}}$ und mit Hilfe dieser Matrix dann $V$ berechnet. Der Aufwand
daf"ur betr"agt insgesamt
\[   \gamma_S
     \lc
         \lc \log(m) \rc^2 + 
         \left( 1 + \frac{1}{\epsilon} \right) \lc \log(m) \rc + 
         \frac{1}{\epsilon} + 0.5
     \rc
\] 
Schritte und
\[ 
   \gamma_P \lb m^{2+\gamma+\epsilon} + \lc 0.5 m^{2.5+\gamma} \rc \rb
\] Prozessoren.

Um den nach \equref{EquRMSVektor} gesuchten Vektor $T$ zu erhalten, 
m"ussen 
noch die Elemente der Vektoren $U$ und $V$, die ja ihrerseits wiederum 
Vektoren darstellen, miteinander multipliziert werden.
Die Vektoren $U$ und $V$ besitzen eine L"ange von $m^{0.5}$.
Die Multiplikation zweier Elemente dieser Vektoren k"onnen
analog zur Matrizenmultiplikation in \ref{SatzAlgMatMult} in 
\[ \lc \log(m) \rc + 1 \] Schritten von \[ m \] Prozessoren erledigt 
werden. Insgesamt sind \[ m^{0.5} * m^{0.5} = m \] solcher Multiplikationen
durchzuf"uhren. Die Berechnung von $T$ aus $U$ und $V$ kann
also in 
\[ \lc \log(m) \rc + 1 \] Schritten von \[ m^2 \] Prozessoren 
durchgef"uhrt werden.

Betrachtet man den Gesamtaufwand zur Berechnung von $U$, $V$ und $T$,
kommt man auf
\[ \gamma_S
   \lc
       \lc \log(m) \rc^2
       + \left( 1 + \frac{1}{\gamma_S} 
       + \frac{1}{\epsilon} \right) \lc \log(m) \rc
       + \frac{1}{\epsilon} + \frac{1}{\gamma_S} + 0.5
   \rc
\] 
Schritte und  
\Beq{EquBerkProzT}
   \gamma_P \lb m^{2+\gamma+\epsilon} + \lc 0.5 m^{2.5+\gamma} \rc \rb
      \leq
   \gamma_P \lb m^{2+\gamma+\epsilon} + 0.5 m^{2.5+\gamma} + 1 \rb
\Eeq
Prozessoren. 

Nach der
obigen Analyse der Berechnung einer der Vektoren kann die parallele
Berechnung aller Vektoren $T_1$ bis $T_{n-1}$ in
\Beq{TermBerkSchritte}
   \gamma_S
   \lc
       \lc \log(n-2) \rc^2 
       + \left( 1 + \frac{1}{\gamma_S}
       + \frac{1}{\epsilon} \right) \lc \log(n-2) \rc + 
       \frac{1}{\epsilon} + \frac{1}{\gamma_S} + 0.5
   \rc
\Eeq
Schritten durchgef"uhrt werden. Da die Berechnung eines Vektors $T_i$ f"ur 
$i>1$ bei gleichem $\epsilon$ schneller ist als die Berechnung von 
$T_1$, ist es m"oglich, dadurch Prozessoren zu sparen, da"s man 
$\epsilon$ f"ur jeden Vektor $T_i$ verschieden w"ahlt, und zwar als Funktion
von 
\begin{itemize}
\item
      der Gr"o"se $n$ der Eingabematrix $A$,
\item 
      der L"ange $m+1$ des jeweiligen Vektors $T_i$ und
\item
      dem $\epsilon$, da"s zur Berechnung des Vektors $T_1$
      verwendet wird.
\end{itemize}
Das separat f"ur jeden Vektor $T_i$ zu w"ahlende $\epsilon$ wird 
mit\footnote{ Es wurde bereits definiert: $m:= n-i-1$.}
$\epsilon_m$ bezeichnet.

Da die Vektoren $T$ f"ur $m \leq n-2$ berechnet werden sollen, mu"s f"ur
jedes $\epsilon_m$ mit $\epsilon_m \neq \epsilon$ die 
Bedingung $m \leq n-3$ erf"ullt sein.
Da gleichzeitig $m \geq 1$ erf"ullt sein mu"s, wird f"ur die folgenden
Analysen $n \geq 4$ angenommen. Andernfalls ist die Anwendung der Idee
zur Wahl der $\epsilon_m$ nicht sinnvoll.

Wie $\epsilon_m$ zu w"ahlen ist, ergibt sich aus
Term \equref{TermBerkSchritte}. Es mu"s gelten:
\[
   \lc \log(m) \rc^2 
   + \left( 1 + \frac{1}{\gamma_S} 
   + \frac{1}{\epsilon_m} \right) \lc \log(m) \rc +
   \frac{1}{\epsilon_m}
       \leq
   \lc \log(n-2) \rc^2 
   + \left( 1 + \frac{1}{\gamma_S} 
   + \frac{1}{\epsilon} \right) \lc \log(n-2) \rc +
   \frac{1}{\epsilon}
\]
L"ost man diese Ungleichung nach $\epsilon_m$ auf erh"alt man:
\Beq{EquWaehleEpsilonM}
   \epsilon_m
       \geq
   \frac{
       \lceil \log(m) \rceil + 1
   }{
       \lceil \log(n-2) \rceil^2 +
       \lb 1 + \frac{1}{\gamma_S} 
              + \frac{1}{\epsilon} 
       \rb \lceil \log(n-2) \rceil +
       \frac{1}{\epsilon} -
       \lceil \log(m) \rceil^2 -
       \lb 1 + \frac{1}{\gamma_S} \rb \lceil \log(m) \rceil
   }
\Eeq
Die Gau"sklammern in dieser Ungleichung f"uhren zu einigen wichtigen
Konsequenzen f"ur $n$, $m$ und $\epsilon_m$. Es gelte dazu 
\begin{eqnarray*}
    & k \in \Nat & \\
    & 1 \leq 2^k < m_1 \leq 2^{k+1} \leq n-2 & \\
    & 1 \leq 2^k < m_2 \leq 2^{k+1} \leq n-2 & \MyPunkt
\end{eqnarray*}
Aus \equref{EquWaehleEpsilonM} folgt dann
\[
    \epsilon_{m_1} = \epsilon_{m_2} \MyPunkt
\]
Das bedeutet insbesondere, da"s es u. U. einige $m$ mit 
$m \leq n-3$ gibt, f"ur die gilt 
\[ \epsilon_m = \epsilon \MyPunkt \] Der ung"unstigste Fall tritt f"ur
\[ n-2 = 2^{k+1} \] ein. Bei diesem Fall ist nur f"ur 
\[ m \leq \frac{n-2}{2} \]
die Bedingung \[ \epsilon_m < \epsilon \] erf"ullt.

F"ur die weitere Analyse ist es an dieser Stelle sinnvoll, die Gau"sklammern
im Term auf der rechten Seite von \equref{EquWaehleEpsilonM} zu beseitigen. 
Dazu wird
der Term nach oben abgesch"atzt.

Terme, die in Gau"sklammern eingefa"st sind, kann man mit Hilfe der
Beziehung
\Beq{EquSchaetzeGauss}
    a \leq \lc a \rc \leq a + 1
\Eeq
absch"atzen. Es gilt jedoch
\begin{eqnarray*}
   &      & \lc \log(x) \rc \\
   & \leq & \log(x) + 1     \\
   & =    & \log(2x) \MyPunkt
\end{eqnarray*} 
Zu beachten sind hier die Konsequenzen, wenn die Absch"atzung mit Hilfe
von \equref{EquSchaetzeGauss} vorgenommen werden. Falls $n-2$ eine 
Zweierpotenz ist, ergibt die auf diese Weise abgesch"atzte Ungleichung 
\equref{EquWaehleEpsilonM} nur
f"ur \[ m \leq \frac{n-2}{4} \] Werte f"ur $\epsilon_m$, so da"s 
\[ \epsilon_m < \epsilon \MyPunkt \]
Eine Verbesserung dieser Absch"atzung der Gau"sklammerfunktion ist 
w"unschenswert.

Dazu wird definiert, da"s eine Funktion $f(x)$
\index{konkav} {\em konkav auf einem Intervall I} ist, falls f"ur ihre
zweite Ableitung $f''(x)$ gilt:
\[ \forall x \in I: \: f''(x) \leq 0 \MyPunkt \]

Soll die Gau"sklammer einer konkaven Funktion $h(x)$ gebildet und die
Fl"ache unter der resultierenden Kurve berechnet werden, so l"a"st sich
der Ausdruck auch durch
\[ \int \lc h(x) dx \rc \leq \int (h(x) + 0.5) dx \]
nach oben absch"atzen. In Abbildung \ref{PicKonkav} ist dies verdeutlicht.
Dort ist Fl"ache 1 gr"o"ser als Fl"ache 2.
\begin{figure}[htb]
\begin{center}
    %
% You need 'bezier.sty'
% You need 'epic.sty'
%
\setlength{\unitlength}{1mm}
\makeatletter
\def\Thicklines{\let\@linefnt\tenlnw \let\@circlefnt\tencircw
\@wholewidth4\fontdimen8\tenln \@halfwidth .5\@wholewidth}
\makeatother
\begin{picture}(100.00,75.00)
\drawline(95.00,69.75)(70.25,69.75)(70.25,39.75)(15.00,39.75)(15.00,9.75)(5.00,9.75)
\bezier{100}(15.00,24.75)(40.00,54.75)(70.25,54.75)
\put(50.25,45.00){Fl"ache 1}
\bezier{100}(15.00,9.75)(40.00,39.75)(70.25,39.75)
\put(19.50,35.50){\vector(1,-4){0}}
\drawline(19.50,35.50)(15.50,52.00)
\put(10.75,52.75){Fl"ache 2}
\put(70.75,54.75){\vector(-4,3){0}}
\drawline(70.75,54.75)(81.25,46.75)
\put(70.50,39.75){\vector(-2,-1){0}}
\drawline(70.50,39.75)(81.25,45.50)
\put(82.50,44.75){$+0.5$}
\end{picture}
%

    \caption{Integration der Gau"sklammer einer konkaven Funktion}
    \label{PicKonkav}
\end{center}
\end{figure}
Somit kommen wir auf
\begin{eqnarray}
    & & \int \lc \log(x) \rc dx \nonumber \\
    & \leq & \int (\log(x) + 0.5 ) dx \label{EquLogNullFuenf} \\
    & = & \int \log(\sqrt{2}x) dx \nonumber \MyPunkt
\end{eqnarray}
Ist der abzusch"atzende Gau"sklammerterm Teil einer Funktion 
\[ h_2(\lc h(x) \rc,\, x ) \MyKomma \] so l"a"st sich die beschriebene
Absch"atzung durchf"uhren, falls $h_2$ monoton ist. Diese Bedingung ist bei 
den folgenden Anwendungen erf"ullt.

F"ur die folgenden Untersuchungen wird die Funktion
\[ g : \, \Rationals \rightarrow \Rationals \] eingef"uhrt. Sie sei
{ \em im Riemann'schen Sinne integrierbar } \cite{BS87} (S. 289)
auf dem Intervall \[ ( - \infty, \infty ) \] und
wird als Platzhalter f"ur die Funktion verwendet, die schlie"slich zur 
Absch"atzung der Gau"sklammerfunktion benutzt wird.
Je genauer sie die Gau"sklammerfunktion absch"atzt, umso besser werden
die Analyseergebnisse.

Mit Hilfe von \equref{EquWaehleEpsilonM} wird folgende Funktion zur 
Berechnung von $\epsilon_m$ bei gegebenen $n$ und $\epsilon$ definiert:
\[ f(m):=
   \frac{
       g(\log(m))+ 1
   }{
       g^2(\log(n-2))
       + \lb 1 + \frac{1}{\gamma_S} 
               + \frac{1}{\epsilon} 
         \rb g(\log(n-2))
       + \frac{1}{\epsilon} -
       g^2(\log(m))-
       \lb 1 + \frac{1}{\gamma_S} \rb g(\log(m))
   }
\]
Mit Hilfe dieser Funktion kommt man anhand von \equref{EquBerkProzT}
f"ur die Anzahl der Prozessoren zur Berechnung aller Vektoren $T$ auf:
\begin{eqnarray}
\nonumber
   & & 
   4 + (n-2)^{2+\gamma+\epsilon} + \frac{ (n-2)^{2.5+\gamma} }{ 2 } +
   \sum_{m=2}^{n-3}
       \left( m^{2+\gamma+f(m)} + \frac{ m^{2.5+\gamma} }{ 2 } + 1
       \right)
\\ 
\nonumber
   & \leq & 
   \overbrace{
       4 + (n-2)^{2+\gamma+\epsilon} + \frac{ (n-2)^{2.5+\gamma} }{ 2 }
   }^{t_1:=}
       +
   \int\limits_{2}^{n-2}
       \left( m^{2+\gamma+f(m)} + \frac{ m^{2.5+\gamma} }{ 2 } + 1
       \right) dm
\\ 
%%\label{EquBerkProzInt1} wird nicht ben"otigt
   & = &
       \overbrace{ 
           t_1 + \left. m \right|_2^{n-2} +
           \left. \left(
               \frac{1}{7+2\gamma} m^{3.5+\gamma}
           \right) \right|_2^{n-2} 
       }^{t_2:=}
       +
       \int\limits_2^{n-2}
           m^{2+\gamma+f(m)} dm
\\
\label{EquBerkProzInt2}
   & = & t_2 +
       \int\limits_2^{n-2}
       \overbrace{
           \MathE^{(2+\gamma+f(m))\ln(m)}
       }^{t_3:=} \, dm
\end{eqnarray}
F"ur die Integration von $t_3$ sind 3 Methoden von Bedeutung:
\begin{MyDescription}
\MyItem{Numerische Berechnung}
    Diese Methode ist f"ur gegebene $n$ und $\epsilon$ eine gangbare
    M"oglichkeit \cite{EM88} (Kap. 9). F"ur eine allgemeine 
    Analyse ist sie jedoch nicht geeignet.
\MyItem{Analytische Berechnung}
    Bezeichne $t_4$ die erste Ableitung von 
    \[ (2+\gamma+f(m))\ln(m) \MyPunkt \]
    Bezeichne $t_5'$ die erste Ableitung des zu bestimmenden Terms $t_5$.
    Da $t_3$ eine Exponentialfunktion ist, mu"s die gesuchte Stammfunktion,
    die Form $t_3 t_5$ besitzen.
    Die Ableitung dieses Ausdrucks ergibt 
    \[ (t_3 t_5)'= t_3 t_4 t_5+t_3 t_5' = t_3 \,
       \underline{ (t_4 t_5 + t_5') } \MyPunkt 
    \]
    Da der unterstrichene Teil den Wert 1 besitzen mu"s, ist die 
    Differentialgleichung \[ t_5' = 1 - t_4 t_5 \] zu l"osen. 
    Dies ist eine {\em explizite gew"ohnliche Differentialgleichung 
    erster Ordnung} \cite{BS87} (S. 414 ff.).

    Man gelangt zu der Vermutung, da"s zu $t_3$ keine Stammfunktion 
    existiert, da sowohl die Integration der Differentialgleichung, 
    als auch die Integration von $t_3$ mit Hilfe der Eigenschaften
    unbestimmter Integrale \cite{BS87} (S. 295) nicht zu 
    einem Ergebnis zu f"uhren scheinen. Unterst"utzt wird diese Vermutung
    durch die Tatsache, da"s f"ur \[ \MathE^{m^2} \] keine Stammfunktion 
    existiert.
\MyItem{Absch"atzung der Stammfunktion nach oben}
    Diese Methode ist am besten geeignet und wird im folgenden benutzt.
    Dazu wird der oben erw"ahnte Term $t_5$ so bestimmt, da"s gilt
    \Beq{EquBerkUngleichung}
        t_4 t_5 + t_5' \geq 1 \MyPunkt
    \Eeq 
\end{MyDescription}

Der Nenner von $f(m)$ wird mit $t_6$ bezeichnet, der Z"ahler mit $t_7$.
Die Ableitung von $t_3$ ergibt:
\begin{eqnarray*}
   \lefteqn{ 
       t_3' = t_3 
       \lb
           \frac{2}{m} + \frac{\gamma}{m}
       \rb
   } \\
   & + &
       \frac{
           \lb 
               g'(\log(m)) \frac{\log(m)}{m} + \frac{ g(\log(m)) }{m} 
               + \frac{1}{m} 
           \rb t_6
       }{ t_6^2 
       } \\
   & + &
       \frac{
           t_7 \ln(m) 
           \lb \frac{ 2 g(\log(m)) g'(\log(m)) }{m \ln(2) }
               + \frac{ \lb 1 + \frac{1}{\gamma_S} \rb
                        g'(\log(m)) 
                 }{ m \ln(2) }
           \rb
       }{ t_6^2
       } 
\end{eqnarray*}
Wie bereits beschrieben wurde, ist es an dieser Stelle m"oglich
\Beq{EquDefGaussNullFuenf}
    g(x) := x + 0.5
\Eeq
als obere Absch"atzung der Gau"sklammerfunktion zu verwenden. Es ist zu
beachten, da"s diese Absch"atzung nicht f"ur die Berechnung von $\epsilon_m$
bei der Anwendung des Algorithmus in einer konkreten Situation
benutzt werden darf. Die Benutzung von \equref{EquDefGaussNullFuenf} an
dieser Stelle ist nur zul"assig, weil die Absch"atzung des gesamten 
Integrals in \equref{EquBerkProzInt2} das Ziel ist.

Wie ebenfalls bereits beschrieben wurde, ist darauf zu achten, da"s die
auftretenden Werte f"ur $\epsilon_m$ kleiner oder gleich $\epsilon$ sind.
Damit dies der Fall ist muss wegen der mit \equref{EquDefGaussNullFuenf} 
gew"ahlten Absch"atzung gelten:
\begin{MyEqnArray}
    \MT  \log( \sqrt{2}m ) \MT \leq \MT \lc \log( n - 2 ) \rc 
\MNl
    \Rightarrow \MT \log(m) \MT \leq \MT
        \lc \log \lb \frac{ n - 2 }{ \sqrt{2} } \rb \rc
\MNl
    \Rightarrow \MT m \MT \leq \MT 
        2^{ \lc \log \lb \frac{ n - 2 }{ \sqrt{2} } \rb \rc }
\end{MyEqnArray}

Ungleichung
\equref{EquBerkUngleichung} wird erf"ullt, wenn man 
\[ t_5 = m \] w"ahlt. Die G"ultigkeit dieser Behauptung ergibt
sich insbesondere aus
der Betrachtung der Gr"o"senordnungen der Z"ahler und Nenner in der
Ableitung von $t_3$.

So erh"alt man durch Absch"atzung von \equref{EquBerkProzInt2} nach oben
f"ur die Anzahl der Prozessoren:
\begin{eqnarray*}
    & & t_2 + 
        \int\limits_{ 2^{\lf \log \lb \frac{n-2}{\sqrt{2}} \rb \rf} }^{n-2} 
            m^{2+\gamma+0.5} dm
        + \int\limits_2^{ 2^{\lf \log \lb \frac{n-2}{\sqrt{2}} \rb \rf} }
            \MathE^{ (2+\gamma+f(m)) \ln(m) } dm \\
    & \leq & t_2 + 
        \left. \frac{1}{3.5+\gamma} m^{3.5+\gamma} 
        \right|_{ 2^{\lf \log \lb \frac{n-2}{\sqrt{2}} \rb \rf} }^{n-2}
        + \left. \lb m^{ 2+\gamma+f(m) } m \rb
          \right|_2^{ 2^{\lf \log \lb \frac{n-2}{\sqrt{2}} \rb \rf} } \\
    & = &  4 + (n-2)^{2+\gamma+\epsilon} + 
               \frac{ (n-2)^{2.5+\gamma} }{ 2 } \\
    & & 
         + \left. m \right|_2^{n-2} +
           \left. \left(
               \frac{1}{7+2\gamma} m^{3.5+\gamma}
           \right) \right|_2^{n-2} 
        +  
        \left. \frac{1}{3.5+\gamma} m^{3.5+\gamma} 
        \right|_{ 2^{\lf \log \lb \frac{n-2}{\sqrt{2}} \rb \rf} }^{n-2} \\
    & & +
         \left. m^{3+\gamma+f(m)}
         \right|_2^{ 2^{\lf \log \lb \frac{n-2}{\sqrt{2}} \rb \rf} }
\end{eqnarray*}
Dieser Term wird mit $t_8$ bezeichnet. An ihm erkennt man, da"s die
Anzahl der Prozessoren mit wachsendem $n$ asymptotisch
\[
    \frac{3}{7+2\gamma} (n-2)^{3.5+\gamma}
\]
betr"agt.

Schlie"slich m"ussen noch die Matrizen $C_1$ bis $C_n$ miteinander
multipliziert werden. Dies geschieht mit Hilfe der Bin"arbaummethode
nach \ref{SatzAlgBinaerbaum}. Es wird definiert
\[ n':= \lf \frac{n}{2} \rf \MyPunkt \]
Da nach \ref{SatzToeplizMult}
bei allen Multiplikationen Dreiecks-Toeplitz-Matrizen verkn"upft werden
und $C_i$ eine $(n-i+2) \times (n-i+1)$-Matrix handelt, k"onnen diese
Multiplikationen in
\Beq{TermBerkSchritteB}
   (\lc \log(n+1) \rc + 1) \lc \log(n) \rc
\Eeq Schritten von weniger als
\begin{eqnarray*}
    & & \sum_{k=1}^{n'} (n-2k+2) * (n-2k+1) \\
    & = & \sum_{k=1}^{n'} (2k(2k-1)) \\
    & = & 4 \sum_{k=1}^{n'} k^2 - 2 \sum_{k=1}^{n'} k \\
    & \MyStack{\ref{SatzSumK}, \, \ref{SatzSumK2} }{ = } & 
        4 \frac{n'(n'+1)(2n'+1)}{6} - 2\frac{n'(n'+1)}{2}
\end{eqnarray*}
Prozessoren durchgef"uhrt werden. F"ur ein gegebenes $n$ ist
der Wert dieses Terms ist kleiner als
der Wert von $t_8$.

In Verbindung mit \ref{SatzDdurchP}
ergibt sich als Endergebnis der Analyse, da"s mit Hilfe des in diesem 
Kapitel vorgestellen Algorithmus die Determinante einer $n \times n$-Matrix
in weniger als\footnote{Summe von \equref{TermBerkSchritte} und
 \equref{TermBerkSchritteB}}
\[
   \gamma_S
   \lc
       \lc \log(n-2) \rc^2 
       + \left( 1 + \frac{1}{\gamma_S}
       + \frac{1}{\epsilon} \right) \lc \log(n-2) \rc + 
       \frac{1}{\epsilon} + \frac{1}{\gamma_S} + 0.5
   \rc +
   (\lc \log(n+1) \rc + 1) \lc \log(n) \rc
\]
Schritten von weniger als $t_8$ Prozessoren berechnet werden kann.

Da in der Praxis f"ur die Matrizenmultiplikation Satz 
\ref{SatzAlgMatMult} statt der in \cite{CW90} angegebenen Methode 
benutzt wird, ist f"ur diesen Fall in allen obigen Termen
\[ \gamma = \gamma_S = \gamma_P = 1 \] zu setzen.

Vergleicht man B-Alg. mit C-Alg., BGH-Alg. und P-Alg., f"allt wiederum
das Fehlen von Fallunterscheidungen auf. Weiterhin werden wie bei BGH-Alg.
keine Divisionen verwendet, so da"s B-Alg. auch in Ringen anwendbar ist.

Betrachtet man die Aufwandsanalyse, so erkennt man, da"s B-Alg. 
leicht schlechter ist als C-Alg. und deutlich besser als BGH-Alg. .
In Kapitel \ref{ChapPan} wird P-Alg. in diese Rangfolge eingereiht.


%
% Datei: pan.tex
%
\MyChapter{Der Algorithmus von Pan}
\label{ChapPan}
%\label{SecIteration}
In diesem Kapitel wird der Algorithmus von V. Pan \cite{Pan85} zur 
Determinantenberechnung vorgestellt. Er kommt ebenfalls ohne
Divisionen\footnote{vgl. Bemerkungen in \ref{SecAlgFrame}}
aus und berechnet die Determinante iterativ. Auf diesen Algorithmus wird
mit {\em P-Alg.} Bezug genommen\footnote{vgl. Unterkapitel \ref{SecBez}}.

Man erh"alt insbesondere durch Variation der in den Unterkapitel
\ref{SecGuessInverse} und \ref{SecNewton} dargestellten Inhalte
einige weitere Versionen des Algorithmus.
F"ur eine vollst"andige Darstellung m"ussen alle diese Varianten beschrieben
und auf ihre Effizienz hin untersucht werden\footnote{N"aheres dazu ist 
insbesondere in \cite{PR85a} zu finden}. Da dies jedoch den Rahmen 
dieses Textes sprengt, beschr"anken sich die folgenden Darstellungen auf
die effizienteste Version des Algorithmus.

P-Alg. bietet erheblich mehr Variationsm"oglichkeiten als C-Alg., BGH-Alg.
und B-Alg., die, wie erw"ahnt, nicht alle hier behandelt werden k"onnen.
Vergleicht man P-Alg. von seiner Methodik her mit den drei
anderen, so erkennt man "Ahnlichkeiten zu BGH-Alg. . In beiden Algorithmen
werden mehrere auch separat bedeutsame teilweise schon l"anger bekannte
Verfahren zusammen verwendet. Von diesen Verfahren hebt sich lediglich
die in Unterkapitel \ref{SecGuessInverse} dargestellte Methode zur
Berechnung einer N"aherungsinversen ab. Sie wurde in \cite{Pan85} erstmalig
ver"offentlicht.

% **************************************************************************

\MySection{Diagonalisierbarkeit}

In diesem Unterkapitel wird die Diagonalisierbarkeit von Matrizen behandelt.
Es ist f"ur das Verst"andnis des in diesem Kapitel dargestellten Algorithmus
zur Determiantenberechnung nicht unbedingt erforderlich und kann daher beim 
Lesen auch "ubersprungen werden. Im folgenden werden jedoch einige 
Hintergr"unde der im Unterkapitel \ref{SecKrylov} dargestellten Methode von 
Krylov n"aher beleuchtet, die ein paar Zusammenh"ange klarer werden lassen.

Literatur zu diesem 
Thema ist neben den in Kapitel \ref{ChapBase} genannten Stellen
auch \cite{Zurm64} S. 169 ff .

Zum Problem der Diagonalisierbarkeit\footnote{Definition s. u.} gelangt
man "uber den Begriff der Basis eines Vektorraumes. Da es sich dabei
um Grundlagen der Linearen Algebra handelt, erfolgt die Darstellung 
vergleichsweise oberfl"achlich.

Sei $K$ ein K"orper und $V$ ein $K$-Vektorraum.
Seien $k_1,\, k_2, \, \ldots, \, k_n \in K \backslash \{ 0 \} $ 
und $v_1, \, v_2, \, \ldots, \, v_n \in V$.
Dann wird
\Beq{EquLinKomb}
   k_1 v_1 + k_2 v_2 + \ldots + k_n v_n 
\Eeq
als \index{Linearkombination} {\em Linearkombination} der Vektoren 
$v_1$ bis $v_n$ bezeichnet.
Sei $0_m$ der Nullvektor in $V$.
Falls f"ur die Vektoren die Bedingung
\[ k_1 v_1 + k_2 v_2 + \ldots + k_n v_n = 0_m \Rightarrow
   k_1 = k_2 = \ldots = k_n = 0
\]
erf"ullt ist, werden sie als 
{\em linear unabh"angig} \index{linear unabh{\Mya}ngig} bezeichnet, 
ansonsten als {\em linear abh"angig} \index{linear abh{\Mya}ngig}.
Falls jedes Element von $V$ als Linearkombination der Vektoren 
$v_1,\ldots,v_n$
darstellbar ist, werden diese Vektoren als 
\index{Basis} {\em Basis} bezeichnet.

F"ur alle Basen gilt die Aussage: \nopagebreak
\begin{quote}
    Je zwei Basen eines $K$-Vektorraumes bestehen aus dergleichen 
    Anzahl von Vektoren.
\end{quote}
Sei $B$ die Basis des $K$-Vektorraumes $V$. Dann wird die Anzahl 
der Vektoren, die $B$ bilden, 
als {\em Dimension von $V$} \index{Dimension} , 
kurz $\dim(V)$, bezeichnet.

Zur Darstellung von Elementen eines Vektorraumes $V$ w"ahlt man sich eine
Basis und beschreibt jedes Element des Vektorraumes als Linearkombination
der Elemente der Basis. Sei $n$ die Dimension des Vektorraumes. Dann
kann man auf diese Weise jedes Element von $V$ als $n$-Tupel von Elementen
des zugrunde liegenden K"orpers $K$ betrachten. Man erh"alt die
Vektorschreibweise:
\Beq{EquVektorSchreibweise}
   \left[ \begin{array}{c} k_1 \\k_2\\ \vdots\\ k_n \end{array} \right]
\Eeq
Die Basis der Form
\begin{quote} % $$$$ Formatierung gepr"uft ?
   $\left[ \begin{array}{c} 1\\ 0\\ 0\\ \vdots\\ 0\end{array} \right]$,
   \hspace{0.7em}
   $\left[ \begin{array}{c} 0\\ 1\\ 0\\ \vdots\\ 0\end{array} \right]$,
   $\ldots , $ \hspace{0.7em}
   $\left[ \begin{array}{c} 0\\ 0\\ 0\\ \vdots\\ 1 \end{array} \right]$
\end{quote}
wird als
{\em kanonische Basis} \index{kanonische Basis} bezeichnet.

Zur Betrachtung der Beziehungen verschiedener Basen zueinander werden diese
Basen ihrerseits bzgl. der kanonischen Basis dargestellt. 

Wird ein Vektor $v$ bzgl. einer Basis $B$ dargestellt, so wird dies
folgenderma"sen ausgedr"uckt:
\[ \left[
       \begin{array}{c}
           v_1 \\ \vdots \\ v_n
       \end{array}
   \right]_B
\]
Werden Vektoren zu Matrizen zusammengefa"st, so wird die gleiche 
Schreibweise auch f"ur diese Matrizen verwendet.

Sei $V$ ein $K$-Vektorraum der Dimension $n$ und $W$ ein $K$-Vektorraum
der Dimension $m$. Ein Ergebnis der Linearen Algebra lautet, da"s dann
die Menge aller $K$-Vektorraumhomomorphismen\footnote{also die Menge
aller {\em strukurvertr"aglichen} linearen Abbildungen} $f$
\[ f : \: V \rightarrow W \]
isomorph ist zur Menge aller $m \times n$-Matrizen $A$, wenn man die 
Abbildung definiert als\footnote{Dies entspricht der 
Matrizenmultiplikation (vgl. \ref{SatzAlgMatMult}), wenn man den 
abzubildenden Vektor $v$ als $n \times 1$-Matrix 
betrachtet (vgl. \ref{SatzAlgMatMult}).}
\[ f \lb
         \left[
             \begin{array}{c} v_1\\ v_2\\ \vdots\\ v_n \end{array}
         \right]
     \rb
   :=
      \left[
          \begin{array}{c}
              \sum_{j=1}^n a_{1,j} v_j \LMatStrut \\
              \vdots \LMatStrut                   \\
              \sum_{j=1}^n a_{m,j} v_j \LMatStrut
          \end{array}
      \right]
\]

Die Untersuchung der $K$-Vektorraumhomomorphismen kann man
also anhand der entsprechenden Matrien vornehmen. Im folgenden 
sind nur quadratische Matrizen von Interesse. Deshalb werden in den 
weiteren Ausf"uhrungen nur diese Matrizen beachtet.

Stellt man die Vektoren einer Basis $B_V$ bzgl. einer anderen Basis $B_W$
(normalerweise der kanonischen Basis) dar und betrachtet sie als
Spaltenvektoren einer Matrix \[ [B_V]_{B_W} \MyKomma \]
so erkennt man beim Vergleich von
\equref{EquLinKomb} und \equref{EquVektorSchreibweise} miteinander,
da"s man einen bzgl. $B_V$ dargestellten Vektor $x$ in seine Darstellung
bzgl. $B_W$ umrechnen kann durch
\Beq{EquBasiswechsel}
    [x]_{B_W}= B_V[x] \MyPunkt
\Eeq
Man erkennt also, da"s man eine Basis auch als Vektorraumhomomorphismus
betrachten kann. Die umgekehrte Betrachtungsweise ist nat"urlich nicht
m"oglich.

Um zum Begriff der {\em Diagonalisierbarkeit} zu gelangen, betrachten wir
nun, was passiert, wenn man Basen austauscht und die Darstellungen bzgl. der
neuen Basen vornehmen will.

Seien $V$ und $W$ jeweils $K$-Vektorr"aume sowie $B_V$ und $B_W$ jeweils
Basen dieser Vektorr"aume. Sei \[ f: \: V \rightarrow W \] ein
Vektorraumhomomorphismus und $A$ die entsprechende Matrix. Es gelte
\Beq{EquVonVNachW}
    \left[
        \begin{array}{c} y_1\\ y_2\\ \vdots\\ y_n \end{array}
    \right]_{B_W}
    = 
    A
    \left[
        \begin{array}{c} x_1\\ x_2\\ \vdots\\ x_n \end{array}
    \right]_{B_V}
\Eeq
Wechselt man nun zu den Basen $\tilde{B_V}$ und $\tilde{B_W}$ und stellt 
die neuen Basen bzgl. der alten durch die Matrizen $C$ und $D$ dar, so gilt
entsprechend \equref{EquBasiswechsel}:
\begin{eqnarray*}
    \left[
        \begin{array}{c} x_1\\ x_2\\ \vdots\\ x_n \end{array}
    \right]_{B_V} 
    & = & 
    C
    \left[
        \begin{array}{c} 
            \tilde{x_1}\\ 
            \tilde{x_2}\\ \vdots \\
            \tilde{x_n}
        \end{array}
    \right]_{\tilde{B_V}}
\\
    \left[
        \begin{array}{c} y_1\\ y_2\\ \vdots\\ y_n \end{array}
    \right]_{B_W}
    & = & 
    D
    \left[
        \begin{array}{c} 
            \tilde{y_1}\\ 
            \tilde{y_2}\\ \vdots \\
            \tilde{y_n}
        \end{array}
    \right]_{\tilde{B_W}}
\end{eqnarray*}
Gleichung \equref{EquVonVNachW} bekommt also folgendes Aussehen:
\[
    D
    \left[
        \begin{array}{c} 
            \tilde{y_1}\\ 
            \tilde{y_2}\\ \vdots \\
            \tilde{y_n}
        \end{array}
    \right]_{\tilde{B_W}}
    =
    A \: C
    \left[
        \begin{array}{c} 
            \tilde{x_1}\\ 
            \tilde{x_2}\\ \vdots \\
            \tilde{x_n}
        \end{array}
    \right]_{\tilde{B_V}}
\]
Multipliziert man beide Seiten mit $D^{-1}$, erh"alt man:
\[
    \left[
        \begin{array}{c} 
            \tilde{y_1}\\ 
            \tilde{y_2}\\ \vdots \\
            \tilde{y_n}
        \end{array}
    \right]_{\tilde{B_W}}
    =
    D^{-1} \: A \: C
    \left[
        \begin{array}{c} 
            \tilde{x_1}\\ 
            \tilde{x_2}\\ \vdots \\
            \tilde{x_n}
        \end{array}
    \right]_{\tilde{B_V}}
\]
Die Abbildung $f$ wird bzgl. der neuen Basen $\tilde{B_V}$ und $\tilde{B_W}$
also durch die Matrix $A':=D^{-1}AC$ dargestellt. W"ahlt man die neuen 
Basen geeignet, so ist es immer m"oglich zu erreichen, da"s 
$A'$ die Form
\[ \left[
   \begin{array}{cccc}
       a_{1,1}' & 0        & \cdots & 0        \MatStrut \\
       0        & a_{2,2}' & \ddots & \vdots   \MatStrut \\
       \vdots   & \ddots   & \ddots & 0        \MatStrut \\
       0        & \cdots   & 0      & a_{n,n}' \MatStrut
   \end{array} \right]
\] bekommt. Eine Matrix dieser Form wird als 
{\em Diagonalmatrix} \index{Diagonalmatrix} bezeichnet.

Betrachtet man nun statt einer Abbildung zwischen den
Vektorr"aumen eine Abbildung in $V$ und wechselt die Basis von $V$, so
besitzt die Abbildung bzgl. der neuen Basis die Form $C^{-1}AC$.
Es stellt sich wiederum die Frage, ob es m"oglich ist, die neue Basis
so zu w"ahlen, da"s $C^{-1}AC$ eine Diagonalmatrix ist. Eine Matrix $A$,
f"ur die das m"oglich ist, wird als
{\em diagonalisierbar} \index{diagonalisierbar} bezeichnet.

Um eine Beziehung zur Methode von Krylov (siehe Unterkapitel 
\ref{SecKrylov}) herstellen zu k"onnen, folgt 
eine Charakterisierung der Diagonalisierbarkeit. Dazu greifen wir auf 
den bereits in
\ref{DefUnterraum} definierten Begriff des {\em Unterraumes} zur"uck.
Anhand dieser Definitionen erkennt man, da"s alle zu einem Eigenwert
geh"orenden Eigenvektoren zusammen mit dem Nullvektor einen Unterraum
des zugrunde liegenden Vektorraumes bilden. Er wird als
{\em Eigenraum} \index{Eigenraum} bezeichnet.

An dieser Stelle werden die Eigenwerte mit der Diagonalisierbarkeit
in Verbindung gebracht:
\begin{satz}
\label{SatzEigenDiagonalBasis}
    Eine $n \times n$-Matrix $A$ ist genau dann diagonalisierbar, wenn der 
    zugrunde liegende $K$-Vektorraum $V$ eine Basis aus Eigenvektoren von 
    $A$ besitzt.
\end{satz}
\begin{beweis}
    Wir verzichten hier auf einen ausf"uhrlichen Beweis. Die Ideen f"ur
    die beiden Beweisrichtungen sind:
    \begin{itemize}
    \item 
          Ist eine Matrix $A$ diagonalisierbar und wechselt man die Basis
          so, da"s $A$ Diagonalgestalt bekommt, werden dadurch die
          Vektoren der kanonischen Basis zu Eigenvektoren der Matrix.
    \item
          Bilden die Eigenvektoren einer Matrix $A$ eine Basis von $V$ und
          benutzt man diese Basis zur Darstellung bekommt $A$ die
          Form einer Diagonalmatrix.
    \end{itemize}
\end{beweis}

\begin{satz}
\label{SatzEigenUnabhaengig}
    Seien $\lambda_1,\, \ldots,\, \lambda_k$ paarweise verschiedene
    Eigenwerte der $n \times n$-Matrix $A$. Sei $v_i$ ein Eigenvektor zu
    $\lambda_i$. Dann sind die Eigenvektoren $v_1,\, \ldots,\, v_k$
    linear unabh"angig.
\end{satz}
\begin{beweis}
    Der Beweis erfolgt durch Induktion nach der Anzahl der verschiedenen
    Eigenwerte $k$:
    \begin{MyDescription}
    \MyItem{$k=1$}
        F"ur diesen Fall ist die Behauptung offensichtlich richtig.
    \MyItem{$k>1$}
        Die Behauptung gelte f"ur $k-1$ und sei f"ur $k$ zu zeigen.
        Induktionsvoraussetzung ist also, da"s
        $v_1, \, \ldots, \, v_{k-1}$ linear unabh"angig sind.
        Angenommen $v_1, \, \ldots,\, v_{k}$ sind linear abh"angig.
        Dann existieren eindeutig bestimmte $r_1, \ldots, r_{k-1} \in K$ 
        mit
        \Beq{EquEigenUnabhaengig}
           v_k = r_1 v_1 + \cdots + r_{k-1} v_{k-1} \MyPunkt
        \Eeq
        Da $v_k$ nicht der Nullvektor sein kann, mu"s mindestens einer 
        der Faktoren $r_1, \dots, r_k$ ungleich Null sein, z. B. $r_i$. 
        
        Betrachtet man $A$ als 
        Abbildung und wendet diese Abbildung auf 
        \equref{EquEigenUnabhaengig} an, so kann man, da es sich um
        Eigenvektoren handelt, auch mit den Eigenwerten multiplizieren
        und erh"alt
        \[
           \lambda_k v_k = \lambda_1 r_1 v_1 + \cdots 
                                             + \lambda_k r_k v_k \MyPunkt
        \]
        Ist nun $\lambda_k = 0$, dann mu"s wegen der Verschiedenheit
        der Eigenwerte $\lambda_i \neq 0$ sein und man erh"alt einen 
        Widerspruch zur linearen Unabh"angigkeit von 
        $v_1,\, \ldots, \, v_k$.
        
        Ist $\lambda_k \neq 0$ erh"alt man einen Widerspruch zur 
        Eindeutigkeit der Darstellung von $v_k$.
    \end{MyDescription}
\end{beweis}

Aus \ref{SatzEigenDiagonalBasis} und \ref{SatzEigenUnabhaengig} ergibt sich:

\begin{korollar}
\label{SatzMaxEigen}
    Die maximale Anzahl verschiedener Eigenwerte einer Matrix ist gleich
    der Dimension des zugrunde liegenden Vektorraumes. 
    
    Besitzt eine Matrix maximal viele verschiedene Eigenwerte, so ist
    sie diagonalisierbar.
\end{korollar}

F"ur die zweite Folgerung ben"otigen wird einen weiteren Betriff. Seien
dazu $T$ und $U$ Unterr"aume des $K$-Vektorraumes $V$. Dann wird die Menge
\[
   \{ w \MySetProperty 
      \exists k,l \in K, \, t \in T, \, u \in U: \: w=kt+lu \}
\]
aller Linearkombinationen zweier Vektoren aus $T$ und $U$ als
{\em direkte Summe von $T$ und $U$} \index{direkte Summe} bezeichnet.
Anhand von \ref{DefUnterraum} erkennt man, da"s die direkte Summe
zweier Unterr"aume von $V$ wiederum ein Unterraum von $V$ ist.

Somit erh"alt man aus
\ref{SatzEigenDiagonalBasis} und \ref{SatzEigenUnabhaengig}:

\begin{korollar}
\label{SatzEigenraum}
    Eine Matrix ist genau dann diagonalisierbar, wenn die direkte Summe
    aller Eigenr"aume der Matrix den zugrundliegenden Vektorraum ergibt.
\end{korollar}

Die Bedeutung der in diesem Unterkapitel dargestellten Sachverhalte
wird deutlich, wenn man sie mit den in Unterkapitel
\ref{SecKrylov} erw"ahnten Einschr"ankungen f"ur die
Verwendbarkeit der Methode von Krylov zur Berechnung der Koeffizienten 
des charakteristischen Polynoms vergleicht.

% ... mehrfache Eigenwerte --> Dimension des Eigenraums

% Welche Beziehungen bestehen zwischen Invertierbarkeit und
%     Dialgonalisierbarkeit?

% **************************************************************************

\MySection{Das Minimalpolynom}
\label{SecMinimalpolynom}

Die Methode von Krylov (siehe \ref{SecKrylov}) dient zur Bestimmung der
Koeffizienten des Minimalpolynoms einer Matrix. Deshalb wird hier dieses
Minimalpolynom zun"achst n"aher betrachtet. Eine tiefergreifende 
Behandlung befindet sich z. B. in \cite{Zurm64} S. 233 ff .

In Satz \ref{SatzCayleyHamilton} wird bewiesen, da"s eine
$n \times n$-Matrix $A$ ihre eigene charakteristische Gleichung erf"ullt.
Diese Beobachtung f"uhrt zu: \nopagebreak[3]
\MyBeginDef
\label{DefMinimalpolynom} 
\index{Minimalpolynom} \index{Minimumgleichung}
    Das Polynom $m(\lambda)$ mit dem kleinsten Grad, f"ur das die 
    Gleichung \[ m(A) = 0_n \] erf"ullt ist, wird 
    {\em Minimalpolynom} genannt. Die Gleichung wird als 
    {\em Minimumgleichung} \index{Minimumgleichung} bezeichnet.
\MyEndDef

Um die Methode von Krylov verstehen zu k"onnen, m"ussen wir verschiedene
Eigenschaften des Minimalpolynoms beleuchten:

\begin{satz}
\label{SatzMinimalpolynomVielfaches}
    Sei $A$ eine $n \times n$-Matrix und $f(\lambda)$ ein Polynom. Es
    gelte \[ f(A) = 0_n \MyPunkt \] Dann ist $f(\lambda)$ ein Vielfaches
    des Minimalpolynoms $m(\lambda)$ von $A$.
\end{satz}
\begin{beweis}
    Angenommen die Behauptung ist falsch. Dann entsteht bei der Division
    von $f(\lambda)$ durch $m(\lambda)$ ein Rest $r(\lambda)$ und f"ur
    ein geeignetes Polynom $q(\lambda)$ gilt:
    \[ f(\lambda) = q(\lambda)m(\lambda) + r(\lambda) \MyPunkt \]
    Der Grad von $r(\lambda)$ ist kleiner als der Grad von $m(\lambda)$.
    Wird nun in diese Gleichung $A$ eingesetzt, erh"alt man
    \[ 0_n = 0_n + r(A) \MyPunkt \] Also mu"s auch gelten
    \[ r(A) = 0_n \MyPunkt \] Da jedoch das Minimalpolynom das Polynom mit
    dem kleinsten Grad ist, das diese Bedingung erf"ullt, f"uhrt dies
    zu einem Widerspruch.
\end{beweis}

Aus \ref{SatzCayleyHamilton} und \ref{SatzMinimalpolynomVielfaches} 
ergibt sich:
\begin{korollar}
\label{SatzVielfaches}
    Das charakteristische Polynom ist ein Vielfaches des Minimalpolynoms.    
\end{korollar}

Da wir die Methode von Krylov zur Berechnung des charakteristischen 
Polynoms verwenden wollen, m"ussen wir wissen, unter welchen Umst"anden
es mit dem Minimalpolynom zusammenf"allt. Diese 
Frage beantwortet der folgende Satz:

\begin{satz}
\label{SatzCharMatGGT}
    Sei $A$ eine $n \times n$-Matrix.
    Es wird definiert:
    \[ C := A - \lambda E_n \MyPunkt \] Sei
    \[ p(\lambda) = \det(C) \] das charakteristische Polynom von $A$.
    Es gelte
    \Beq{Equ20MatGGT}
        m(\lambda) = \frac{ p(\lambda) }{ q(\lambda) }
    \Eeq
    \MyPunktA{25em}
    Das Polynom $m(\lambda)$ ist genau dann das Minimalpolynom von $A$,
    wenn $q(\lambda)$ der gr"o"ste gemeinsame Teiler (ggT) der
    Determinanten aller $(n-1) \times (n-1)$-Untermatrizen von $C$ ist.
\end{satz}
\begin{beweis}
    Der Beweis erfolgt in zwei Schritten:
    \begin{itemize}
    \item Sei zun"achst $q(\lambda)$ der ggT Determinanten der 
          Untermatrizen. Es ist
          zu zeigen, da"s dann $m(\lambda)$ das Minimalpolynom ist.

          Mit Hilfe von Satz \ref{SatzEntw} (Zeilen- und Spaltenentwicklung)
          folgt, da"s $p(\lambda)$ durch $q(\lambda)$ teilbar ist. D. h. es
          gibt ein Polynom $m'(\lambda)$, so da"s
          \Beq{Equ2MatGGT}
              p(\lambda) = m'(\lambda) q(\lambda) \MyPunkt
          \Eeq
          Weiterhin gibt es eine $n \times n$-Matrix
          $M$ aus teilerfremden Polynomen "uber $\lambda$, so da"s gilt:
          \Beq{Equ1MatGGT}
              \adj(C) = M q(\lambda) \MyPunkt
          \Eeq
          Nach Satz \ref{SatzAdj} gilt:
          \Beq{Equ5MatGGT}
              C \, \adj(C) = E_n p(\lambda) \MyPunkt
          \Eeq
          Mit \equref{Equ2MatGGT} folgt aus \equref{Equ5MatGGT}:
          \Beq{Equ4MatGGT}
              C \, \adj(C) = E_n m'(\lambda) q(\lambda) \MyPunkt
          \Eeq
          Mit \equref{Equ1MatGGT} folgt aus \equref{Equ5MatGGT}:
          \Beq{Equ3MatGGT}
              C \, \adj(C) = C M q(\lambda) \MyPunkt
          \Eeq
          Aus \equref{Equ3MatGGT} und \equref{Equ4MatGGT} folgt:
          \[ C M = E_n m'(\lambda) \MyPunkt \]
          Benutzt man die Definition von $C$, erh"alt man
          \[ (A - \lambda E_n) M = E_n m'(\lambda) \MyPunkt \]
          Setzt man nun in dieser Gleichung $A$ f"ur $\lambda$ ein,
          ergibt sich
          \[ m'(A) = 0_n \MyPunkt \]
          Nach Satz \ref{SatzMinimalpolynomVielfaches} ist $m'(\lambda)$
          also ein Vielfaches des Minimalpolynoms $m(\lambda)$.
          
          Angenommen es gibt ein Polynom $m''(\lambda)$ mit
          \[ m''(A) = 0_n \MyKomma \] dessen Grad kleiner ist als der
          Grad von $m'(\lambda)$. Da der Grad des charakteristischen
          Polynoms immer $n$ ist, mu"s dann der Grad von $q(\lambda)$ in
          \equref{Equ2MatGGT} und somit auch in \equref{Equ1MatGGT}
          entsprechend gr"o"ser sein, im Widerspruch dazu, da"s
          $q(\lambda)$ der ggT ist. Also ist $m'(\lambda)$ das
          Minimalpolynom.
    \item Sei nun $m(\lambda)$ das Minimalpolynom. Dann ist zu zeigen, da"s
          $q(\lambda)$ der ggT der Unterdeterminanten ist.
          
          Nach \ref{SatzVielfaches} gibt es ein Polynom $q'(\lambda)$, so
          da"s
          \Beq{Equ6MatGGT}
              p(\lambda) = q'(\lambda) m(\lambda) \MyPunkt
          \Eeq
          Benutzt man f"ur das Minimalpolynom die Koeffizientendarstellung,
          erh"alt man mit geeigneten Koeffizienten $b_i$:
          \begin{eqnarray*}
              \lefteqn{- E_n m(\lambda)} \\
              & = & m(A) - E_n m(\lambda) \\
              & = & b_m(A^m - \lambda^m E_n)
                    + b_{m-1}(A^{m-1} - \lambda^{m-1} E_n) + \cdots +
                    b_1(A - \lambda E_n) \MyPunkt \\
          \end{eqnarray*}
          Also ist $m(\lambda)$ durch $(A-\lambda E_n)$ teilbar und es
          gibt eine Matrix $N$ aus Polynomen "uber $\lambda$, so da"s gilt:
          \Beq{Equ7MatGGT}
              m(\lambda) E_n = (A - \lambda E_n) N \MyPunkt
          \Eeq
          Mutipliziert man beide Seiten mit $q'(\lambda)$, erh"alt man
          \[ p(\lambda) E_n = q'(\lambda) (A - \lambda E_n) N \MyPunkt \]
          Subtrahiert man nun auf beiden Seiten
          \begin{eqnarray*}
               & & p(\lambda) E_n \\
               & \MyStack{nach \ref{SatzAdj}}{=} & C \,\adj(C) \\
               & = & (A - \lambda E_n) \adj(C) \MyKomma
          \end{eqnarray*}
          erh"alt man
          \[
             0_{n,n} =
             \underbrace{ (A - \lambda E_n)
                        }_{ \mbox{$(*1)$} }
             \:
             \underbrace{ (q'(\lambda) N - \adj(C))
                        }_{ \mbox{$(*2)$} }
          \]
          In dieser Gleichung ist Term $(*1)$ f"ur ein beliebig gew"ahltes 
          $\lambda$ ungleich 
          der Nullmatix\footnote{abgesehen von einigen Sonderf"allen, deren
          Existenz den Beweis jedoch nicht beeintr"achtigt}. Also mu"s 
          Term $(*2)$ gleich der Nullmatrix sein, so 
          da"s gilt:
          \[ q'(\lambda) N = \adj(C) \]
          Also ist $q'(\lambda)$ Teiler der Elemente von $\adj(C)$.
          
          Angenommen es gibt ein Polynom $q''(\lambda)$, dessen Grad
          gr"o"ser ist als der Grad von $q'(\lambda)$ und das ebenfalls
          Teiler der Elemente von $\adj(C)$ ist. Da der Grad von
          $p(\lambda)$ immer $n$ ist, mu"s dann der Grad von $m(\lambda)$ 
          in \equref{Equ6MatGGT} kleiner sein, im Widerspruch zu der
          Voraussetzung, da"s $m(\lambda)$ das Minimalpolynom ist.
    \end{itemize}
\end{beweis}

Berechnet man in \equref{Equ7MatGGT} auf beiden Seiten die Determinante,
erh"alt man
\Beq{Equ21MatGGT}
   m^n(\lambda) = p(\lambda) \det(N) \MyPunkt
\Eeq
Aus \equref{Equ20MatGGT} und \equref{Equ21MatGGT} ergibt sich:

\begin{korollar}
\label{SatzMinimalNullGenauDann}
    Es ist $\lambda_i$ genau dann Nullstelle von $m(\lambda)$, wenn es auch
    Nullstelle von $p(\lambda)$ ist.
\end{korollar}

Anders ausgedr"uckt: $m(\lambda)$ und $p(\lambda)$ besitzen die gleichen
Nullstellen mit evtl. verschiedenen Vielfachheiten. Das f"uhrt
zu einer weiteren Schlu"sfolgerung:

\begin{korollar}
\label{SatzPaarweiseVerschieden}
    Besitzt eine $n \times n$-Matrix $n$ paarweise verschiedene Eigenwerte,
    so stimmen ihr Minimalpolynom und ihr charakteristisches Polynom
    "uberein.
\end{korollar}

Falls die Eigenwerte nicht paarweise verschieden sind, k"onnen 
Minimalpolynom und charakteristisches Polynom also verschieden sein, was
eine Einschr"ankung f"ur die Methode von Krylov 
(siehe Unterkapitel \ref{SecKrylov}) bedeutet, wenn man sie zur Berechnung
der Koeffizienten des charakteristischen Polynoms verwendet.
Wann die beiden Polynome verschieden sind, zeigt der folgende Satz:

\begin{satz}
\label{SatzMinimalDimEins}
    Das Minimalpolynom $m(\lambda)$ einer Matrix $A$ stimmt genau dann 
    mit dem charakteristischen Polynom $p(\lambda)$ der Matrix "uberein, 
    wenn die Dimension jedes Eigenraumes $1$ 
    betr"agt\footnote{Es ist zu beachten, da"s diese Aussage nicht dazu 
    "aquivalent ist, da"s die direkte Summe der Eigenr"aume den gesamten 
    Vektorraum ergibt.}.
\end{satz}
\begin{beweis}
    Aus \ref{SatzCharMatGGT} folgt, da"s $m(\lambda)$ und $p(\lambda)$
    genau dann "ubereinstimmen, wenn der 
    ggT der Determinanten aller $(n-1) \times (n-1)$-Untermatrizen
    der charakteristischen Matrix von $A$ gleich $1$ ist.
    
    Betrachtet man die beiden Polynome in ihrer Linearfaktorendarstellung,
    mu"s der genannte ggT, falls er ungleich $1$ ist, mit einem Linearfaktor
    von $p(\lambda)$ "ubereinstimmen. F"ur die entsprechende Nullstelle
    von $p(\lambda)$ verschwinden auch alle 
    $(n-1) \times (n-1)$-Unterdeterminanten. Es gibt also f"ur diese
    Nullstelle keine $n-1$ linear unabh"angigen Spaltenvektoren der 
    charakteristischen Matrix. Der Rangabfall der Nullstelle ist also
    gr"o"ser als $1$ und es gibt mehr als einen linear unabh"angigen 
    Eigenvektor zu diesem Eigenwert.
\end{beweis}

Aus \ref{SatzPaarweiseVerschieden} und \ref{SatzMinimalDimEins} erh"alt
man:
\begin{korollar}
\label{SatzMinimalKomplex}
    Falls die Eigenwerte nicht paarweise verschieden sind und das
    Minimalpolynom mit dem charakteristischen Polynom "ubereinstimmt,
    zerf"allt es im K"orper der reellen Zahlen nicht in Linearfaktoren.
\end{korollar}

Wie bereits mehrfach erw"ahnt, erfolgt nun die Anwendung der Ergebnisse 
dieses Unterkapitels auf die Methode von Krylov.

% $$$ Verbesserung der Benutzung der Methode von Krylov:
% (\det(A) gesucht); bestimme Matrix B, so da"s $\det(B)$ bekannt oder
% leicht zu berechnen und f"ur $A*B$ gilt:
%     die Dimension jedes Eigenraumes ist gleich 1; 
%     (es gen"ugt: die Eigenwerte sind paarweise verschieden)
%     w"unschenswert $\det(B)$ ist 'Einheit' im zugrundeliegenden Ring;
% berechne \det(AB); dann eine zus"atzliche Division:
%  \det(A) = \det(AB) / \det(B)

% **************************************************************************

\MySection{Die Methode von Krylov}
\label{SecKrylov}
\index{Krylov!Methode von}

In diesem Unterkapitel wird die Methode von Krylov zur Bestimmung der
Koeffizienten des Minimalpolynoms einer Matrix beschrieben
(siehe z. B. \cite{Zurm64} ab S. 171 oder \cite{Hous64} ab S. 149;
Originalver"offentlichung \cite{Kryl31} ). Wie in Unterkapitel 
\ref{SecMinimalpolynom} beschrieben wird, ist das Minimalpolynom
unter bestimmten Bedingungen mit dem charakteristischen Polynom identisch.
Da sich unter den Koeffizienten des charakteristischen Polynoms auch die
Determinante der zugrunde liegenden Matrix befindet 
(vgl. \ref{SatzDdurchP}), ist es m"oglich, Krylovs Methode zur 
Determinantenberechnung zu verwenden, was im Algorithmus von Pan 
ausgenutzt wird.

Sei $A$ die $n \times n$-Matrix, deren Minimalpolynom zu
berechnen ist. 
Sei $z_0$ ein geeigneter Vektor der L"ange $n$. Wie $z_0$ beschaffen ist, 
wird noch behandelt. Sei $i \in \Nat$ gegeben. Die 
Vektoren \[ z_1,\, \ldots,\, z_i \] erh"alt man durch
\begin{eqnarray}
    z_1 & := & A z_0 \nonumber \\
    z_2 & := & A z_1 = A^2 z_0 \nonumber \\
    \vdots \nonumber \\
    z_i & := & A z_{i-1} = A^i z_0 \label{EquDefZI} \MyPunkt
\end{eqnarray}
Die Vektoren \[ z_0,\, \ldots ,\, z_i \] werden als {\em iterierte Vektoren}
bezeichnet.
Betrachtet man die iterierten Vektoren als Spaltenvektoren einer Matrix,
erh"alt man eine sogenannte {\em Krylov}-Matrix:
\[ K(A,z_0,i):= [ z_0, z_1, z_2, \ldots, z_{i-1} ] \MyPunkt \]
Zwischen den iterierten Vektoren besteht eine lineare Abh"angigkeit
besonderer Form, die von Krylov \cite{Kryl31} f"ur das hier zu
beschreibende Verfahren entdeckt wurde.

Das Minimalpolynom von $A$ wird mit $m(\lambda)$ bezeichnet.
Es gilt also
\Beq{EquKrylovPolynom}
    m(A) = 0_{n,n} \MyPunkt
\Eeq Das Polynom habe den Grad $j$.
Seien $c_0,\, \ldots\, c_{j-1}$ die Koeffizienten des Polynoms. 
Definiert man
\[
   c := \left[
        \begin{array}{c} c_0 \\ c_1 \\ \vdots \\ c_{j-1} \end{array}
        \right]
\]
dann ergibt sich
\begin{eqnarray} % $$$$ Formatierung gepr"uft ?
    m(A) & = & 0_{n,n} \nonumber
\\ \Leftrightarrow \hspace{1.2mm} \nonumber
    A^j + c_{j-1} A^{j-1} + \ldots + c_1 A + c_0 E_n & = & 0_{n,n}
\\ \Leftrightarrow \hspace{9.1mm} \nonumber
    c_{j-1} A^{j-1} + \ldots + c_1 A + c_0 E_n & = & - A^j
\\ \Leftrightarrow \nonumber
    c_0 E_n z_0 + c_1 A z_0 + \ldots + c_{j-1} A^{j-1} z_0 &
                                                         = & - A^j z_0
\\ \Leftrightarrow \hspace{3.6cm} \label{EquKrylovEqu}
    K(A,z_0,j) c & = & - A^j z_0
\end{eqnarray}
Gleichung \equref{EquKrylovEqu} kann man als lineares Gleichungssystem
in Matrizenschreibweise betrachten. Multipliziert man die rechte Seite
dieser Gleichung aus, erh"alt man einen Vektor der L"ange $n$.
Nach \ref{SatzRangGleich} ist das entsprechende Gleichungssystem
genau dann l"osbar, wenn
\[ \rg(K(A,z_0,j) = \rg([K(A,z_0,j),\, A^j z_0]) \MyPunkt \]
Wir suchen eine eindeutige L"osung des Gleichungssystems und erhalten diese
durch Verwendung von \ref{SatzGenauEine}, wonach der bis hierhin
nicht n"aher beschriebene Vektor $z_0$ so gew"ahlt werden mu"s,
da"s die Spaltenvektoren von $K(A,z_0,j)$ linear unabh"angig und
die Spaltenvektoren von $[K(A,z_0,j),\, A^j z_0]$ linear abh"angig sind.

An dieser Stelle wird deutlich, da"s $m(\lambda)$ das Polynom mit dem
kleinsten Grad sein mu"s, das \equref{EquKrylovPolynom} erf"ullt, damit
\equref{EquKrylovEqu} eine eindeutige L"osung besitzt. Falls ein
Polynom $m_1(\lambda)$ existiert, da"s \equref{EquKrylovPolynom} erf"ullt
und dessen Grad kleiner ist als der Grad von $m(\lambda)$, dann ist die
lineare Abh"angigkeit unabh"angig von der Wahl von $z_0$ bereits f"ur
weniger als $j$ iterierte Vektoren gegeben.

Da nach \equref{EquKrylovEqu} die ersten $j+1$ iterierten Vektoren linear
abh"angig sind, bleibt zu zeigen, da"s die ersten $j$ dieser Vektoren linear
unabh"angig sind. Dazu betrachten wir das $s \in \Nat$ mit der Eigenschaft,
da"s $s$ paarweise verschiedene Eigenvektoren von $A$ immer linear 
unabh"angig und $s+1$ von ihnen immer linear abh"angig sind. Aus den 
Grundlagen "uber Eigenvektoren und lineare Unabh"angigkeit geht hervor, 
da"s ein solches $s$ gibt.

Seien somit
\[ x_1,\, \ldots ,\, x_s \] linear 
unabh"angige Eigenvektoren von $A$.
Sei der iterierte Vektor $z_0$ darstellbar als Linearkombination einer 
maximalen Anzahl (also $s$) von Eigenvektoren von $A$. 
Demnach gilt f"ur geeignete 
\[ d_1,\, \ldots,\, d_s \MyKomma \] ungleich Null\footnote{Eine 
ausf"uhrliche Diskussion der Eigenschaften der 
iterierten Vektoren, insbesondere ihrer Beziehung zu den Eigenvektoren, 
befindet sich in \cite{Bode59} (Teil 2, Kapitel 2).}:
\Beq{EquDefZNull}
    z_0 = d_1 x_1 + \cdots + d_s x_s \MyPunkt
\Eeq 
Eine lineare Abh"angigkeit zwischen den ersten $j$ iterierten Vektoren
hat die Form
\Beq{EquIteriertLinAbh}
    h(z_0):= e_0 z_0 + e_1 z_1 + \cdots + e_{j-1} z_{j-1} = 0_n \MyKomma
\Eeq
wobei nicht alle $e_i$ gleich Null sind. Mit Hilfe der
Gleichungen \equref{EquDefZI} und \equref{EquDefZNull} in Verbindung mit
den Eigenwertgleichungen\footnote{vgl. Gleichung \equref{EquEigenMotiv}}
\[ A x_i = \lambda_i x_i \]
erh"alt man:
\begin{eqnarray*} % $$$$ Formatierung gepr"uft ?
    z_0 & = & \left. d_1 x_1 + \cdots + d_s x_s 
                        \: \hspace{14.2mm} \right| * e_0 \\
    z_1 & = & 
        \left. \lambda_1 d_1 x_1 + \cdots + 
                   \lambda_s d_s x_s \: \hspace{7.1mm} \right| * e_1 \\
    z_2 & = & 
         \left. \lambda_1^2 d_1 x_1 + \cdots + \lambda_s^2 d_s x_s 
              \: \hspace{7mm} \right| * e_2 \\
    & \vdots \\
    z_{j-1} & = &
         \left. \lambda_1^{j-1} d_1 x_1 
                       + \cdots + \lambda_s^{j-1} d_s x_s \: \right| * 1
\end{eqnarray*}
Diese Gleichungen f"ur die $z_i$ werden mit den am rechten Rand angegebenen 
Werten (vgl. \equref{EquIteriertLinAbh}) multipliziert und anschlie"send 
addiert. Definiert man
\[
   g(\lambda):= e_0 + e_1 \lambda + \cdots + e_{j-1} \lambda^{j-1} \MyKomma
\]
so lautet das Ergebnis in Verbindung mit \equref{EquIteriertLinAbh} :
\[
    h(z_0)= g(\lambda_1) d_1 x_1 + \cdots + g(\lambda_s) d_s x_s
          = 0_n \MyPunkt
\]
Da die $x_k$ nach Voraussetzung linear unabh"angig sind, mu"s also gelten
\[ \forall 1 \leq k \leq s :\: g(\lambda_k)d_k = 0 \MyPunkt \]
Da wiederum nach Voraussetzung $z_0$ eine Linearkombination aller $s$
linear unabh"angigen Eigenvektoren $x_i$ (s. o.) ist, folgt
\Beq{Equ2BewLinUnabh}
    \forall 1 \leq i \leq s :\: g(\lambda_i)= 0 \MyPunkt
\Eeq
Da die Dimension jedes Eigenraumes mindestens $1$ betr"agt, folgt mit Hilfe
von \ref{SatzMinimalNullGenauDann}, da"s
die maximale Anzahl linear unabh"angiger Eigenvektoren $s$ mindestens
so gro"s ist wie der Grad des Minimalpolynoms $j$.

Da $g(\lambda)$ als Polynom vom Grad $j-1$
nur maximal $j-1$ Nullstellen besitzen kann, folgt aus 
\equref{Equ2BewLinUnabh}
\[ e_0 = e_1 = \cdots = e_{j-1} = 0 \MyKomma \]
Daraus wiederum folgt mit \equref{EquIteriertLinAbh},
da"s die iterierten Vektoren linear unabh"angig sind.

Der bis hierhin gef"uhrte Beweis der linearen Unabh"angigkeit
iterierter Vektoren verwendet die maximale Anzahl $s$ linear unabh"angiger
Eigenvektoren. Betrachten wir deshalb diesen Wert $s$ genauer.
Es gibt zwei F"alle:
\begin{itemize}
\item
      Minimalpolynom und charakteristisches Polynom sind identisch.
      Satz \ref{SatzMinimalDimEins} f"uhrt zu zwei Unterf"allen:
      \begin{itemize}
      \item
            Die Eigenwerte sind paarweise verschieden. In diesem Fall
            ist $s$ gleich dem Grad $n$ des charakteristischen Polynoms
            und somit gleich $j$.
      \item
            Die Eigenwerte sind nicht paarweise verschieden. Nach
            \ref{SatzMinimalDimEins} ist $s$ gleich der Anzahl verschiedener
            Eigenwerte und damit in $\Rationals$ kleiner als $n$ und 
            mindestens $1$. In diesem Fall l"a"st sich die lineare
            Unabh"angigkeit nur f"ur weniger als $j$ iterierte Vektoren
            beweisen und die Methode von Krylov ist zur Bestimmung der
            Koeffizienten des Minimalpolynoms nicht anwendbar.
      \end{itemize}
\item 
      Minimalpolynom und charakteristisches Polynom sind nicht identisch.
      Es gibt wiederum zwei Unterf"alle:
      \begin{itemize}
      \item 
            Die direkte Summe der Eigenr"aume ergibt den gesamten 
            Vektorraum. In diesem Fall ist $s = n >j$.
      \item
            Die direkte Summe der Eigenr"aume ergibt nicht den gesamten
            Vektorraum. In diesem Fall ist $s \geq j$.
      \end{itemize}
\end{itemize}
W"ahlt man also $z_0$ als Linearkombination einer maximalen Anzahl linear
unabh"angiger Eigenvektoren, so sind die ersten $j$ iterierten Vektoren
linear unabh"angig, es sei denn, Minimalpolynom und charakteristisches
Polynom sind identisch und die Eigenwerte nicht paarweise verschieden.

Das nun noch verbliebene Problem ist die Wahl von $z_0$ f"ur eine gegebene 
Matrix $A$, da im Normalfall die Eigenvektoren nicht bekannt sind. Diese
Schwierigkeit
kann dadurch "uberwunden werden, da"s man die Methode von Krylov mit 
verschiedenen Vektoren $z_0$ auf die Matrix $A$ anwendet und dabei die
Vektoren $z_0$ so ausw"ahlt, da"s mindestens einer unter ihnen eine 
Linearkombination aller Eigenvektoren $x_1$ bis $x_s$ ist.

W"ahlt man eine Basis des zugrunde liegenden Vektorraumes, bestehend aus $n$
Vektoren, sowie einen
weiteren Vektor so aus, da"s je $n$ dieser $n+1$ Vektoren linear unabh"angig
sind und der $n+1$-te jeweils eine Linearkombination aller $n$ anderen ist,
so besitzt mindestens einer dieser Vektoren die geforderte Eigenschaft.
Dies erkennt man durch folgende "Uberlegung:
Stellt man die $n$ Vektoren als Linearkombinationen der Eigenvektoren dar,
so wird jeder Eigenvektor mindestens einmal ben"otigt. Da der $n+1$-te
Vektor eine Linearkombination der anderen $n$ ist, ist er also auch eine
Linearkombination aller beteiligen Eigenvektoren. Ein Beispiel f"ur 
$n+1$ Vektoren, die offensichtlich diese Eigenschaft besitzen, sind die
Vektoren der kanonische Basis und deren Summe:
\[ 
   \left[
       \begin{array}{c}
           1 \\ 0 \\ 0 \\ \vdots \\ 0
       \end{array}
   \right]
   \left[
       \begin{array}{c}
           0 \\ 1 \\ 0 \\ \vdots \\ 0
       \end{array}
   \right] \: \ldots  \:
   \left[
       \begin{array}{c}
           0 \\ 0 \\ \vdots \\ 0 \\ 1
       \end{array}
   \right] \: 
   \left[
       \begin{array}{c}
           1 \\ 1 \\ 1 \\ \vdots \\ 1
       \end{array}
   \right]  
\]

Uns interessiert die Berechnung der Determinante mit Hilfe der 
Methode von Krylov. Zusammengefa"st sieht das Vorgehen dazu folgenderma"sen
aus:
\begin{itemize}
\item 
      Zun"achst werden auf die soeben beschriebene Weise $n+1$ f"ur den 
      iterierten Vektor $z_0$ bestimmt. In der praktischen Anwendung 
      beschr"ankt man
      sich in der Regel auf einen Vektor, da die Anzahl der F"alle, in 
      denen dieser Vektor nicht ausreicht, so gering ist, da"s diese F"alle
      vernachl"assigt werden k"onnen.

      Die folgenden Schritte werden mit jedem Vektor $z_0$ parallel 
      durchgef"uhrt. 
      
      Falls mehrere der parallelen Zweige ein Ergebnis 
      liefern, m"ussen diese Ergebisse gleich sein. Falls sie nicht 
      gleich sind, wurde die Rechnung nicht korrekt durchgef"uhrt.

      Falls keiner der Zweige ein Ergebnis liefert, sind entweder die 
      Eigenwerte der zugrunde liegenden Matrix nicht paarweise verschieden,
      oder die Matrix ist nicht invertierbar. Diese beiden Unterf"alle 
      k"onnen mit dem hier dargestellten Algorithmus nicht unterschieden 
      werden.
\item 
      Es werden die iterierten Vektoren von $z_1$ bis $z_n$ berechnet.
\item
      Das Gleichungssystem \equref{EquKrylovEqu} wird dadurch gel"ost, 
      da"s die aus den iterierten Vektoren bestehende Krylov-Matrix
      inveritert wird. Ist die Krylov-Matrix nicht invertierbar, so ist 
      der Berechnungsversuch aus den bereits beschriebenen Gr"unden
      ein Fehlschlag und wird abgebrochen.
\item
      Die Koeffizienten des charakteristischen Polynoms, und somit auch
      die Determinante, werden dadurch berechnet, da"s die Inverse
      der Krylov-Matrix mit dem iterierten Vektor $z_n$ multipliziert wird
      (vgl. \equref{EquKrylovEqu}).
\end{itemize}

% mehr iterierte Vektoren, als der Grad des Minimalpolynoms angibt sind
% immer linear abh"angig!!!! (--> L"osbarkeit linearer Gleichungssysteme);

% **************************************************************************

\MySection{Vektor- und Matrixnormen}
\label{SecNorm}

Die Darstellungen der Wahl einer N"aherungsinversen in Unterkapitel 
\ref{SecGuessInverse} und der iterativen Matrizeninvertierung 
in Unterkapitel \ref{SecNewton} benutzen Normen von Matrizen.
Deshalb werden im vorliegenden Unterkapitel die Normen eingef"uhrt, die
dort zur Beschreibung erforderlich sind. 
Literatur dazu ist z. B. \cite{GL83} ab S. 12 
oder \cite{Isaa73} ab S. 3.

Umgangssprachlich formuliert, stellt der Begriff der {\em Norm} eine 
Verallgemeinerung des Begriffs der {\em L"ange} dar. Um zu Matrixnormen 
zu gelangen, kl"aren wir zun"achst, was eine Vektornorm ist:
\MyBeginDef
\index{Norm!eines Vektors}
\label{DefVektorNorm}
    Eine Funktion
    \[ f: \Rationals^n \rightarrow \Rationals \MyKomma \]
    die die Bedingungen
    \begin{enumerate} % $$$$ Formatierung geprueft ?
    \item
         $ \forall x \in \Rationals: f(x) \geq 0, 
            f(x) = 0 \Leftrightarrow x = 0_n 
         $
    \item
         $ \forall x,y \in \Rationals: f(x + y) \leq f(x) + f(y) $
    \item
         $ \forall a \in \Rationals, x \in \Rationals^n: f(ax)= |a| f(x) $
    \end{enumerate}
    erf"ullt, hei"st {\em Norm "uber $\Rationals^n$}. 
\MyEndDef

\MyBeginDef
\label{DefPNorm}
\index{H{\Myo}ldernorm} \index{p-Norm}
    Sei \[ p \in \Nat \] 
    fest gew"ahlt und \[ x \in \Rationals^n \] beliebig.
    \[ ||x||_p := (|x_1|^p + \ldots + |x_n|^p)^{1/p} \]  
    Die so definierte Funktion hei"st { \em $p$-Norm }.

    \[ ||x||_{\infty}:= \max_{1\leq i \leq n} |x_i| \]
    Diese Funktion wird mit {\em $\infty$-Norm} bezeichnet.

    Die $p$-Normen sowie die $\infty$-Norm 
    werden als { \em H"oldernormen } bezeichnet.
\MyEndDef

Mit der vorangegangenen Definition werden zwar einige Begriffe angegeben,
es ist jedoch nicht selbstverst"andlich, da"s es sich bei den Funktionen
auch wirklich um Normen handelt:
\begin{satz}
\label{SatzPNorm}
    Die in \ref{DefPNorm} definierten Funktionen 
    sind Normen "uber $\Rationals^n$.
\end{satz}
\begin{beweis}
    Die Funktionen erf"ullen die Bedingungen aus \ref{DefVektorNorm}.

    Der Beweis dieser Behauptung ist f"ur
    die $1$-Norm, $2$-Norm und $\infty$-Norm in
    \cite{Isaa73} ab S. 4 und
    f"ur die anderen Normen in \cite{Achi67} S. 4-7 angegeben.
    % (Isaa73 -> [30]) Achi67; BM b260/Achi
\end{beweis}

Den Begriff der Norm kann man auch auf Matrizen ausdehnen. F"ur uns 
gen"ugt die Betrachtung quadratischer Matrizen.
\MyBeginDef
\index{Norm!einer Matrix} \index{Matrixnorm}
\label{DefMatrixNorm}
    Eine Funktion 
    \[ f: \Rationals^{n^2} \rightarrow \Rationals \MyKomma \]
    die die Bedingungen
    \begin{eqnarray*}
        \forall A \in \Rationals^{n^2} & : & f(A) \geq 0,
             f(A) = 0 \leftrightarrow A=0_n \\
        \forall A,B \in \Rationals^{n^2} & : & f(A+B) \leq f(A) + f(B) \\
        \forall c \in \Rationals, A \in \Rationals^{n^2} & : &
             f(cA) = |c|f(A) \\
        \forall A,B \in \Rationals^{n^2} & : & f(AB) \leq f(A) * f(B) \\
    \end{eqnarray*}
    erf"ullt, hei"st {\em Matrixnorm "uber $\Rationals^{n^2}$ }.
\MyEndDef
Die vierte der obigen Bedingungen wird in der Literatur nicht immer f"ur 
Matrixnormen gefordert. In solchen F"allen wird unterschieden zwischen
Matrixnormen, die diese Bedingungen erf"ullen, und solchen, die diese
Bedingung nicht erf"ullen (vgl. \cite{Isaa73} S. 8 und \cite{GL83}
S. 14). F"ur uns sind diese Unterschiede nicht von Bedeutung.

Die von uns benutzten Matrixnormen sind folgenderma"sen definiert:
\MyBeginDef
\label{DefInduzierteNorm}
\index{Operatornorm} \index{Matrixnorm!induzierte}
\index{Matrixnorm!nat{\Myu}rliche} \index{Norm!nat{\Myu}rliche}
    Sei \[ A \in \Rationals^{n^2} \] Sei \[ x \in \Rationals^n \] Es gelte
    \[ ||x|| = 1 \] f"ur eine fest gew"ahlte Vektornorm.
    Die Funktion \[ ||A||:= ||Ax|| \] hei"st dann
    {\em durch die Vektornorm induzierte Matrixnorm }.
\MyEndDef
Sie ist in der Literatur auch noch unter den Namen {\em nat"urliche Norm}
und {\em Operatornorm} bekannt und wird h"aufig noch anders definiert
(vgl. \cite{Isaa73} S. 8). Das hat jedoch f"ur unsere Anwendungen keine
Bedeutung.

\begin{satz}
\label{SatzInduzierteNorm}
    Die in \ref{DefInduzierteNorm} definierte Funktion ist eine
    Matrixnorm.
\end{satz}
\begin{beweis}
    Die Funktion erf"ullt die Bedingungen in \ref{DefMatrixNorm}
    (\cite{Isaa73} ab S. 8 ).
\end{beweis}

Es folgen Beispiele f"ur induzierte
Matrixnormen, die im weiteren Text benutzt werden. Dazu ben"otigen wir
vorher noch einen weiteren Begriff:

Seien $\lambda_1, \, \ldots, \, \lambda_n$ die Eigenwerte von $A$. Dann
wird der Spektralradius \index{Spektralradius} $\rho(A)$ definiert als
\[ \rho(A) := \max\{ |\lambda_1|,\, \ldots,\, |\lambda_n| \} \MyPunkt \]

Durch
Indizes wird jeweils kenntlich gemacht, durch welche
Vektornorm die jeweilige Matrixnorm induziert wird\footnote{Bei der
Vielzahl der in diesem Unterkapitel auftauchenden Normen, sollte man
nicht vergessen, da"s $|x|$ f"ur einen Skalar $x$ einfach nur den
Absolutwert bezeichnet.}.
\begin{eqnarray}
    ||A||_1 & = & \max_j\sum_{k=1}^n |a_{k,j}| \label{EquMatNormEins} \\
    ||A||_2 & = & \sqrt{\rho(A * A)} \label{EquMatNormZwei} \\
    ||A||_\infty & = & \max_i\sum_{k=1}^n |a_{i,k}| 
                                              \label{EquMatNormInfty} \\
\end{eqnarray}
Die Beweise der Gleichungen \equref{EquMatNormEins},
\equref{EquMatNormZwei} und \equref{EquMatNormInfty} sind in
\cite{Isaa73} ab S. 9 zu finden.

Falls es nicht im Einzelfall anders festgelegt ist, gilt im weiteren 
Text $||\:||=||\:||_2$.

% $$$$ hier Satz "uber maximale Gr"o"se der Eigenwerte  ( <--> Normen )
%       (\det = \prod \lambda !!!)
%      
% ... sehr interessant, jedoch nicht 100-prozentig erforderlich
% ... zu finden in Isaa73 (S. 12)

% **************************************************************************

\MySection{Wahl einer N"aherungsinversen}
\label{SecGuessInverse}
\index{Inverse!einer Matrix} \index{Matrizeninvertierung}
\index{N{\Mya}herungsinverse}

In dem in Kapitel \ref{ChapPan} vorzustellenden Algorithmus wird die
Krylov-Matrix dadurch invertiert, da"s eine N"aherungsinverse berechnet
und diese dann iterativ verbessert wird. In diesem Unterkapitel
wird die Wahl der N"aherungsinversen beschrieben.
Literatur dazu sind \cite{PR85} und \cite{PR85a}.

F"ur die weiteren Betrachtungen ben"otigen wir:
\begin{eqnarray}
    t & := & \frac{1}{ ||A^TA||_1 } \label{EquPanDefT} \\
    B & := & t A^T \label{EquPanDefB} \\
    R(B) & := & E_n - B A \label{EquDefResidual} 
\end{eqnarray}

Im folgenden wird in mehreren Schritten eine Ungleichung f"ur
$||R(B)||$ bewiesen, aus der folgt, da"s das in Unterkapitel \ref{SecNewton}
beschriebene Verfahren effizient auf $B$ angewendet werden kann, um eine
Inverse von $A$ mit zufriedenstellender N"aherung zu erhalten.

\MyBeginDef
\label{DefSymmetrisch}
\index{symmetrisch} \index{Matrix!symmetrische}
    Gilt f"ur eine Matrix $A$ \[ A = A^T \MyKomma \] dann wird sie als
    {\em symmetrisch} bezeichnet.
\MyEndDef

Da die beiden folgenden Lemmata aus den Grundlagen "uber Normen stammen,
werden die Beweise auf einen Literaturverweis beschr"ankt. Sie sind
in \cite{Atki78} ab S. 416 zu finden.

\begin{lemma}
\label{SatzSymmetrischSpektral}
    F"ur jede symmetrische Matrix $A$ gilt:
    \[ ||A||_2 = \rho(A) \MyPunkt \]
\end{lemma}
            
\begin{lemma}
\label{SatzAtkinson}
\[
    ||A^T A||_2 = \rho(A^T A) = ||A||^2 \leq 
    ||A^T A||_1 \leq \max_i \sum_j |a_{i,j}| \max_j \sum_i |a_{i,j}| 
    \leq n ||A^T A||
\]
\end{lemma}

\begin{lemma}
\label{SatzInverseEigenwert}
    Sei $\lambda$ ein Eigenwert der invertierbaren $n \times n$-Matrix $A$. 
    Dann ist $1/\lambda$ ein Eigenwert von $A^{-1}$.
\end{lemma}
\begin{beweis}
    Da $A$ invertierbar ist, ist $\lambda \neq 0$. Sei $v$ ein Eigenvektor
    zu $\lambda$. Dann gilt $v \neq 0_n$, sowie
    \begin{eqnarray*}
        Av & \neq & 0_n \\
        Av & = & \lambda v \MyPunkt
    \end{eqnarray*}
    Damit ergibt sich:
    \[ A^{-1}(Av) = v = (1/\lambda) \lambda v = (1/\lambda) A v \MyKomma\]
    woraus die Behauptung folgt.
\end{beweis}

\begin{lemma}
\label{SatzEigenUngleichung}
    Sei $A$ invertierbar. Sei $\lambda$ ein Eigenwert von $A^T A$.
    Dann gilt:
    \Beq{EquEigenUngleichung}
        \frac{ 1 }{  ||A^{-1}||^2 }
             \leq \lambda \leq ||A||^2
             \MyPunkt
    \Eeq
\end{lemma}
\begin{beweis}
    Die rechte Ungleichung von \equref{EquEigenUngleichung} folgt aus
    \ref{SatzAtkinson}.
    
    Die linke Ungleichung von \equref{EquEigenUngleichung} folgt aus
    \ref{SatzInverseEigenwert} in Verbindung mit \ref{SatzAtkinson}.
\end{beweis}

\begin{satz}
\label{SatzResidualEigenwert}
    Seien $t$, $B$ und $R(B)$ entsprechend der Gleichungen
    \equref{EquPanDefB}, \equref{EquPanDefT} und \equref{EquDefResidual}
    definiert. Sei $\mu$ ein Eigenwert von $R(B)$. Dann
    gilt
    \[
        0 \leq \mu \leq
        1- \frac{ 1
                }{ ||A^T A||_1 ||A^{-1}||^2
                } \MyPunkt
    \]
\end{satz}
\begin{beweis}
    Sei $v$ Eigenvektor von $\mu$ (d. h. $v$ ist nicht der Nullvektor).
    Dann gilt:
    \[ R(B)v = \mu v \MyPunkt \]
    Mit Hilfe von \equref{EquDefResidual} und \equref{EquPanDefB}
    erh"alt man:
    \[ (E_n - t A^T A)v = v - tA^TA v = \mu v \MyPunkt \]
    Daraus folgt
    \[ A^T A v = \lambda v, \: \lambda = \frac{ 1-\mu }{ t } \]
    Also ist $\lambda$ ein Eigenwert von $A^T A$ und mit
    \ref{SatzEigenUngleichung} folgt
    \begin{eqnarray*}
        \frac{ 1 }{ ||A^{-1}||^2
             } \leq \lambda =
            \frac{ 1 - \mu }{ t } \leq ||A||^2 \\
        \Leftrightarrow
            1 - t||A||^2 \leq \mu \leq 1 - \frac{ t }{ ||A^{-1}||^2 }
    \end{eqnarray*}
    Mit Hilfe von Lemma \ref{SatzAtkinson} und \equref{EquPanDefT}
    ergibt sich die Behauptung.
\end{beweis}

Da die Matrix \[ R(B) = E_n - tA^T A \] symmetrisch ist, folgt aus
\ref{SatzSymmetrischSpektral} und \ref{SatzResidualEigenwert} eine
Ungleichung, die es erlaubt, die in \equref{EquPanDefB} definierte Matrix 
$B$ f"ur unsere Zwecke zu verwenden:
\begin{korollar}
\label{SatzNormNaheInvers}
    \[
        ||R(B)|| \leq
            1-\frac{ 1
                   }{ ||A^T A||_1 ||A^{-1}||^2
                   }
    \]
\end{korollar}
Die Benutzung dieser Folgerung wird in Unterkapitel \ref{SecNewton}
beschrieben.

% **************************************************************************

\MySection{Iterative Matrizeninvertierung}
\label{SecNewton}
\index{Iterationsverfahren} \index{Matrizeninvertierung} 
\index{Inverse!einer Matrix}
\index{Newton!Iterationsverfahren von}

In diesem Unterkapitel wird beschrieben, wie man eine gegebene 
N"aherungsinverse $B$ einer invertierbaren Matrix $A$ iterativ schrittweise
verbessern kann. Diese Methode ist in der Literatur als
{\em Newton-Verfahren} bekannt und wird in
\cite{PR85} und \cite{PR85a} sowie in \cite{Hous64} ab S. 64
behandelt. 

Um zu einem Iterationsverfahren zu gelangen, nehmen wir einige Umformungen
an einer Gleichung vor, in der das mit \equref{EquDefResidual} $R(B)$ 
vorkommt:
\begin{MyEqnArray}
                    \MT R(B)          \MT = \MT 0_{n,n} \MNl
    \Leftrightarrow \MT R(B) + E_n    \MT = \MT E_n \MNl
    \Leftrightarrow \MT (R(B) + E_n)B \MT = \MT B \MNl
    \Leftrightarrow \MT (2E_n- BA)B   \MT = \MT B
\end{MyEqnArray}
Man definiert mit Hilfe der letzten dieser Gleichungen die Iteration
\begin{eqnarray}
    B_0 & := & B \nonumber \\
    B_i & := & (2E_n - B_{i-1}A)B_{i-1} \label{EquPanDefNewton} \MyPunkt
\end{eqnarray}
Betrachtet man nun \equref{EquDefResidual} f"ur $B_i$ statt $B$, bekommt
man eine Aussage "uber die St"arke der Konvergenz der Iteration:
\begin{eqnarray*}
    R(B_i) & = & E_n - B_iA \\
           & = & E_n - (2E_n - B_{i-1}A)B_{i-1}A \\
           & = & E_n - 2B_{i-1}A + (B_{i-1}A)^2 \\
           & = & (R(B_{i-1}))^2
\end{eqnarray*}
Daraus folgt, da"s f"ur jede Matrixnorm gilt:
\[ ||R(B_i)|| = ||(R(B_{i-1}))^2|| \MyPunkt \]

Unter der Voraussetzung \Beq{EquResidualLessOne} ||R(B)|| < 1 \Eeq
konvergiert $||R(B_i)||$ also quadratisch gegen Null und die
$B_i$ entsprechend gegen $A^{-1}$.

Um das Iterationsverfahren einsetzen zu k"onnen, m"ussen wir noch 
feststellen, wieviele Iterationen erforderlich sind, um eine bestimmte
Genauigkeit zu erreichen. Dazu definieren wir als Abk"urzung
\[ q:= ||R(B)|| , \: q_i:= ||R(B_i)||, \: q_0 := q \MyPunkt \]
Damit das Verfahren "uberhaupt in praktisch nutzbarer Weise konvergiert,
darf $q$ nicht beliebig nahe bei $1$ liegen. Sei $c \in \Nat$ gegeben. 
Dann nehmen wir an, da"s f"ur $q$ gilt:
\Beq{EquSupposeQ} 
    q = 1 - \frac{1}{n^c} \MyPunkt 
\Eeq
Da das Verfahren quadratisch konvergiert, gilt f"ur $k \in \Nat_0$:
\[
    q_k = q^{2^k} \MyPunkt
\]
Mit \equref{EquSupposeQ} erhalten wir:
\[
    q_k = \lb 1 - \frac{1}{n^c} \rb^{2^k} \MyPunkt
\]
F"ur unsere Anwendung sei
\[ q_k < \frac{1}{d} , \: d \in \Nat \] ausreichend. Mit 
\[ 
    k := e \log(n), \: e \in \Rationals_{>0} 
\]  erhalten wir:
\[ 
    \lb 1 - \frac{1}{n^c} \rb^{n^e} = \frac{1}{d} \MyPunkt
\]
Da uns die Anzahl der Iterationen interessiert, l"osen wir diese Gleichung
nach $e$ auf und nehmen dazu $n \rightarrow \infty$ an:
\begin{MyEqnArray}
    \MT \lb 1 - \frac{1}{n^c} \rb^{n^e} \MT = \MT \frac{1}{d} \MNl
    \Rightarrow \MT
    \lb \lb 1 - \frac{1}{n^c} \rb^{n^c} \rb^{n^e / n^c} \MT = \MT 
                                                       \frac{1}{d} \MNl
    \Rightarrow \MT
    \MathE^{- (n^e / n^c) } \MT \approx \MT \frac{1}{d} \MNl
    \Rightarrow \MT
    \ln(d) \MT \approx \MT n^{e-c} \MNl
    \Rightarrow \MT
    \frac{ \ln(\ln(d)) }{ \ln(n) } + c \MT \approx \MT e    
\end{MyEqnArray}
Wir k"onnen also f"ur wachsende $n$ annehmen, da"s gilt
\[ c \approx e \MyPunkt \]
Um die --- bei gegebenem $n$ --- von $e$ bestimmte Anzahl der Iterationen
zu erhalten, m"ussen wir $c$ n"aher bestimmen. Vergleichen wir dazu
\equref{EquSupposeQ} mit der Aussage von \ref{SatzNormNaheInvers}.
Aus Grundlagen "uber Normen (z. B. \cite{Isaa73} ab S. 12) erhalten wir
f"ur eine beliebige Matrix $W$:
\[
    ||W||_2 \leq \sqrt{||W^2||_1} \MyPunkt
\]
Wir k"onnen also die Berechnung von $c$ auf die Bestimmung der 1-Norm, also
der maximalen Betragsspaltensumme, einer Matrix zur"uckf"uhren und erkennen,
da"s $c$ von der Gr"o"se der Matrizenelemente abh"angt.

Somit erkennen wir eine weitere Einschr"ankung f"ur die Anwendbarkeit des
Algorithmus. Die Matrix mu"s die Bedingung
\Beq{EquWellConditioned}
    ||A^T A||_1 ||A^{-1}||^2 \leq n^c
\Eeq f"ur ein $c \in \Nat$ erf"ullen. Die Wahl von $c$ h"angt davon ab, 
auf welche Matrizen man
das Verfahren anwendet. Um P-Alg. mit den anderen Algorithmen 
vergleichen zu k"onnen, nehmen wir im weiteren Verlauf des Textes $c=1$ an.

% **************************************************************************

\MySection{Determinatenber. mit Hilfe der Methoden von Krylov und Newton}
\label{SecAlgPan}

In diesem Unterkapitel werden die in den vorangegangenen Abschnitten
dargestellten Methoden zu einem Algorithmus zur Determinantenberechnung
zusammengefa"st, auf den mit {\em P-Alg.} Bezug genommen wird\footnote{
vgl. Unterkapitel \ref{SecBez}}. Weiterhin wird die Anzahl der Schritte und
der Prozessoren analysiert. 

Im folgenden ist mit $n^x$ jeweils $\lc n^x \rc$ gemeint.

Im folgenden wird mit Hinweis auf die Bemerkungen auf S.
\pageref{PageAlg2MatMult} f"ur die Multplikation zweier 
$n \times n$-Matrizen ein Aufwand von \[ \gamma_S (\lc\log(n)\rc + 1) \]
Schritten und \[ \gamma_P n^{2+\gamma} \] Prozessoren in Rechnung gestellt.

Es sei die Determinante der $n \times n$-Matrix $A$ zu berechnen.
Zuerst wird die Krylov-Matrix entsprechend der Darstellung in Abschnitt
\ref{SecKrylov} berechnet. Es ist "ublich, f"ur den dort erw"ahnten
Vektor $z$ zur Berechnung der iterierten Vektoren den Einheitsvektor
zu verwenden, dessen s"amtliche Elemente gleich $1$ sind. Aufgrund der
aus theoretischer Sicht ohnehin bereits eingeschr"ankten Verwendbarkeit
der Methode von Krylov f"ur unsere Zwecke, bedeutet die Beschr"ankung auf
diesen Vektor nur eine unbedeutende Verschlechterung des Algorithmus.

Zuerst sind die iterierten Vektoren anhand des Vektors $z$ und der Matrix
$A$ zu berechnen. Dies geschieht anhand der nachstehenden Gleichungen
(\cite{Pan85}, \cite{BM75} S. 128, \cite{Kell82}). Auf den rechten Seiten 
werden nur Ergebnisse vorangegangener Gleichungen oder $z$ bzw. $A$ 
verwendet. Auf den linken Seiten stehen neu berechnete Ergebnisse. Die Terme 
auf den rechten Seiten beschreiben Matrizenmultiplikationen und die linken 
Seiten deren Ergebnisse.
\begin{eqnarray*}
    [ A^3v,\, A^2v ] & = & A^2[ Av,\,v ] \\ \relax
    [ A^7v,\, A^6v,\, A^5v,\, A^4v ] & = & 
                                      A^4[ A^3v,\, A^2v,\, Av,\, v ] 
                                       \\ \relax
    & \vdots & \\ \relax
    [ A^{2*2^h}v,\, \ldots,\, A^{2^h}v ] & = & 
                              A^{2^h}[ A^{2^h-1}v,\, \ldots,\,v ] 
\end{eqnarray*}
Um die jeweils n"achste Gleichung mit Hilfe der schon bekannten Ergebnisse
aufzustellen, sind zwei Matrizenmultiplikationen erforderlich. Mit der
ersten wird die n"achste ben"otigte Potenz von $A$ berechnet. Mit der 
zweiten wird der jeweilige Term der rechten Seite der Gleichung ausgewertet.
Die $n$ gesuchten iterierten Vektoren k"onnen auf diese Weise in
\[ 
    \gamma_S (\lc\log(n)\rc + 1) \lc\log(n)\rc
\] 
Schritten von
\[ 
    \gamma_P n^{2+\gamma}
\] 
Prozessoren berechnet werden.

Anschlie"send ist f"ur die aus iterierten Vektoren bestehende Krylov-Matrix
eine N"aherungsinverse entsprechend der Ausf"uhrungen in Unterkapitel
\ref{SecGuessInverse} zu berechnen. Dazu werden die Gleichungen
\equref{EquPanDefB} und \equref{EquPanDefT} verwendet. Betrachtet man 
diese Gleichungen, erkennt man, da"s zur Berechnung der dortigen Matrix $B$
aus der Matrix $A$ eine Matrizenmultiplikation, eine Berechnung der
$1$-Norm einer Matrix, eine Division durch einen Skalar sowie eine 
Matrix--Skalar--Multiplikation erforderlich ist. An dieser Stelle wird
deutlich, da"s der Algorithmus nicht ohne Divisionen auskommt.

Um die $1$-Norm zu erhalten, sind $n$ Summen von je $n$-Matrizenelementen
zu berechnen und miteinander zu vergleichen. Dies kann mit Hilfe der 
Bin"arbaummethode nach Satz \ref{SatzAlgBinaerbaum} in
\[
    2\lc\log(n)\rc
\] Schritten von
\[ 
    n \lf \frac{n}{2} \rf
\] Prozessoren durchgef"uhrt werden.

Insgesamt kann die Berechnung der N"aherungsinversen also in
\[
    \gamma_S (\lc\log(n)\rc+1) + 2\lc\log(n)\rc + 2
\] Schritten von
\[
    \gamma_P n^{2+\gamma}
\] Prozessoren geleistet werden.

Nachdem sie zur Verf"ugung steht, kann sie mit Hilfe des in Unterkapitel
\ref{SecNewton} beschriebenen Verfahrens verbessert werden.

Ein Iterationsschritt mit Hilfe von Gleichung \equref{EquPanDefNewton}
erfordert zwei Multiplikationen und eine Addtition von Matrizen. Setzt man
f"ur die Anzahl der durchzuf"uhrenden Iterationen zun"achst die Unbestimmte
$I$ ein, kann das Iterationsverfahren in
\[
    I * 2 \gamma_S (\lc\log(n)\rc + 1) + 1
\] Schritten von
\[
    \gamma_P n^{2+\gamma}
\] Prozessoren auf die N"aherungsinverse angewendet werden um diese
bis auf eine ausreichende Genauigkeit an die gesuchte Inverse anzun"ahern.

Schlie"slich ist noch eine Matrix--Vektor--Multiplikation durchzuf"uhren,
um die Methode von Krylov zur Berechnung der Koeffizienten des
charakteristischen Polynoms zu vollenden. Betrachtet man den Vektor wiederum
als Matrix, kann dies in
\[
    \gamma_S(\lc\log(n)\rc+1)
\] Schritten von
\[
    \gamma_P n^{2+\gamma}
\] Prozessoren erledigt werden. Nach \ref{SatzDdurchP} hat man 
mit den Koeffizienten auch die Determinante berechnet.

Hier wird eine Einschr"ankung f"ur den Algorithmus deutlich. Da ein 
N"aherungsverfahren verwendet wird, ist es nicht m"oglich, die Determinante
f"ur Matrizen mit Elementen aus $\Rationals$ genau zu berechnen. Diese
m"ussen deshalb aus $\Integers$ stammen, denn in diesem Fall sind
die Koeffizienten des charakteristischen Polynoms ebenfalls ganzzahlig und
die Determinante kann bei gen"ugend genau durchgef"uhrtem Newton-Verfahren
durch Rundung genau ermittelt werden.

Betrachtet man die Einschr"ankungen f"ur die Verwendbarkeit der Methode
von Krylov (vgl. Unterkaptel \ref{SecKrylov}) sowie die Bedingung
\equref{EquWellConditioned} auf S. \pageref{EquWellConditioned}, erkennt
man, da"s insgesamt nicht unerhebliche Anforderungen an die Matrix,
deren Determianten zu berechnen ist, gestellt werden m"ussen, damit der
in diesem Kapitel dargestellte Algorithmus anwendbar ist.

Als Gesamtaufwand f"ur den Algorithmus erh"alt man f"ur die
Anzahl der Schritte:
\begin{eqnarray*}
   & & \gamma_S (\lc\log(n)\rc + 1) \lc\log(n)\rc \\
   & + & \gamma_S (\lc\log(n)\rc +1 ) + 2\lc\log(n)\rc + 2 \\
   & + & I * 2 \gamma_S (\lc\log(n)\rc + 1) + 1 \\
   & + & \gamma_S(\lc\log(n)\rc+1) \\
   & = &
      \gamma_S \lb \lc\log(n)\rc^2 + 5 \lc\log(n)\rc
    + I \lb 2 \lc\log(n)\rc + 2 \rb + \frac{3}{\gamma_S} + 2 \rb
\end{eqnarray*}
Mit Verweis auf Unterkapitel \ref{SecNewton} nehmen wir $I= \log(n)$ an.
Dadurch lautet der Term f"ur die Anzahl der Schritte:
\[
    \gamma_S \lb 3 \lc\log(n)\rc^2 + 7 \lc\log(n)\rc
    + \frac{3}{\gamma_S} + 2 \rb \MyPunkt
\]
Vergleicht man die Annahme f"ur $I$ mit den Ausf"uhrungen in Unterkapitel
\ref{SecNewton}, erkennt man, da"s dies der beste mit dem Algorithmus zu
erreichende Wert ist. Bei schlechteren Randbedingungen erh"alt man f"ur
die Schritte einen entsprechend schlechteren Wert.

F"ur die Prozessoren ergibt sich:
\[ 
    \gamma_P n^{2+\gamma} \MyPunkt
\]
Da die Matrizenmultiplikation nach \cite{CW90} nur f"ur sehr gro"se $n$
Vorteile bringt, wird \[ \gamma_P = \gamma_S = \gamma = 1 \] gesetzt,
um eine f"ur die Anwendung des Algorithmus realistische 
Vergleichsm"oglichkeit mit den anderen Algorithmen zu bekommen.

Vergleicht man P-Alg. mit C-Alg., BGH-Alg. und B-Alg., so erkennt man, da"s
die Anzahl der Prozessoren in P-Alg. gleich ist mit der Anzahl der
Prozessoren f"ur die Matrizenmultiplikation ist und somit um eine Potenz
geringer als beim besten der drei anderen Algorithmen.

Ein gravierender Nachteil von P-Alg. sind die Einschr"ankungen
f"ur die Benutzbarkeit. Die Bedingung, da"s die Matrizenelemente ganzzahlig
sein m"ussen, l"a"st sich durch Ausnutzung der Eigenschaften einer 
Determinaten (siehe Definition \ref{DefDet}) ausgleichen. Dazu werden
die Matrizenelemente durch einen geeigneten Faktor multipliziert, so da"s
die resultierende Matrix nur ganzzahlige Elemente enth"alt. Nachdem der
Algorithmus auf diese Matrix angewendet worden ist, kann man dann aus
dem Ergebnis auf die Determinate der urspr"unglichen Matrix schlie"sen.

Durch die in Unterkapitel \ref{SecKrylov} beschriebenen Einschr"ankungen,
kann f"ur P-Alg. nicht garantiert werden, da"s er f"ur jede invertierbare
Matrix eine Determinante ungleich Null liefert. Die Wahrscheinlichkeit f"ur
einen solchen Fall ist zwar gering, jedoch unterscheidet diese 
Beschr"ankung P-Alg. von den anderen drei Algorithmen.

Weiterhin mu"s man bei P-Alg. nat"urlich wiederum auf die Existenz von 
Divisionen und das Fehlen von Fallunterscheidungen hinweisen.


%
% Datei: implemen.tex
%
\MyChapter{Implementierung}
\label{ChapImplemen}

In diesem Kapitel wird die Implementierung der in den Kapitel 
\ref{ChapCsanky} bis \ref{ChapPan} behandelten Algorithmen 
beschrieben\footnote{Der Algorithmus nach dem Entwicklungssatz von
Laplace (siehe Unterkapitel \ref{SecDivCon}) wird hier nicht 
ber"ucksichtigt.}.

\MySection{Erf"ullte Anforderungen}
\label{SecAnford}

In diesem Unterkapitel werden die wesentlichen Eigenschaften des
implementierten Programms beschrieben um einen "Uberblick "uber dessen
Leistungsmerkmale zu geben.

Die Algorithmen sind in Modula-2 auf einem Rechner mit einem Prozessor
implementiert\footnote{Megamax Modula-2 auf einem ATARI ST}. 
Als Literatur "uber die Programmiersprache Modula-2 ist
z. B. \cite{DCLR86} empfehlenswert.

Alle Algorithmusteile, die parallel ausgef"uhrt werden sollen,
werden mit Hilfe von Schleifen nacheinander ausgef"uhrt. Mit Hilfe von
Z"ahlprozeduren (siehe Modul `Pram') wird w"ahrend der Programmausf"uhrung
ermittelt, in wieviel Schritten und mit Hilfe von wieviel Prozessoren der
Algorithmus durch eine PRAM abgearbeitet werden kann.

Das Programm erm"oglicht es, Matrizen anhand von Parametervorgaben 
zuf"allig zu erzeugen. Die Parameter sind
\begin{itemize}
\item Gr"o"se der Matrix,
\item
      Rang (um auch Matrizen mit der Determinante Null gezielt 
      erzeugen zu k"onnen),
\item
      Wahl einer der Mengen\footnote{durch geeignete Parameterwahl mit dem 
      Befehl `param'}  $\Nat$, $\Integers$, $\Rationals$ oder
      $\Rationals^+$ \footnote{alle Elemente aus $\Rationals$, die
      gr"o"ser oder gleich Null sind} f"ur die Matrizenelemente,
      da Algorithmen evtl. nicht auf jede dieser Mengen anwendbar sind,
      und
\item 
      Vielfachheiten der Eigenwerte (wichtig f"ur den Algorithmus von Pan).
\end{itemize}

F"ur erzeugte Matrizen kann jeder der implementierten Algorithmen aufgerufen
werden. Eine Matrix wird zusammen mit den durch die verschiedenen 
Algorithmen berechneten Determinanten und den Me"sergebnissen der 
Z"ahlprozeduren unter einem Namen auf dem Hintergrundspeicher abgelegt. 
Diese Verwaltung geschieht automatisch. Dazu mu"s der Benutzer vor jedem
neuen Anlegen eines aus den obigen Daten bestehenden Datensatzes einen
Namen f"ur diesen Datensatz angeben.

Das Programm ist kommandoorientiert. D. h. nach dem Start wird der 
Benutzer aufgefordert, Befehle einzugeben, die nacheinander ausgef"uhrt 
werden.

Die Lesbarkeit des Quelltextes wird durch h"aufige Kommentare gesteigert.

% **************************************************************************

\MySection{Bedienung des Programms}
\label{SecBedienung}

In diesem Unterkapitel wird die Benutzung des Programms beschrieben.

Nach dem Programmstart mu"s zun"achst mit Hilfe des Befehls {\em find}
ein Datensatz festgelegt werden, der bearbeitet werden soll. Dies kann
ein bereits existierender oder ein neu anzulegender sein.

Bei einem neu angelegten Datensatz m"ussen danach mit Hilfe des Befehls 
{\em param} die Parameter f"ur die zu erzeugende Matrix festgelegt werden.

Anschlie"send kann man mit {\em gen} eine neue Matrix erzeugen lassen.
Nachdem mit {\em param} die Parameter festgelegt sind, kann mit mit {\em gen}
zu jeder Zeit eine neue Matrix generieren lassen, wodurch jedoch die 
vorangegangene Matrix verloren geht.

Wenn eine Matrix erzeugt worden ist, kann man mit Hilfe der Befehle
{\em berk, bgh, csanky} und {\em pan} die entsprechenden Algorithmen auf 
die generierte Matrix anwenden.

Mit {\em show} kann man sich zu jeder Zeit den aktuellen mit {\em find}
bestimmten Datensatz auf dem Bildschirm anzeigen lassen. Die generierte
Matrix wird nur ausgegeben, wenn eine Matrixzeile in eine Bildschirmzeile 
pa"st. Gr"o"sere Matrizen kann man mit {\em mshow} trotzdem ausgeben 
lassen.

Falls man einen weiteren Datensatz anlegen oder einen bereits vorhandenen
wieder bearbeiten will, benutzt man erneut den Befehl {\em find}. Die
Speicherung des alten Datensatzes geschieht automatisch.

Weitere Befehle, die man nach dem Programmstart zu jedem Zeitpunkt angeben
kann, sind: {\em del, exit, h, help, hilfe, ?, ls} und {\em q} . Ihre
Bedeutung ist in der folgenden Liste aller erlaubten Befehle erkl"art:

\begin{MyDescription}
\MyItem{\em berk}
    Der Algorithmus von Berkowitz aus Kapitel \ref{ChapBerk} wird auf
    die Matrix des aktuellen Datensatzes angewendet. Die berechnete
    Determinante sowie die Ergebisse der Z"ahlprozeduren 
    (siehe Modul `Pram') werden im aktuellen Datensatz abgelegt.
\MyItem{\em bgh}
    Die Wirkung dieses Befehls ist analog zu der des Befehls {\em berk},
    jedoch bezogen auf den Algorithmus von Borodin, von zur Gathen und
    Hopcroft aus Kapitel \ref{ChapBGH}.
\MyItem{\em csanky}
    Die Wirkung dieses Befehls ist analog zu der des Befehls {\em berk},
    jedoch bezogen auf den Algorithmus von Csanky aus Unterkapitel 
    \ref{SecAlgFrame}.
\MyItem{\em del}
    Es wird nach einem Datensatznamen gefragt. Der zugeh"orige Datensatz
    wird im Hauptspeicher und auf dem Hintergrundspeicher gel"oscht.
\MyItem{\em exit}
    Das Programm wird beendet. Der aktuelle mit {\em find} festgelegte
    Datensatz wird automatisch auf dem Hintergrundspeicher abgelegt.
\MyItem{\em find}
    Es wird nach einem Datensatznamen gefragt. Falls ein Datensatz mit 
    diesem Namen bereits im Hauptspeicher abgelegt ist, wird dieser erneut
    zum aktuellen Datensatz. Falls dies nicht der Fall ist und auf dem 
    Hintergrundspeicher ein Datensatz mit diesem Namen abgelegt ist,
    wird dieser in den Hauptspeicher geladen und zum aktuellen Datensatz.
    Falls beide F"alle nicht zutreffen, wird ein neuer Datensatz mit dem 
    angegebenen Namen im Hauptspeicher angelegt und als aktueller 
    Datensatz betrachtet.
\MyItem{\em gen}
    Entsprechend der mit {\em param} festgelegten Parameter wird eine neue
    Matrix f"ur den aktuellen Datensatz generiert. Die dort durch die
    Befehle {\em berk, bgh, csanky} und {\em pan} abgelegten Daten werden
    gel"oscht.
\MyItem{\em h, help, hilfe, ?}
    Durch diese Befehle wird eine Kurzbeschreibung aller erlaubten Befehle
    auf den Bildschirm ausgegeben.
\MyItem{\em ls}
    Auf dem Bildschirm wird eine Liste der Namen der im Hauptspeicher
    befindlichen Datens"atze ausgegeben.
\MyItem{\em mshow}
    Die Matrix des aktuellen Datensatzes wird auf dem Bildschirm ausgegeben.
\MyItem{\em pan}
    Die Wirkung dieses Befehls ist analog zu der des Befehls {\em berk},
    jedoch bezogen auf den Algorithmus von Pan aus Kapitel \ref{ChapPan}.
\MyItem{\em param}
    Es wird nach den Parametern f"ur die mit Hilfe von {\em gen} zu
    generierende Matrix gefragt. Alle zuvor im aktuellen Datensatz
    abgelegten Daten werden gel"oscht.
\MyItem{\em q}
    Die Wirkung dieses Befehls ist mit der des Befehls {\em exit} identisch.
\MyItem{\em show}
    Der aktuelle Datensatz wird auf dem Bildschirm ausgegeben. Die 
    Matrix wird nur ausgegeben, falls eine Matrixzeile in eine 
    Bildschirmzeile pa"st. Mit dem Befehl {\em mshow} kann die Matrix 
    dennoch ausgegeben werden.
\end{MyDescription}

% **************************************************************************

\MySection{Die Modulstruktur}
\label{SecModule}

In diesem Unterkapitel wird die Struktur des implementierten Programms 
beschrieben. Dazu wird zu jedem Modul dessen Aufgabe und evtl. dessen
Beziehung zu anderen Modulen angegeben. Alle Beschreibungen von Details
der Implementierung, die f"ur die Benutzung des Programms und f"ur 
das Verst"andnis von dessen Gesamtstruktur unwichtig sind, erfolgen durch 
Kommentare innerhalb des Quelltextes (siehe Anhang).

Eine Beschreibung des Programmoduls {\em main} entspricht einer Erkl"arung
der Benutzung des Programms. Diese ist in Unterkapitel \ref{SecBedienung}
zu finden.

Die Beschreibung der Programmodule zum Test einzelner Teile des 
Gesamtprogramms ist von untergeordnetem Interesse und beschr"ankt sich 
deshalb auf kurze Bemerkungen "uber ihren Zweck ihm Rahmen der 
alphabetischen Auflistung (s. u.).

Die vorrangige Aufmerksamkeit des an der Implementierung der Algorithmen
Interessierten sollte sich auf die Module {\em Det} und {\em Pram} sowie
auf das Programmodul {\em algtest} richten. 

Die genannten vorrangig interessanten Module sind im Anhang 
\ref{ChapImplDet} zusammen mit dem Modul {\em main} gesammelt. Die
weniger interessanten Testprogrammodule sind im Anhang \ref{ChapTest}
aufgef"uhrt. Alle weiteren Module sind in alphabetischer Reihenfolge in
Anhang \ref{ChapSupport} zu finden.

Zun"achst wird anhand eines {\em reduzierten Ebenenstrukturbildes}
(Erkl"arung s. u.) ein "Uberblick "uber die Programmstruktur geben. 
Anschlie"send erfolgt eine alphabetische Auflistung der Module und 
ihrer Erkl"arungen.

In das erw"ahnte Strukturbild sind alle Module nach folgenden Regeln
eingetragen:
\begin{itemize}
\item
      Die Eintragung erfolgt ebenenweise. Die niedrigste Ebene ist im
      Bild unten zu finden und die h"ochste oben. Jedes Modul geh"ort 
      genau einer Ebene an.
\item
      Jedes Modul wird im Rahmen der Ma"sgaben durch die anderen Regeln
      in einer m"oglichst niedrigen Ebene eingetragen.
\item
      Von jedem Modul A aus, das ein Modul B benutzt, z. B. durch Aufruf von
      Prozeduren des Moduls B, wird im Rahmen der Einschr"ankungen durch
      andere Regeln ein Pfeil auf dieses Modul $B$ gerichtet.
\item 
      Jedes Modul wird so eingetragen, da"s kein Pfeil von ihm auf ein 
      Modul auf der gleichen oder einer h"oheren Ebene gerichtet ist.
\item
      Alle Pfeile von einem Modul aus auf Module, die nicht genau eine
      Ebene tiefer angeordnet sind, werden weggelassen.
\end{itemize}
Durch die letzte Regel gewinnt das Strukturbild erheblich an
"Ubersichtlichkeit, ohne wesentlichen Informationsgehalt zu verlieren.
Aus dem Quelltext jedes Moduls ist zu entnehmen, welche Module, au"ser
den im Bild angegebenen, sonst noch benutzt werden. Die aufgef"uhrten 
Regeln liefern das in Abbildung \ref{PicModule} angegebene Bild.

\begin{figure}[htb]
\begin{center}
    %
% You need 'epic.sty'
%
\setlength{\unitlength}{1mm}
\makeatletter
\def\Thicklines{\let\@linefnt\tenlnw \let\@circlefnt\tencircw
\@wholewidth4\fontdimen8\tenln \@halfwidth .5\@wholewidth}
\makeatother
\begin{picture}(105.00,115.00)
\put(35.75,9.00){SysMath}
\put(60.75,9.25){Sys}
\put(36.00,20.50){Func}
\put(32.75,30.25){Rnd}
\put(48.00,30.00){Str}
\put(21.00,39.00){rndtest}
\put(42.75,39.00){Type}
\put(60.50,38.75){strtest}
\put(15.00,47.75){Frag}
\put(32.75,48.00){List}
\put(51.00,48.00){Simptype}
\put(78.00,48.00){typetest}
\put(15.00,57.00){Mat}
\put(30.00,57.25){Cali}
\put(44.75,57.00){Inli}
\put(60.00,57.25){Reli}
\put(17.75,66.00){Hash}
\put(30.50,66.00){listtest}
\put(54.00,66.00){Pram}
\put(35.75,74.75){Rema}
\put(54.00,75.00){pramtest}
\put(27.00,84.25){Data}
\put(42.50,84.50){Mali}
\put(42.00,94.00){Det}
\put(40.50,102.00){main}
\put(45.00,96.75){\vector(-1,-4){0}}
\drawline(45.25,102.00)(45.00,96.75)
\put(45.00,88.00){\vector(-1,-4){0}}
\drawline(45.25,93.50)(45.00,88.00)
\put(38.25,78.25){\vector(4,-3){0}}
\drawline(31.00,84.00)(38.25,78.25)
\put(39.50,78.00){\vector(-1,-1){0}}
\drawline(46.00,83.75)(39.50,78.00)
\put(58.75,69.00){\vector(-1,-4){0}}
\drawline(59.00,74.75)(58.75,69.00)
\put(53.75,69.50){\vector(3,-1){0}}
\drawline(39.25,74.25)(53.75,69.50)
\put(31.75,61.00){\vector(2,-1){0}}
\drawline(22.25,65.50)(31.75,61.00)
\put(33.50,60.75){\vector(-1,-3){0}}
\drawline(35.25,65.50)(33.50,60.75)
\put(36.00,61.25){\vector(-4,-1){0}}
\drawline(57.25,66.00)(36.00,61.25)
\put(62.25,61.00){\vector(1,-1){0}}
\drawline(58.50,65.25)(62.25,61.00)
\put(18.25,51.00){\vector(1,-4){0}}
\drawline(18.00,56.50)(18.25,51.00)
\put(36.25,51.00){\vector(1,-3){0}}
\drawline(34.25,57.25)(36.25,51.00)
\put(37.75,51.75){\vector(-2,-1){0}}
\drawline(47.25,56.75)(37.75,51.75)
\put(40.25,51.50){\vector(-4,-1){0}}
\drawline(63.75,56.75)(40.25,51.50)
\put(43.50,42.75){\vector(4,-1){0}}
\drawline(19.25,47.00)(43.50,42.75)
\put(45.75,42.50){\vector(3,-2){0}}
\drawline(37.75,48.00)(45.75,42.50)
\put(47.00,42.75){\vector(-2,-1){0}}
\drawline(57.50,47.50)(47.00,42.75)
\put(48.75,42.50){\vector(-4,-1){0}}
\drawline(82.50,47.75)(48.75,42.50)
\put(34.00,33.75){\vector(1,-1){0}}
\drawline(28.75,38.75)(34.00,33.75)
\put(49.75,33.50){\vector(3,-4){0}}
\drawline(46.00,38.50)(49.75,33.50)
\put(51.00,33.50){\vector(-3,-1){0}}
\drawline(65.25,38.75)(51.00,33.50)
\put(39.00,24.00){\vector(1,-2){0}}
\drawline(35.50,30.25)(39.00,24.00)
\put(40.50,24.00){\vector(-2,-1){0}}
\drawline(50.50,29.50)(40.50,24.00)
\put(40.75,12.00){\vector(1,-4){0}}
\drawline(39.75,20.00)(40.75,12.00)
\end{picture}
%

    \caption{reduziertes Ebenenstrukturbild}
    \label{PicModule}
\end{center}
\end{figure}

Es folgen die Kurzbeschreibungen der einzelnen Module in alphabetischer 
Reihenfolge.

\begin{MyDescription}
\MyItem{\bf algtest (Programmodul)}
    Dieses Modul dient zum Test der Algorithmen zur 
    Determinantenberechnung ohne Behinderung durch Anforderungen 
    irgendwelcher Art, insbesondere
    ohne Beachtung der Parallelisierung und der Ma"sgabe, Matrizen
    beliebiger Gr"o"se zu verarbeiten.
\MyItem{Cali (CArdinal LIst)}
    In diesem Modul sind lineare Listen positiver ganzer Zahlen 
    implementiert. Es st"utzt sich auf das Modul `List'.
\MyItem{Data}
    Das Modul `Data' dient der Verwaltung der Datens"atze bestehend aus 
    Matrizen und ihren Parametern, sowie der berechneten Determinanten 
    und der dabei gez"ahlten Schritte und Prozessoren.
\MyItem{Det}
    In diesem Modul sind die Algorithmen zur parallelen
    Determinantenberechnung implementiert.
\MyItem{Frag (array FRAGments)}
    Im Modul `Frag' sind Felder beliebiger variabler L"ange und beliebigen
    Inhalts implementiert. Das Modul ist erforderlich, um Matrizen 
    verarbeiten zu k"onnen, deren Gr"o"se durch den Benutzer erst w"ahrend 
    der Laufzeit des Programms festgelegt wird.
    
    Das Modul profitiert von der Verwaltung von 
    Elementen beliebiger Typen durch das Modul `Type'.
\MyItem{Func (FUNCtions)}
    In diesem Modul sind verschiedene Prozeduren und Funktionen 
    insbesondere f"ur mathematische Zwecke zusammengefa"st.
\MyItem{Hash}
    Durch dieses Modul werden Prozeduren zur Streuspeicherung, auch unter
    dem Namen `Hashing' bekannt, zur Verf"ugung gestellt. Das Modul wird 
    durch den Algorithmus von Borodin, von zur Gathen und Hopcroft im 
    Modul `Det' ben"otigt, um Zwischenergebnisse bei der parallelen 
    Berechnung von Termen zu speichern.
    
    Das Modul `Hash' erlaubt es, beliebige Daten zu speichern. Dabei wird
    auf das Modul `Type' zur Verwaltung von Elementen beliebiger Typen
    zur"uckgegriffen.
\MyItem{Inli (INteger LIst)}
    In diesem Modul sind lineare Listen ganzer Zahlen 
    implementiert. Es st"utzt sich auf das Modul `List'.
\MyItem{List}
    Das Modul `List' stellt Prozeduren zur Verwaltung von linearen 
    doppelt verketteten Listen bliebiger Elemente zur Verf"ugung. Analog
    zu den Modulen `Frag', `Hash' und `Mat' benutzt `List' das Modul
    'Type` zur Verwaltung von Elementen beliebiger Typen.
    
    Auf dem Modul `List' bauen verschiedene Module zur Implementierung von
    Listen spezieller Typen auf.
\MyItem{\bf listtest (Programmodul)}
    Dieses Programmodul dient zum Test des Moduls `List'. Es verwendet
    dazu das Modul `Cali'.
\MyItem{\bf main (Programmodul)}
    Dies ist das Hauptmodul des gesamten Programms. Es nimmt die Befehle
    des Benutzers entgegen und ruft die entsprechenden Prozeduren auf.
    Die Benutzung ist in Unterkapitel \ref{SecBedienung} beschrieben.
\MyItem{Mali (MAtrix LIst)}
    In diesem Modul sind lineare Listen von Matrizen
    implementiert. Es st"utzt sich auf die Module `List' und `Mat'.
\MyItem{Mat (MATrix)}
    Dieses Modul stellt Prozeduren zur Verwaltung von zweidimensionalen 
    Matrizen beliebiger Gr"o"se f"ur beliebige Elemente zur Verf"ugung.
    Es st"utzt sich auf das Modul `Frag' zur Verwaltung der Felder
    beliebiger Gr"o"se und auf das Modul `Type' zur Verwaltung von 
    Elementen beliebiger Typen.
\MyItem{Pram}
    Das Modul `Pram' stellt die Z"ahlprozeduren zur Verf"ugung, die zur 
    Ermittlung des Aufwandes f"ur
    eine PRAM zur Abarbeitung der verschiedenen Algorithmen zur 
    Determiantenberechnung erforderlich sind. Das Modul wird durch die
    Module 'Det' und 'Rema' benutzt und verwendet seinerseits insbesondere
    das Modul `Cali' f"ur Verwaltungsaufgaben.
\MyItem{\bf pramtest (Programmodul)}
    Dieses Programmodul dient zum Test des Modul `Pram'.
\MyItem{Reli (REal LIst)}
    In diesem Modul sind lineare Listen von Flie"skommazahlen
    implementiert. Es st"utzt sich auf das Modul `List'.
\MyItem{Rema (REal MAtrix)}
    Dieses Modul implementiert Matrizen aus Flie"skommazahlen. Es st"utzt
    sich dazu auf das Modul \nopagebreak[3] `Mat'.
\MyItem{Rnd  (RaNDomize)}
    Das Modul 'Rnd' erlaubt es, Zufallszahlen nach der linearen 
    Kongruenzmethode zu erzeugen. Es wird vom Modul `Rema' dazu benutzt,
    anhand von verschiedenen Parametern zuf"allige Matrizen zu generieren.
\MyItem{\bf rndtest (Programmodul)}
    Dieses Programmodul dient zum Test des Moduls `Rnd'.
\MyItem{simptype (SIMPle TYPE)} \sloppy
    Dieses Modul stellt Verwaltungsprozeduren f"ur die einfachen 
    Datentypen \newline[3] `LONGCARD', `LONGINT' und `LONGREAL' zur 
    Verf"ugung, damit
    sie in Verbindung mit dem Modul `Type' verwendet werden k"onnen.
    \fussy
\MyItem{Str (STRing)}
    Im Modul `Str' sind diverse Prozeduren zur Verarbeitung von 
    Zeichenketten implementiert.
\MyItem{\bf strtest (Programmodul)}
    Dieses Programmodul dient zum Test des Moduls `Str'.
\MyItem{Sys}
    Dieses Modul stellt Prozeduren zum Ablegen von Daten auf dem 
    Hintergrundspeicher zur Verf"ugung. Da die Behandlung der Massenspeicher
    auf den verschiendenen Rechnersystemen unterschiedlich ist, mu"s
    das Modul 'Sys' bei der Portierung des Programms auf einen
    anderen Rechner neu implementiert werden. 
\MyItem{SysMath}
    Die zur Verf"ugung gestellten mathematischen Funktionen sind von 
    System zu System unterschiedlich. Deshalb sind im Modul `SysMath' die
    Funktionen gesammelt, die im Programm benutzt werden. Bei der Portierung
    des Programms auf ein anderes Computersystem mu"s dieses Modul evtl.
    angepa"st werden.
\MyItem{Type}
    Dieses Modul dient der Verwaltung von Elementen beliebiger Datentypen.
    Ein neuer Typ wird im Rahmen dieses Moduls durch die Angabe 
    verschiedener Verwaltungsprozeduren definiert. Das Modul "ubernimmt 
    auf diese Weise die Sammlung der Eigenschaften verschiedener Typen, um 
    so die "Ubersichtlichkeit zu steigern. Ohne dieses Modul mu"s jedes 
    der Module `Frag', `Hash', `List' und `Mat' eine entsprechende 
    Verwaltung separat enthalten.
\MyItem{\bf typetest (Programmodul)}
    Dieses Programmodul dient zum Test des Moduls `Type'.
\end{MyDescription}

% **************************************************************************

\MySection{Anmerkungen zur Implementierung}

An dieser Stelle werden einige praktische Gesichtspunkte der
Implementierung kommentiert.

Vergleicht man das Modul `algtest' mit dem Rest des Quelltextes, so erkennt 
man, da"s insbesondere die Anforderung der {\em Pseudoparallelisierung}
die L"ange des Quelltextes stark vergr"o"sert. Bei den Algorithmen im
Modul `Det' handelt es sich ungef"ahr um eine Vergr"o"serung 
um den Faktor 5.

Weiterhin zeigt sich, da"s eine flexible auch nachtr"aglich 
erweiterbare Programmstruktur, die unter dem Gesichtspunkt sich evtl. 
anschlie"sender Arbeiten w"unschenswert ist, nicht unerheblichen Aufwand
bedeutet. So machen die eigentlich interessierenden Programmteile nur
ca. 30 Prozent des Quelltextes\footnote{Gesamtl"ange ca. 9500 Zeilen} aus. 
Der gesamte weitere Aufwand ergibt sich
einerseits aus verschiedenen Anforderungen an Leistungen und Struktur des 
Programms, andererseits aus der Notwendigkeit, Datentypen zu implementieren,
die die verwendete Sprache nicht standardm"a"sig zur Verf"ugung stellt.

Auf eine Implementierung auf leistungsf"ahigeren Rechnern wurde verzichtet,
da der ben"otigte Speicherplatz quadratisch mit der Anzahl der Zeilen und
Spalten der Matrizen w"achst. Der zus"atzliche Speicherplatz f"uhrt nicht
zu einer Steigerung der Matrizengr"o"se von weitreichendem Interesse.
Diese Beschr"ankung erlaubt es, mit der relativ geringen Rechenleistung
eine ATARI ST auszukommen.

Gr"o"sere Matrizen sind auch aus einem weiteren Grund nicht ohne
erheblichen weiteren Aufwand sinnvoll. Die Standardarithmetiken
verschiedener Implementierungen von Programmiersprachen erlauben eine
Rechengenauigkeit von typischerweise ca. 19 Stellen. Dies reicht nicht
aus, um Determinanten
gr"o"serer Matrizen "uberhaupt darzustellen. Deshalb ist es f"ur eine
deutliche Steigerung der Matrizengr"o"se erforderlich, eine eigene
Flie"skommaarithmetik zu implementieren, die es erm"oglicht, mit
beliebiger Genauigkeit\footnote{im Rahmen der physikalischen Grenzen}
zu rechnen.

Die praktischen Erfahrungen in verschiedenen Bereichen der angewandten
Informatik zeigen, da"s in bestehenden in der Regel zufriedenstellend
laufenden Programmen eine Restquote an Programmierfehlern im
Quelltext von ca. einem Fehler pro 1000 Zeilen Quelltext existiert.
Beim gegenw"artigen Stand der Technik ist es nicht m"oglich, Programme
wesentlich fehlerfreier zu bekommen.

Besonders bei mathematischen Programmen ist es in der Regel erforderlich,
trotzdem nahezu Fehlerfreiheit zu erreichen, was die Implementierung
solcher Programme zus"atzlich erschwert. Ein praktisches Beispiel
f"ur diese Probleme sind die implementierten Algorithmen zur
Determiantenberechnung. Die pseudoparallelen Algorithmen im Modul
`Det' besitzen eine Gesamtl"ange von ca. 2300 Zeilen, erheblich mehr
als die Implementierungen im Programmodul 'algtest'.

Da die Dauer einer Fehlersuche schwer abzusch"atzen ist und nur begrenzte
Zeit zur Verf"ugung stand, haben die genannten Schwierigkeiten dazu
gef"uhrt, da"s zwar alle Algorithmen lauff"ahig sind, jedoch
die im Anhang zu findenden Implementierungen leider keine
Determinanten berechnen:
\begin{itemize}
\item P-Alg. im Programmodul `algtest' sowie
\item BGH-Alg., B-Alg. und P-Alg. im Modul `Det'.
\end{itemize}
Durch umfangreiche Testl"aufe kann ausgeschlossen werden, da"s die Fehler
au"serhalb der Module zu suchen sind. Es mu"s sich jeweils um fehlerhafte
Implementierung der Algorithmusbeschreibungen in den jeweiligen Kapiteln
handeln (z. B. Vorzeichenfehler oder falsche Indizes).


%
% Datei: endbem.tex
%
\MyChapter{Nachbetrachtungen}
\label{ChapEndbem}

In diesem Kapitel werden die Ergebnisse dieser Arbeit abschlie"send 
betrachtet und bewertet.

\MySection{Vergleich der Algorithmen}
% Csanky: schnellster, am einfachsten zu implementieren, jedoch Divisionen
% BGH: relativ gesehen sehr langsam, Implementierung sehr aufw"andig,
%      nur f"ur positive ganze Zahlen (mit Tricks: ...)
% Berk: etwas langsamer als 'Csanky', etwas schwieriger zu implementieren,
%       keine Divisionen!
% Pan: nicht f"ur alle Matrizen verwendbar; nur ganze Zahlen

In diesem Unterkapitel werden die vier haupts"achlich in dieser Arbeit 
behandelten Algorithmen zusammenfassend miteinander verglichen.

Auf die Algorithmen wird, wie in Unterkapitel \ref{SecBez} definiert 
wird, mit Hilfe der Anfangsbuchstaben ihrer Autoren
Bezug genommen: C-Alg., BGH-Alg., B-Alg. und P-Alg..

Der Vergleich wird nach folgenden Kriterien durchgef"uhrt:
\begin{itemize}
\item
      Werden Divisionen benutzt?
\item
      Werden Fallunterscheidungen durchgef"uhrt?
\item
      Wie gro"s ist die Anzahl der Schritte?
\item
      Wie gro"s ist die Anzahl der Prozessoren?
\item
      Welche Einschr"ankungen f"ur die Anwendbarkeit gibt es?
\item
      Wie aufwendig ist die Implementierung?
\item
      Welche "Ahnlichkeiten in der Methodik gibt es? 
\end{itemize}
Im Anschlu"s an diesen Vergleich wird eine Bewertung der Algorithmen 
anhand der genannten Gesichtspunkte vorgenommen.

Zun"achst sei angemerkt, da"s keiner der Algorithmen Fallunterscheidungen
verwendet. Dies vereinfacht eine Betrachtung aus der Sicht des 
Schaltkreisentwurfs.

Es kommen lediglich B-Alg. und BGH-Alg. ohne Divisionen aus und k"onnen 
somit in beliebigen Ringen angewendet werden. C-Alg. und P-Alg. k"onnen nur
in K"orpern verwendet werden, falls man exakte Ergebnisse verlangt

P-Alg. darf nur auf ganzzahlige Matrizen
angewendet werden. Da P-Alg. Divisionen verwendet, ist es jedoch auch dann
nicht gew"ahrleistet, da"s er die Determinante exakt liefert. Hinzu kommt
bei P-Alg. als erheblicher Nachteil die eingeschr"ankte Verwendbarkeit, auch 
wenn sich dies in der praktischen Anwendung nicht sehr stark auswirkt. 
Diese Eigenschaft von P-Alg. ist mit dem Laufzeitverhalten von `Quicksort'
vergleichbar, das im {\em Average Case} sehr gut, jedoch im {\em Worst 
Case} sehr schlecht ist. P-Alg. ist theoretisch nur eingeschr"ankt 
verwendbar, praktisch jedoch nahezu uneingeschr"ankt, da zuf"allig
zusammengestellte Matrizen mit sehr hoher Wahrscheinlichkeit invertierbar
sind und die Bedingungen f"ur P-Alg. erf"ullen.

Betrachtet man die Anzahl der Schritte, stellt man fest, da"s C-Alg. 
am schnellsten ist, gefolgt von B-Alg. und P-Alg.. Das Schlu"slicht 
bildet mit einigem Abstand BGH-Alg. 

Bez"uglich der Anzahl der Prozessoren hat P-Alg. die Nase vorn, gefolgt
von C-Alg. und B-Alg.. Das Schlu"slicht bildet BGH-Alg. wiederum mit
Abstand. Zu beachten ist der sehr gute Wert f"ur die Prozessoren bei P-Alg.,
der sich vermutlich kaum weiter verbessern l"a"st.

Auf einem Rechner mit einem Prozessor ist P-Alg. der effizienteste, da
er insgesamt die wenigsten Operationen ben"otigt. Er wird gefolgt von 
C-Alg. B-Alg. liegt hier an dritter Stelle gefolgt von BGH-Alg..

Betrachtet man den Aufwand f"ur die Implementierung gemessen in Anzahl der
Quelltextzeilen\footnote{Dies besitzt in der Praxis in Verbindung mit
den anderen Eigenschaften der Algorithmen eine nicht zu untersch"atzende 
Bedeutung.}, stellt man fest, da"s C-Alg. bei weitem am einfachsten 
zu implementieren ist. Ihm folgt P-Alg. dichtauf. Die Implementierung
von B-Alg. ist bereits etwas aufwendiger, aber noch zumutbar. BGH-Alg.
bildet auch in diesem Punkt mit relativ gro"sem Abstand das Schlu"slicht.

Bei der Betrachtung der Verfahren, die die einzelnen Algorithmen verwenden,
erkennt man Parallelen zwischen C-Alg. und B-Alg. sowie zwischen 
BGH-Alg. und P-Alg.. Sowohl in C-Alg. als auch in B-Alg. wird jeweils
ein Satz verwendet, der zum Zeitpunkt der Ver"offentlichung der Algorithmen
bereits seit mehreren Jahrzehnte bekannt war. Die beiden S"atze sind 
durch diverse Umformungen f"ur die parallele Determinanteberechnung nutzbar
gemacht worden. In BGH-Alg. und P-Alg. hingegen werden jeweils mehrere
auch separat bedeutsame Verfahren zu einem Algorithmus in Verbindung 
gebracht.

Unter dem Gesichtspunkt, da"s BGH-Alg. nicht der einzige divisionsfreie
Algorithmus mit polynomiellem Aufwand ist, besitzt er wegen des
erforderlichen Implementierungsaufwandes und seiner relativen Langsamkeit
nur theoretisches Interesse. 

C-Alg. ist am brauchbarsten f"ur schnelle Erstellung einer Implementierung,
falls das Vorhandensein von Divisionen nicht weiter st"ort.

P-Alg. sollte verwendet werden, falls es im wesentlichen auf Geschwindigkeit
ankommt und die Einschr"ankungen f"ur die Anwendbarkeit sowie die
Existenz von Divisionen nicht st"oren.

B-Alg. ist der Algorithmus unter den vieren mit den ausgewogensten
Leistungsmerkmalen und geht insgesamt als Sieger aus dem Vergleich hervor.
Er besitzt eine ausreichende Effizienz, kommt ohne Divisionen aus und
unterliegt keinen sonstigen Einschr"ankungen, wie z. B. P-Alg.. Im
Zweifelsfall sollte immer B-Alg. verwendet werden.

Um zum Abschlu"s einen Eindruck von der Effizienz der
Algorithmen im direkten Vergleich
zu liefern, sind f"ur Tabelle \ref{TabVergleich} die aus den
Aufwandsanalysen hervorgegangene Terme beispielhaft ausgewertet worden.
In der Tabelle werden bei den Anzahlen der Prozessoren die guten
Werte f"ur P-Alg. und die auff"allig schlechten Werte f"ur BGH-Alg.
besonders deutlich.

\begin{table}[htb]
    \begin{center}
    \begin{tabular}{|c||r|r||r|r||r|r||r|r|}
        \hline
        n & \multicolumn{2}{|c||}{C-Alg.}   & 
            \multicolumn{2}{|c||}{BGH-Alg.} &
            \multicolumn{2}{|c||}{B-Alg.}   &
            \multicolumn{2}{|c|}{P-Alg.} \\
        \cline{2-9}
          & Schr. & Proz. & Schr. & Proz. &
            Schr. & Proz. & Schr. & Proz. \\
        \hline
        2 & 9 & 8  & 16 & 24  & 12 & 4  & 15 & 8 \\ \hline
        4  & 16 & 128  & 55 & 1486  & 19 & 27  & 31 & 64 \\ \hline
        6  & 25 & 648  & 99 & 15343  & 27 & 361  & 53 & 216 \\ \hline
        8  & 25 & 2048  & 118 & 81284  & 36 & 2234  & 53 & 512 \\ \hline
        10  & 36 & 5000  & 172 & 298460  & 41 & 6417  & 81 & 1000 \\ \hline
        12  & 36 & 10368  & 180 & 867715  & 49 & 15670  & 81 & 1728 \\ \hline
        14  & 36 & 19208  & 196 & 2145346  & 49 & 41922  & 81 & 2744 \\ \hline
        16  & 36 & 32768  & 205 & 4708392  & 53 & 72317  & 81 & 4096 \\ \hline 
        18  & 49 & 52488  & 265 & 9431425  & 59 & 120961  & 115 & 5832 \\ \hline
        20  & 49 & 80000  & 275 & 17574870  & 69 & 194580  & 115 & 8000 \\ \hline
    \end{tabular}
    \end{center}
    \caption{Vergleich der Algorithmen}
    \label{TabVergleich}
\end{table}

% $$$ Reihenentwicklung Grad >n fuer BGH-Alg. sch"adlich
%     Algorithmen auch ohne Parallelisierung interessant

\MySection{Ausblick}

Zum Schlu"s folgt noch eine kurze Liste weiterer Themen, auf die man bei der 
Bearbeitung der vorliegenden Arbeit st"o"st:

\begin{itemize}
\item Analyse von Schaltkreisen f"ur die beschriebenen Algorithmen
      % \cite{ ... Ingos Buch}
\item Betrachtung von Matrizen mit Elementen aus $\Complex$
      % Diagonalisierbarkeit in $\Complex$ !? --> Methode von Krylov
\item Analyse verschiedener Varianten der beschriebenen Algorithmen
      % Pan: 3 Verfahren zur iterativen Invertierung
      %      etliche Verfahren zur Bestimmung der N"aherungsinversen
      % Berk: Wahl von \epsilon; Wahl der Parallelisierung von
      %       Algorithmusteilen
      % BGH: etliche M"oglichkeiten, die Konvergenz der Potenzreihen
      %      sicherzustellen
\item Analyse des Aufwandes f"ur die Aufgabenverteilung zwischen mehreren
      Prozessoren
\item Betrachtung anderer Rechnermodelle (insbesondere Rechnermodelle
      ohne gemeinsamen Speicher f"ur die Prozessoren)
      % CREW, Arten der Zugriffsregelung
      % kein gemeinsamer Speicher (Analyse des Kommunikationsaufwandes)
% \item Suche weiterer Algorithmen  % selbstverst"andlich ...
\item Analyse des Speicherplatzverbrauchs
%\item Implementierung von Flie"skommazahlen mit einer beliebigen Anzahl von
%      Stellen
% Implementierung:  Zahlen beliebiger L"ange 
% $$$$$            (--> Satz "uber maximale Gr"o"se der Eigenwerte)
\end{itemize}

Es bleibt also einiges zu tun... . F"ur dieses Mal soll das jedoch alles 
sein.
\vspace{5ex}

Wenn die Gedanken wieder leichter flie"sen...
\begin{verse}
    Himmlische Stille rauscht durch die Nacht \\
    Samtenes Schweigen str"omt durch die Luft \\
    glitzernd breitet sich das Tal in der Ferne \\[2ex]
    
    kein Laut \\
    ohne Hast l"a"st die Ruhe verbreiten ihr Gl"uck
\end{verse}


%
% Datei: index.tex (Literaturliste, Stichwortverzeichnis)
% 
\markboth{}{}
\addcontentsline{toc}{chapter}{Literatur}
\bibliography{diplom}
\addcontentsline{toc}{chapter}{Stichwortverzeichnis}
\printindex


%
% Datei: anhang.tex (Listings)
%
\begin{appendix}
\sloppy
\thispagestyle{empty}
\vspace*{6cm}
\begin{center}
   \LARGE
   Algorithmen zur parallelen Determinantenberechnung \\[1cm]
   \Large
   Holger Burbach \\[1cm]
   Oktober 1992 \\[1.5cm]
   \LARGE Anhang (Quelltexte)
\end{center}

\MyChapter{Implementierung der parallelen Determinantenberechnung}
\label{ChapImplDet}

In diesem Kapitel sind alle Module gesammelt, denen das vorrangige
Interesse der Implementierung gilt.

%
%
% Datei: main.tex
%
% Haupt-Tex-Datei der Diplomarbeit
%
%%
% noch zu bearbeitende Stellen sind mit $$$$ markiert;
% Anmerkungen sind mit $$$ markiert;
% die aktuell in Berarbeitung befindliche Stelle ist mit $$$$$ markiert;
%%
% - Invertierung von Dreiecksmatrizen aus Csan74
% 
% - Warum laufen die Algorithmen von Csanky nur in K"orper
%       der Charakteristik 0 ?
%   Antwort: Weil Divisionen benutzt werden !!!!!!!!!!!!!!
% - Csan76 : K"orper mit Charakteristik 0
%   BGH82  : beliebige K"orper
%   Berk84 : beliebige K"orper
%   Pan85  : Iterationsverfahren f"ur det(A)
% - Schaltkreise???
%%
% ********************************
% Lesen von zusaetzlichen Dateien:
% ********************************
%%\input amssym.def
%%\input amssym
%%
%==============================================================
\documentstyle[german,ifthen,din_a4,makeidx,bezier,epic]{book}
%==============================================================
%%
% *************************
% Stiloptionen im Vorspann:
% *************************
\pagestyle{myheadings} % vgl. \newcommand{\MySection}{ ... } (s. u.)
\makeindex
\frenchspacing % europ"aische Behandlung der Satzenden
% Nummerierung der Textuntergliederung bis einschlie"slich 'section':
\setcounter{secnumdepth}{1}
\setcounter{tocdepth}{1}
%%
% ******************************
% Ausnahmen von Trennungsregeln:
% ******************************
\hyphenation
    { CRCW Pa-ral-lel-rech-ner Mo-dell De-ter-mi-nan-ten-be-rech-nung
      Ar-beits-spei-cher Fourier-trans-for-ma-tion PRAM
    }
%%
% ***********************
% Auswahl von Textteilen:
% ***********************
\typeout{}
\typein[\eingabe]{Textteile auswaehlen (j/n)?}
\ifthenelse{\equal{\eingabe}{j}
}{
    \typeout{}
    \typeout{Textteile:}
    \typeout{ inhalt, vorbem, csanky, bgh, berk}
    \typeout{ pan, implemen, endbem, index, anhang}
    \typeout{}
    % Auswahl der Textteile eingeben:
    \typein[\auswahl]{Welche Textteile?}

    \includeonly{\auswahl}
% ****Datei-Ein/Ausgabe funktioniert nicht****
%    \newwrite\AuswahlAusgabeDatei
%    \immediate\openout\AuswahlAusgabeDatei=\jobname.aus
%    \ifthenelse{\equal{\auswahl}{a}
%    }{ 
%        \typeout{ ...in then}
%        \write\AuswahlAusgabeDatei{
%            \typeout{Alle Textteile sind ausgewaehlt.}
%        }
%    }{
%        \typeout{ ...in else}
%        \write\AuswahlAusgabeDatei{
%            \typeout{Ausgewaehlte Textteile:}
%            \typeout{\auswahl}
%            \includeonly{\auswahl}
%        }
%    }
%    \closeout\AuswahlAusgabeDatei
}{}

%\newread\AuswahlEingabeDatei
%\openin\AuswahlEingabeDatei=\jobname.aus
%\read\AuswahlEingabeDatei to \ZeileI
%\ZeileI
%\closein\AuswahlEingabeDatei

%%
%===============
\begin{document}
%===============
%%
% *************************
% Stiloptionen im Textteil:
% *************************
\bibliographystyle{mygalpha}
\parindent0pt  % Absatzanf"ange nicht einr"ucken
\parskip2ex plus0.4ex minus0.4ex % Abst"ande zwischen Abs"atzen 2ex +-0.4ex
%%
% *******************************
% eigene Dokumentuntergliederung:
% *******************************
% bei "Anderung der Gliederung sind evtl. die Aufrufe von 
% \addcontentsline in 'tail.tex' anzupassen
\newcommand{\MyMark}[1]{ \thesection \hspace{0.5em} \sc #1 }
\newcommand{\MyChapter}[1]{\chapter{#1}}
\newcommand{\MySection}[1]{ 
                            \section{#1}
                            \markboth{ \MyMark{#1} }{ \MyMark{#1} }
                          }
\newcommand{\MySectionA}[2]{
                            \section[#1]{#2}
                            \markboth{ \MyMark{#1} }{ \MyMark{#1} }
                           }
\newcommand{\MySubSection}[1]{\subsection{#1}}
\newcommand{\MySubSectionA}[2]{\subsection[#1]{#2}}
\newcommand{\MySubSubSection}[1]{\subsubsection{#1}}
\newcommand{\MyParagraph}[1]{\paragraph{#1}}
%%
% ************************
% diverse neue Umgebungen:
% ************************
% Auswahl von 'subsection' in folgender Zeile ggf. anzupassen:
\newtheorem{satz}{Satz}[section]
\newtheorem{lemma}[satz]{Lemma}
\newtheorem{korollar}[satz]{Folgerung}
\newtheorem{definition}[satz]{Definition}
\newcommand{\MyBeginDef}{\begin{definition} \rm}
\newcommand{\MyEndDef}{\end{definition} \vspace{2ex}}
\newtheorem{algorithmus}[satz]{Algorithmus}
\newtheorem{bemerkung}[satz]{Bemerkung}
\newenvironment{beweis}{\medbreak {\bf Beweis} \quad
                       }{ \hfill $ \Box $ \bigbreak }
% f"ur den Anhang (Listings):
\newenvironment{MyListing}{ \small % '\normalsize' ist zu gross
                          }{ }
\newenvironment{DefModul}[1]{ \MySection{Definitionsmodul '#1`}
                              \begin{MyListing}
                            }{ \end{MyListing} }
\newenvironment{ImpModul}[1]{ \MySection{Implementierungsmodul '#1`}
                              \begin{MyListing}
                            }{ \end{MyListing} }
\newenvironment{ProgModul}[1]{ \MySection{Programmodul '#1`}
                               \begin{MyListing}
                             }{ \end{MyListing} }
% ***********************
% eigene Listen-Umgebung:
% ***********************
\newenvironment{MyDescription}{ \begin{list}{ $\bullet$
                                      }{ \leftmargin3.51em \labelsep0.5em
                                         \labelwidth3em \listparindent0em
                                         \rightmargin0em \itemsep3ex
                                         \parsep2ex
                                      }
                              }{ \end{list} }
\newcommand{\MyItem}[1]{\item[#1] \hspace{1em} \\} % Item f"ur MyDescription
% ***********************
% eigene Gleichungsliste:
% ***********************
\newcommand{\DS}{\displaystyle}
%               Abk"urzung f"ur die Verwendung in 'array'-Umgebung f"ur
%               mehrzeilige Formeln
% Umgebung:
\newenvironment{MyEqnArray}{   \[ \begin{array}{lrcl} \DS \MatStrut
                           }{  \end{array} \]
                           }
% Tabulator f"ur Umgebung:
\newcommand{\MT}{ & \DS } %MyTab
% Zeilenende f"ur Umgebung:
\newcommand{\MNl}{ \\ \DS \MatStrut} %MyNewline
%%
% ************************
% Schreibweisen (Symbole):
% ************************
% Zahlenmengenzeichen aus 'lsii_la.tex':
\font\sanss=cmss10
\newcommand{\Integers}{ \! \hbox{\sanss { Z\kern-.4em Z}} } %\IZ
\newcommand{\Nat}{ \hbox{\sanss {I\kern-.14em N}} }   %\IN
\newcommand{\Rationals}{ \hbox{\vrule width 0.6pt height 6pt depth 0pt
                         \hskip -3.0pt{\sanss Q}}
                       } % \IR
\newcommand{\Complex}{ \hbox{\vrule width 0.6pt height 6pt depth 0pt 
                       \hskip -3.0pt{\sanss C}}
                     } % \IC
% Zahlenmengenzeichen aus den Euler-Fonts:
%%\newcommand{\Integers}{  \Bbb{Z} }
%%\newcommand{\Nat}{       \Bbb{N} }
%%\newcommand{\Rationals}{ \Bbb{R} }
%%\newcommand{\Complex}{   \Bbb{C} }
% eigenes:
\newcommand{\proc}{\cal P \mit \,}  % Anzahl zu besch"aftigender Prozessoren
\newcommand{\permut}{\cal S \mit \! } % Menge aller Permutationen
\newcommand{\base}{\cal B \mit \,}   % Basis der logarithmischen
%                                       Zahlendarstellung
\newcommand{\accuracy}{\cal A \mit \,}
%                                   Schreibweise f"ur Anzahl der Stellen,
%                                       mit denen gerechnet wird
%                                       (accuracy <-> Genauigkeit)
\newcommand{\ExpBound}{\cal E \mit \,}
%                                   Schranke f"ur Exponenten in der
%                                       logarithmischen Darstellung
\newcommand{\LogRep}{\cal L \mit \,}
%                                   logarithmische Darstellung
%                                       (logarithmic representation)
\newcommand{\round}{\cal R \mit \,}  % Symbol f"ur Rundungsfunktion
\newcommand{\RepErr}{\cal F \mit \,} % Symbol f"ur Darstellungs Fehler
\newcommand{\necess}{\cal N \mit \,} % Symbol f"ur Anzahl n"otiger Stellen
\newcommand{\PRing}{R \, [[]]}
%           Potenzreihenring R (Liste der Unbestimmten in [[]] weggelassen)
\newcommand{\MathE}{\mbox{\rm e}} % Konstante 2.718...
%%
% ***************
% Funktionsnamen:
% ***************
\newcommand{\adj}{ \mbox{\rm adj} \,}   % Funktionsname 'adj'
\newcommand{\tr}{ \mbox{\rm tr} \,}     % Funktionsname 'tr'
\newcommand{\sgn}{ \mbox{\rm sgn} \,}   % Funktionsname 'sgn'
\newcommand{\sig}{ \mbox{\rm sig} }     % Signatur einer Permutation
\newcommand{\rg}{ \mbox{\rm rg} \,}     % Rang einer Matrix
\newcommand{\MyKer}{ \mbox{\rm ker} \,} % Kern einer Matrix
\newcommand{\MyDim}{ \mbox{\rm dim} \,} % Dimension eines Vektorraumes
\newcommand{\cond}{ \mbox{\rm cond} \,} % -> 'Pan' ...
%%
%****************
% eigene Befehle:
%****************
\newcommand{\MatStrut}{\mbox{\rule[-2ex]{0ex}{5ex}}}
\newcommand{\LMatStrut}{\mbox{\rule[-4ex]{0ex}{7ex}}}
%               St"utzen f"ur Matrizen
\newcommand{\equref}[1]{\mbox{(\ref{#1})}}
%               Verweis auf Gleichungen: Nummer in Klammern
%
\newcommand{\Mya}{"a} % ... zur Benutzung von Umlauten in Index-Begriffen
\newcommand{\Myo}{"o}
\newcommand{\Myu}{"u}
\newcommand{\Mys}{"s}
\newcommand{\MyPunkt}{ \mbox{\hspace{0.5em}.} }
\newcommand{\MyPunktA}[1]{ \nopagebreak \mbox{\hspace{#1}.} \\ }
\newcommand{\MyKomma}{ \mbox{\hspace{0.5em},} }
\newcommand{\MyKommaA}[1]{ \nopagebreak \mbox{\hspace{#1},} \\ }
%               falls ein Punkt oder ein Komma als Satzzeichen 
%               direkt hinter einer abgesetzten Gleichung stehen soll
\newcommand{\MyChoose}[2]{ \left( { #1 \atop #2 } \right) }
%               statt TeX-Befehl \choose (sieht besser aus)
\newcommand{\MySetProperty}{ \: | \: }
%               f"ur Mengen: Trennsymbol zwischen Mengenelement und
%                            Eigenschaftschaftsangabe f"ur Element
\newcommand{\lc}{\left\lceil}
\newcommand{\rc}{\right\rceil}
\newcommand{\lf}{\left\lfloor}
\newcommand{\rf}{\right\rfloor}
\newcommand{\lb}{\left(}
\newcommand{\rb}{\right)}
\newcommand{\Beq}[1]{\begin{equation} \label{#1}}
\newcommand{\Eeq}{\end{equation}}
%               Abk"urzungen
\newcommand{\und}{\wedge}
\newcommand{\oder}{\vee}
%               Verbesserung der Lesbarkeit
\newcommand{\MyStack}[2]{ \stackrel{ \mbox{\scriptsize\rm #1} }{ #2 } }
%               fuer Hinweise ueber Relationszeichen in Gleichungen
% **********************
% Text der Diplomarbeit:
% *********************
% 
% Datei: inhalt.tex (Titelseite, Referenzen und Inhaltsverzeichnis)
%
% Titelseite:
\begin{titlepage}
    \begin{center}
        \vspace*{6cm}
        \LARGE Algorithmen zur parallelen Determinantenberechnung 
                                                             \\[1.5cm]
        \Large    Holger Burbach \\[1cm]
        Oktober 1992 \\[4cm]
        Diplomarbeit, geschrieben am Lehrstuhl Informatik II \\
        der Universit"at Dortmund \\[1cm]
        Betreut von Prof. Ingo Wegener
    \end{center}
\end{titlepage}
\setcounter{page}{2}
%
% Inhaltsverzeichnis:
{  \parskip0ex plus 0.5ex
   \tableofcontents
}


%
% Datei: vorbem.tex
%
\MyChapter{Vorbemerkungen}
\label{ChapIntro}

% **************************************************************************

\MySection{Einf"uhrung}

Die Berechnung der Determinante einer quadratischen Matrix ist ein
Problem,
dessen effiziente L"osung in vielen Bereichen von Interesse ist, in der
Informatik z. B. in der Computergrafik und der Kodierungstheorie.

Ein Problem in der analytischen Geometrie ist es, die 
Lage von geometrischen Objekten zueinander festzustellen und 
Schnittpunkte oder -ebenen zu berechnen. Ein Teilproblem dabei ist die
Pr"ufung der linearen Unabh"angigkeit von Vektoren. Es
l"a"st sich auf die Berechnung einer Determinante zur"uckf"uhren.

Ein weiterer Bereich, in dem die Determiantenberechnung angewendet wird,
ist
die theoretische Physik. Dort wird in vielen Theorien auf Matrizen zu
Beschreibung der verschiedenen Sachverhalte zur"uckgegriffen. In der
Einstein'sche Relativit"atstheorie z. B. wird die {\em Tersorrechnung},
die die Eigenschaften sich von Koordinatensystem zu Koordinatensystem
"andernder Ma"szahlen untersucht, ausgibig verwendet. Die Determinante ist
eine solche Ma"szahl. Beim Studium von Literatur, die diese Thematiken
behandelt (z. B. \cite{BS86} ab S. 70), st"o"st man immer wieder auf
Matrizen und ihre Determinanten.

Seitdem
sich die Forschung im Bereich der Informatik zunehmend mit 
Parallelrechnern
be\-sch"af\-tigt, werden f"ur alle bekannten Probleme Algorithmen gesucht, 
die die Tatsache, da"s auf einem Parallelrechner mehrere Prozessoren 
gleichzeitig an der L"osung desselben Problems arbeiten, besonders 
effizient ausnutzen.

Betrachtet man eine Matrix aus der Sicht der Informatik als Datenstruktur,
so dr"angt sich die Benutzung dieser Datenstruktur in Parallelrechnern 
geradezu auf, denn intuitiv, ohne zun"achst alle Probleme ausgearbeitet 
zu haben, kann man auf die Idee kommen, die Matrizenelemente jeweils
einzelnen Prozessoren oder Gruppen von Prozessoren zuzuordnen, die das 
zugrundeliegende Problem f"ur dieses Matrizenelement bearbeiten. 
Selbstverst"andlich ist die praktische Verwendung dieser Idee nicht in 
jedem Fall ganz so einfach.

So war die effiziente Parallelisierung der Determinantenberechnung lange
Zeit ein ungel"ostes Problem, bis 1976, als Laszlo Csanky einen in jenen 
Tagen "uberraschenden Algorithmus ver"offentlichte \cite{Csan76}. Es 
folgten eine Reihe weiterer Algorithmen verschiedener Autoren mit 
vergleichbaren Leistungsmerkmalen.

In all diesen Ver"offentlichungen wird vorrangig die Gr"o"senordnung 
der Laufzeiten und Anzahlen der Prozessoren betrachtet. Es werden in
einzelnen Ver"offentlichungen auch bereits einige Vergleiche mit den 
anderen Algorithmen durchgef"uhrt. So ist es w"unschenswert einen 
"Uberblick "uber die existierenden Algorithmen zu bekommen und sie 
insgesamt miteinander zu vergleichen.

Die vorliegende Diplomarbeit behandelt vier Algorithmen zur 
parallelen De\-ter\-mi\-nan\-ten\-be\-rech\-nung\footnote{ \cite{Csan76}, 
\cite{BGH82}, \cite{Berk84} und \cite{Pan85}}. Dabei wird auf die 
Verwendung von Gr"o"senordnungen ( O-Notation ) in Aufwandsanalysen 
weitgehend verzichtet.
Um den Einsteig in das Thema zu 
erleichtern, wird zus"atzlich noch der Entwicklungssatz von
Laplace zur Berechnung der Determinante erw"ahnt.

Die Darstellung der vier Algorithmen in den zugeh"origen Kapiteln
\ref{ChapCsanky} bis \ref{ChapPan} umfa"st neben den Grundlagen und der 
Algorithmen selbst, jeweils eine Analyse der Rechenzeit und des Grades der
Parallelisierung\footnote{Diese Begriffe sind in Kapitel \ref{SecBez}
definiert.}. Da in der Praxis die Gr"o"se des ben"otigen 
Speicherplatzes kein vorrangiges Problem mehr darstellt, wird dieser 
Wert nicht analysiert. Matrizen mit Elementen aus $\Complex$ werden nicht
betrachtet. In diesen vier Kapitel wird versucht, auf Unterschiede
und Gemeinsamkeiten der Algorithmen einzugehen.

Im Anschlu"s an die Darstellung der Algorithmen wird in Kapitel 
\ref{ChapImplemen} ihre Implementierung beschrieben. Die Quelltexte 
sind im Anhang zu finden.
Schlie"slich erfolgt in Kapitel \ref{ChapEndbem}
ein zusammenfassender Vergleich der Algorithmen.

Der Text soll es erm"oglichen, die Algorithmen ohne weitere Literatur 
anhand
von Grundkenntnissen aus der Mathematik und Informatik zu verstehen. 
Aus diesem Grund und um einheitliche Bezeichnungen zu vereinbaren sind 
Grundlagen, insbesondere aus der Linearen Algebra, an den ben"otigten 
Stellen aufgef"uhrt. F"ur den Fall, da"s die im Text enthaltenen
Informationen nicht ausreichen, sind die benutzten Quellen an den
jeweiligen Stellen angegeben.

Alle Betrachtungen abstrahieren von technischen Problemen bei der 
Konstruktion von Parallelrechnern. Dazu wird das Rechnermodell der
PRAM benutzt. Die Beschreibung dieses Modells erfolgt in 
Kapitel \ref{SecModell}. 

Vor anderen Teilen dieser Diplomarbeit sollten zun"achst die Kapitel 
\ref{SecModell} und \ref{SecBez} gelesen werden. Alle 
weiteren Teile von Kapitel \ref{ChapIntro} sowie Kapitel \ref{ChapBase}
sind als Sammlung von Grundlagen zu verstehen, auf die bei Bedarf 
zur"uckgegriffen werden kann\footnote{Im Text kommen h"aufig 
Punkte und Kommata als Satzzeichen direkt im Anschlu"s an
abgesetzte Gleichungen vor. An einigen Stellen, besonders hinter
Vektoren und Matrizen, fehlen diese Satzzeichen aus technischen 
Gr"unden.}.

% **************************************************************************

\MySection{Das Berechnungsmodell}
\label{SecModell}
\index{Berechnungsmodell} \index{PRAM} \index{CRCW}
\index{Modellrechner}
In diesem Kapitel wird der f"ur Komplexit"atsbetrachtungen verwendete
Modellrechner beschrieben. Es ist die 
{\em Arbitrary Concurrent Read Concurrent Write Parallel Random Access 
  Machine ( arbitrary CRCW PRAM) }.
Sie besteht aus
gleichen Prozessoren, die alle auf denselben Arbeitsspeicher zugreifen.
Innerhalb einer Zeiteinheit k"onnen diese
Prozessoren, und zwar alle gleichzeitig,
zwei Operanden aus dem Speicher lesen, eine der in Tabelle \ref{Csan76Tab2}
aufgef"uhrten Operationen ausf"uhren und
das Ergebnis wieder im Speicher ablegen. Falls beim
Schreiben mehrere Prozessoren auf eine Speicherzelle zugreifen, mu"s der
Algorithmus unabh"angig davon korrekt sein, welcher Prozessor seinen
Schreibzugriff tats"achlich ausf"uhrt.
\begin{table}[htb]
    \begin{center}
    \begin{tabular}{|p{4cm}|c|}
        \hline
        Operation & Symbol \\
        \hline
        \hline
        Addition & $+$ \\
        \hline
        Subtraktion & $-$ \\
        \hline
        Multiplikation & $*$ \\
        \hline
        Division (Ergebnis in $\Rationals$) & $/$ \\
        \hline
        Division (Ergebnis in $\Integers$) & div \\
        \hline
        $ x - (x \; \mbox{div} \; y) * y $ & $x$ mod $y$ \\
        \hline
    \end{tabular}
    \end{center}
    \caption{Operationen des Modellrechners}
    \label{Csan76Tab2}
\end{table}
Da die
Komplexit"at der vier haupts"achlich interessierenden 
Algorithmen nicht nur auf ihre Gr"o"senordnung hin
untersucht wird, sondern die Ausdr"ucke zur Beschreibung der Komplexit"at
genau angegeben werden sollen, ist es erforderlich, von Details
der Implementierung, die die Konstanten
beeinflussen, zu abstrahieren, so da"s die Aussagen allgemeing"ultig sind.
Aus diesem Grund wird
\begin{itemize}
    \item f"ur die Verarbeitung von Schleifenbedingungen,
    \item f"ur die Verarbeitung von Verzweigungsbedingungen,
    \item f"ur die Ein- und Ausgabe von Daten,
    \item f"ur die Initialisierung von Speicherbereichen,
    \item und f"ur komplexe Adressierungsarten
          (z. B. indirekte Adressierung) beim Zugriff auf Speicherbereiche
\end{itemize}
kein zus"atzlicher Aufwand in Rechnung gestellt. Es werden also nur die
arithmetischen Operationen gez"ahlt.

Zu beachten ist, da"s bei einer PRAM jeder Aufwand zur Verteilung von
Aufgaben auf verschiedene Prozessoren vernachl"assigt wird. Diese 
Eigenschaft bietet die M"oglichkeit zur Kritik, da so jedes
Problem deutlich vereinfacht wird, jedoch in der Praxis die Organisation der 
Aufgabenverteilung nicht unerheblichen Aufwand erfordert.
Eine genaue Analyse der 
Auswirkungen dieser Vernachl"assigung ist umfangreich und nicht 
Thema des vorliegenden Textes.

% **************************************************************************

\MySection{Bezeichnungen}
\label{SecBez}

In diesem Kapitel werden die verwendeten Begriffe und Symbole 
definiert.

Um auf die in der Arbeit haupts"achlich behandelten Algorithmen 
einfach Bezug
nehmen zu k"onnen, werden mit Hilfe der Namen ihrer Autoren die
folgenden Abk"urzungen vereinbart:
\begin{itemize}
\item
      C-Alg. steht f"ur den Algorithmus von Csanky 
      (Unterkapitel \ref{SecAlgFrame} ab S. \pageref{SecAlgFrame}).
\item
      BGH-Alg. steht f"ur den Algorithmus von Borodin, von zur Gathen 
      und Hopcroft (Unterkapitel \ref{SecAlgBGH} ab 
      S. \pageref{SecAlgBGH}).
\item
      B-Alg. steht f"ur den Algorithmus von Berkowitz 
      (Unterkapitel \ref{SecAlgBerk} ab S. \pageref{SecAlgBerk}).
\item
      P-Alg. steht f"ur den Algorithmus von Pan
      (Unterkapitel \ref{SecAlgPan} ab S. \pageref{SecAlgPan}).
\end{itemize}

\index{Bezeichnungen!nat{\Myu}rliche Zahlen}
Die Menge der positiven ganzen Zahl {\em ohne Null} wird mit \[ \Nat \]
bezeichnet. Die Menge der
positiven ganzen Zahl einschlie"slich der Null wird mit \[ \Nat_0 \]
bezeichnet. Falls nicht im Einzelfall anders festgelegt erfolgen alle
Darstellungen von Zahlen zur Basis $10$.

\index{Bezeichnungen!Schritt}
Der Vorgang, in dem beliebig viele Prozessoren gleichzeitig je
zwei Operanden aus dem Arbeitsspeicher lesen, aus diesen Operanden
ein Ergebnis berechnen und dieses Ergebnis wieder im
Arbeitsspeicher ablegen, wird als ein {\em Schritt} bezeichnet.

\index{Zeitkomplexit{\Mya}t}
\index{Parallelisierungsgrad}
Die {\em parallele Zeitkomplexit"at eines Algorithmus} bezeichnet die
Anzahl der Schritte, die dieser ben"otigt, um die L"osung\footnote{Die
von uns betrachteten Probleme besitzen nur eine L"osung.}
f"ur das
zugrunde liegende Problem zu berechnen.
Die maximale Anzahl der Prozessoren, die dabei gleichzeitig
besch"aftigt werden, wird mit {\em Parallelisierungsgrad des
Algorithmus} bezeichnet.

Falls nicht im Einzelfall anders festgelegt, gilt folgende 
Regelung: Gro"sbuchstaben bezeichnen
Matrizen und Kleinbuchstaben Zahlen oder Vektoren, $A$ bezeichnet eine
$n \times n$-Matrix, indizierte
Kleinbuchstaben beziehen sich auf die Elemente der mit dem zugeh"origen 
Gro"sbuchstaben bezeichneten
Matrix.

Die in Tabelle \ref{Csan76Tab1} aufgelisteten Schreibweisen werden
benutzt.
\index{Bezeichnungen!Indizierung}
\index{Einheitsmatrix} \index{Nullmatrix}
\index{Permutation}
\begin{table}[htb]
    \begin{center}
    \begin{tabular}{|p{10cm}|c|}
        \hline
            Begriff & Schreibweise \\
        \hline\hline
            Element in Zeile $i$, Spalte $j$ von $A$ & $ a_{i,j} $ \\
        \hline
            $i$-tes Element des Vektors $v$ & $v_i$ \\
        \hline
            Matrix, die aus A durch Streichen der Zeilen $v$ und der
            Spalten $w$ entsteht (dabei seien $v$ und $w$
            echte Teilmengen der Menge der Zahlen von $1$ bis $n$; diese
            Mengen werden hier als durch Kommata getrennt Zahlenfolge
            geschrieben) & $ {A}_{(v|w)} $ \\
        \hline
            Einheitsmatrix (Elemente der Haupt\-di\-ago\-na\-len gleich $1$;
            alle anderen Elemente gleich $0$)
            mit $n$ Zeilen und Spalten & $E_n$ \\
        \hline
            Einheitsmatrix (Anzahl der
            Zeilen und Spalten aus dem Zusammenhang klar) & $E$ \\
        \hline
            Nullmatrix (alle Elemente sind gleich 0) mit $m$ Zeilen und
            $n$ Spalten & $0_{m,n}$ \\
        \hline
            Nullvektor (alle Elemente sind gleich 0) der L"ange $m$ &
            $0_m$ \\
        \hline
            Logarithmus von $x$ zur Basis $2$      & $\log(x)$ \\
        \hline
            Menge aller $n$-stelligen Permutationen & $\permut_n$ \\
        \hline
            \begin{minipage}{10em}
                \begin{math} \displaystyle
                     \lim_{n\rightarrow \infty}
                     \left( 1 + \frac{1}{n} \right)^n
                \end{math}
            \end{minipage} \LMatStrut
            $(\: = 2.718281\ldots)$ &  \MathE  \\
        \hline
            Logarithmus von $x$ zur Basis $\MathE$ & $\ln(x)$ \\
        \hline
            Anzahl der Elemente der Menge $M$ & $|M|$ \\
        \hline
    \end{tabular}
    \end{center}
    \caption{Bezeichnungen}
    \label{Csan76Tab1}
\end{table}

% **************************************************************************

\MySection{Das Pr"afixproblem}
Von einer effizienten L"osung des Pr"afixproblems wird an verschiedenen
Stellen Gebrauch gemacht. Es ist also von "ubergreifendem Interesse und
wird deshalb hier behandelt (\cite{LF80},
\cite{Wege89} S. 83 ff.). Es l"a"st sich folgenderma"sen formulieren:
\begin{quote}
\index{Pr{\Mya}fixproblem}
\label{PagePraefixproblem}
    Gegeben sei die Halbgruppe \[ (M,\circ) \MyPunkt \] 
    D. h. die Verkn"upfung 
    $\circ$ ist assoziativ auf $M$. Weiterhin seien 
    \[ x_1,x_2,x_3,\ldots,x_{n} \]
    Elemente aus $M$. Es wird definiert
    \[ p_i := x_1 \circ x_2 \circ x_3 \circ \ldots \circ x_i \MyPunkt \]
    Das Pr"afixproblem besteht darin, alle Elemente der Menge
    \[ \{ p_i | 1 \leq i \leq n \} \] zu berechnen.
\end{quote}
Es sind u. a. zwei M"oglichkeiten\footnote{die sich zu einer dritten 
zusammenfassen lassen (Satz \ref{SatzAlgPraefix})} denkbar, dies mit 
parallelen Algorithmen zu erreichen.
\begin{itemize}
    \item Die erste M"oglichkeit:
        \begin{enumerate}
            \item L"ose das Pr"afixproblem parallel f"ur
                  \[ x_1,\ldots,x_{\lceil n/2 \rceil} \]
                  und
                  \[ x_{\lceil n/2+1 \rceil},\ldots,x_n \MyKomma \]
                  so da"s nach diesem Schritt
                  \[ p_1,\ldots,p_{\lceil n/2 \rceil} \]
                  bereits berechnet sind.
            \item
                  Berechne aus \[ p_{\lceil n/2 \rceil} \] und
                  der L"osung des Problems f"ur
                  \[ x_{\lceil n/2+1 \rceil},\ldots,x_n \]
                  parallel in einem weiteren Schritt
                  \[ p_{\lceil n/2+1 \rceil},\ldots,p_n \]
        \end{enumerate}
    \item
        Die zweite M"oglichkeit, die hier kurz dargestellt werden soll,
        sieht folgenderma"sen aus (o. B. d. A. sei $n$ eine Zweierpotenz):
        \begin{enumerate}
            \item
                Berechne parallel in einem Schritt
                \[ x_1 \circ x_2, x_3 \circ x_4, \ldots,
                   x_{n-1} \circ x_n \]
            \item
                L"ose das Pr"afixproblem f"ur diese
                $n/2$ Werte. Damit werden alle $p_i$ mit geradem $i$
                berechnet.
            \item
                Die noch fehlenden $p_i$ f"ur ungerade $i$ k"onnen nun
                parallel in einem weiteren Schritt aus der L"osung f"ur die
                $n/2$ Werte und den $x_i$ mit ungeradem $i$ berechnet
                werden.
        \end{enumerate}
\end{itemize}
Diese beiden M"oglichkeiten k"onnen zu einem Algorithmus zusammengefa"st
werden:
\begin{satz}
\label{SatzAlgPraefix}
\index{Algorithmus!Pr{\Mya}fixproblem}
    Gegeben sei die Halbgruppe \[ (M,\circ) \MyPunkt \] 
    Das Pr"afixproblem
    f"ur $n$ Elemente \[ x_1,x_2,\ldots,x_n \] von $M$ l"a"st sich
    von \[ \lf \frac{3}{4}n \rf \] Prozessoren in
    \[ \lceil \log(n) \rceil \] Schritten l"osen.
\end{satz}
\begin{beweis}
    O. B. d. A. sei $n$ eine Zweierpotenz. In dem Fall, da"s $n$ keine
    Zweierpotenz ist, wird $n$ durch die n"achst h"ohere Zweierpotenz $n'$
    ersetzt und alle Verkn"upfungen mit Elementen $x_i$ von $M$ f"ur
    \[ i>n \] werden nicht durchgef"uhrt. 

    Benutze folgenden Algorithmus:
    \begin{enumerate}
    \item
          Wenn \[ n=1 \] dann ist $x_1$ das Ergebnis.
    \item
          Wenn \[ n=2 \] dann ist \[ x_1, x_1 \circ x_2 \]
          das Ergebnis.
    \item
          Schritte \ref{StepPraefix3a} und \ref{StepPraefix3b} parallel:
          \begin{enumerate}
          \item \label{StepPraefix3a}
                \begin{enumerate}
                \item \label{StepPraefix3a1}
                      Berechne parallel in einem Schritt
                      \[ x_1 \circ x_2,\ldots,
                         x_{n/2-1} \circ x_{n/2}
                      \]
                \item \label{StepPraefix3a2}
                      Benutze den Algorithmus rekursiv zur L"osung des
                      Problems f"ur die in Schritt \ref{StepPraefix3a1}
                      erhaltenen $n/4$ Werte. Auf diese Weise sind die
                      $p_i$ f"ur
                      \[ 1 \leq i \leq n/2 \] mit geraden $i$, u. a.
                      auch $p_{n/2}$, bereits berechnet.
                \end{enumerate}
          \item \label{StepPraefix3b}
                Benutze den Algorithmus rekursiv zur L"osung des Problems
                f"ur \[ x_{n/2+1},\ldots,x_n \]
          \end{enumerate}
    \item Schritte \ref{StepPraefix4a} und \ref{StepPraefix4b} parallel:
          \begin{enumerate}
          \item \label{StepPraefix4a}
               F"ur $i$ gelte \[ 1 \leq i \leq n/2 \MyPunkt \]
               Wenn $n/2 > 2$, dann
               berechne parallel in einem Schritt mit Hilfe der $p_i$
               aus \ref{StepPraefix3a2} und der $x_i$ mit ungeradem $i$
               die fehlenden $p_i$ mit ungeradem $i$.
          \item \label{StepPraefix4b}
                Berechne aus $p_{n/2}$ und den Ergebnissen von 
                \ref{StepPraefix3b} die $p_i$ mit
                \[ n/2+1 \leq i \leq n \MyPunkt \]
          \end{enumerate}
    \end{enumerate}
    Zur Analyse des Algorithmus bezeichnet $s(n)$ die Anzahl der Schritte,
    die er ben"otigt, um das Pr"afixproblem f"ur $n$ Eingabewerte zu
    l"osen, und $p(n)$ die Anzahl der Prozessoren, die dabei besch"aftigt
    werden k"onnen.
    \begin{itemize}
    \item Hier wird zun"achst die Anzahl der Schritte betrachtet.
          Es gilt \[ s(1) = 0,\,s(2) = 1,\, s(4) = 2 \MyPunkt \]
          Bei der Betrachtung des Algorithmus erkennt man, da"s folgende
          Rekursionsgleichung G"ultigkeit besitzt:
          \Beq{Berk84Equ6}
              \forall n>4: \: s(n) = \max(s(n/4)+1,\,s(n/2)) + 1 \MyPunkt
          \Eeq
          Wenn man diese Formel auf $s(n/2)$ anwendet und das Ergebnis in
          die obige Formel einsetzt, erh"alt man
          \[ \forall n>4: \: s(n) = 
                 \max(\: s(n/4)+1,\, \max(s(n/8)+1,\, s(n/4))+1 \:) \:+ 1 
          \]
          Aufgrund der Assoziativit"at der $\max$-Funktion ist dies
          gleichbedeutend mit
          \begin{MyEqnArray}
             \MT \forall n>4: \: s(n) \MT = \MT
                 \max(s(n/4)+1,\, s(n/8)+2,\, s(n/4)+ 1) + 1 \MNl
             \Rightarrow 
             \MT \forall n>4: \: s(n) \MT = \MT s(n/4)+2 
          \end{MyEqnArray}
          Es gilt also f"ur jedes $i$:
          \[ \forall n>4: \: s(n) = s(n/2^{2i}) + 2i \]
          Mit \[ i = \frac{\log(n)}{2} \] erh"alt man als Endergebnis 
          \[ s(n) = \log(n) \MyPunkt \]
    \item F"ur die Anzahl der besch"aftigten Prozessoren $p(n)$ gilt:
          \[ p(1) = 0, p(2) = 1, p(4)= 2 \]
          Ferner gilt offensichtlich folgende Rekursionsgleichung:
          \Beq{Berk84Equ8}
              \forall n>4:\: p(n)= 
              \max(n/2+n/4,\, p(n/4)+p(n/2),\, n/4+p(n/2))
          \Eeq
          Beim Ausrechnen der Werte von $p(8)$, $p(16)$ und $p(32)$
          mit Hilfe dieser Rekursionsgleichung gelangt man zu der
          Vermutung, da"s gilt:
          \Beq{Berk84Equ7}
              \forall n>4: \: p(n) = \frac{3}{4} n 
          \Eeq
          Dies wird durch Induktion bewiesen. Zu beachten
          ist, da"s nach Voraussetzung nur die Potenzen von $2$ als Werte
          f"ur $n$ in Frage kommen. 
          
          Sei also nun \[ n>4 \] und es gelte
          \begin{eqnarray*}
              p(n) & = & \frac{3}{4}n \\
              p(n/2) & = & \frac{3}{8}n \MyPunkt
          \end{eqnarray*}
          Es ist zu zeigen, da"s dann auch
          \[ p(2n)= \frac{3}{2}n \]
          richtig ist.
          Nach der Rekursionsgleichung \equref{Berk84Equ8} gilt
          \[ p(2n) = \max(3/2*n, p(n/2)+p(n), n/2 + p(n)) \]
          Die Anwendung der Induktionsvoraussetzung f"uhrt zu
          \[ p(2n) = \max(3/2*n, 3/8*n + 3/4*n, n/2 + 3/4*n) \] und somit zu
          \[ p(2n) = \frac{3}{2}n \] was zu zeigen war. F"ur die Anzahl der 
          ben"otigen Prozessoren gilt also
          \[ \forall n > 4: \: p(n) = \frac{3}{4}n \]
          Da die L"osung des Problems f"ur \[ n \leq 4 \] einfach ist, wird
          die Quantifizierung nicht weiter beachtet.
    \end{itemize}
    Damit die Aussagen nicht nur f"ur Zweierpotenzen, werden die Werte
    f"ur $s(n)$ und $p(n)$ mit Gau"sklammern versehen. F"ur $p(n)$ ist
    dies ohne weitere Begr"undung problematisch. Betrachtet man den 
    Algorithmus jedoch genauer, stellt man fest, da"s beim ersten 
    ausgef"uhrten Schritt die meisten Prozessoren besch"aftigt werden. 
    Die Anzahl dieser Prozessoren gibt der Term in Gau"sklammern an.
\end{beweis}

% **************************************************************************

\MySection{L"osungen grundlegender Probleme}
In diesem Kapitel werden Algorithmen zur L"osung einiger
grundlegender Probleme angegeben und auf ihre Komplexit"at hin untersucht.

\begin{satz}[Bin"arbaummethode]  % $$$ wird benutzt (nicht loeschen)
\label{SatzAlgBinaerbaum}
    Wird das Pr"afixproblem (siehe Beschreibung Seite
    \pageref{PagePraefixproblem}) dahingehend vereinfacht, da"s nur
    $p_n$ zu berechnen ist, so l"a"st sich dieses vereinfachte Problem in
    \[ \lc \log(n) \rc \] Schritten von \[ \lf \frac{n}{2} \rf \] 
    Prozessoren l"osen.
\end{satz}
\begin{beweis}
    Verkn"upfe die $x_i$ nach dem Schema in Abbildung \ref{PicBinBaum}.
    \begin{figure}[htb]
    \begin{center}
        \input{bilder/pbinbaum}
        \caption{Bin"arbaummethode}
        \label{PicBinBaum}
    \end{center}
    \end{figure}
    Falls $n$ keine Zweierpotenz ist, werden die Verkn"upfungen mit den 
    $x_j$, f"ur die gilt \[ n < j \leq 2^{\lc \log(n) \rc} \MyKomma \]
    nicht durchgef"uhrt. Die L"osung des Problems erfordert offensichtlich
    den angegebenen Aufwand.
    \mbox{ \hspace{4em} \hfill }
\end{beweis}

\begin{korollar}[Parallele Grundrechenarten]
\label{SatzAlgRechnen}             % $$$ wird benutzt (nicht loeschen)
\index{Algorithmus!parallele Grundrechenarten}
    Seien $n$ Zahlen durch die gleiche Rechenoperation miteinander zu
    verkn"upfen. Diese Rechenoperation sei eine der Grundrechenarten
    Addition oder Multiplikation. Die Verkn"upfung kann in
    \[ \lc \log(n) \rc \] Schritten von
    \[ \lf \frac{n}{2} \rf \] Prozessoren durchgef"uhrt
    werden.
\end{korollar}
\begin{beweis}
    Aufgrund der Assoziativit"at der Rechenoperationen folgt
    dies direkt aus Satz \ref{SatzAlgBinaerbaum}. 
    \mbox{ \hspace{4em} \hfill }
\end{beweis}

\begin{satz}[Parallele Matrizenmultiplikation]
\label{SatzAlgMatMult}            % $$$ wird benutzt (nicht loeschen)
\index{Algorithmus!parallele Matrizenmultiplikation}
    Sei $A$ eine $m \times p$-Matrix und $B$ eine $p \times n$-Matrix.
    Sie lassen sich in
    \[ \lceil \log(p) \rceil + 1 \] Schritten von
    \[ m * n * p \] Prozessoren miteinander
    multiplizieren.
\end{satz}
\begin{beweis}
    Sei $C$ die $m \times n$-Ergebnismatrix. 
    Sie wird mit Hilfe der Gleichung
    \[ c_{i,j} = \sum_{k=1}^p a_{i,k} b_{k,j} \]
    berechnet. Dazu werden zuerst parallel in einem Schritt
    \[ d_{i,k,j} := a_{i,k} b_{k,j} \] mit
    \begin{eqnarray*}
         1 \leq & i & \leq m \\
         1 \leq & j & \leq n \\
         1 \leq & k & \leq p 
    \end{eqnarray*} von
    \[ m * n * p \] Prozessoren berechnet.
    Die Ergebnismatrix erh"alt man dann nach der Gleichung
    \[ c_{i,j} = \sum_{k=1}^p d_{i,k,j} \]
    Die Berechnung der Matrix $C$ aus den $d_{i,k,j}$ kann nach
    \ref{SatzAlgRechnen} f"ur ein Matrizenelement in
    \[ \lceil \log(p) \rceil \] Schritten von
    \[ \lf \frac{p}{2} \rf \] Prozessoren durchgef"uhrt
    werden, also f"ur
    die gesamte Matrix in genauso vielen Schritten von
    \[ m * n * \lf \frac{p}{2} \rf \] Prozessoren.
    Die Werte
    f"ur Schritte und Prozessoren zusammengenommen ergeben die Behauptung.
\end{beweis}

Zwei $n \times n$-Matrizen lassen sich also in 
\[ \lceil \log(n) \rceil + 1 \] Schritten von
\[ n^3 \] Prozessoren miteinander multiplizieren.

Die \label{PageAlg2MatMult}
Matrizenmultiplikation l"a"st sich asymptotisch, d. h. f"ur $n \to \infty$
auch mit
\[ O(n^{2+\gamma}), \: \gamma = 0.376 \]
Prozessoren durchf"uhren \cite{CW90}. Gegen"uber \ref{SatzAlgMatMult} 
ergibt sich wegen des erheblichen konstanten Aufwandes nur f"ur gro"se $n$ 
eine Verbesserung. Es wird jeweils gesondert darauf hingewiesen, falls
auf diese M"oglichkeit zur"uckgegriffen wird.

% **************************************************************************
% **************************************************************************
% **************************************************************************

\MyChapter{Grundlagen aus der Linearen Algebra}
\label{ChapBase}

In diesem Kapitel werden die f"ur den gesamten weiteren Text wichtigen 
Begriffe und S"atze aus der Linearen Algebra behandelt. Falls in sp"ateren
Kapiteln an einzelnen Stellen weitergehende Grundlagen insbesondere aus 
anderen Bereichen n"otig sind, werden diese an den jeweiligen Stellen 
behandelt.

Da es sich bei dem Inhalt dieses Kapitels um
Grundlagen handelt, sind einige Beweise etwas oberfl"achlicher bzw. 
fehlen ganz.

Literatur:
\begin{itemize}
\item
      \cite{MM64} Kapitel 1 und 2
\item
      \cite{Doer77} Kapitel 6, 9 und 12
\item
      \cite{BS87} ab Seite 148
\end{itemize}

Im folgenden sind $A$ und $B$ $n \times n$-Matrizen. F"ur uns reichen
Betrachtungen im K"orper der rationalen Zahlen aus.

% **************************************************************************

\MySection{Matrizen und Determinanten}
\label{SecMatUndDet}

In diesem Kapitel werden die grundlegendsten Begriffe "uber Matrizen
und Determinanten aufgef"uhrt, um eine Grundlage f"ur den weiteren 
Text zu vereinbaren. 

\MyBeginDef
\index{invertierbar}
\label{DefInvertierbar}
    $A$ hei"st {\em invertierbar}, wenn es eine
    Matrix $B$ gibt, so da"s \[ AB = BA = E_n \]
    In diesem Fall hei"st $B$ {\em Inverse von $A$} und wird auch mit
    \[ A^{-1} \] bezeichnet.
\MyEndDef

\MyBeginDef
\index{Transponierte}
    Falls f"ur die Matrizen $A$ und $B$ gilt
    \[ b_{i,j} = a_{j,i} \] so hei"st $B$ { \em Transponierte von $A$}.
    F"ur die Transponierte von $A$ wird auch $A^T$ geschrieben.
\MyEndDef

\MyBeginDef
\label{DefTr}
\index{Spur}
    \[ \tr(A) := \sum_{i=1}^n a_{i,i} \]
    hei"st {\em Spur der Matrix $A$}.
\MyEndDef

\MyBeginDef
\index{Permutation}
\index{Inversion}
    Eine bijektive Abbildung
    \[ f : \{1,\ldots,n \} \rightarrow \{1, \ldots, n \} \]
    hei"st {\em $n$-Permutation}.
    Sei \[ 1 \leq i < j \leq n \] Falls gilt
    \[ f(i) > f(j) \MyKomma \] so hei"st diese Bedingung {\em Inversion der
    $n$-Permutation}.
\MyEndDef

\[ \permut_n \] bezeichnet die Menge aller $n$-Permutationen. Zusammen mit
der Konkatenation von Abbildungen bildet sie eine Gruppe, die
{\em symmetrische Gruppe $\permut_n$}.

\MyBeginDef
\index{Signatur einer Permutation}
\label{DefSig}
    Sei $f$ eine $n$-Permutation. Dann hei"st
    \begin{equation}
    \label{EquDefSig}
        \sig(f) := \prod_{1 \leq i < j \leq n} \frac{f(i)-f(j)}{i - j}
    \end{equation}
    {\em Signatur von $f$}.
\MyEndDef

Die so definierte Signatur besitzt folgende Eigenschaften:
\begin{itemize}
\item 
      Es gilt
      \[ \forall f \in \permut_n: \sig(f) \in \{ 1,-1 \} \]
      Aus der Permutationseigenschaft ergibt sich, da"s es f"ur jede 
      Differenz, die als Faktor im Z"ahler von \equref{EquDefSig} auftaucht,
      eine Differenz im Nenner mit dem gleichen Betrag existiert, so da"s 
      der Wert des gesamten Produktes den Betrag $1$ besitzt. Das Vorzeichen
      wird durch die Anzahl der Inversionen beeinflu"st.
\item
      Falls die Anzahl der Inversionen der $n$-Permutation $f$ gerade ist,
      so gilt \[ \sig(f)= 1 \MyKomma \] andernfalls gilt
      \[ \sig(f) = -1 \]
\item

      Die Anzahl der $n$-Permutationen $f$ mit \[ \sig(f)=1 \MyKomma \]
      ist gleich
      der Anzahl der $n$-Permutationen $g$ mit \[ \sig(g)=-1 \MyPunkt \]
      (\cite{Doer77} Seite 196)
\end{itemize}

Die Permutationen mit \[ \sig(f)=1 \] nennt man {\em gerade}, die anderen 
{\em ungerade}.

\MyBeginDef
\index{Determinante!Definition}
\label{DefDet}
    Seien \[ A,B \in \Rationals^{n^2} \]
    Sei \[ \det : \: \Rationals^{n^2} \rightarrow \Rationals \]
    eine Abbildung mit folgenden Eigenschaften:
    \begin{MyDescription}
    \MyItem{D1}
        Entsteht $B$ aus $A$ durch Multiplikation einer Zeile mit
        \[ r \in \Rationals \] so gilt:
        \[ \det(B) = r \det(A) \]
    \MyItem{D2}
        Enth"alt $A$ zwei gleiche Zeilen, so gilt:
        \[ \det(A) = 0 \]
    \MyItem{D3}
        Entsteht $B$ aus $A$ durch Addition des $r$-fachen einer Zeile zu
        einer anderen, so gilt:
        \[ \det(B) = \det(A) \]
    \MyItem{D4} F"ur die Einheitsmatrix gilt:
        \[ \det(E_n) = 1 \]
    \end{MyDescription}
    Dann hei"st
    \[ \det(A) \] {\em Determinante der Matrix $A$}.
\MyEndDef

\MyBeginDef
\label{DefZeilenOp}
\index{Zeilenoperationen!elementare}
\index{Spaltenoperationen!elementare}
    Die auf einer Matrix definierten Operationen
    \begin{itemize}
    \item 
          Vertauschung zweier Zeilen 
    \item 
          Multiplikation einer Zeile mit einem 
          Faktor\footnote{vgl. D1 in \ref{DefDet}}
    \item 
          Addition des Vielfachen einer Zeile zu einer 
          anderen\footnote{vgl. D3 in \ref{DefDet}}
    \end{itemize}
    werden {\em elementare Zeilenoperationen} genannt. Die entsprechenden
    Operationen auf Matrizenspalten werden 
    {\em elementare Spaltenoperationen} genannt.
\MyEndDef

\begin{satz}
\label{SatzDetPermut}
    \begin{equation}
    \label{EquDet}
       g(A) =
       \sum_{f \in \permut_n} 
           \sig(f) a_{1,f(1)} a_{2,f(2)} \ldots a_{n,f(n)}
    \end{equation}
    besitzt die Eigenschaften aus \ref{DefDet}.
\end{satz}
\begin{beweis}
    Da es sich hier um Grundlagen handelt, wird der Beweis weniger
    ausf"uhrlich angegeben:
    \begin{MyDescription}
    \MyItem{D1}
         F"ur jede Permutation kommt im entsprechenden Summanden in
         \equref{EquDet} aus jeder Zeile der Matrix genau ein Element als
         Faktor vor. Falls eine Zeile mit $r$ multipliziert wurde, kann
         man also aus jedem Summanden $r$ ausklammern und erh"alt die
         Behauptung.
    \MyItem{D2}
         Seien Zeile $i$ und Zeile $j$ gleich $(i \neq j)$.
         Berechne die
         Summe in \equref{EquDet} getrennt f"ur die ungeraden und die
         geraden Permutationen.

         Die
         $n$-Permutation $g$ vertausche $i$ mit $j$ und lasse alles andere
         gleich. Aus den Grundlagen der Theorie der
         Halbgruppen und Gruppen ergibt sich, da"s man die ungeraden
         $n$-Permutationen erh"alt, indem man die geraden
         $n$-Permutationen jeweils einzeln mit der Permutation $g$
         zusammen ausf"uhrt.

         Deshalb entsprechen sich die Summanden der beiden Teilsummen
         paarweise und unterscheiden sich nur durch das Vorzeichen. Der
         Gesamtausdruck besitzt also den Wert $0$.
    \MyItem{D3}
         Es werde das $r$-fache von Zeile $i$ zu Zeile $j$ addiert. Dadurch
         enth"alt jeder Summand in \equref{EquDet} genau einen Faktor,
         der seinerseits wieder die Summe zweier Matrizenelemente ist.
         Deshalb kann man die gesamte Summe in zwei Summen aufteilen.
         Die eine entspricht genau der Summe in \equref{EquDet}, die andere
         enth"alt in jedem Summanden zwei gleiche
         Faktoren, sowie den Faktor $r$. 
         Diesen Faktor kann man ausklammern (mit Hilfe
         von \mbox{D1}). Nach \mbox{D2} ist der Wert dieser Summe dann
         gleich Null, und nur die andere bleibt "ubrig.
    \MyItem{D4}
         Au"ser f"ur die identische Abbildung enth"alt jeder Summand
         der entsprechenden Permutation in \equref{EquDet} mindestens zwei
         Nullen als Faktoren und ist deshalb gleich Null. Der Summand, der
         der identischen Abbildung entspricht, hat den Wert $1$.
    \end{MyDescription}
    \nopagebreak
\end{beweis}

% **************************************************************************

\MySection{Der Rang einer Matrix}
\label{SecRang}
\index{Rang}

Zum Verst"andnis des weiteren Textes wird in diesem Kapitel der
Begriff des {\em Rangs} einer Matrix eingef"uhrt. Da dieser Begriff
f"ur die Untersuchung linearer Gleichungssysteme wichtig ist, werden
teilweise auch nichtquadratische Matrizen betrachtet\footnote{ Literatur: 
siehe \ref{SecMatUndDet} } . 

Da in diesem Kapitel 
wiederum Grundlagen aus der Linearen Algebra behandelt werden, ist
die Darstellung auf die f"ur uns wichtigen Aspekte beschr"ankt.

% $$$ Kenntnis der Begriffe 'linear unabh"angig' und 'Linearkombination'
%     wird hier vorausgesetzt
\MyBeginDef
    Die maximale Anzahl linear unab"angiger Spaltenvektoren einer Matrix
    wird mit {\em Spaltenrang} \index{Spaltenrang} bezeichet.
    
    Die maximale Anzahl linear unabh"angiger Zeilenvektoren einer Matrix
    wird mit {\em Zeilenrang} \index{Zeilenrang} bezeichnet.
\MyEndDef

Die folgenden Betrachtungen gelten f"ur den Zeilenrang analog.

Die $m \times n$-Matrix $A$ habe den Spaltenrang $r$. Es gilt also
\[ 0 \leq r \leq n \MyPunkt \] Die Spaltenvektoren von $A$ werden mit
\[ a_1, \, \ldots , \, a_n \] bezeichnet. Seien die Spaltenvektoren
\[a_{i_1}, \, \ldots, \, a_{i_r}\] linear unabh"angig. Das bedeutet, aus
\begin{eqnarray}
  & & d_1 a_{i_1} + \cdots + d_r a_{i_r} \nonumber \\
  & = & \left[
            \begin{array}{c}
                d_1 a_{1,{i_1}} \\ \vdots \\ d_1 a_{n,{i_1}}
            \end{array}
        \right]
        + \cdots +
        \left[
            \begin{array}{c}
                d_r a_{1,{i_r}} \\ \vdots \\ d_r a_{n,{i_r}}
            \end{array}
        \right]
        \nonumber \\
  & = & \left[
             \begin{array}{c}
                 d_1 a_{1,{i_1}} + \cdots + d_r a_{1,{i_r}} \\
                 \vdots \\
                 d_1 a_{n,{i_1}} + \cdots + d_r a_{n,{i_r}}
             \end{array}
        \right] \label{EquRangZeilentausch} \\
  & = & 0_n \nonumber
\end{eqnarray}
folgt
\[ d_1 = \ldots = d_r = 0 \MyPunkt \]
Vertauscht man zwei Elemente des Vektors \equref{EquRangZeilentausch},
bleibt die G"ultigkeit dieser Bedingung davon unber"uhrt.
Daraus folgt, da"s die Vertauschung zweier Matrixzeilen den Spaltenrang
der Matrix unber"uhrt l"a"st. 

Ebenso verh"alt es sich mit der Multiplikation einer Zeile der Matrix
mit einem Faktor $c \neq 0$.
Analog zur Argumentation bei der Vertauschung zweier Zeilen erh"alt man
\[
        \left[
             \begin{array}{c}
                 c (d_1 a_{1,{i_1}} + \cdots + d_r a_{1,{i_r}}) \\
                 \vdots \\
                 (d_1 a_{n,{i_1}} + \cdots + d_r a_{n,{i_r}})
             \end{array}
        \right]
\]
\MyPunktA{35em}
Die Bedingung $d_1 = \ldots = d_r = 0$ bleibt durch den Faktor unber"uhrt.

Das gleiche Ergebnis erh"alt man f"ur die Addition des Vielfachen einer
Zeile zu einer anderen.

Die elementaren Zeilenoperationen\footnote{siehe \ref{DefZeilenOp}}
lassen den Spaltenrang also unver"andert.
Analog verh"alt es sich mit den elementare Spaltenoperationen und
dem Zeilenrang.

Durch die Zeilen- und Spaltenoperationen l"a"st sich jede Matrix in
Diagonalform "uberf"uhren\footnote{nicht zu verwechseln mit
Diagonalisierbarkeit!}, ohne da"s dadurch der Rang ver"andert wird.

F"ur eine Matrix, bei der nur die Elemente der Hauptdiagonalen von Null
verschieden sein k"onnen, stimmen Zeilen- und Spaltenrang offensichtlich
"uberein. Da die Zeilen und Spaltenoperationen den Rang unver"andert
lassen, erh"alt man:
\begin{korollar}
\label{SatzRang}
    F"ur jede Matrix stimmen Zeilen- und Spaltenrang "uberein.
\end{korollar}
Aus diesem Grund ist die folgende Definition sinnvoll:

\MyBeginDef
\label{DefRang} \index{Rang}
    Die Anzahl linear unabh"angiger Spalten einer Matrix $A$ wird als
    {\em Rang von $A$} bezeichnet, kurz
    \[ \rg(A) \MyPunkt \]
\MyEndDef

Es gilt offensichtlich:
\[ rg(A) \leq \min(m,n) \MyPunkt \]

An dieser Stelle k"onnen wir drei wichtige Begriffe zueinander in
Beziehung setzen:

\begin{satz}
\label{SatzRgDetInv}
    F"ur eine $n \times n$-Matrix $A$ sind folgende Aussagen "aquivalent:
    \begin{itemize}
    \item
         Matrix $A$ ist invertierbar.
    \item
         F"ur die Determinante gilt:
         \[ \det(A) \neq 0 \]
    \item
         F"ur den Rang gilt:
         \[ \rg(A) = n \]
    \end{itemize}
\end{satz}
\begin{beweis}
    Es ist zu beachten, da"s jede Matrix durch elementare Zeilen-
    und Spaltenoperationen in eine Diagonalmatrix "uberf"uhrt werden kann,
    ohne da"s sich die Invertierbarkeitseigenschaft, der Betrag der 
    Determinante oder der Rang dadurch ver"anderen. Man kann also 
    o. B. d. A. davon ausgehen, da"s $A$ Diagonalform besitzt.
    
    Durch diese "Uberlegung wird die G"ultigkeit der Aussage offensichtlich.
\end{beweis}

% **************************************************************************

\MySection{L"osbarkeit linearer Gleichungssysteme}
\label{SecLinEqu}
\index{Gleichungssystem}

Da lineare Gleichungssysteme eine wichtige Rolle beim Verst"andnis
noch folgender Ausf"uhrungen spielen, werden sie in diesem Kapitel
n"aher betrachtet. Die nun folgenden Grundlagen aus der Linearen 
Algebra sind in der in \ref{SecMatUndDet} aufgelisteten Literatur
ausf"uhrlich behandelt. Wir beschr"anken uns hier auf die f"ur
den nachfolgenden Text wichtigen Sachverhalte.

F"ur die weiteren Beschreibungen drei grundlegende Begriffe aus der 
Linearen Algebra von Bedeutung, deren Definitionen deshalb hier
angegeben sind:

Sei $K$ ein K"orper. Eine Menge $V$ zusammen mit zwei Verkn"upfungen
\begin{eqnarray}
    + & : & V \times V \rightarrow V \\
    * & : & K \times V \rightarrow V \MyKomma
\end{eqnarray}
die die folgenden Bedingungen erf"ullt:
\begin{itemize}
\item Die Menge $V$ in Verbindung mit $+$ ist eine Gruppe.
\item F"ur alle $v,w \in V$ und alle $r,s \in K$ gelten die 
      Gleichungen
      \begin{eqnarray*}
          (r+s)v = rv +sv \\
          r(v+w) = rv + rw \\
          (rs)v = r(sv) \\
          1v = v \MyPunkt 
      \end{eqnarray*}
\end{itemize}
wird als $K$-Vektorraum bezeichnet.

\MyBeginDef
\label{DefUnterraum}
    Eine nichtleere Teilmenge $U$ eines $K$-Vektorraumes $V$ wird als 
    {\em Unterraum} \index{Unterraum} von $V$ bezeichnet, falls gilt
    \[
        \forall u,v \in U, \, r \in K : \: 
        \left\{
            \begin{array}{rcl}
                u + v & \in & U \\
                r u & \in & U \MyPunkt 
            \end{array}
        \right.
    \]
\MyEndDef

\MyBeginDef
\label{DefKern}
\index{Kern!einer linearen Abbildung}
    Die Menge aller Vektoren $x$,
    f"ur die bei einer gegebenen $m \times n$-Matrix $A$ gilt
    \Beq{EquKern}
        A x = 0_n
    \Eeq
    wird als {\em Kern der Matrix A} , kurz $\MyKer(A)$,
    bezeichnet\footnote{ In der Literatur wird h"aufig an dieser
    Stelle der Kern einer linearen Abbildung betrachtet. Da die 
    theoretischen Hintergr"unde hier nicht von Interesse sind, ist in
    unser Betrachtung von Matrizen die Rede.} .
\MyEndDef

F"ur unsere Zwecke ben"otigen wir noch zwei weitere Feststellungen:

\begin{bemerkung}
\label{SatzKernUnterraum}
    Betrachtet man \ref{DefUnterraum}, erkennt man,
    da"s alle $x$, die Gleichung \equref{EquKern} er\-f"ul\-len,
    einen Unterraum bilden.
\end{bemerkung}

Der Rang einer Matrix und die Dimension von $\MyKer(A)$ h"angen auf
eine f"ur uns wichtige Weise zusammen:

\begin{lemma}
\label{SatzDimKer}
    Sei $V$ ein $K$-Vektorraum der Dimension $n$ und
    $W$ ein $K$-Vektorraum der Dimension $m$. Sei
    \[ f: V \rightarrow W \] eine lineare Abbildung mit der
    entsprechenden $m \times n$-Matrix $A$. Dann gilt:
    \[ \rg(A) + \MyDim(\MyKer(A)) = \MyDim(V) \MyPunkt \]
\end{lemma}
\begin{beweis}
    Zum Beweis der obigen Gleichung wird die Dimension von 
    $\MyKer(A)$ in Abh"angigkeit vom Rang von $A$ betrachtet.

    Die Matrix $A$ habe den Rang $r$. Die maximale Anzahl linear
    unabh"angiger Spaltenvektoren betr"agt also $r$. O. B. d. A. seien
    die ersten $r$ Spaltenvektoren linear unabh"angig.
    
    Es wird in zwei Schritten gezeigt, da"s in $\MyKer(A)$ die maximale 
    Anzahl der linear unabh"angigen Vektoren, also die Dimension von $A$,
    $n-r$ betr"agt.
    
    \begin{itemize}
    \item Bezeichne $a_i$ den $i$-ten Spaltenvektor von $A$. Der Kern 
          von $A$ ist die Menge
          \begin{eqnarray*}
              & & \{ x \in V \MySetProperty A x = 0_m \} \\
              & = & \{ x \in V \MySetProperty
               a_1 x_1 + a_2 x_2 + \cdots + a_n x_n = 0_m \} \MyPunkt
          \end{eqnarray*}
          Die Dimension dieser Menge ist die maximale Anzahl linear
          unabh"angiger Vektoren $x$, die in ihr enthalten sind. Die
          Elemente eines Vektors $x$ kann man, wie an der obigen
          Darstellung ersichtlich ist, als Faktoren in einer
          Linearkombination von Spaltenvektoren von $A$ betrachten.
          Anhand der Voraussetzungen, die f"ur die Spaltenvektoren
          gelten, lassen sich Aussagen f"ur die Vektoren $x$ machen.
          
          Da $r+1$
          Spaltenvektoren von $A$ immer linear abh"angig sind, ist es
          m"oglich, durch Linearkombination von jeweils $r+1$ 
          Spaltenvektoren den Nullvektor zu erhalten. Sind die "ubrigen 
          $n-(r+1)$ Spaltenvektoren nicht an einer Linearkombination 
          beteiligt, entspricht das einer Null als entsprechendes Element 
          von $x$.
          
          Es gibt $n-r$ M"oglichkeiten,
          zu den ersten $r$ linear unabh"angigen Spaltenvektoren von $A$
          genau einen weiteren
          auszusuchen. Aufgrund der Verteilung der Nullen in den 
          entsprechenden Vektoren $x$ mu"s es mindestens $n-r$ linear 
          unabh"angige Vektoren im Kern von $A$ geben.
    \item 
          Angenommen, die Anzahl der linear unabh"angigen Vektoren 
          in $\MyKer(A)$ ist gr"o"ser als $n-r$. O. B. d. A. seien die 
          ersten $n-r$ dieser Vektoren
          so gew"ahlt, da"s bei jedem dieser Vektoren unter den
          Vektorelementen $x_{r+1}$ bis $x_n$ genau eines ungleich Null ist.
          Aus den vorangegangenen Ausf"uhrungen folgt, da"s solche
          Vektoren existieren m"ussen. Die Vektoren werden mit 
          $v_{r+1}, \, \ldots, \, v_n$ bezeichnet. F"ur den Vektor $v_i$
          sei das $i$-te Vektorelement ungleich Null.
          
          Da die ersten $r$ Spaltenvektoren von $A$ linear unabh"angig sind,
          gibt es f"ur die ersten $r$ Elemente jedes Vektors $v_i$
          keine Wahlm"oglichkeit, sobald das $i$-te Element im
          Rahmen der Randbedingungen dem Wert nach festliegt.

          Sei $w$ ein weiterer Vektor aus $\MyKer(A)$, der zu den Vektoren
          $v$ linear unabh"angig ist. Wird das $j$-te Element eines Vektors
          $v_i$ mit $v_{i,j}$ bezeichnet, und das $k$-te Element von $w$
          mit $w_k$, dann gibt es Faktoren $d_{r+1}, \, \ldots , \, d_n$,
          so da"s
          \[ \forall r+1 \leq k \leq n : \: w_k = d_k * v_{k,k} \MyPunkt \]
          
          Ebenso wie f"ur die Vektoren $v$ liegen damit auch die
          ersten $r$ Elemente des Vektors $w$ fest. Sie k"onnen anhand
          der obigen Beziehung aus den Elementen $r+1$ bis $n$ der
          Vektoren $v$ und $w$ berechnet werden, sofern $w$ "uberhaupt die
          Rahmenbedinungen erf"ullt und in $\MyKer(A)$ ist.

          Damit $w$ dennoch zu den Vektoren $v$ linear unabh"angig ist,
          mu"s es einen $r+1$-ten Spaltenvektor von $A$ geben, der zu den
          ersten $r$ Spaltenvektoren linear unabh"angig ist und der
          bei der Zusammenstellung der Vektoren $v$ mit Hilfe der
          Betrachtung von Linearkombinationen (s. o.) noch nicht benutzt
          wurde. Dies ist jedoch im Widerspruch zu den Voraussetzungen.
    \end{itemize}
    Somit folgt:
    \[ \rg(A) + \dim(\MyKer(A)) = r + (n-r) = n = \dim(V) \MyPunkt \]
\end{beweis}
Mit Hilfe von \ref{SatzKernUnterraum} und \ref{SatzDimKer} k"onnen wir
die f"ur uns wichtigen Eigenschaften linearer Gleichungssysteme
betrachten.

Ein lineares Gleichungssystem besteht aus $n$ Gleichungen mit $m$
Unbekannten:
\begin{eqnarray*}
    a_{1,1} x_1 + a_{1,2} x_2 + \cdots + a_{1,n} x_n & = & b_1 \\
    \vdots & \vdots & \vdots \\
    a_{n,1} x_1 + a_{n,2} x_2 + \cdots + a_{m,n} x_n & = & b_m
\end{eqnarray*}
Die $x_i$ sind die zu bestimmenden Unbekannten. Betrachtet man die
gegebenen Konstanten $a_{i,j}$ und $b_i$ als Elemente einer Matrix
bzw. eines Vektors, kann man das Gleichungssystem auch kompakter
darstellen:
\[ A x = b \MyPunkt \]
Dabei ist $A$ ein $n \times m$-Matrix und $b$ ein Vektor der L"ange $m$.
Ist $b= 0_m$, so bezeichnet man das Gleichungssystem als
homogen \index{Gleichungssystem!homogenes}.
Ist $b \neq 0_m$, so bezeichnet man das Gleichungssystem als
inhomogen \index{Gleichungssystem!inhomogenes}.

Uns interessieren die L"osungsmengen eines solchen Gleichungssystems.

\begin{satz}
\label{SatzLoesungsraum}
    Sei $V$ der zugrundeliegende $K$-Vektorraum der Dimension $n$.
    Die L"osungsmenge $L(A,0_m)$ des homogenen Gleichungssystems
    \[ A x = 0_m \] ist ein Unterraum von $V$ der Dimension
    \[ n - \rg(A) \MyPunkt \]
\end{satz}
\begin{beweis}
    Die L"osungsmenge $L(A,0_m)$ ist der Kern von $A$. Damit folgt
    die Behauptung aus \ref{SatzKernUnterraum} und
    \ref{SatzDimKer}.
\end{beweis}

Ist $\rg(A)=n$, gilt also $L(A,0_m) = \{ 0_n \}$.

Uns interessieren auch die L"osungen inhomogener Gleichungssysteme.
Dazu betrachtet man die erweiterte Matrix $[A,b]$, die aus der Matrix
$A$ und dem Vektor $b$ als $n+1$-te Spalte besteht:

\begin{satz}
\label{SatzRangGleich}
    Die Gleichung
    \Beq{EquInhomogen}
        A x = b \MyPunkt
    \Eeq
    ist genau dann l"osbar, wenn gilt
    \Beq{EquRangGleich}
        \rg(A) = \rg([A,b]) \MyPunkt
    \Eeq
\end{satz}
\begin{beweis}
    Man kann die linke Seite von \equref{EquInhomogen} als 
    Linearkombination von Spaltenvektoren von $A$ betrachten. Die Faktoren 
    der Linearkombination bilden die Elemente des L"osungsvektors $x$.

    Der Vektor $b$ ist genau dann als Linearkombination der
    Spaltenvektoren von $A$ darstellbar, wenn die maximale Anzahl linear
    unabh"angiger Vektoren in $A$ und $[A,b]$ gleich ist. Dies ist
    gleichbedeutend mit der G"ultigkeit von \equref{EquRangGleich}.
\end{beweis}

Dieser Satz f"uhrt zu einer f"ur uns bedeutsamen Charakterisierung
der L"osbarkeit:

\begin{korollar}
\label{SatzGenauEine}
    Ist \equref{EquInhomogen} l"osbar und $\rg(A) = n$,
    also $n \leq m$, dann sind die Spaltenvektoren von $A$
    linear unabh"angig
    und werden alle f"ur die Linearkombination zur Darstellung von $b$
    ben"otigt. Es gibt in diesem Fall genau einen Vektor $x$, der
    \equref{EquInhomogen} l"ost.
\end{korollar}

Ist $\rg(A) = m$, also $m \leq n$, dann ist \equref{EquInhomogen} f"ur
jedes $b$ l"osbar, denn es gilt allgemein
\[ \rg(A) \leq \rg([A,b]) \leq m \MyPunkt \] F"ur $\rg(A) = m$ folgt
daraus mit Hilfe der vorangegangenen Bemerkungen die L"osbarkeit.

% $$$ hier evtl. noch Wegener Satz 7.4 und Hilfssatz 7.7

% **************************************************************************

\MySection{Das charakteristische Polynom}
\label{ChapCharPoly}

Betrachtet \label{PageEigenMotiv}
man $A$ als lineare Abbildung im $n$-dimensionalen Vektorraum,
so lautet eine interessante Fragestellung:
\begin{quote}
    Gibt es einen Vektor $x$ ungleich dem Nullvektor sowie einen Skalar 
    $\lambda$, so da"s gilt
    \Beq{EquEigenMotiv}
        Ax=\lambda x \mbox{\hspace{1em}?}
    \Eeq
\end{quote}
Man kann \equref{EquEigenMotiv} umformen in
\[ (A - \lambda E_n)x = 0 \MyPunkt \] Dies ist eine homogenes lineares
Gleichungssystem mit $n$ Gleichungen in $n$ Unbekannten. 
Aus \ref{SatzLoesungsraum} und \ref{SatzRgDetInv} folgt, da"s es genau
dann eine nichttriviale L"osung besitzt, wenn gilt
\[ \det(A - \lambda E_n) = 0 \MyPunkt \]
Dies motiviert die folgende Definition:

\MyBeginDef
\label{DefCharPoly}
\index{Polynom!charakteristisches}
\index{Gleichung!charakteristische}
    \[ \det(A-\lambda E_n) \] hei"st\footnote{Die Form
    $\det(\lambda E_n - A)$ ist in der Literatur auch gebr"auchlich. Welche
    Form gew"ahlt wird, h"angt von den Vorlieben im Umgang mit den
    Vorzeichen ab. Bis auf die Vorzeichen bleibt die G"ultigkeit von
    Aussagen unber"uhrt. F"ur uns ist die in der Definition angegebene
    Form bequemer.} {\em charakteristisches Polynom der Matrix $A$}. Die
    obige Darstellung als Determinante einer Matrix 
    hei"st \index{Matrizendarstellung!des charakteristischen Polynoms} 
    {\em Matrizendarstellung} des charakteristischen Polynoms.
    \[ \det(A-\lambda E_n) = 0 \] hei"st {\em charakteristische Gleichung
    der Matrix $A$}.
    Die Matrix \[ A - \lambda E_n \] wird als 
    {\em charakteristische Matrix} \index{Matrix!charakteristische}
    bezeichnet.
\MyEndDef

Das charakteristische Polynom einer Matrix l"a"st sich auch in der
\index{Koeffizientendarstellung} {\em Koeffizientendarstellung} 
\Beq{EquKoeffDarst}
    c_n \lambda^n + c_{n-1} \lambda^{n-1} + \cdots + c_1 \lambda + c_0
\Eeq
angeben.

In der Literatur sind als
Vorzeichen von $c_i$ nicht nur $+$, sondern auch $-$ oder $(-1)^i$
gebr"auchlich. Mancherorts herrscht auch die Gewohnheit,
die Indizierung der Koeffizienten in der Form
$c_0 \lambda^n + c_1 \lambda^{n-1} + \cdots + c_{n-1} \lambda + c_n$
durchzuf"uhren. Da diese Vielfalt der Gebr"auche in der Literatur nicht 
nur auf die Koeffizientendarstellung beschr"ankt ist, kann es u. U. sein, 
da"s $c_n$ sowohl das Vorzeichen $+$ als auch immer den Wert $1$ besitzt.
In diesen F"allen wird $c_n$ auch h"aufig weggelassen.
F"ur uns ist die gew"ahlte Form \equref{EquKoeffDarst} der Darstellung
am sinnvollsten.
% $$$ in Verbindung mit meinem anderen Krempel: c_n = (-1)^n

\MyBeginDef 
\label{DefEigenwert}
\index{Eigenwert} \index{Eigenvektor}
    Die Nullstellen des charakteristischen Polynoms einer Matrix hei"sen
    {\em Eigenwerte} der Matrix. Ein Vektor $x$ der
    Gleichung \equref{EquEigenMotiv} zusammen mit einem Eigenwert
    $\lambda$ erf"ullt, hei"st {\em Eigenvektor}. Der Nullvektor ist
    als Eigenvektor nicht zugelassen.
\MyEndDef

\MyBeginDef
    Sei $\lambda_i$ ein Eigenwert der $n \times n$-Matrix $A$. Dann wird
    \[ \rg(A) - \rg(A - \lambda_i E_n) \] als
    {\em Rangabfall} \index{Rangabfall} des Eigenwertes $\lambda_i$ 
    bezeichnet.
\MyEndDef

Die Determinante und die Spur findet man unter den Koeffizienten des
charakteristischen Polynoms wieder. Besonders f"ur die Berechnung der
Determinante ist dies eine wichtige Beobachtung:
\begin{bemerkung}
\label{SatzDdurchP}
    \begin{eqnarray}
       \det(A) & = & c_0               \nonumber
    \\ \tr(A) & = & (-1)^{n-1} c_{n-1} \label{EquTrCoefficient}
    \end{eqnarray}
\end{bemerkung}
Wenn $p(\lambda)$ das charakteristische Polynom der Matrix $A$ ist, gilt
also: \[ \det(A)= p(0) \MyPunkt \]

Rechnet man die Matrizendarstellung des charakteristischen Polynoms um in
die Koeffizientendarstellung, wird dabei die G"ultigkeit von
\equref{EquTrCoefficient} deutlich. Der Wert von $c_{n-1}$ wird nur durch
die Matrixelemente in der Hauptdiagonalen beeinflu"st. Das Produkt dieser
Elemente besteht aus $n$ Linearfaktoren der Form
\[ a_{i,i} - \lambda \MyPunkt \] Berechnet man den Wert dieses Produktes,
so wird $c_n$ durch alle Summanden der Form
\[ (-1)^{n-1} a_{i,i}\lambda^{n-1} \] beeinflu"st. Insgesamt gilt also
\[ c_{n-1} = (-1)^{n-1} \sum_{i=1}^n a_{i,i} \MyKomma \] so da"s man mit
\equref{EquTrCoefficient} die Spur erh"alt.

Ein Algorithmus zur Berechnung der Koeffizienten des charakteristischen
Polynoms einer Matrix berechnet somit u. a. auch deren Determinante und
deren Spur.



%
% Datei: csanky.tex (Textteile nach 'Csan74' und 'Csan76')
%

\MyChapter{Die Algorithmen von Csanky}
\label{ChapCsanky}

In diesem Kapitel werden zwei von L. Csanky\footnote{ \cite{Csan74} und 
\cite{Csan76} } vorgeschlagene Algorithmen 
behandelt. Der erste davon verwendet eine relativ bekannte Methode zur
Determinantenberechnung\footnote{Entwicklungssatz von Laplace}. 
Die Effizienz dieser Methode ist jedoch nicht befriedigend. Ihre 
Darstellung ist als Einstieg gedacht.

\MySection{Die Stirling'schen Ungleichungen}

Als Vorarbeit f"ur den ersten Algorithmus werden in diesem Unterkapitel 
weitere Grundlagen behandelt.

\begin{satz}[Stirling'sche Ungleichungen]
\label{SatzStirling}
\index{Stirling'sche Ungleichungen}
    Sei \[ n \in \Nat \] Dann gilt
    \begin{eqnarray}
        \label{Equ1SatzStirling}
        n! & \geq & \sqrt{2\pi n}
                    \left( \frac{n}{\MathE} \right)^n
                    \MathE^{\frac{1}{12n+1}}
    \\  
        \label{Equ2SatzStirling}
        n! & \leq & \sqrt{2\pi n}
                    \left( \frac{n}{\MathE} \right)^n
                    \MathE^{\frac{1}{12n}}
    \end{eqnarray}
\end{satz}
\begin{beweis}
    Die Ungleichungen folgen aus \cite{Mang67} S. 154.
\end{beweis}

Die Ungleichungen werden in den folgenden Lemmata angewendet. Die Richtung
der Ab\-sch"a\-tzung der jeweiligen Werte wird durch deren Verwendung
im sp"ateren Text bestimmt.

\begin{lemma}
\label{SatzStirling2Anwendung}
    \[ \MyChoose{ n }{ \frac{n}{2} }
           \geq
       \frac{2^n}{ \sqrt{\pi \frac{n}{2}} \: \MathE }
    \]
\end{lemma}
\begin{beweis}
    Es soll nach unten abgesch"atzt werden. Deshalb wird der Z"ahler mit
    Hilfe von \equref{Equ1SatzStirling} und der Nenner mit Hilfe von
    \equref{Equ2SatzStirling} abgesch"atzt. Man erh"alt:
    \begin{eqnarray}
       \MyChoose{ n }{ \frac{n}{2} }
           & = &
       \frac{n!}{ \left( \frac{n}{2} \right)^2 } \nonumber
    \\ \label{Equ1LemmaStirling2}
           & \geq &
       \frac{
            \sqrt{2\pi n}
            \left( \frac{n}{\MathE} \right)^n
            \MathE^{\frac{1}{12n+1}}
       }{
            \left(
                \sqrt{\pi n}
                \left( \frac{ \frac{n}{2} }{\MathE} \right)^{\frac{n}{2}}
                \MathE^{\frac{1}{6n}}
            \right)^2
       }
    \\     & \geq &
       \frac{
           \sqrt{2}
       }{
           \sqrt{\pi n} \left( \frac{1}{2} \right)^n
       }
       \, \MathE^{ \frac{ -18n-2 }{ (12n + 1)6n } }
    \\     & \geq & 
       \frac{2^n}{ \sqrt{\pi \frac{n}{2}} \: \MathE 
       } \nonumber
    \end{eqnarray}
\end{beweis}

Auf die gleiche Weise, wie im vorangegangenen Lemma
\ref{SatzStirlingAnwendung} eine Absch"atzung nach unten erfolgt,
erh"alt man eine Absch"atzung von
\[ \log \MyChoose{ n }{ \frac{n}{2} } \] nach oben:

\begin{lemma}
\label{SatzStirlingAnwendung}
    \[ \log{ n \choose \frac{n}{2} }
           \leq
       n - \frac{\log \left( \frac{n}{2} \right) + 1}{2}
    \]
\end{lemma}
\begin{beweis}
    Aus \equref{Equ1SatzStirling} folgt:
    \[ \left( \frac{n}{2} \right) !
           \geq
       \sqrt{\pi n}
       \left( \frac{ \frac{n}{2} }{\MathE} \right)^{\frac{n}{2}}
       \MathE^{\frac{1}{6n+1}}
    \]
    Also gilt:
    \begin{eqnarray}
    \label{EquStirling1Schaetzung}
       \frac{n!}{\left( \frac{n}{2}! \right)^2 }
           & \leq &
       \frac{
            \sqrt{2\pi n}
            \left( \frac{n}{\MathE} \right)^n
            \MathE^{\frac{1}{12n}}
       }{
            \left(
                \sqrt{\pi n}
                \left( \frac{ \frac{n}{2} }{\MathE} \right)^{\frac{n}{2}}
                \MathE^{\frac{1}{6n+1}}
            \right)^2
       }
    \\ \label{EquStirling2Schaetzung}
           & = &
       \frac{
           \sqrt{2}
       }{
           \sqrt{\pi n} \left( \frac{1}{2} \right)^n
       }
       \MathE^{ \frac{ 1-18n }{ 12n(6n+1) } }
    \\ \nonumber
        & \leq & \frac{2^n}{ \sqrt{ \pi \frac{n}{2} }  }
    \end{eqnarray}
    Bildet man nun auf beiden Seiten der Ungleichungskette den
    Logarithmus, erh"alt man mit Hilfe einiger Logarithmengesetze
    \[ \log{ n \choose \frac{n}{2} }
           \leq
       n - \frac{1}{2} \log\left( \frac{n}{2} \right) -
                                                   \frac{1}{2} \log(\pi)
           \leq
       n - \frac{ \log \left( \frac{n}{2} \right) + 1 }{2}
    \]
\end{beweis}

% **************************************************************************

\MySection{Der Entwicklungssatz von Laplace}
\label{SecLaplace}

F"ur den ersten im Kapitel \ref{ChapCsanky} darzustellenden Algorithmus
wird ein bekannter Satz verwendet. Er wird in diesem Unterkapitel zusammen 
mit einer h"aufig benutzten Folgerung dargestellt.

\begin{satz}[Entwicklungssatz von Laplace]
\label{SatzLaplace}
\index{Laplace!Entwicklungssatz von}
    Sei $k$ eine nat"urliche Zahl mit
    \[ 1 \leq k \leq n-1 \] Sei $D_n$ die Determinante der Matrix $A$.

    F"ur die nat"urliche Zahl $i$ gelte \[ 1 \leq i \leq {n \choose k} \]
    Sei $D_k^{(2i-1)}$ die Determinante einer Untermatrix $A_k^{(2i-1)}$ von
    $A$, die aus $k$ Spalten der ersten $k$ Zeilen gebildet werde.
    von
    $A$, die aus den f"ur $A_k^{(2i-1)}$ nicht gew"ahlten $n-k$ Zeilen und
    Spalten gebildet werde.

    F"ur jedes $i$ werde eine andere der \[ {n \choose k} \] m"oglichen
    Auswahlen f"ur die $k$ Spalten f"ur $A^{(2i-1)}$ getroffen.

    F"ur eine Untermatrix $A_k^{(2i-1)}$ bezeichne \[ f(A_k^{(2i-1)}) \] die
    $n$-Permutation, die die Spalten von $A$ so vertauscht, da"s
    $A_k^{(2i-1)}$ aus den ersten $k$ und $A_{n-k}^{(2i)}$
    aus den weiteren $n-k$ Zeilen und Spalten von $A$ besteht.

    Dann gilt:
    \begin{equation}
    \label{EquSatzLaplace}
    D_n = \sig(f(A_k^1)) D_k^1 D_{n-k}^2 + \sig(f(A_k^3)) D_k^3 D_{n-k}^4
          + \cdots
          + \sig \left( f \left( A_k^{2{n \choose k}-1} \right) \right)
            D_k^{2{n \choose k}-1} D_k^{2{n \choose k}}
    \end{equation}
\end{satz}
\begin{beweis}
    (vergl. \cite{Csan74} Seite 21)
    Es ist zu zeigen, da"s die Berechnung der Determinante nach
    \equref{EquSatzLaplace} mit der Berechnung nach \equref{EquDet}
    "ubereinstimmt.

    Die einzelnen Determinanten in \equref{EquSatzLaplace} werden auf
    die in \equref{EquDet} angegebene Weise berechnet.
    Dazu ist zu beachten, da"s $\permut_n$ genau $n!$
    Elemente besitzt.
    Es gen"ugt zu zeigen, da"s in \equref{EquSatzLaplace}
    ebenso viele Permutationen auf die Indizes von $1$ bis $n$ angewendet
    werden und alle voneinander verschieden sind.

    Berechnet man einen Summanden
    \[ \sig\lb f\lb A_k^{(2i-1)}\rb\rb D_k^{(2i-1)} D_{n-k}^{(2i)} \]
    in \equref{EquSatzLaplace} anhand von \equref{EquDet},
    kann man die zwei Summen durch Ausmultiplizieren zu einer
    zusammenfassen. Den $k!$ Permutationen zur Berechnung von $D_k^{(2i-1)}$
    und den $(n-k)!$ Permutationen zur Berechnung von $D_{n-k}^{(2i)}$
    entsprechen \[ k! \, (n-k)! \] $n$-Permutationen zur Berechnung der
    zusammengefa"sten Summen f"ur diesen einen Summanden. Die Menge dieser
    $n$-Permutationen wird mit $M_i$ bezeichnet.

    Die $k$-Permutationen bzw. $(n-k)$-Permutationen zur Berechnung aller
    Determinanten $D_k$ und $D_{n-k}$ werden jeweils auf
    verschiedene Mengen von Indizes angewendet, d. h. f"ur zwei beliebige
    dieser Mengen von Indizes gibt es in der einen mindestens einen Index,
    der in der anderen nicht enthalten ist. Das bedeutet, da"s alle Mengen
    $M_i$ voneinander verschieden sind.

    Die Anzahl der Mengen $M_i$, die man auf diese Weise f"ur
    \equref{EquSatzLaplace} erh"alt, betr"agt \[ {n \choose k} \]
    Ihre Vereinigung ergibt eine Menge mit 
    \[ {n \choose k} k! \, (n-k)! \]
    also \[ n! \] Elementen, was zu zeigen war.
\end{beweis}

Aus diesem Satz erh"alt man mit $k=1$:

\begin{korollar}[Zeilen- und Spaltenentwicklung]
\label{SatzEntw}
\index{Zeilenentwicklung}
\index{Spaltenentwicklung}
\index{Entwicklung! nach einer Zeile oder Spalte}
     Seien \[ 1 \leq p \leq n \] und \[ 1 \leq q \leq n \] beliebig.
     Dann gilt die {\em Entwicklung nach Zeile $p$}
     \[ \det(A)= \sum_{j=1}^n (-1)^{p+j} a_{p,j} \det(A_{(p|j)}) \]
     und die {\em Entwicklung nach Spalte $q$}
     \[ \det(A)= \sum_{i=1}^n (-1)^{i+q} a_{i,q} \det(A_{(i|q)}) \]
\end{korollar}

% **************************************************************************

\MySection{Determinantenberechnung durch 'Divide and Conquer'}
\label{SecDivCon}
\index{Divide and Conquer}
\index{Csanky!Algorithmus von}
\index{Algorithmus!von Csanky}

Der Algorithmus (\cite{Csan74} S. 21 ff.), der hier betrachtet wird,
berechnet die Determinante
rekursiv mit Hilfe der Methode {\em Divide and Conquer}, d. h. durch
die Berechnung der Determinanten von Untermatrizen der gegebenen Matrix.

Da es sich hierbei nur um ein einleitendes Beispiel handelt, wird zur
Vereinfachung angenommen,
da"s die Anzahl der Zeilen und Spalten $n$ der
Matrix eine Zweierpotenz ist. Falls dies nicht der Fall ist, wird sie um
entsprechend viele Zeilen und Spalten erweitert, so da"s die neuen
Elemente der Hauptdiagonalen jeweils gleich $1$ und alle weiteren
neuen Elemente jeweils gleich $0$ sind.

Zur rekursiven Berechnung der Determinate wird Satz \ref{SatzLaplace}
benutzt. Man w"ahlt
\[ k := \frac{1}{2}n \MyPunkt \]
Somit gilt auch \[ n-k = \frac{1}{2}n \MyPunkt \]

Die Anzahl der Schritte, die der Algorithmus ben"otigt, um mit Hilfe 
dieses Satzes die Determinante einer $n \times n$-Matrix zu berechnen, 
wird mit \[ s(n) \] bezeichnet. Die Anzahl der Prozessoren, die dabei
besch"aftigt werden, wird mit \[ p(n) \] bezeichnet.

Bei der Berechnung wird Gleichung
\equref{EquSatzLaplace} rekursiv ausgewertet. Das f"uhrt dazu, da"s
\[ \MyChoose{n}{ \frac{n}{2} } \] Determinanten von Untermatrizen zu
berechnen sind. Die Berechnung einer Determinanten erfordert
\[ s\left( \frac{n}{2} \right) \] Schritte und
\[ 2 \MyChoose{n}{ \frac{n}{2} } p\left( \frac{n}{2} \right) \] Prozessoren.

Aus \ref{SatzAlgRechnen} folgt, da"s die in
\equref{EquSatzLaplace} auftretenden Additionen in
\[ \log \MyChoose{ n }{ \frac{n}{2} } \] Schritten von
\[ \frac{ \MyChoose{ n }{ \frac{n}{2}} }{2} \] Prozessoren erledigt werden
k"onnen. Die Multiplikationen k"onnen in einem Schritt von ebenfalls
\[ \frac{ \MyChoose{ n }{ \frac{n}{2}} }{2} \] Prozessoren durchgef"uhrt
werden. Also gilt f"ur die Anzahl der Schritte, die der Algorithmus 
ben"otigt, folgende Rekursionsgleichung:
\[
   s(n) = \left\{
              \begin{array}{lcr}
                  0 & : & n = 1
              \\ s\left( \frac{n}{2} \right) + 1 +
                 \log \MyChoose{ n }{ \frac{n}{2} }
                   & : & n > 1
              \end{array}
          \right.
\]
Mit \ref{SatzStirlingAnwendung} folgt
\[ s(n)
       \leq
   s\left( \frac{n}{2} \right) + 1 +
   n - \frac{\log \left( \frac{n}{2} \right) + 1}{2}
\]
Das ist "aquivalent zu
\[
   s(n)
       \leq
   s\left( \frac{n}{2} \right) + 1 +
   n - \frac{\log(n)}{2}
\]
Die Aufl"osung der Rekursion ergibt:
\[
   s(n)
       \leq
   \log(n) + \sum_{j=1}^{\log(n)} \frac{n}{j} -
   \frac{1}{2} \sum_{k=1}^{\log(n)} \log\left( \frac{n}{k} \right)
\]
Man kann nun die folgenden Absch"atzungen vornehmen:
\begin{eqnarray*}
   \sum_{j=1}^{\log(n)} \frac{1}{j}
       & = &
   1 + \frac{1}{2} + \frac{1}{3} + \cdots + \frac{1}{\log(n)}
\\
   & \leq & 1 + \int\limits_1^{\log(n)} \frac{1}{x} \, dx
\\
   & = & [ \ln(x) ]_1^{\log(n)} = 1 + \ln(\log(n))
\\ 
   & \leq & 1 + \log(\log(n))
\end{eqnarray*}
und
\[
   \sum_{k=1}^{\log(n)} \log(k)
       \leq
   \sum_{k=1}^{\log(n)} \log(\log(n))
       =
   \log(n)\log(\log(n))
\]
und kommt so in Verbindung mit einigen Logarithmengesetzen auf
\[
   s(n)
       \leq
   \log(n) + n (1 + \lc \log(\log(n)) \rc ) 
   - \frac{1}{2} \log^2(n) + \log(n) \lc \log(\log(n)) \rc
\] % Probe: s(2) \leq 2.5
Also gilt \[ s(n) = O(n\log(\log(n))) \]

F"ur die Anzahl der besch"aftigten Prozessoren gilt
\[ p(n) = \left\{
              \begin{array}{lcr}
                  0 & : & n = 1
              \\  2 & : & n = 2
              \\  \max\left(
                         2 \MyChoose{ n }{ \frac{n}{2} }
                         p\left( \frac{n}{2} \right)
                      ,  \frac{ \MyChoose{ \scriptstyle n }{ \frac{n}{2} } 
                              }{2}
                      \right)
                    & : & \mbox{sonst}
              \end{array}
          \right.
\]
Bei diesem Beispielalgorithmus interessiert uns nur die Gr"o"senordnung
der Anzahl besch"aftigter Prozessoren. Deshalb wird diese Anzahl grob
nach unten abgesch"atzt durch
\[ p(n) > \MyChoose{ n }{ \frac{n}{2} } \]
Mit \ref{SatzStirling2Anwendung} folgt
\[ p(n) > \frac{ 2^n }{ \sqrt{\pi \frac{n}{2}} \: \MathE } \]
Da nach unten abgesch"atzt wurde, folgt aus dieser Ungleichung
\[ p(n) = \Omega\left( \frac{2^n}{\sqrt{n}} \right) \]
Die Anzahl der Prozessoren ist also von exponentieller Gr"o"senordnung.

Es wird sich bei den noch zu betrachtenden Algorithmen zeigen, da"s 
sowohl f"ur die Anzahl der Schritte als auch f"ur die
Anzahl der besch"aftigten Prozessoren deutlich bessere Werte m"oglich sind.

% **************************************************************************

\MySection{Die Linearfaktorendarstellung}

In diesem Unterkapitel werden einige Aussagen behandelt, die sich in
Verbindung mit einer weiteren Darstellungsm"oglichkeit f"ur das 
charakteristische Polynom ergeben. Diese Aussagen werden f"ur den 
Beweis des Satzes von Frame (siehe \ref{SecFrame}) ben"otigt. Literatur
dazu ist bereits in Kapitel \ref{ChapBase} aufgelistet.

Neben der Matrizendarstellung und der Koeffizientendarstellung gibt es 
noch eine dritte f"ur uns wichtige Darstellung f"ur das charakteristische
Polynom, die 
\index{Linearfaktorendarstellung} {\em Linearfaktorendarstellung}. 
Ein Polynom $n$-ten Grades
kann man auch als Produkt von $n$ Linearfaktoren darstellen, so da"s das
charakteristische Polynom folgendes Aussehen bekommt:
\Beq{TermLinearfaktoren}
    (\lambda_1 - \lambda) (\lambda_2 - \lambda) \ldots
    (\lambda_n - \lambda)
\Eeq
Dabei sind die $\lambda_i$ die Eigenwerte der zugrunde liegenden
$n \times n$-Matrix.
Sie sind {\em nicht} paarweise verschieden.
Diese Tatsache f"uhrt uns zum Begriff der
{\em Vielfachheit}. Man kann in der obigen Linearfaktorendarstellung
gleiche Faktoren mit Hilfe von Potenzen beschreiben. Dazu gelte
\[ k \in \Nat, \: 1 \leq k \leq n \MyPunkt \]
Die Eigenwerte seien
\[ \lambda'_1, \lambda'_2, \ldots, \lambda'_k \MyKomma \]
jedoch alle paarweise voneinander verschieden. So bekommt
\equref{TermLinearfaktoren} folgendes Aussehen:
\[ (\lambda'_1 - \lambda)^{m(\lambda'_1)}
   (\lambda'_2 - \lambda)^{m(\lambda'_2)} \ldots
   (\lambda'_k - \lambda)^{m(\lambda'_k)}
\]
\index{Vielfachheit}
In dieser Darstellung wird $m(\lambda'_i)$ mit {\em Vielfachheit} des
Eigenwertes $\lambda'_i$ bezeichnet.

Die $n$ Eigenwerte einer $n \times n$-Matrix besitzen zur Determinante 
und zur Spur jeweils eine wichtige Beziehung, welche in den n"achsten 
beiden S"atzen dargestellt wird:
\begin{satz}
    \[ \det(A)= \prod_{i=1}^{n} \lambda_i \]
\end{satz}
\begin{beweis}
    Die G"ultigkeit der Behauptung wird bei Betrachtung der Berechnung der
    Koeffizientendarstellung des charakteristischen Polynoms aus dessen
    Linearfaktorendarstellung deutlich. Beim Ausmultiplizieren der 
    Linearfaktoren wird der Wert von $c_0$ nur durch das Produkt der 
    Eigenwerte beeinflu"st.
\end{beweis}

\begin{satz}
\label{SatzTrEigenwerte}
    \[ \tr(A)= \sum_{i=1}^{n} \lambda_i \]
\end{satz}
\begin{beweis}
    Betrachtet man die Berechnung der Koeffizientendarstellung des
    charakteristischen Polynoms aus dessen Linearfaktorendarstellung durch
    Ausmultiplizieren, erkennt man, da"s gilt:
    \[ c_{n-1} = (-1)^{n-1} \sum_{i=1}^n \lambda_i \MyPunkt \]
    Mit \ref{SatzDdurchP} folgt daraus die Behauptung.
\end{beweis}

Die Eigenwerte besitzen weiterhin folgende interessante Eigenschaft:
\begin{satz}
\label{SatzEigenPotenz}
    Seien $\lambda_1, \lambda_2, \ldots, \lambda_n$ die Eigenwerte der
    $n \times n$-Matrix $A$. Sei $k \in \Nat$. Dann gilt:
    \[ \lambda_1^k, \lambda_2^k, \ldots, \lambda_n^k \] sind die
    Eigenwerte von $A^k$.
\end{satz}
\begin{beweis}
    Ein Skalar $\lambda$ ist genau dann Eigenwert von $A$, wenn es einen
    Vektor $x \neq 0_n$ gibt, so da"s gilt\footnote{vgl. 
    Erl"auterungen auf S. \pageref{PageEigenMotiv}}
    \Beq{Equ1EigenPotenz}
        Ax = \lambda x \MyPunkt
    \Eeq
    Mit Hilfe dieser Beziehung wird die Behauptung per Induktion
    bewiesen.

    Nach Voraussetzung ist die Behauptung f"ur $k=1$ erf"ullt. Es gelte
    also nun
    \[ \forall \lambda= \lambda_1, \ldots, \lambda_n \exists x : \:
       A^k x = \lambda^k x \MyPunkt
    \]
    Zu zeigen ist, da"s dann auch gilt
    \[ \forall \lambda= \lambda_1, \ldots, \lambda_n \exists x : \:
       A^{k+1} x = \lambda^{k+1} x
    \]
    Diese Gleichung kann man umformen in
    \[ \forall \lambda= \lambda_1, \ldots, \lambda_n \exists x : \:
       A^k \underbrace{A x}_{(*)}= \lambda^k \underbrace{\lambda x}_{(*)}
    \]
    Nach Voraussetzung sind die Terme $(*)$ gleich. Sie werden mit $y$ 
    bezeichnet. Die Gleichung bekommt dann folgendes Aussehen:
    \[ \forall \lambda= \lambda_1, \ldots, \lambda_n \exists y: \:
       A^k y= \lambda^k y
    \]
    Dies ist wiederum nach Voraussetzung richtig.
\end{beweis}

Aus diesem Satz ergibt sich mit Hilfe von \ref{SatzTrEigenwerte}
eine f"ur uns wichtige Beziehung:
\begin{korollar}
\label{SatzTraceLambda}
    \[ \tr(A^k) = \sum_{i=1}^n \lambda_i^k \]
\end{korollar}

% **************************************************************************

\MySection{Die Newton'schen Gleichungen f"ur Potenzsummen}
\label{SecNewtonPotenz}

Mit den {\em Newton'schen Gleichungen f"ur Potenzsummen} werden in diesem
Unterkapitel weitere Grundlagen f"ur den Beweis der S"atze von Frame
(siehe \ref{SecFrame}) behandelt. Die gesamten Hintergr"unde f"ur diese
Gleichungen werden z. B. in \cite{Haup52} (Kapitel 7 und 8) behandelt.

Eine {\em Potenzsumme} ist eine Summe von Potenzen einer oder mehrerer
Unbestimmter. Auf Seite \ref{SatzSumK} sind weitere Beispiele f"ur 
einfachere Potenzsummengleichungen zu finden.

Da uns diese Gleichungen im Zusammenhang mit dem charakteristischen
Polynom einer Matrix interessieren, werden sie anhand dieses Polynoms
entwickelt. Dazu werden folgende Vereinbarungen getroffen:
\begin{itemize}
\item
      Das charakteristische Polynom der $n \times n$-Matrix $A$ sei
      \[
         p(\lambda) := c_n \lambda^n + c_{n-1} \lambda^{n-1} + \cdots
                       + c_1 \lambda + c_0 \MyPunkt 
      \]
\item
      Die erste Ableitung von $p(\lambda)$ wird mit \[ p'(\lambda) \]
      bezeichnet.
\item Die Eigenwerte von $A$ seien
      \[ \lambda_1, \lambda_2, \ldots, \lambda_n \MyPunkt \]
      Es wird definiert
      \[ s_k := \sum_{i=1}^n \lambda_i^k \MyPunkt \]
\end{itemize}

Zun"achst besch"aftigen wir uns mit der 
Polynomdivision. \index{Polynomdivsion}
Sei dazu ein $i$
mit $ 1 \leq i \leq n $ gegeben. Da $\lambda_i$ Eigenwert von $A$ und somit
Nullstelle von $p(\lambda)$ ist, l"a"st sich $p(\lambda)$ ohne Rest durch
\[ \lambda_i - \lambda \] teilen. Das Ergebnis ist ein Polynom vom Grad
$n-1$. Das folgende Lemma liefert eine Aussage "uber dessen
Koeffizienten:

\begin{lemma}
\label{SatzPolynomDiv}
    Gegeben sei die Gleichung:
    \Beq{Equ1SatzPolynomDiv}
        \frac{ p(\lambda) }{ (\lambda_i - \lambda) }
            =
        \hat{c}_{n-1}\lambda^{n-1} +
        \hat{c}_{n-2}\lambda^{n-2} + \ldots +
        \hat{c}_1 \lambda + \hat{c}_0
    \Eeq
    Dann gilt f"ur $1 \leq k \leq n-1$:
    \Beq{Equ2SatzPolynomDiv}
        \hat{c}_k =
        - \sum_{j=1}^{n-k} \lambda_i^{j-1} c_{k+j}
%        = - ( \lambda_i^{n-k-1} c_n + \lambda_i^{n-k-2} c_{n-2} +
%              \cdots + \lambda c_{k+2} + c_{k+1} )
    \Eeq
\end{lemma}
\begin{beweis}
     Multipliziert man beide Seiten von Gleichung
     \equref{Equ1SatzPolynomDiv} mit \[ (\lambda_i - \lambda) \] und
     multipliziert die rechte Seite der so gewonnenen Gleichung aus,
     ergibt sich:
     \begin{eqnarray*}
         \lefteqn{c_n \lambda^n + c_{n-1} \lambda^{n-1} + \cdots
                       + c_1 \lambda + c_0 }
     \\ & = &
         - \hat{c}_{n-1} \lambda^n +
         (\lambda_i \hat{c}_{n-1} - \hat{c}_{n-2}) \lambda^{n-1} +
         (\lambda_i \hat{c}_{n-2} - \hat{c}_{n-3}) \lambda^{n-2} + \cdots +
         (\lambda_i \hat{c}_1 - \hat{c}_0) \lambda +
         \lambda_i \hat{c}_0
     \end{eqnarray*}
     Setzt man \[ \hat{c}_n = \hat{c}_{-1} = 0 \MyKomma \]
     erh"alt man durch Koeffizientenvergleich:
     \begin{eqnarray}
        \nonumber
            & \forall 1 \leq l \leq n :
            c_k = \lambda_i \hat{c}_k - \hat{c}_{k-1}
     \\ \label{Equ3SatzPolynomDiv}
            \Leftrightarrow &
                \forall 1 \leq l \leq n :
                \hat{c}_{l-1} = \lambda_i \hat{c}_l - c_l
     \end{eqnarray}
     Gleichung \equref{Equ2SatzPolynomDiv} wird durch
     Induktion\footnote{Um die Art und Weise, wie die Koeffizienten der
     Polynome indiziert sind, einheitlich zu halten, verl"auft die
     Induktion etwas ungewohnt.} bewiesen.
     F"ur $k = n-1$ folgt die G"ultigkeit von \equref{Equ2SatzPolynomDiv} 
     aus \equref{Equ3SatzPolynomDiv}. 
     
     Gelte \equref{Equ2SatzPolynomDiv} also nun f"ur $k = l$.
     Es ist zu zeigen, da"s die Gleichung dann auch f"ur $k = l-1$ gilt:
     \[
        \hat{c}_{l-1} = - \sum_{j=1}^{n-(l-1)} \lambda_i^{j-1} c_{l-1+j}
     \]
     Zieht man $c_l$ aus der Summe heraus und indiziert neu, erh"alt man:
     \[
        \hat{c}_{l-1} = - c_l - \sum_{j=1}^{n-l} \lambda_i^{j} c_{l+j}
     \]
     Zieht man nun noch $\lambda_i$ aus der Summe heraus, erh"alt man:
     \[
        \hat{c}_{l-1} = - c_l - \lambda_i 
                                   \sum_{j=1}^{n-l} \lambda_i^{j-1} c_{l+j}
     \]
     Nach Induktionsvoraussetzung ist dies gleichbedeutend mit:
     \[
        \hat{c}_{l-1} = - c_l + \lambda_i \hat{c}_l
     \]
     Die G"ultigkeit dieser Gleichung folgt aus \equref{Equ3SatzPolynomDiv}.
\end{beweis}

Eine an dieser Stelle wichtige Regel f"ur das Differenzieren 
lautet (\cite{Haup52} S. 160):

\begin{bemerkung}
\label{SatzProdDiff}
\index{Differenzierung!von Produkten}
    Seien \[ g_1(x), g_2(x), \ldots, g_n(x) \]
    auf einem Intervall $I$ differenzierbare Funktionen. Es gelte 
    \[ f(x) = \prod_{i=1}^n g_i(x) \MyPunkt \]
    Dann ist auch $f(x)$ auf $I$ differenzierbar mit
    \[ f'(x) = \sum_{i=1}^n g_1(x) g_2(x) \cdots 
                            g_{i-1}(x) g_i'(x) g_{i+1}(x) \cdots
                            g_{n-1}(x) g_n(x) \MyPunkt
    \]
\end{bemerkung}

Falls gilt \[ \forall x \in I, 1 \leq i \leq n: \: g_i(x) \neq 0 \MyKomma \]
l"a"st sich diese Regel auch einfacher formulieren:
\[ f'(x) = \sum_{i=1}^n \frac{f(x)}{g_i(x)} g_i'(x) \]

Betrachtet man das charakteristische Polynom $p(\lambda)$ in 
Linearfaktorendarstellung und beachtet, da"s die Ableitung von
\[ (\lambda_i - \lambda) \] $-1$ ergibt, dann erh"alt man mit Hilfe von 
\ref{SatzProdDiff}:

\begin{korollar}
\label{SatzCharPolyDiff}
    \Beq{EquCharPolyDiff} 
        p'(\lambda)= - \sum_{i=1}^n 
                               \frac{ p(\lambda) }{ (\lambda_i - \lambda) }
    \Eeq
\end{korollar}

Mit Hilfe von \ref{SatzPolynomDiv} und \ref{SatzCharPolyDiff} erh"alt man
nun die gesuchten Gleichungen (\cite{Haup52} S. 181):

\begin{satz}[Newton'sche Gleichungen f"ur Potenzsummen]
\label{SatzNewtonPotenz}
\index{Newton!Gleichungen von}
    \hfill \mbox{\hspace{1cm}} \\ 
    F"ur die Koeffizienten des charakteristischen Polynoms 
    gilt\footnote{Die Terme $s_i$ sind am Beginn des Unterkapitels 
    definiert.}:
    \[
       \forall 0 \leq k \leq n-1 : \:
       - (n - (k+1)
         ) c_{k+1}
       - \sum_{j=2}^{n-k} s_{j-1} c_{k+j}
            = 0
    \]
\end{satz}
\begin{beweis}
    Die $\hat{c}_k$ aus Lemma \ref{SatzPolynomDiv} sind abh"angig vom
    gew"ahlten $\lambda_i$. Deshalb definieren wir diese Koeffizienten
    als Funktionen von $\lambda_i$:
    \[ \hat{c}_k(\lambda_i) :=
           - \sum_{j=1}^{n-k} \lambda_i^{j-1} c_{k+j} \MyPunkt
    \]

    Dann folgt aus Lemma \ref{SatzPolynomDiv}:
    \Beq{Equ4SatzNewtonPotenz}
       \sum_{i=1}^n \hat{c}_k(\lambda_i)
           = \overbrace{- \sum_{j=1}^{n-k} s_{j-1} c_{k+j}
                       }^{ \dot{c}_k := } \MyPunkt
    \Eeq
    "Ubertr"agt man diese Beziehung zwischen den Koeffizienten auf die
    entsprechenden Polynome erh"alt man:
    \Beq{Equ1SatzNewtonPotenz}
       \sum_{i=1}^n \frac{ p(\lambda) }{ \lambda_i - \lambda }
           =
       \sum_{k=0}^{n-1} \dot{c}_k \lambda^k
    \Eeq
    Die erste Ableitung des charakteristischen Polynoms in der
    Koeffizientendarstellung sieht folgenderma"sen aus:
    \Beq{Equ2SatzNewtonPotenz}
       p'(\lambda) = n c_n \lambda^{n-1} + (n-1) c_{n-1} \lambda^{n-2}
                     + \cdots +
                     2 c_2 \lambda + c_1
    \Eeq
    Aus den drei Gleichungen \equref{EquCharPolyDiff},
    \equref{Equ1SatzNewtonPotenz} und \equref{Equ2SatzNewtonPotenz} folgt:
    \[ - \sum_{k=0}^{n-1} \dot{c}_k \lambda^k
           =
       n c_n \lambda^{n-1} + (n-1) c_{n-1} \lambda^{n-2}
           + \cdots +
       2 c_2 \lambda + c_1
    \]
    Durch Koeffizientenvergleich erh"alt man aus dieser Gleichung:
    \[ 
        \forall 0 \leq k \leq n-1 : \: - \dot{c}_k = (k+1) c_{k+1} 
    \]
    Setzt man f"ur $\dot{c}_j$ den entsprechenden Term aus 
    Gleichung \equref{Equ4SatzNewtonPotenz} ein, ergibt sich:
    \begin{MyEqnArray}
       \MT
       \forall 0 \leq k \leq n-1 : \: 
         \sum_{j=1}^{n-k} s_{j-1} c_{k+j}
           \MT = \MT 
       (k+1) c_{k+1}
    \MNl \Leftrightarrow \MT
       \forall 0 \leq k \leq n-1 : \: 
         n c_{k+1} + \sum_{j=2}^{n-k} s_{j-1} c_{k+j}
           & \DS = & \DS
       (k+1) c_{k+1}
    \MNl \Leftrightarrow \MT 
       \forall 0 \leq k \leq n-1 : \: 
         (n-(k+1)) c_{k+1} + \sum_{j=2}^{n-k} s_{j-1} c_{k+j}
           \MT = \MT 0 
    \end{MyEqnArray}
\end{beweis}

% **************************************************************************

\MySection{Die Adjunkte einer Matrix}
\label{SecAdj}

Bei den Beweisen des Satzes von Frame in Unterkapitel \ref{SecFrame} 
spielt die Adjunkte einer Matrix eine bedeutende Rolle und wird
deshalb hier behandelt.

\MyBeginDef
\label{DefAdj}
\index{Adjunkte}
    Sei $A$ eine $n \times n$-Matrix.
    Erh"alt man die Matrix $B$ aus der Matrix $A$ nach
    \[
        b_{i,j} := (-1)^{i+j} \det(A_{(j|i)}) \MyKomma
    \] so hei"st $B$ {\em Adjunkte der Matrix $A$}. Die Adjunkte wird mit
    \[ \adj(A) \] bezeichnet.
\MyEndDef

Zum Beweis einer uns besonders interessierenden Eigenschaft der Adjunkten
ben"otigen wir zun"achst noch ein Lemma. Es
behandelt den Fall einer Zeilen- bzw. Spaltenentwicklung\footnote{ vgl. 
\ref{SatzEntw} }, bei der jedoch
als Faktoren f"ur die Unterdeterminanten die Matrizenelemente nicht aus
der Zeile bzw. Spalte entnommen werden, nach der die Determinante 
entwickelt wird:

\begin{lemma}
\label{SatzFalscheEntw}
    Seien \[ 1 \leq p,p' \leq n \] und \[ 1 \leq q,q' \leq n \] mit
    \[ p \neq p' \] und \[ q \neq q' \] Dann gilt:
    \[ \sum_{j=1}^n (-1)^{p+j} a_{p',j} \det(A_{(p|j)}) = 0 \] und
    \[ \sum_{i=1}^n (-1)^{i+q} a_{i,q'} \det(A_{(i|q)}) = 0 \]
\end{lemma}
\begin{beweis}
    Betrachtet man die Berechnung der Unterdeterminanten in den obigen
    Gleichungen nach \equref{EquDet}, erkennt man beim
    Vergleich mit der Berechnung der Determinante einer Matrix, die
    zwei gleiche Zeilen enth"alt, da"s die Terme beider Berechnungen 
    nach einigen Vereinfachungen "ubereinstimmen. Nach Satz
    \ref{SatzDetPermut} ist die Determinante in diesem Fall gleich $0$.
\end{beweis}

Mit Hilfe von \ref{SatzFalscheEntw} erhalten wir nun:

\begin{satz}
\label{SatzAdj}
    \begin{eqnarray}
        \label{EquSatzAdj1}
        A * \adj(A) = E_n * \det(A) 
     \\ \label{EquSatzAdj2}
        \adj(A) * A = E_n * \det(A)
    \end{eqnarray}
\end{satz}
\begin{beweis}
    Spalte $k$ der Matrix $\adj(A)$ sieht so aus:
    \[ 
        \left[
        \begin{array}{c}
            (-1)^{1+k} \det(A_{(k|1)}) 
         \\ (-1)^{2+k} \det(A_{(k|2)})
         \\ \vdots
         \\ (-1)^{n+k} \det(A_{(k|n)})
        \end{array} 
        \right]
    \]
    Das Element an der Stelle $(i,k)$ der Produktmatrix \[ A * \adj(A) \]
    ist also gleich \[ \sum_{j=1}^n a_{i,j} (-1)^{j+k} A_{(k|j)} \]
    Dies ist nach \ref{SatzEntw} gleich $\det(A)$ f"ur \[ i = k \] und
    nach \ref{SatzFalscheEntw} gleich $0$ f"ur \[ i \neq k \]
    Daraus folgt die G"ultigkeit von \equref{EquSatzAdj1}. 
    Die Argumentation
    f"ur \equref{EquSatzAdj2} verl"auft analog.
\end{beweis}

% **************************************************************************

\MySection{Der Satz von Frame}
\label{SecFrame}

In diesem Unterkapitel wird eine Methode von J. S. Frame (\cite{Fram49};
\cite{Dwye51} S. 225-235) vorgestellt\footnote{ Die
Originalver"offentlichung \cite{Fram49} enth"alt keinen Beweis. Dieser
Beweis ist schwer zu bekommen. Er wird hier deshalb
frei nachvollzogen und d"urfte sich vom Original nicht wesentlich 
unterscheiden. In diesem Zusammenhang m"ochte ich mich bei R. T. Bumby 
f"ur seinen Hinweis auf die Newton'schen Gleichungen f"ur Potenzsummen
bedanken.} , die es u. a. erlaubt, die
Determinante einer Matrix zu berechnen. Diese Methode ist im wesentlichen
eine Neuentdeckung der 
Methode von Leverrier \index{Leverrier!Methode von}
aus dem 19. Jahrhundert zur 
Bestimmung der Koeffizienten des charakteristischen Polynoms 
(z. B. \cite{Hous64} S. 166 ff. ). Die Darstellung wird hier auf die Teile 
beschr"ankt, die f"ur die Determinantenberechnung wichtig sind.

Die Adjunkte von
\Beq{TermFrameAminusLambda} 
    A - \lambda E_n
\Eeq 
besteht aus lauter Determinanten von $(n-1) \times (n-1)$-Matrizen. Sie
kann deshalb durch ein Polynom vom Grad $n-1$ dargestellt werden
(siehe dazu auch \ref{DefCharPoly} und \ref{DefAdj}).
Dies motiviert die folgende Vereinbarung zus"atzlich zu den in
\ref{SecNewtonPotenz} aufgef"uhrten Bezeichnungen:
\begin{quote}
    Seien \[ B_i \: , \: 0 \leq i \leq n-1 \] geeignet gew"ahlte 
    $n \times n$-Matrizen.
    Dann bezeichnet
    \[
       c(\lambda) := B_{n-1} \lambda^{n-1} + B_{n-2} \lambda^{n-2} + 
                     \cdots + B_2 \lambda^2 + B_1 \lambda + B_0 \MyPunkt
    \] die Adjunkte von \equref{TermFrameAminusLambda}.
\end{quote}

\begin{lemma}
\label{SatzFrame1}
    \begin{eqnarray*}
       B_{n-1} & = & (-1) E_n
    \\ \forall n-2 \geq i \geq 0 : \:
       B_{i} & = & A B_{i+1} - c_{i+1} E_n
    \end{eqnarray*}
\end{lemma}
\begin{beweis}
    Aus Satz \ref{SatzAdj} in Verbindung mit Definition \ref{DefCharPoly}
    folgt:
    \[ (A - \lambda E_n) c(\lambda) = p(\lambda) E_n \]
    Setzt man die Koeffizientendarstellungen von $c(\lambda)$ und
    $p(\lambda)$ in diese Gleichung ein, erh"alt man
    \[
    \begin{array}{p{1.8em}p{3.6em}l}
    & \multicolumn{2}{l}{
          \DS (A - \lambda E_n) (B_{n-1} \lambda^{n-1} + 
               B_{n-2} \lambda^{n-2} + \cdots + 
               B_2 \lambda^2 + B_1 \lambda + B_0) \MatStrut
      } 
    \\ & & \DS \MatStrut
        = (c_n \lambda^n + c_{n-1} \lambda^{n-1} + \cdots
                       + c_2 \lambda^2 + c_1 \lambda + c_0 ) E_n
    \\ \mbox{ $\DS \Leftrightarrow$ } & 
       \multicolumn{2}{l}{ \MatStrut
           \DS AB_{n-1} \lambda^{n-1} + AB_{n-2} \lambda^{n-2} +
                       \cdots + AB_2 \lambda^2 + AB_1 \lambda + AB_0
       }
    \\ &
       \multicolumn{2}{l}{ \MatStrut
           \DS - B_{n-1} \lambda^{n} - B_{n-2} \lambda^{n-1} -
                       \cdots - B_2 \lambda^3 - B_1 \lambda^2 - B_0 \lambda
       }
    \\ & & \DS \MatStrut
       = (c_n \lambda^n + c_{n-1} \lambda^{n-1} + \cdots
                   + c_2 \lambda^2 + c_1 \lambda + c_0 ) E_n
    \\ \mbox{ $\DS \Leftrightarrow$ } & 
       \multicolumn{2}{l}{ \MatStrut
           \DS - B_{n-1} \lambda^{n} + (AB_{n-1} - B_{n-2}) \lambda^{n-1} +
               (AB_{n-2} - B_{n-3}) \lambda^{n-2} + \cdots 
       }
    \\ &
       \multicolumn{2}{l}{ \MatStrut
           \DS + (AB_2 - B_1) \lambda^2 + (AB_1 - B_0) \lambda
       }
    \\ & & \DS \MatStrut
       = (c_n \lambda^n + c_{n-1} \lambda^{n-1} + \cdots
                         + c_2 \lambda^2 + c_1 \lambda + c_0 ) E_n
    \end{array}
    \]
    Koeffizientenvergleich ergibt die Behauptung.
\end{beweis}

\begin{lemma}
\label{SatzFrame2}
    Es gilt\footnote{Da in dieser Arbeit verschiedene Arten von hoch und
    tiefgestellten Indizes und Markierungen verwendet werden, sei hiermit
    exiplizit darauf hingewiesen, da"s mit $p'(\lambda)$ ,wie allgemein
    "ublich, die erste Ableitung von $p(\lambda)$ gemeint ist.}
    \[ p'(\lambda) = - \tr(c(\lambda)) \MyPunkt \]
\end{lemma}
\begin{beweis}
    Zu zeigen ist:
    \[ \sum_{j=1}^n j c_j \lambda^{j-1} = 
       - \tr\left( \sum_{j=1}^{n} B_{j-1} \lambda^{j-1} \right)
    \]
    Durch Koeffizientenvergleich erh"alt man:
    \[ 
       \forall 1 \leq j \leq n: \: j c_j = - \tr(B_{j-1}) 
    \]
    Durch wiederholte Anwendung von \ref{SatzFrame1} ergibt sich daraus:
    \begin{eqnarray*}
       j c_j & = & - \tr(A B_{j} - c_{j} E_n)
    \\ \Leftrightarrow
       (n-j) c_j & = & \tr(A B_j)
    \\ \Leftrightarrow
       (n-j) c_j & = & \tr(A (A B_{j+1} - c_{j+1} E_n))
    \\ \Leftrightarrow
       (n-j) c_j & = & \tr(A^{n-j}B_{n-1}) 
                       - \sum_{k=1}^{n-j-1} \tr(A^k) c_{j+k}
    \\ \Leftrightarrow
       (n-j) c_j & = & - \tr(A^{n-j}) 
                       - \sum_{k=1}^{n-j-1} \tr(A^k) c_{j+k}
    \\ \Leftrightarrow
       (n-j) c_j & = & - \tr(A^{n-j}) 
                       - \sum_{k=1}^{n-j-1} \tr(A^k) c_{j+k}
    \end{eqnarray*}
    Nach \ref{SatzTraceLambda} ist dies gleichbedeutend mit
    \[
       (n-j) c_j + s_{n-j}
                   + \sum_{k=1}^{n-j-1} s_k c_{j+k} = 0
    \]
    Da f"ur das charakteristische Polynom $c_n=1$ gilt, ist diese 
    Gleichung nach Satz \ref{SatzNewtonPotenz} richtig.
    \\ \hspace{10em} % damit das Viereck auf der rechten Seite steht
\end{beweis}

\begin{lemma}
\label{SatzFrame3}
    \[ \forall 1 \leq i \leq n: \: c_i = \frac{1}{n-i} \tr(A B_i) \]
\end{lemma}
\begin{beweis}
    Wie in \ref{SatzFrame1} folgt zun"achst aus
    \ref{SatzAdj} in Verbindung mit Definition \ref{DefCharPoly}:
    \begin{MyEqnArray}
       \MT (A - \lambda E_n) c(\lambda) \MT = \MT p(\lambda) E_n 
    \MNl \Leftrightarrow \MT
        A c(\lambda) - \lambda c(\lambda) \MT = \MT p(\lambda) E_n
    \MNl \Leftrightarrow \MT
       \tr(A c(\lambda)) \MT = \MT 
       \tr(\lambda c(\lambda)) + \tr(p(\lambda) E_n) \MatStrut
    \end{MyEqnArray}
    Mit Hilfe von \ref{SatzFrame2} folgt:
    \[
    \begin{array}{p{1.8em}rcl}
       & \multicolumn{3}{l}{ 
             \DS \tr(A c(\lambda)) \mbox{ $\DS =$ }
             \DS n p(\lambda) - \lambda p'(\lambda) 
         }
    \\ \mbox{ $\DS \Leftrightarrow$ } &
       \multicolumn{3}{l}{ \MatStrut
          \DS \tr(AB_{n-1} \lambda^{n-1} + AB_{n-2} \lambda^{n-2} + 
                       \cdots + AB_2 \lambda^2 + AB_1 \lambda + AB_0)
       }
    \\ & \mbox{ $\DS = $ } & \MatStrut
        \DS n ( \lambda^n c_n + c_{n-1} \lambda^{n-1} + \cdots
                       + c_2 \lambda^2 + c_1 \lambda + c_0)
    \\ & & \MatStrut
        \DS - (n \lambda^n c_n + (n-1) c_{n-1} \lambda^{n-1} + \cdots
                       + 2 c_2 \lambda^2 + c_1 \lambda )
    \end{array}
    \]
    Koeffizientenvergleich ergibt
    \[ \forall 1 \leq i \leq n: \: \tr(AB_i) = n c_i - i c_i \MyPunkt \]
\end{beweis}

\begin{satz}[Frame]
\label{SatzFrame}
\index{Frame!Satz von}
    \Beq{Equ3SatzFrame}
        \det(A)= \frac{ \tr(A B_0) }{ n }
    \Eeq
\end{satz}
\begin{beweis}
    Man erh"alt die Behauptung aus \ref{SatzFrame1}, \ref{SatzFrame3} und
    \ref{SatzDdurchP}.
\end{beweis}

% **************************************************************************

\MySection{Determinantenberechnung mit Hilfe des Satzes von Frame}
\label{SecAlgFrame}
\index{Algorithmus!von Csanky}
\index{Csanky!Algorithmus von}

In diesem Unterkapitel wird eine effiziente Methode zur parallelen
Determinantenberechnung \cite{Csan76} vorgestellt (abgek"urzt mit 
{\em C-Alg.};
vgl. Unterkapitel \ref{SecBez}). Sie benutzt Divisionen und kann deshalb
nur angewendet werden, wenn die Berechnungen in einem K"orper stattfinden.
Dies ist problematisch,
weil in realen Rechnern nur mit begrenzter Genauigkeit gearbeitet werden 
kann und somit immer auf die eine oder andere Weise modulo gerechnet wird. 
Z. B. besitzt 6 im Ring $ \Integers_8 $ kein multiplikatives Inverses.

Dies motiviert den Entwurf von Algorithmen, die ohne Divisionen
auskommen, und somit auch in Ringen anwendbar sind, wie BGH-Alg. und
B-Alg. . P-Alg. benutzt wie C-Alg. ebenfalls Divisionen.

Die Determinantenberechnung erfolgt in dem Algorithmus, der in diesem
Unterkapitel vorgestellt wird, mit Hilfe des Satzes von Frame
(Satz \ref{SatzFrame}). Der Satz nutzt die bereits in \ref{SatzDdurchP}
erw"ahnte Tatsache aus, da"s sich unter den Koeffizienten des
charakteristischen Polynoms auch die Determiante befindet. Diese Eigenschaft
des charakteristischen Polynoms wird auch in B-Alg. und P-Alg. in
Verbindung mit anderen Verfahren zur Bestimmung der Koeffizienten
verwendet.

Die Berechnung der Determinante nach Satz \ref{SatzFrame} erfolgt mit Hilfe
einer Rekursionsgleichung. F"ur eine effiziente parallele Berechnung ist
dies nicht befriedigend. Deshalb ist sind einige Umformungen \cite{Csan76}
erforderlich. F"ur diese Umformungen wird ein Operator ben"otigt. Dazu seien
$M$ und $N$ jeweils $n \times n$-Matrizen:

\MyBeginDef
\index{Spuroperator}
\label{DefSpurOp}
    Der Operator $T$ wird definiert durch: \[ {T}{N} := \tr(N) \]
    Er wird {\em Spuroperator} genannt.
\MyEndDef

Es gilt also \[ (E + MT)N = N + {M}{T}{N} = N + M \tr(N) \MyPunkt \]

F"ur die Determinantenberechnung nach Satz \ref{SatzFrame} ist die dort
auftretende Matrix $B_0$ zu berechnen. Mit Hilfe der Lemmata
\ref{SatzFrame1} und \ref{SatzFrame2} sowie des soeben definierten
Operators $T$ erh"alt diese Berechnung folgendes Aussehen:
\begin{eqnarray*}
   B_0 & = & A B_1 - c_1 E_n
\\     & = & A B_1 - \frac{E_n}{n-1} \tr(A B_1)
\\     & = & \left( E_n - \frac{E_n}{n-1} T \right) A B_1
\\     & \vdots & 
\\     & = & \left( E_n - \frac{E_n}{n-1}T \right)
             \left\{A
                 \left[
                     \left(E_n - \frac{E_n}{n-2}T \right)
                     \left\{A
                         \left[
                             \cdots
                             (E_n - {E_n}{T})
                             \{A[E_n]\}
                             \cdots
                         \right]
                     \right\}
                 \right]
             \right\}
\end{eqnarray*}
Mit Hilfe der Assoziativit"at der Matrizenmultiplikation erh"alt man:
\Beq{Equ1SatzCsanky}
    B_0 = \left(
              \underbrace{
                  A - \overbrace{ \frac{E}{n-1} }^{\mbox{Term 1}}
                  \overbrace{ {T}{A} }^{\mbox{Term 2}}
              }_{\mbox{Term 3}}
          \right)
          \left(A - \frac{E}{n-2} {T}{A} \right)
          \cdots
          \left(A - \frac{E}{2} {T}{A} \right)
          (A - {E}{T}{A})
\end{equation}
Da $E_n$ nur in der Hauptdiagonalen von $0$ verschiedene Elemente
besitzt,
l"a"st sich Term 1 in einem Schritt von \[ n \] Prozessoren berechnen.
Parallel dazu l"a"st sich Term 2 nach Satz \ref{SatzAlgRechnen} in
\[ \left\lceil \log(n) \right\rceil \] Schritten von
\[ \left\lfloor \frac{n}{2} \right\rfloor \] Prozessoren berechnen.

Anschlie"send ist die Ergebnismatrix von Term 1 mit dem Ergebnis von
Term 2 zu multiplizieren. Dies kann, wie bei der Berechnung von Term 1
in einem Schritt von \[ n \] Prozessoren durchgef"uhrt werden. Die
darauf folgende Matrizensubtraktion zur Berechnung von Term 3 kann in
einem Schritt von \[ n^2 \] Prozessoren erledigt werden.

Insgesamt kann Term 3 also in
\[ \left\lceil \log(n) \right\rceil + 2 \] Schritten von
\[ n^2 \] Prozessoren berechnet werden.

Zur Berechnung von $B_0$ sind $n$ Terme auf die gleiche Weise wie
Term 3 zu berechnen. Term 2 braucht f"ur all diese Terme nur einmal
berechnet zu werden. Insgesamt kann die Berechnung der $n$ Terme in
\[ \left\lceil \log(n) \right\rceil + 2 \] Schritten von
\[ n^3 \] Prozessoren erledigt werden.

Um das Endergebnis $B_0$ zu erhalten sind schlie"slich noch die
Ergebnismatrizen der $n$ Terme miteinander zu multiplizieren. Die
Anzahl der Schritte und Prozessoren daf"ur folgt aus den S"atzen
\ref{SatzAlgBinaerbaum} und \ref{SatzAlgMatMult}.
Zu beachten ist dabei,
da"s eine Verkn"upfung nicht in einem Schritt von
einem Prozessor durchgef"uhrt wird, sondern nach \ref{SatzAlgMatMult} in
\[ \lceil \log(n) \rceil + 1 \] Schritten von \[ n^3 \] Prozessoren.
Deshalb werden diese Matrizenmultiplikationen in
\[ (\lceil \log(n) \rceil + 1) \lceil \log(n) \rceil \] Schritten von
\[ n^3 \left\lfloor \frac{n}{2} \right\rfloor \] Prozessoren
durchgef"uhrt.

Insgesamt wird die Berechnung von $B_0$ also in
\[ \lceil \log(n) \rceil^2 + 2 \lceil \log(n) \rceil + 2 \] Schritten
von \[ n^3 \left\lfloor \frac{n}{2} \right\rfloor \] Prozessoren
durchgef"uhrt.

Um die Determinante zu berechnen, sind noch nacheinander durchzuf"uhren:
\begin{enumerate}
\item
      eine Matrizenmultiplikation nach Satz \ref{SatzAlgMatMult} in
      \[ \lceil \log(n) \rceil + 1 \] Schritten von \[ n^3 \] 
      Prozessoren,
\item 
      die Berechnung der Spur\footnote{siehe Definition
      \ref{DefTr} } einer Matrix nach Satz \ref{SatzAlgBinaerbaum} in
      \[ \lc \log(n) \rc \] Schritten von
      \[ \lf \frac{n}{2} \rf \]
      Prozessoren und
\item
      eine Division in einem Schritt von einem Prozessor.
\end{enumerate}
Diese drei Berechnungsstufen werden insgesamt in
\[ 2 \lceil \log(n) \rceil + 2 \] Schritten von \[ n^3 \] Prozessoren
durchgef"uhrt.

Die Berechnung der Determinanten mit Hilfe von Satz \ref{SatzFrame}
kann also in
\[ \lc \log(n) \rc^2 + 4 \lc \log(n) \rc + 4 \] Schritten
durchgef"uhrt werden. Die Anzahl der Prozessoren betr"agt
\begin{eqnarray*}
   &      & n^3 \lf \frac{n}{2} \rf
\\ & \leq & \lc \frac{n^4}{2} \rc
\end{eqnarray*}

Man erkennt, da"s C-Alg. keine Fallunterscheidungen verwendet. Dies ist
ein Vorteil bei der Konstruktion von Schaltkreisen, da somit keine
Teilschaltkreise f"ur einzelne Zweige entworfen werden m"ussen. Dadurch
wird ein Schaltkreis zur Determinantenberechnung mit Hilfe von C-Alg.
nicht unn"otig vergr"o"sert. Es wird sich zeigen, da"s die anderen
Algorithmen (BGH-Alg., B-Alg. und P-Alg.) die Eigenschaft fehlender 
Fallunterscheidungen ebenfalls besitzen. In dieser Hinsicht besitzt keiner
der Algorithmen einen Vorteil gegen"uber den anderen.

Vergleicht man den Aufwand an Schritten und Prozessoren mit dem der anderen
Algorithmen, zeigt sich, da"s C-Alg. bereits sehr effizient ist.


%
% Datei: bgh.tex (Textteile nach 'BGH82')
%

\MyChapter{Der Algorithmus von {Borodin,} Von zur Gathen und Hopcroft}
\label{ChapBGH}

Der Algorithmus \cite{BGH82}, der in diesem Kapitel dargestellt wird,
verbindet die Vermeidung von Divisionen \cite{Stra73}, das Gau"s'sche
Eliminationsverfahren (z. B. \cite{BS87} S. 735)
und die parallele Berechnung von Termen \cite{VSBR83} miteinander, um
die Determinante einer Matrix zu berechnen. Auf diesen Algorithmus wird
mit {\em BGH-Alg.} Bezug genommen (vgl. Unterkapitel \ref{SecBez}).

Er unterschiedet sich in
seiner Methodik von den anderen Algorithemen (C-Alg., B-Alg. und P-Alg.)
vor allem dadurch, da"s er die Koeffizienten des charakteristischen
Polynoms in keiner Weise beachtet (vgl. \ref{SatzDdurchP}), sondern die
Determinante durch miteinander verkn"upfte Transformationen, nicht zuletzt
auch durch Ausnutzung von Satz \ref{SatzDetPermut}, direkt
berechnet.

Wie sich in diesem Kapitel zeigen wird, besitzt der Algorithmus durch
die Verbindung der drei o. g. Verfahren eine gewisse Eleganz, besonders,
was die Handhabung der Konvergenz von Potenzreihen angeht. 

Ein Nachteil des Algorithmus ist die vergleichsweise schlechte Effizienz.

% **************************************************************************

\MySection{Das Gau"s'sche Eliminationsverfahren}
\label{SecGauss}

\index{Gau{\Mys}s'sches Eliminationsverfahren}
Das Gau"s'sche Eliminationsverfahren wird dazu benutzt, lineare
Gleichungssysteme der Form \[ Ax=b \] zu l"osen. Dazu wird die sogenannte
{\em erweiterte Koeffizientenmatrix} betrachtet. Sie ist eine
$n \times (n+1)$-Matrix, deren erste $n$ Spalten aus den Spalten der
Koeffizientenmatrix $A$ bestehen und deren $(n+1)$-te Spalte aus dem
Vektor $b$ besteht.

Die Idee des Gau"s'schen Eliminationsverfahrens ist es, die erweiterte
Koeffizientenmatrix so zu transformieren,
da"s die darin enthaltene Matrix $A$ die Form einer {\em oberen
Dreiecksmatrix} \index{Dreiecksmatrix} bekommt. F"ur eine solche
$n \times n$-Matrix gilt:
\[ \forall 1 \leq j < i \leq n: a_{i,j} = 0 \]
Falls f"ur die Matrix
\[ \forall 1 \leq i < j \leq n: a_{i,j} = 0 \]
erf"ullt ist, nennt man sie {\em untere Dreiecksmatrix}.
Zur Transformation werden \index{Zeilenoperationen!elementare}
{\em elementare Zeilenoperationen} verwendet.
Sie werden in Definition \ref{DefDet} der Determinanten einer Matrix
unter D1 und D3 beschrieben. Sie haben nicht nur die dort genannten
Beziehungen zur Determinanten einer Matrix, sondern noch zus"atzlich die
Eigenschaft, da"s sie, angewandt auf die erweiterte Koeffizientenmatrix,
die L"osungsmenge des linearen Gleichungssystems unver"andert lassen.

F"ur die Determinantenberechnung wird die erweiterte Koeffizientenmatrix
nicht weiter beachtet. Alle Operationen beziehen sich nur auf die Matrix
$A$. Die Matrizenelemente unterhalb der
Hauptdiagonalen\footnote{ Die Hauptdiagonale bilden $a_{1,1}$ bis
$a_{n,n}$.} werden spaltenweise durch Nullen ersetzt, beginnend mit der
ersten Spalte. Die Transformationen werden durch folgende Gleichungen
beschrieben\footnote{ Das Gau"s'sche Eliminationsverfahren wird im
weiteren Text so modifiziert, da"s Divisionen durch Null nicht
vorkommen k"onnen. Dieser Fall wird deshalb schon hier au"ser Acht
gelassen.}:
\begin{eqnarray}
    \label{Equ1GaussDef}
    a_{i,j}^{(0)} & := & a_{i,j}
\\  \label{Equ2GaussDef}
    a_{i,j}^{(k)} & := & \left\{
                            \begin{array}{lcr}
                                a_{i,j}^{(k-1)} & : & i \leq k
                            \\  a_{i,j}^{(k-1)}-a_{k,j}^{(k-1)}
                                    \frac{ a_{i,k}^{(k-1)} }{
                                           a_{k,k}^{(k-1)}  }
                                               & : & i > k
                            \end{array}
                        \right.
\end{eqnarray}
Die so gewonnene Matrix $A^{(n)}$ ist die gesuchte obere Dreiecksmatrix.
Betrachtet man Satz \ref{SatzDetPermut}, erkennt man, da"s sich die
Determinante dieser Dreiecksmatrix dadurch berechnen l"a"st, da"s man
die Elemente der Hauptdiagonalen miteinander multipliziert. Da man nur
die in \ref{DefDet} erw"ahnten Operationen verwendet hat, erh"alt man so
auch die Determinante der Matrix $A$.

% **************************************************************************

\MySection{Potenzreihenringe}
\label{SecPotRing}

Im darzustellenden Algorithmus spielen Potenzreihenringe eine wichtige
Rolle. Deshalb werden in diesem Unterkapitel die f"ur uns interessanten
Eigenschaften dieser Ringe behandelt. F"ur unsere Betrachtungen sind Ringe
mit einer zus"atzlichen Eigenschaft von besonderem Interesse:

\MyBeginDef
\label{DefEinheit}
    Sei $R$ ein Ring. Ein $x \in R$ wird als \index{Einheit!in einem Ring}
    {\em Einheit}\footnote{nicht zu verwechseln mit Einselement}
    bezeichnet, wenn es ein $y \in R$ gibt, so da"s
    \[ x * y = 1 \MyPunkt \] Gibt es in $R$ solche Elemente, so wird $R$ 
    als { \em Ring mit Division durch Einheiten } bezeichnet. 
\MyEndDef

Falls in diesem
Kapitel von Ringen die Rede ist, sind immer Ringe mit Division durch
Einheiten gemeint, falls nicht ausdr"ucklich etwas anderes angegeben wird.

Sei $M$ eine Menge von Unbestimmten:
\[ M:= \{ x_1,\,x_2,\, \ldots,\, x_u \} \MyPunkt \]
Dann hei"st $ R[M] $ \index{Ring!{\Myu}ber {\mit M}}
{\em Ring "uber $M$}. F"ur $R[M]$ schreiben wir auch abk"urzend $R[]$.
Die Elemente von $R[]$ sind Terme, in denen neben den
Elementen von $R$ zus"atzlich Elemente von $M$ als Unbestimmte auftreten
d"urfen. 

Analog zur Definition von $R[]$ wird $R[[M]]$ definiert als {\em
Potenzreihenring "uber $M$}. \index{Potenzreihenring!{\Myu}ber {\mit M}}
F"ur $R[[M]]$ wird auch $R[[]]$ geschrieben.
Die Elemente von $R[[]]$
besitzen folgendes Aussehen:
\begin{itemize}
\item
      Sei $T$ eine Teilmenge\footnote{$T$ kann unendlich gro"s sein}
      von $\Nat^{n^2}$.
\item
      F"ur ein $e\in T$ bezeichne $e_i$ das $i$-te Element.
\item
      F"ur $e \in T$ sei \[ k_{e_1,e_2,\ldots,e_{n^2}} \in R \]
\item
      Jedes $u \in R[[]]$ hat f"ur geeignete $k_i$ und eine geeignete Menge
      $T$ die Form:
      \Beq{EquAllgemeinePotenzreihe}
         \sum_{e: \{e_1,e_2,\ldots,e_u \} \in T}
                                                 k_{e_1,e_2,\cdots,e_u}
             \prod_{i=1}^u x_i^{e_i}
      \Eeq
\end{itemize}
Die Summe der Glieder von $u$, f"ur die gilt
   \[ \sum_{i=1}^u e_i = p \]
wird {\em homogene Komponente vom Grad
$p$} \index{homogene Komponente} genannt. Die homogene Komponente
vom Grad $0$ wird auch {\em konstanter Term} \index{konstanter Term}
genannt.

Der Ring $R[]$ enth"alt $R[[]]$ als Unterring und dieser wiederum als
Unterring den Ring der Polynome "uber den Unbestimmten $M$.

\sloppy
Der Potenzreihenring $R[[]]$ besitzt eine f"ur uns besonders interessante
Eigenschaft. Dazu zu\-n"achst der folgende Satz: \fussy
\begin{satz}[Taylor]
\index{Taylor!Satz von}
\label{SatzTaylor}
    Eine Funktion $f$ sei in \[ (x_0-\alpha,x_0+\alpha) \] mit
    \[ \alpha > 0 \] $(n+1)$-mal differenzierbar. Dann gilt f"ur
    \[ x \in (x_0-\alpha,x_0+\alpha) \] die {\em Taylorentwicklung}
    \[
        f(x)= \sum_{\nu = 0}^n \frac{ f^{(\nu)}(x_0) }{ \nu! }
                  (x-x_0)^{\nu} + R_n(x)
    \]
    mit
    \[
        R_n(x):= \frac{ f^{(n+1)}(x_0+ \vartheta(x-x_0)) }{ (n+1)! }
                 (x-x_0)^{n+1}
    \]
    wobei \[ \vartheta \in (0,1) \] und $x_0$ der sogenannte
    {\em Entwicklungspunkt} ist.
\end{satz}
\begin{beweis}
    \cite{Hild74} S. 33-35
\end{beweis}
Weitere Literatur zum Thema 'Taylorreihen' ist z. B. \cite{BS87} (S. 31 und
269). Ein Beispiel f"ur die Anwendung von Satz \ref{SatzTaylor} ist die
Funktion
\Beq{Equ1TylorBeispiel}
    f_1(x) := \frac{1}{1-x} \MyPunkt
\Eeq
Sie ist unendlich oft
differenzierbar mit dem Entwicklungspunkt $x_0=0$ erh"alt man die
Potenzreihe
\Beq{Equ2TaylorBeispiel}
    f_2(x) = \sum_{i=0}^{\infty} x^i \MyPunkt
\Eeq
Der {\em Konvergenzradius} \index{Konvergenzradius} (\cite{BS87} S. 366)
betr"agt $1$, d. h. nur f"ur
\[ |x| < 1 \] gilt \[ f_1(x)=f_2(x) \MyPunkt \]
F"ur den Konvergenzradius $k$ wird das Intervall 
\[ (k,-k) \] als {\em Konvergenzbereich} \index{Konvergenzbereich} 
bezeichnet.

Satz \ref{SatzTaylor} l"a"st sich auch auf mehrere Unbestimmte
verallgemeinern. F"ur uns ist dabei nur folgendes interessant:
\begin{quote}
     Seien \[ f,g \in R[[]] \MyPunkt \]
     Der konstante Term von $g$ sei gleich Null.
     F"ur die Unbestimmten gelte 
     \Beq{Equ2Konvergenz}
         x_1,\ldots, x_u \in (-1,1) \MyPunkt
     \Eeq
     Sei $g$ konvergent.
     Dann folgt aus Satz \ref{SatzTaylor}, da"s sich
     in $R[[]]$ Divisionen der Form
     \Beq{Equ1ZuErsetzen}
         \frac{f}{1-g}
     \Eeq
     ersetzen lassen durch
     \Beq{Equ1StattDivision}
        f*(
              \underbrace{1+g+g^2+\ldots}_{ (*) }
          ) \MyPunkt
     \Eeq
\end{quote}
Die Potenzreihe $g$ wird als {\em innere} Reihe bezeichnet.
Die Terme $(*)$
sind ebenfalls Potenzreihen und werden als
{\em "au"sere} Reihen bezeichnet. Setzt man die {\em innere} Reihe in eine
der {\em "au"seren} ein, erh"alt man wiederum eine Potenzreihe. Diese wird
als {\em Gesamtreihe} bezeichnet.

Im obigen Beispiel konvergiert die Gesamtreihe, falls die innere
Reihe konvergiert und ihre Unbestimmten innerhalb des Konvergenzradius
der "au"seren liegen. Da diese Bedingungen erf"ullt sind, folgt die
Konvergenz der Reihe \equref{Equ1StattDivision}. Um die Konvergenz
in praktisch nutzbarem Ma"se sicherzustellen, sollte der Betrag der Werte,
die f"ur die Unbestimmten eingesetzt werden, nicht beliebig nahe bei $1$
liegen.

Konvergenz ist beim Umgang mit Potenzreihen ein wichtiges Thema. 
Besonders beim Ver\-kn"u\-pfen von Potenzreihen mit mehreren 
Unbestimmten, wie
im vorliegenden Fall, sind Konvergenzbetrachtungen u. U. komplex.
Allgemeine Betrachtungen der Konvergenz von Potenzreihen mit mehreren
Unbestimmten f"uhren an dieser Stelle zu weit und sind z. B. in
\begin{itemize}
\item
      \cite{BT70} ab S. 1 sowie ab S. 49 \hspace{2em} und
\item
      \cite{Hoer73} ab S. 34 
\end{itemize}
zu finden.

Bei praktischen Berechnungen k"onnen Potenzreihen nicht beliebig weit
entwickelt werden, da die Rechenleistung beschr"ankt ist. Deshalb mu"s
ein Grad festgelegt werden, bis zu dem die Potenzreihen entwickelt werden.
Dieser Grad ist i. A. besonders von der St"arke der Konvergenz der Reihe
abh"angig, die entwickelt werden soll. Die Festsetzung eines solchen
Grades erfordert eine Analyse des jeweiligen Problems, das mit Hilfe der
Entwicklung in Potenzreihen gel"ost werden soll. So kann eine Potenzreihe
als Endergebnis mehrerer hintereinander durchgef"uhrter Verkn"upfungen von
Potenzreihen u. U. auch dann konvergieren, wenn als Zwischenergebnis
auftretende Reihen divergieren\footnote{In dem Algorithmus zur
Determinantenberechnung, der in diesem Kapitel vorgestellt wird,
tritt diese Besonderheit auf. In \cite{BGH82} wird darauf in keiner Weise
eingegangen, was sich bei der Bearbeitung als st"orend herausgestellt
hat. }

Da f"ur uns an dieser Stelle weitere allgemeine Betrachtungen uninteressant 
sind, erfolgt die Konvergenzanalyse im Zusammenhang mit der Anwendung 
der Potenzreihenentwicklung auf unser Problem der Determinantenberechnung.

% $$$ hier behandeln, wie Potenzreihen verkn"upft werden ?
%     (vgl. \ref{SecAlgBGH}  (-> letztes Unterkapitel) )

% **************************************************************************

\MySection{Das Gau"s'sche Eliminationsverfahren ohne Divisionen}
\label{SecGaussOhneDiv}

Die M"oglichkeiten zur Vermeidung von Divisionen wurden von V. Strassen
\cite{Stra73} allgemein untersucht. In diesem Unterkapitel wird dargestellt,
wie sich Strassens Ergebnisse auf das Gau"s'sche Eliminationsverfahren
anwenden lassen.

Die Hauptidee zur Vermeidung von Divisionen ist es, alle Berechnungen nicht
in einem Ring $R$ mit Division durch Einheiten
durchzuf"uhren, sondern im zugeh"origen Potenzreihenring $R[[]]$, wobei
Matrizenelemente als Unbestimmte auftreten. Um die
Berechnungen in diesen Ring zu "ubertragen, wird das
Kroneckersymbol \index{Kroneckersymbol} definiert als
\[ \delta_{i,j} :=
       \left\{
           \begin{array}{rcl}
               1 & : & i = j \\
               0 & : & i \neq j
           \end{array}
       \right.
\]
Es sei die Determinante der $n \times n$-Matrix $A$ zu berechnen. Ihre
Elemente werden mit Hilfe der Definition
\Beq{EquDefBGHErsetzung}
    a_{i,j}' := \delta_{i,j} - a_{i,j}
\Eeq
ersetzt. Das bedeutet, jedes
Matrizenelement $a_{i,j}$ wird ersetzt durch
\[ \delta_{i,j} - a_{i,j}' \MyPunkt \]
Wendet man nun das Gau"s'sche Eliminationsverfahren an, bekommt jede
Division die Form \equref{Equ1ZuErsetzen} und kann somit ersetzt werden
durch \equref{Equ1StattDivision}, wie durch das Beispiel in
Unterkapitel \ref{SecBeispielOhneDiv} deutlich wird.

Berechnet man mit Hilfe des Eliminationsverfahrens die Determinante von
$A$, wie in \ref{SecGauss} beschrieben ist, und rechnet dabei in 
$R[[]]$ statt in $R[]$, erh"alt man als Endergebnis eine Potenzreihe $d'$
"uber den Unbestimmten $a_{i,j}'$,
die die Determinante von $A$ beschreibt.

In der praktischen Berechnung ersetzt man die $a_{i,j}'$ mit Hilfe von 
\equref{EquDefBGHErsetzung} durch konkrete Werte und wertet die 
Potenzreihe $d'$ aus, um die Determinante als Element von $R$ zu erhalten.

Ein bis hierhin ungel"ostes Problem ist die Sicherstellung der Konvergenz
von $d'$. Dazu ist die Frage zu beantworten:
\begin{quote}
    Wie gro"s ist der Konvergenzradius von $d'$?
\end{quote}

Hierf"ur m"ussen wir zun"achst eine Eigenschaft der Determinante n"aher
betrachten\footnote{Literatur zu diesem Lemma ist die in Kapitel 
\ref{ChapBase} aufgelistete Grundlagenliteratur.}:
\begin{lemma}
\label{SatzDetEindeutig}
    Es gibt genau eine Abbildung, die jeder Matrix ihre Determinante
    zuordnet.
\end{lemma}
\begin{beweis}
    Der Beweis wird anhand der Matrix $A$ gef"uhrt.

    Seien $f$ und $\hat(f)$ zwei voneinander verschiedene Abbildungen, mit
    den in der Definition \ref{DefDet} der Determinante beschriebenen
    Eigenschaften.

    Es werden zwei F"alle unterschieden:
    \begin{itemize}
    \item
          Bei \[ \rg(A) < n \] gilt nach Satz \ref{SatzRgDetInv}
          \[ f(A) = \hat{f}(A) = 0 \MyPunkt \]
    \item
          Sei \Beq{Equ1SatzDetEindeutig}  \rg(A) = n \MyPunkt \Eeq
          Entsteht die Matrix $B$ aus $A$ durch Zeilenumformungen 
          entsprechend D1 in Definition \ref{DefDet}, dann gibt es 
          ein $c \neq 0$, so da"s gilt: 
          \[ f(B) = c * f(A) \MyPunkt \]
          Das gleiche gilt auch f"ur $\hat{f}$:
          \[ \hat(B) = c * \hat(A) \MyPunkt \]
          Wegen \equref{Equ1SatzDetEindeutig} ist es m"oglich, durch
          Zeilenumformungen \[ B = E_n \] zu erreichen. Aus D4 in 
          Definition \ref{DefDet} folgt dann:
          \[ f(A) = \frac{1}{c} f(E_n) = \frac{1}{c} 
                  = \frac{1}{c}\hat{f}(E_n) = \hat{f}(A) \MyPunkt
          \]
    \end{itemize}
    In beiden F"allen gilt also $ f = \hat(f) $ im Widerspruch zur 
    Voraussetzung, da"s $f$ und $\hat(f)$ verschieden sind.
\end{beweis}

Mit der Unterst"utzung durch dieses Lemma gelangt man zu einer
wichtigen Aussage:
\begin{satz}
\label{SatzBGHKonvergenz}
    Bezeichne $d$ die Potenzreihe "uber den Unbestimmten $a_{i,j}$, die
    man aus $d'$ (s. o.) dadurch erh"alt, da"s man alle 
    Unbestimmten $a_{i,j}'$ mit Hilfe von \equref{EquDefBGHErsetzung} 
    ersetzt.

    F"ur $d$ gilt:
    \begin{quote}
         Alle homogenen Komponenten mit einem Grad ungleich $n$ sind
         gleich Null.
    \end{quote}
\end{satz}
\begin{beweis}
    Aus der Richtigkeit der im vorliegenden Kapitel beschriebenen Verfahren
    folgt, da"s $d$ eine Determinante von $A$ entsprechend der 
    Definition \ref{DefDet} beschreibt.

    Bezeichne $f$ die nach Satz \ref{SatzDetPermut} berechnete Determinante
    von $A$ als Summe, deren Summanden jeweils aus einem Produkt von $n$
    Matrizenelementen bestehen.

    Nach Lemma \ref{SatzDetEindeutig} gilt:
    \[
        d = f \MyPunkt
    \]
    Betrachtet man die Termstruktur von $d$ und beachtet, da"s f"ur die 
    Matrizenelemente keine zus"atzlichen Eigenschaften vorausgesetzt werden,
    folgt aus dieser Gleichung die Behauptung.
\end{beweis}
Der Satz wird in Unterkapitel \ref{SecBeispielOhneDiv} an einer
$3 \times 3$-Matrix demonstriert.

Sowohl $d$ als auch $d'$ beschreiben die Determinante von $A$. 
Der Konvergenzradius von beiden Reihen ist also {\em Unendlich}.

Aus Satz \ref{SatzBGHKonvergenz} folgt insbesondere, da"s sich alle 
homogenen Komponenten mit einem Grad gr"o"ser als $n$ gegenseitig aufheben.
Da alle Divisionen durch Additionen und Multiplikationen ersetzt worden 
sind, gehen diese Komponenten nicht in den Wert von Komponenten geringeren
Grades ein. Komponenten mit einem bestimmten Grad beeinflussen
im Verlauf der Rechnungen lediglich die Werte von Komponenten gleichen oder
h"oheren Grades. 

Also ist es unn"otig, die homogenen Komponenten mit einem Grad gr"o"ser
als $n$ "uberhaupt zu berechnen. Dies ist ein wichtiges Ergebnis f"ur die
Analyse der Effizienz des Algorithmus.

% **************************************************************************

\MySection{Beispiel zur Vermeidung von Divisionen}
\label{SecBeispielOhneDiv}

F"ur eine $3 \times 3$-Matrix wird in diesem Unterkapitel
gezeigt, wie die Determinante mit Hilfe des Gau"s'schen
Eliminationsverfahrens ohne Divisionen berechnet
wird\footnote{Das Beispiel wurde mit Hilfe eines Programms zur 
symbolischen Manipulation von Termen berechnet. Wegen der vielen Indizes
ist das Nachrechnen ohne Computer nicht ratsam.}. Wie im
vorangegangenen Unterkapitel \ref{SecGaussOhneDiv} begr"undet ist, werden
bei allen Potenzreihen nur die homogenenen Komponenten bis maximal zum
Grad $3$ betrachtet.

Es ist die Determinante von
\Beq{Equ1BGHBeispiel}
    \left[
        \begin{array}{ccc}
            a_{1,1} & a_{1,2} & a_{1,3} \MatStrut \\
            a_{2,1} & a_{2,2} & a_{2,3} \MatStrut \\
            a_{3,1} & a_{3,2} & a_{3,3} \MatStrut
        \end{array}
    \right]
\Eeq 
zu berechnen.
Die Ersetzung mit Hilfe von Gleichung \equref{EquDefBGHErsetzung} ergibt:
\[
    \left[
        \begin{array}{ccc}
            1 - a_{1,1}' & 0 - a_{1,2}' & 0 - a_{1,3}' \MatStrut \\
            0 - a_{2,1}' & 1 - a_{2,2}' & 0 - a_{2,3}' \MatStrut \\
            0 - a_{3,1}' & 0 - a_{3,2}' & 1 - a_{3,3}' \MatStrut
        \end{array}
    \right]
    \begin{array}{c}
        \MatStrut \\ \MatStrut \\ \MyPunkt \MatStrut
    \end{array}
\]
Nun werden Vielfache der ersten Zeile von
den folgenden Zeilen subtrahiert, und man erh"alt:
\[
    \left[
        \begin{array}{ccc}
            1 - a_{1,1}'
        &   0 - a_{1,2}'
        &   0 - a_{1,3}' \MatStrut
        \\     0
        &   (1 - a_{2,2}') - (0 - a_{1,2}')
            \frac{ (0 - a_{2,1}') }{ (1 - a_{1,1}') }
        &   (0 - a_{2,3}')  - (0 - a_{1,3}')
            \frac{ (0 - a_{2,1}') }{ (1 - a_{1,1}') } \MatStrut
        \\     0
        &   (0 - a_{3,2}') - (0 - a_{1,2}')
            \frac{ (0 - a_{3,1}') }{ (1 - a_{1,1}') }
        &   (1 - a_{3,3}') - (0 - a_{1,3}')
            \frac{ (0 - a_{3,1}') }{ (1 - a_{1,1}') } \MatStrut
        \end{array}
    \right]
    \begin{array}{c}
        \MatStrut \\ \MatStrut \\ \MyPunkt \MatStrut
    \end{array}
\]
Durch Ersetzung der Divisionen und Vereinfachung der Terme erh"alt man:
\[
    \left[
        \begin{array}{ccc}
            1 - a_{1,1}'
        &   0 - a_{1,2}'
        &   0 - a_{1,3}' \MatStrut
        \\     0
        &   \begin{array}{c}
                (1 - a_{2,2}') - a_{1,2}'a_{2,1}' *
            \\  (1 + a_{1,1}' + (a_{1,1}')^2 + (a_{1,1}')^3)
            \end{array}
        &   \begin{array}{c}
                (0 - a_{2,3}')  - a_{1,3}'a_{2,1}' *
            \\  (1 + a_{1,1}' + (a_{1,1}')^2 + (a_{1,1}')^3)
            \end{array} \LMatStrut
        \\     0
        &   \begin{array}{c}
                (0 - a_{3,2}') - a_{1,2}'a_{3,1}' *
            \\  (1 + a_{1,1}' + (a_{1,1}')^2 + (a_{1,1}')^3)
            \end{array}
        &   \begin{array}{c}
                (1 - a_{3,3}') - a_{1,3}'a_{3,1}' *
            \\  (1 + a_{1,1}' + (a_{1,1}')^2 + (a_{1,1}')^3)
            \end{array}
        \end{array}
    \right]
    \begin{array}{c}
        \MatStrut \\ \MatStrut \\ \MyPunkt \MatStrut
    \end{array}
\]
Da nur die homogenen Komponenten bis maximal zum Grad $3$ ber"ucksichtigt
werden, erh"alt man durch weitere Vereinfachung der Terme:
\[
    \left[
        \begin{array}{ccc}
            1 - a_{1,1}'
        &   0 - a_{1,2}'
        &   0 - a_{1,3}' \MatStrut
        \\     0
        &   \begin{array}{c}
                1 - (a_{2,2}' + a_{1,2}'a_{2,1}' +
            \\   a_{1,2}'a_{2,1}'a_{1,1}')
            \end{array}
        &   \begin{array}{c}
                0 - (a_{2,3}'  + a_{1,3}'a_{2,1}' +
            \\  a_{1,3}'a_{2,1}'a_{1,1}')
            \end{array} \LMatStrut
        \\     0
        &   \begin{array}{c}
                0 - (a_{3,2}' + a_{1,2}'a_{3,1}' +
            \\  a_{1,2}'a_{3,1}'a_{1,1}')
            \end{array}
        &   \begin{array}{c}
                1 - (a_{3,3}' + a_{1,3}'a_{3,1}' +
            \\  a_{1,3}'a_{3,1}'a_{1,1}')
            \end{array} \LMatStrut
        \end{array}
    \right]
    \begin{array}{c}
        \MatStrut \\ \LMatStrut \\ \MyPunkt \LMatStrut
    \end{array}
\]
Man erkennt, da"s alle Elemente der Hauptdiagonalen wieder die Form
\[ 1 - g_{i,j} \] und alle anderen Elemente wieder die Form
\[ 0 - g_{i,j} \] besitzen, wobei der konstante Term der $g_{i,j}$ jeweils
gleich $0$ ist. Bei der Fortsetzung des Eliminationsverfahrens k"onnen 
auftretende Divisionen also wiederum auf die gleiche Weise ersetzt werden.

Als n"achstes wird ein Vielfaches der zweiten Zeile von der dritten
subtrahiert. Dazu sei
\begin{eqnarray*}
    a_{2,2}'' & := &
        1 - (a_{2,2}' + a_{1,2}'a_{2,1}' + a_{1,2}'a_{2,1}'a_{1,1}') \\
    a_{2,3}'' & := &   
        0 - (a_{2,3}'  + a_{1,3}'a_{2,1}' + a_{1,3}'a_{2,1}'a_{1,1}') \\
    a_{3,2}'' & := &
        0 - (a_{3,2}' + a_{1,2}'a_{3,1}' + a_{1,2}'a_{3,1}'a_{1,1}') \\
    a_{3,3}'' & := &
        1 - (a_{3,3}' + a_{1,3}'a_{3,1}' + a_{1,3}'a_{3,1}'a_{1,1}') \\
    a_{3,3}''' & := &
        a_{3,3}'' - \frac{ a_{3,2}'' }{ a_{2,2}'' } a_{2,3}'' 
        \MyPunkt
\end{eqnarray*}
Man erh"alt die Matrix:
\[
    \left[
        \begin{array}{ccc}
            1 - a_{1,1}'
        &   0 - a_{1,2}'
        &   0 - a_{1,3}' \MatStrut
        \\     0
        &   a_{2,2}''
        &   a_{2,3}'' \MatStrut
        \\     0
        &      0
        &   a_{3,3}''' \MatStrut
        \end{array}
    \right]
    \begin{array}{c}
        \MatStrut \\ \MatStrut \\ \MyPunkt \MatStrut
    \end{array}
\]
Da nur die homogenen Komponenten bis zum Grad $3$ betrachtet werden sollen,
erh"alt man durch Ersetzung der Division in der beschriebenen Weise und
Vereinfachung der Terme\footnote{Da alle Prokukte aus mehr als $3$ 
Unbestimmten sofort weggelassen werden, d"urfen die Rechenschritte nicht
durch $=$ verbunden werden.} f"ur $a_{3,3}''$:
\begin{eqnarray*} % mit 'form' geprueft (Dateien: bgh2.for bgh2.log)
    & & a_{3,3}'' - \frac{ a_{3,2}'' }{ a_{2,2}'' } a_{2,3}'' \\
    & \rightarrow &
        a_{3,3}'' - a_{3,2}'' \\
    & & * (1 + a_{2,2}' + a_{1,2}'a_{2,1}' + a_{2,2}'^2 + a_{1,2}'a_{2,1}'a_{1,1}' \\
    & & \: \: + 2 a_{2,2}'a_{1,2}'a_{2,1}' + a_{2,2}'^3) \\
    & & * a_{2,3}'' \\
    & \rightarrow &
        1 - a_{1,1}'a_{1,3}'a_{3,1}' - a_{1,2}'a_{2,3}'a_{3,1}'
        - a_{1,3}'a_{2,1}'a_{3,2}' \\
    & & - a_{1,3}'a_{3,1}' - a_{2,2}'a_{2,3}'a_{3,2}'
        - a_{2,3}'a_{3,2}' - a_{3,3}' \MyPunkt
% Form:
%        1 - a[11]*[a13]*[a31] - [a12]*[a23]*[a31] - [a13]*[a21]*[a32]
%          - [a13]*[a31] - [a22]*[a23]*[a32] - [a23]*[a32] - [a33];
\end{eqnarray*}
Um die Determinante zu berechnen, werden die Elemente der Hauptdiagonalen
$a_{1,1}'$, $a_{2,2}''$ und $a_{3,3}'''$ miteinander multipliziert.
Wiederum werden die Komponenten mit zu gro"sem Grad weggelassen. Man
erh"alt:
\begin{eqnarray*}
   & & a_{1,1}' a_{2,2}'' a_{3,3}''' \\
   & \rightarrow &
   1-a_{1,1}'a_{2,2}'a_{3,3}'+ a_{1,1}'a_{2,2}' + a_{1,1}'a_{2,3}'a_{3,2}'
   + a_{1,1}'a_{3,3}' - a_{1,1}' + a_{1,2}'a_{2,1}'a_{3,3}' \\
   & &
   - a_{1,2}'a_{2,1}' - a_{1,2}'a_{2,3}'a_{3,1}' - a_{1,3}'a_{2,1}'a_{3,2}'
   + a_{1,3}'a_{2,2}'a_{3,1}' - a_{1,3}'a_{3,1}' \\
   & &
   + a_{2,2}'a_{3,3}' - a_{2,2}'  - a_{2,3}'a_{3,2}' - a_{3,3}' \MyPunkt
%Form:
%  det =
%     1 - [a11]*[a22]*[a33] + [a11]*[a22] + [a11]*[a23]*[a32] + [a11]*[a33] -
%       [a11] + [a12]*[a21]*[a33] - [a12]*[a21] - [a12]*[a23]*[a31] - [a13]*
%       [a21]*[a32] + [a13]*[a22]*[a31] - [a13]*[a31] + [a22]*[a33] - [a22] -
%       [a23]*[a32] - [a33];
\end{eqnarray*}
Um die gesuchte Determinante zu erhalten, setzt man die mit Hilfe von
Gleichung \equref{EquDefBGHErsetzung} aus den $a_{i,j}$ erhaltenen
Werte f"ur die $a_{i,j}'$ ein.

Zum Beweis, da"s wir richtig gerechnet haben, machen wir nun die
durch \equref{EquDefBGHErsetzung} definierte Substitution im obigen Term
wieder r"uckg"angig. Nach der Vereinfachung des Terms lautet das Ergebnis,
ohne da"s zus"atzlich irgendwelche Teilterme weggelassen worden sind:
\begin{eqnarray*}
  \lefteqn{ a_{1,1}' a_{2,2}'' a_{3,3}''' = } \\
 & & a_{1,1}a_{2,2}a_{3,3} + a_{1,2}a_{2,3}a_{3,1} + a_{1,3}a_{2,1}a_{3,2} \\
 & & - a_{1,1}a_{2,3}a_{3,2} - a_{1,2}a_{2,1}a_{3,3} - a_{1,3}a_{2,2}a_{3,1}
    \MyPunkt
\end{eqnarray*}
Die Richtigkeit dieses Ergebnisses wird beim Vergleich mit Satz
\ref{SatzDetPermut} deutlich.

% **************************************************************************

\MySection{Parallele Berechnung von Termen}
\label{SecVSBR}

Durch die Methode von Strassen zur Vermeidung von Divisionen entstehen
Terme, die es parallel auszuwerten gilt. In diesem Unterkapitel wird
ein Verfahren \cite{VSBR83} beschrieben, welches diese Auswertung
erm"oglicht. Die
Beschreibung des Verfahrens ist auf die Verwendung im Rahmen des
Kapitels angepa"st. Eine ausf"uhrliche Beschreibung ist auch in
\cite{Wald87} ab S. 22 zu finden.

Zun"achst wird die Berechnung von Termen formalisiert.
Dazu wird die Menge \[  \{v_i \MySetProperty 1 \leq i \leq c \} \]
mit $V$ bezeichnet. Die Menge der
Elemente $a_{i,j}$ der $n \times n$-Matrix $A$, die hier als
Unbestimmte auftreten, wird mit $X$ bezeichnet. Sei $R$ der Ring, in dem
alle Rechnungen durchgef"uhrt werden. Es wird definiert
\[ \bar{V} := V \cup X \cup R \MyPunkt \]

\MyBeginDef
\label{DefProgramm}
    Sei $R[]$ der bereits erw"ahnte Ring "uber den Elementen von $X$.
    Sei $c \in \Nat$ gegeben. Seien $+$ und $*$ die
    beiden Ringoperatoren f"ur Addition bzw. Multiplikation.
    Es gelte \[ \circ \in \{+,*\} \MyPunkt \] Weiterhin gelte
    \[ \forall 1\leq i\leq c: \: v_i', v_i'' \in \bar{V}
           \backslash \{ v_i, \, v_{i+1}, \, \ldots, \, v_c \} \MyPunkt
    \]
    Jede Folge der Form
    \[ v_i := v_i' \circ v_i'', \: i = 1, \, \ldots, \, c \]
    hei"st dann { \em Programm "uber R[] } . \index{Programm "uber R[]}
    Ein Element einer solchen Folge wird {\em Anweisung} genannt. Abh"angig
    davon, ob $\circ$ die Addition oder die Multiplikation bezeichnet, wird
    das $v_i$ auch als {\em Additions-} bzw.
    {\em Multiplikationsknoten} bezeichnet. Falls das genaue Aussehen von
    Anweisungen von untergeordnetem Interesse ist, werden diese zur
    Abk"urzung durch ihren Additions- bzw. Multiplikationsknoten
    repr"asentiert.
\MyEndDef

Falls in diesem Unterkapitel im Einzelfall nichts anderes festgelegt wird,
ist mit $v_i'$ bzw. $v_i''$ jeweils der erste bzw. zweite Operand der 
Anweisung $v_i$ gemeint. Dies gilt auch dann, wenn andere Buchstaben
benutzt werden oder keine Indizierung erfolgt.

Jeder Term "uber $R[]$ l"a"st sich durch ein Programm "uber $R[]$
berechnen. Um Aussagen "uber solche Programme machen zu k"onnen, sind
eine Reihe weiterer Vereinbarungen erforderlich, die im folgenden
aufgef"uhrt sind.

Der durch eine Anweisung $v_i$ berechnet Term wird mit $f(v_i)$
bezeichnet.

Sei $x \in \bar{V}$. Seien $x',x'' \in V$.
Dann wird der {\em Grad von $x$} mit $g(x)$
bezeichnet und folgenderma"sen definiert:
\[ g(x) := \left\{
           \begin{array}{rcl}
               0 & : & x \in R \\
               1 & : & x \in X \\
               g(x') + g(x'') & : & (x \in V) \und (x := x' * x'') \\
               \max(g(x'),\, g(x'')) & : &
                                    (x \in V) \und (x := x' + x'')
           \end{array}
           \right.
\]
Der Grad von $x$ stimmt nicht mit dem Grad des Polynoms "uberein,
da"s dem Term $f(x)$ entspricht. Dazu ein Beispiel: \nopagebreak[3]
\[
    \begin{array}{ccc}
        v_1:= y * (-1) & f(v_1) = -y & g(v_1)=1 \\  
        v_2:= y + v_1  & f(v_2) = 0  & g(v_2)=1
    \end{array}
\]

Es wird o. B. d. A. angenommen, da"s f"ur jede Anweisung 
\[ x := x' \circ x'' \] die Bedingung \[ g(x') \geq g(x'') \]
erf"ullt ist.

F"ur alle $a \in \Nat$ wird definiert:
\begin{eqnarray*}
   V_a & := & \{ u \in V \MySetProperty
                 g(u) > a, \, u:= u' * u'', \, g(u') \leq a \}  \\
   V_a'& := & \{ u \in V \MySetProperty
                 g(u) > a, \, u:= u' + u'', \, g(u'') \leq a \}
\end{eqnarray*}

\MyBeginDef
\label{DefTiefe}
    Sei $v\in V$. Sei $v_1, \, \ldots , \, v_k$ die l"angste Folge von 
    Elementen von $\bar{V}$, so da"s gilt
    \begin{eqnarray*}
        & & v_1 = v \\
        \forall 1 \leq i \leq k-1 & : & (v_{i+1} = v_i') \oder 
                                        (v_{i+1} = v_i'') \\
        & & v_k \in F \cup X
        \MyPunkt
    \end{eqnarray*}
    Dann bezeichnet $d(v)=k$ die {\em Tiefe von $v$}.
\MyEndDef

\MyBeginDef
\label{Deffvw}
    Sei $v,w \in \bar{V}$.
    Dann wird $f(v;w) \in R[]$ wie folgt definiert:

    Bezeichnen $v$ und $w$ denselben Knoten, so gilt:
    \[ f(v;w) := 1 \MyPunkt \]
    Falls dies nicht erf"ullt ist und $w \in R \cup X$, dann gilt:
    \[ f(v;w) := 0 \MyPunkt \]
    Falls dies ebenfalls nicht erf"ullt ist und
    \[ w:= w' + w'' \MyKomma \]
    dann gilt \[ f(v;w) := f(v;w') + f(v;w'') \MyPunkt \]
    Falls auch dies nicht erf"ullt ist, bleibt nur noch der Fall "ubrig
    da"s gilt
    \[ w:= w' * w'' \MyPunkt \] Daf"ur wird definiert
    \[ f(v;w) := f(v;w') * f(w'') \MyPunkt \]
\MyEndDef
Durch die Art und Weise, wie $f(v;w)$ definiert ist, ergibt sich eine
besondere Eigenschaft f"ur den Fall, da"s $g(w) < 2g(v)$ erf"ullt ist.
Falls n"amlich in dem Programm, zu dem $v$ und $w$ geh"oren,
der Knoten $v$ durch eine neue 
Unbestimmte $v'$ ersetzt wird, dann ist $f(v;w)$ der Koeffizient von 
$v'$ in $f(w)$. 

In Verbindung mit $f(v;w)$ besitzen die Funktionen $g()$ und $d()$ 
eine Eigenschaft, die weiter unten von Bedeutung ist:
\begin{lemma}
\label{SatzGrad}
    \[ g(v) > g(w) \Rightarrow f(v;w) = 0 \]
\end{lemma}
\begin{beweis}
    Der Beweis erfolgt durch Induktion nach $d(w)$.
    \begin{MyDescription}
    \MyItem{$ d(w) = 0 $ }
        Es gilt:
        \begin{eqnarray*}
            g(w) = 0 & \Rightarrow & w \in R \\
            g(v) > g(w) & \Rightarrow & v \in V \cup X 
        \end{eqnarray*}
        Also ist $f(v;w) = 0$ .
    \MyItem{$ d(w) > 0 $ }
        Das Lemma gelte f"ur alle $ u\in\bar{V}, \, d(u)<d(w)$.
        Aus \ref{DefTiefe} folgt direkt, da"s
        f"ur jede Anweisung \[ w:= w' \circ w'' \] gilt 
        \[ d(w) > d(w'), \: d(w) > d(w'') \MyPunkt \]
        Mit Hilfe von \ref{Deffvw} folgt daraus die G"ultigkeit
        des Lemmas.
    \end{MyDescription}
\end{beweis}

\begin{lemma}
\label{SatzTiefe}
    \[ d(v) > d(w) \Rightarrow f(v;w) = 0 \]
\end{lemma}
\begin{beweis}
    analog zu \ref{SatzGrad}
\end{beweis}

Es lassen sich nun zwei Aussagen formulieren.
Dazu gelte jeweils $v,w \in V$ und $0 < g(v) \leq a < g(w)$.

\begin{lemma}
\label{Satz1VSBR}
    \[
       f(v;w) =
           \sum_{u\in V_a} (f(v;u) * f(u;w)) +
           \sum_{u\in V_a'} (f(v;u'') * f(u;w))
    \]
\end{lemma}
\begin{beweis}
    Der Beweis erfolgt durch Induktion nach $d(w)$. Aufgrund der
    Struktur der zu beweisenden Aussage sind die Beweise von
    Induktionsanfang und Induktionsschlu"s nicht voneinander 
    getrennt.
    
    Wegen der Voraussetzung \[ 0 < a < d(w) \] folgt aus 
    \[ d(w) \leq 1 \Rightarrow w \in R \cup X \MyKomma \]
    da"s $d(w)= 1$ nicht auftreten kann. Sei im folgenden also $d(w)>1$.

    Vier F"alle sind zu unterscheiden:
    \begin{MyDescription}
    \MyItem{ $ w:= w' + w'', \: g(w'') \leq a $ }
        Das Lemma gelte f"ur $w'$.
        Aus der Voraussetzung folgt:
        \[ w \in V'_a \MyPunkt \]
        Aus \ref{SatzTiefe} folgt:
        \[ f(w;w') = 0 \MyPunkt \]
        Au"serdem gilt:
        \[ g(w'') \leq a \Rightarrow
           \forall u \in V'_a: \: f(u;w'') = 0 \MyPunkt
        \]
        Nach \ref{Deffvw} gilt: \[ f(w,w) = 1 \MyPunkt \]
        So ergibt sich:
        \begin{eqnarray*}
            f(v;w') & = &
                \sum_{u \in V_a} (f(v;u) * f(u,w')) \\
              & & + \sum_{u \in V'_a} (f(v;u'') * f(u;w')) \\
            & = &
                \sum_{u \in V_a} (f(v;u) * (f(u,w') + f(u;w''))) \\
              & & + \sum_{u \in V'_a} (f(v;u'') * (f(u;w') + f(u;w''))) \\
            & = &
                \sum_{u \in V_a} (f(v;u) * f(u,w)) \\
              & & + \sum_{u \in V'_a \backslash \{w\} } (f(v;u'') * f(u;w))
        \end{eqnarray*}
        Es folgt mit Hilfe von \ref{Deffvw}:
        \begin{eqnarray*}
            f(v;w) & = & f(v;w') + f(v;w'') \\
            & = & 
                f(v;w') + f(v;w'') * f(w;w) \\
            & = & 
                \sum_{u \in V_a} (f(v;u) * f(u,w)) \\
              & & + \sum_{u \in V'_a \backslash \{w\} }
                     (f(v;u'') * f(u;w)) + f(v;w'') * f(w;w) \\
            & = &
                \sum_{u \in V_a} (f(v;u) * f(u,w)) \\
              & & + \sum_{u \in V'_a} (f(v;u'') * f(u;w))
        \end{eqnarray*}
    \MyItem{ $ w:= w' + w'', \: g(w'') > a $ }
        Das Lemma gelte f"ur $w'$ und $w''$.
        \begin{eqnarray*}
            f(v;w) & = & f(v;w') + f(v;w'') \\
            & = &
                \sum_{u \in V_a} (f(v;u)*f(u;w')) +
                \sum_{u \in \bar{V_a}} (f(v;u'')*f(u;w')) \\
            & & + \sum_{u \in V_a} (f(v;u)*f(u;w'')) +
                  \sum_{u \in \bar{V_a}} f(v;u'')*f(u;w'')) \\
            & = &
                \sum_{u \in V_a} (f(v;u) * (f(u;w') + f(u;w''))) \\
            & & + \sum_{u \in \bar{V_a}} (f(v;u'')*(f(u;w') + f(u;w''))) \\
            & = &
                \sum_{u \in V_a} (f(v;u) * f(u;w)) +
                \sum_{u \in \bar{V_a}} (f(v;u'') * f(u;w))
        \end{eqnarray*}
    \MyItem{ $ w:= w' * w'', \: g(w') \leq a $ }
        Es gilt: 
        \begin{eqnarray*}
            w & \in & V_a \\
            f(v;w) & = & f(v;w) * f(w;w) \MyPunkt
        \end{eqnarray*}
        Andererseits gilt:
        \begin{eqnarray*}
            \forall u \in V_a \backslash \{w\} & : & f(u;w') = 0 \\
            \Rightarrow
            \forall u \in V_a \backslash \{w\} & : &
                f(u;w) = f(w'') * f(u;w') = f(w'') * 0 = 0
        \end{eqnarray*}
        Also folgt:
        \[ \sum_{u \in V_a} (f(v;u) * f(u;w)) = f(v;w) * f(w;w) = f(v;w)
           \MyPunkt
        \]
        Weiterhin folgt aus \ref{SatzGrad}:
        \begin{eqnarray*}
            \lefteqn{ \forall u \in V'_a : \: f(u;w') = 0 } \\
            & \Rightarrow &
               \sum_{u \in V'_a} (f(u'') * f(u;w)) \\
            & & = \sum_{u \in V'_a} (f(u'') * f(w'') * f(u;w')) = 0
        \end{eqnarray*}
        Also ist das Lemma f"ur diesen Fall richtig.    
    \MyItem{ $ w:= w' * w'', \: g(w') > a $ }
         Das Lemma gelte f"ur $w'$.
         \begin{eqnarray*}
            f(v;w) & = & f(w'') * f(v;w') \\
            & = & f(w'') *
                \left(
                    \sum_{u\in V_a} (f(v;u) * f(u;w')) +
                    \sum_{u\in V_a'} (f(v;u'') * f(u;w'))
                \right) \\
            & = &
                \sum_{u\in V_a} (f(v;u) * f(w'') * f(u;w')) +
                \sum_{u\in V_a'} (f(v;u'') * f(w'') * f(u;w')) \\
            & = &
                \sum_{u\in V_a} (f(v;u) * f(u;w)) +
                \sum_{u\in V_a'} (f(v;u'') * f(u;w))
        \end{eqnarray*}
   \end{MyDescription}
\end{beweis}
  
\begin{lemma}
\label{Satz2VSBR}
    \[
       f(w) = 
           \sum_{u\in V_a} (f(u) * f(u;w)) +
           \sum_{u\in V_a'} (f(u'') * f(u;w))
    \]
\end{lemma}
\begin{beweis}
    Bis auf den Unterschied, da"s die auftretenden Terme entsprechend
    unterschiedlich sind, ist der Beweis identisch zum Beweis von
    \ref{Satz1VSBR}.
\end{beweis}

Mit Hilfe von \ref{Satz1VSBR} und \ref{Satz2VSBR} l"a"st sich ein
Verfahren zur parallelen Berechnung von Termen angeben, das im
folgenden beschrieben wird.

Gegeben sei ein Programm der L"ange $c$,
d. h. \[ V = \{v_1, \, v_2, \, \ldots, v_c\} \MyPunkt \]
Es ist $f(v_c)$ zu berechnen. Die Berechnung erfolgt stufenweise.
Seien $v,w \in V$.
In Stufe $0$ werden alle $f(w)$ mit \[ g(w)=1 \]  und alle
$f(v;w)$ mit \[ g(w) - g(v) = 1 \] berechnet.

In Stufe $i$ werden alle $f(w)$ mit
\[ 2^{i-1} < g(w) \leq 2^i \] und alle $f(v;w)$ mit
\[ 2^{i-1} < g(w) - g(v) \leq 2^i \] berechnet. Dabei werden die Ergebnisse
der vorangegangenen Stufen benutzt.

Auf diese Weise ist $f(v_c)$ nach
\[ \lc \log(g(v_c)) \rc \] Stufen berechnet.

In Stufe $i$ werden zun"achst die $f(w)$ mit Hilfe von \ref{Satz2VSBR}
berechnet. Dazu wird $a=2^{i-1}$ gew"ahlt:
\begin{eqnarray}
    f(w) \nonumber
    & = & \nonumber
        \sum_{u\in V_a} (f(u) * f(u;w)) +
        \sum_{u\in V_a'} (f(u'') * f(u;w)) \\
    & = & \label{EquStepIfw}
        \sum_{u\in V_a} (f(u')*f(u'')* f(u;w)) +
        \sum_{u\in V_a'} (f(u'') * f(u;w))
\end{eqnarray}
Anhand der Definitionen erkennt man, da"s f"ur alle auftretenden 
$f(\ldots)$ gilt: \[ g(f(\ldots)) \leq 2^{i-1} \MyPunkt \]
Also wurden alle zu benutzenden Terme bereits in einer der vorangegangenen
Stufen berechnet. 

Man erkennt anhand der bisher angestellten Betrachtungen "uber Programme
zur Berechnung von Termen, da"s der Aufwand f"ur alle Programme der
L"ange $c$ gleich ist. Da eine Aufgabe, die in $a$ Schritten von 
$b$ Prozessoren erledigt wird, auch in $2a$ Schritten von $b/2$ 
Prozessoren erledigt werden kann, erfolgt die Analyse des Aufwandes 
f"ur eine bestimmte Stufe $i$ zun"achst mit Hilfe der durchschnittlich f"ur
eine Stufe zu erwartenden erforderlichen Operationen\footnote{Diese 
Betrachtungsweise kennt man in der Literatur unter dem Begriff 
{\em Rescheduling}.}.

F"ur eine bestimmte Stufe l"a"st sich die Gr"o"se der Mengen $V_a$ und
$V_a'$ nicht genau vorherbestimmen. Falls die Berechnung in $z$ Stufen 
durchgef"uhrt wird, dann gilt jedoch
\begin{eqnarray*}
    & & 0 \leq i,j \leq z \\
    & & i \neq j \\
    & & a_k := 2^{k-1} \\
    & & V_{a_i} \cap V_{a_k} = V'_{a_i} \cap V'_{a_k} = \emptyset \\
    & & \sum_{0 \leq i \leq z} |V_{a_i}| \leq c \\
    & & \sum_{0 \leq i \leq z} |V'_{a_i}| \leq c 
\end{eqnarray*}
Die Mengen $V_a$ und $V'_a$ besitzen also durchschnittlich h"ochstens
\[ \frac{c}{z} = \frac{c}{ \lc \log(g(v_c)) \rc } \]
Elemente. Dieser Wert wird mit $m$ bezeichnet.

Aus den vorangegangenen "Uberlegungen folgt, da"s \equref{EquStepIfw} in
\[ \lc \log(m) \rc + 3 =
   \lc \log \lb \frac{c}{ \lc \log(g(v_c)) \rc } \rb \rc + 3
\]
Schritten von
\[ 2m = 2 \frac{c}{ \lc \log(g(v_c)) \rc } \] 
Prozessoren berechnet werden kann.

Nachdem in Stufe $i$ die $f(w)$ berechnet worden sind, werden die $f(v;w)$
mit Hilfe von \ref{Satz1VSBR} ausgerechnet. Dazu wird $a= g(v) + 2^{i-1}$ 
gew"ahlt:
\begin{eqnarray*}
    f(v;w)
    & = &
        \sum_{u\in V_a} (f(v;u) * f(u;w)) +
        \sum_{u\in V_a'} (f(v;u'') * f(u;w)) \\
    & = &
        \sum_{u\in V_a} (f(u'') * f(v;u') * f(u;w)) +
        \sum_{u\in V_a'} (f(v;u'') * f(u;w)) \\
\end{eqnarray*}
Anhand der Definitionen erkennt man, da"s alle $f(v;u')$, $f(u;w)$ und
$f(v;u'')$ bereits berechnet wurden. F"ur $f(u'')$ gibt es kritische
F"alle, die separat untersucht werden m"ussen:
\begin{MyDescription}
\MyItem{ $g(u') \geq g(u'') > 2^i, \: f(v;u') = 0$ }
    Der Fall ist kein Problem, da der Wert des jeweiligen gesamten Terms
    gleich Null ist.
\MyItem{ $g(u') \geq g(u'') > 2^i, \: f(v;u') \neq 0$ }
    Es mu"s gelten:
    \[ g(u') \geq g(v) \MyPunkt \]
    Daraus ergibt sich:
    \begin{eqnarray*}
        g(u) & = & g(u') + g(u'') \\
             & > & g(v) + 2^i \\
             & \geq & g(w) \\
             & \Rightarrow & f(u;w) = 0
    \end{eqnarray*}
    Der Wert des Terms ist also wiederum gleich Null.
\end{MyDescription}

F"ur die Analyse des Aufwandes gelten die gleichen Bemerkungen wie f"ur
die Berechnung der $f(w)$.

Insgesamt kann $f(v_c)$ in
\Beq{Equ1VSBRAnalyse}
        2\lc \log(g(v_c)) \rc (\lc \log(m) \rc + 3) =
        2\lc \log(g(v_c)) \rc
        \lb \lc \log \lb \frac{c}{ \lc \log(g(v_c)) \rc } 
                     \rb 
             \rc + 3
        \rb
\Eeq
Schritten von
\Beq{Equ2VSBRAnalyse}
   2m = 2 \frac{c}{ \lc \log(g(v_c)) \rc } 
\Eeq
Prozessoren berechnet werden.

F"ur das bis hierhin beschriebene und analysierte Verfahren gibt es einen
Sonderfall, der mit geringerem Aufwand gel"ost werden kann.
\MyBeginDef
\label{DefHomogen}
    Sei $v_1,\, \ldots ,\, v_c$ ein Programm im Sinne von \ref{DefProgramm}.
    Falls f"ur alle Additionsknoten $v_i:= v_i'+ v_i''$ dieses Programms 
    gilt:
    \[ g(v_i') = g(v_i'') \MyKomma \]
    so wird das Programm als {\em homogen} bezeichnet.
\MyEndDef
F"ur homogene Programme sind alle Mengen $V_a'$ leer. Mit dieser 
Feststellung ergeben sich aus \ref{Satz1VSBR} und \ref{Satz2VSBR} zwei
Folgerungen f"ur homogene Programme:
\begin{korollar}
\label{Satz3VSBR}
    \[
       f(v;w) =
           \sum_{u\in V_a} (f(v;u) * f(u;w))
    \]
\end{korollar}

\begin{korollar}
\label{Satz4VSBR}
    \[
       f(w) = 
           \sum_{u\in V_a} (f(u) * f(u;w))
    \]
\end{korollar}

Werden im angegebenen Verfahren zur parallelen Berechnung von Termen 
die Folgerungen
\ref{Satz3VSBR} und \ref{Satz4VSBR} statt der Lemmata \ref{Satz1VSBR} und
\ref{Satz2VSBR} benutzt, so f"uhrt das zu leicht verringertem
Berechnungsaufwand. Dann kann $f(v_c)$ analog zur obigen Analyse 
f"ur $f(v_c)$ bei nicht homogenen Programmen in
\Beq{Equ3VSBRAnalyse}
        2\lc \log(g(v_c)) \rc
        \lb \lc \log \lb \frac{c}{ \lc \log(g(v_c)) \rc } 
                     \rb 
             \rc + 2
        \rb
\Eeq Schritten von 
\Beq{Equ4VSBRAnalyse}
    \frac{c}{ \lc \log(g(v_c)) \rc } 
\Eeq Prozessoren berechnet werden.

% **************************************************************************

\MySection{Das Gau"s'sche Eliminationsverfahren parallelisiert}
\label{SecAlgBGH}

Das Thema dieses Unterkapitels ist es, wie das in \ref{SecVSBR}
beschriebene Verfahren benutzt werden kann, um mit Hilfe des in
\ref{SecGaussOhneDiv} angegebenen
Gau"s'schen Eliminationsverfahrens ohne Divisionen parallel die
Determinante einer $n \times n$-Matrix zu berechnen. Auf den so
entstehenden Algorithmus wird mit BGH-Alg. Bezug genommen
(vgl. Unterkapitel \ref{SecBez}).

Da keine Divisionen durchgef"uhrt werden, ist BGH-Alg. ebenso wie
B-Alg. auch in Ringen anwendbar. In dieser Hinsicht besitzen die
beiden Algorithmen gegen"uber C-Alg. und P-Alg. einen Vorteil.

Im folgenden wird beschrieben, wie ein Programm im Sinne von
\ref{SecVSBR} anhand der Ergebnisse von \ref{SecGaussOhneDiv}
aufgebaut wird. Um die Auswirkungen des Grades, bis zu dem Potenzreihen
entwickelt werden, auf die Effizienz der Rechnung besser demonstrieren 
zu k"onnen, werden die folgenden Betrachtungen zun"achst unabh"angig von
einem konkreten Grad durchgef"uhrt. F"ur alle Potenzreihen werden
die homogenen Komponenten bis maximal zum Grad $s$ betrachtet.

Es gibt drei wesentliche Elementaroperationen f"ur Potenzreihen, die 
zun"achst auf ihren Aufwand hin untersucht werden. Da sich nach
\ref{SecVSBR} die Anzahl der Prozessoren und der Schritte aus der
Programml"ange ergibt, wird im folgenden nur die Anzahl der Anweisungen
im Sinne von \ref{DefProgramm} betrachtet:
\begin{MyDescription}
\MyItem{Addition}
    Dieser Fall gilt f"ur {\em Subtraktion} analog. Es werden die
    homogenen Komponenten gleichen Grades addiert. Da die homogenen 
    Komponenten bis zum Grad $s$ betrachtet werden, sind hierf"ur 
    $s+1$ Anweisungen erforderlich.
\MyItem{Multiplikation}
    Seien $a$ und $b$ die zu multiplizierenden Potenzreihen. F"ur eine
    Potenzreihe $x$ bezeichne $x_i$ deren homogene Komponente vom
    Grad $i$. Das Ergebnis der zu Multiplikation von $a$ und $b$ sei $c$.
    Man erh"alt $c$ mit:
    \[ c_i := \sum_{j=0}^{i} a_j * b_{i-j} \MyPunkt \]
    Da f"ur $c$ auch nur die homogenen Komponenten bis zum
    Grad $s$ berechnet werden m"ussen, folgt mit Hilfe der Gleichung
    \[ 2^i = \sum_{j=0}^{i-1} 2^j + 1 \] f"ur die Anzahl der Anweisungen
    bei Benutzung der Bin"arbaummethode nach \ref{SatzAlgBinaerbaum}:
    \begin{eqnarray*}
        & & \sum_{i=0}^s 
            \lb 
                i + \sum_{j=0}^{\lc \log(i) \rc-1} 2^j 
            \rb \\
        & = &
            \sum_{i=0}^s \lb i + 2^{\lc \log(i) \rc} - 1 \rb \\
        & \leq &
            \sum_{i=0}^s \lb i + 2^{\log(i) + 1} - 1 \rb \\
        & = &
            \sum_{i=0}^s \lb 3i - 1 \rb \\
        & = & 3 \sum_{i=0}^s i - (s+1) \\
        & = & \frac{3}{2} s (s + 1) - (s+1) \\
        & = & \frac{3s^2 + s - 2}{2} \MyPunkt
    \end{eqnarray*}
\MyItem{Division}
    Die Divisionen werden entsprechend der Ausf"uhrungen in
    \ref{SecPotRing} und \ref{SecGaussOhneDiv} durch Additionen und
     Multiplikationen ersetzt (vgl. S. \pageref{Equ1ZuErsetzen}).
    Da nur die homogenen Komponenten bis zum Grad $s$ betrachtet werden,
    erfolgt die Potenzreihenentwicklung wie in \equref{Equ1StattDivision}
    nur bis zum $s$-ten Glied.

    Somit sind
    $s$ Multiplikationen und $s-1$ Additionen von Potenzreihen sowie
    die Addition des konstanten Terms durchzuf"uhren. In Verbindung mit
    den vorangegangenen Analysen von Addition und Multiplikation ergibt
    sich f"ur die Anzahl der Anweisungen:
    \begin{eqnarray*}
        &   & s * \lb \frac{3s^2 + s - 2}{2} \rb + (s-1) * (s+1) + 1 \\
        & = & \frac{3s^3 + 3s^2 - 2s}{2} \MyPunkt
    \end{eqnarray*}
\end{MyDescription}

Als n"achstes wird untersucht, wieviele der einzelnen Elementaroperationen
zur Berechnung der Determinante benutzt werden. Dazu werden zwei 
Gleichungen benutzt:

\begin{bemerkung}
\label{SatzSumK}
    Sei $n \in \Nat_0$. Dann gilt:
    \[ \sum_{k=1}^n k = \frac{ n(n+1) }{ 2 } \]
\end{bemerkung}

\begin{bemerkung}
\label{SatzSumK2}
    Sei $n \in \Nat_0$. Dann gilt:
    \[ \sum_{k=1}^n k^2 = \frac{ n(n+1)(2n+1) }{ 6 } \]
\end{bemerkung}

Das in \ref{SecGauss} beschriebene Verfahren verwendet die Gleichungen
\equref{Equ1GaussDef} und \equref{Equ2GaussDef}. Werden mit Hilfe dieser 
Gleichungen zun"achst alle Matrizenelemente transformiert, betr"agt 
die Anzahl der Berechnungen neuer Elemente:
\begin{eqnarray*}
   &   & \sum_{i=1}^{n-1} \sum_{j=i+1}^n (n-(j-1)) \\
   & = & \sum_{i=2}^n \sum_{j=i}^n (n-(j-1)) \\
   & = & \sum_{i=2}^n \sum_{j=1}^{n-(i-1)} j \\
   & \MyStack{nach \ref{SatzSumK}}{=} &
         \sum_{i=2}^n \frac{ (n-(i-1))*((n-(i-1))+1) }{2} \\
   & = & \frac{1}{2} \sum_{i=2}^n ((n-i+1)*(n-i+2)) \\
   & = & \frac{1}{2} \sum_{i=2}^n (n^2+3n-2ni+i^2-3i+2) \\
   & = & \frac{1}{2}
         \lb (n-1)(n^2+3n+2)
             + \sum_{i=2}^n i^2
             - \sum_{i=2}^n 2ni
             - \sum_{i=2}^n 3i
         \rb \\
   & \MyStack{nach \ref{SatzSumK},\ref{SatzSumK2}}{=} &
         \frac{1}{2}
         \lb (n^3+2n^2-n-2)
             + \frac{1}{6} n(n+1)(2n+1) - 1 \right. \\
   & &   \left.
             - 2n \lb\frac{ n(n+1) }{ 2 } - 1 \rb
             - 3 \lb \frac{ n(n+1) }{ 2 } - 1 \rb
         \rb \\
   & = &
          \frac{1}{2}
          \lb (n^3+2n^2-n-2)
              + \frac{ 2n^3+3n^2+n }{ 6 } - 1 \right. \\
   & &   \left.
              - ( n^3+n^2 - 2n)
              - \lb \frac{ 3(n^2+n) }{ 2 } - 3 \rb
          \rb \\
   & = & \frac{1}{6}( n^3- n)
\end{eqnarray*}
F"ur jede einzelne Transformation eines Matrixelements
werden nach \equref{Equ2GaussDef}
eine Subtraktion, eine Multiplikation und eine Division durchgef"uhrt.
Da alle Rechnungen in $R[[]]$ erfolgen, werden dabei Potenzreihen 
miteinander verkn"upft, wof"ur der Aufwand
gemessen in durchzuf"uhrenden Anweisungen bereits analysiert worden 
ist (s. o.).
F"ur die identische Abbildung nach \equref{Equ1GaussDef} wird kein Aufwand
in Rechnung gestellt. So kommt man auf
\begin{eqnarray}
   & & \nonumber
       \frac{1}{6}( n^3 - n) *
       \lb
           s + 1
           + \frac{3s^2 + s - 2}{2}
           + \frac{3s^3 + 3s^2 - 2s}{2}
       \rb \\
  & = & \label{AnwNeueElem}
% Form:
%    1/4*n^3*s^3 + 1/2*n^3*s^2 + 1/12*n^3*s - 1/4*n*s^3 - 1/2*n*s^2
%    - 1/12*n*s;
         \frac{1}{4}
       \lb n^3 s^3 + 2n^3 s^2 + \frac{n^3 s}{3} - n s^3 - 2 n s^2
       - \frac{n s}{3} \rb
\end{eqnarray}
Anweisungen, um eine gegebene Matrix mit Hilfe des Gau"s'schen Verfahrens
in eine obere Dreiecksmatrix zu "uberf"uhren. Zu Berechnung der
Determinante sind im Anschlu"s daran noch die Elemente der Hauptdiagonalen
miteinander zu multiplizieren. Dies kann mit $n-1$ Multiplikationen
geleistet werden, denen
\begin{eqnarray}
   & & \nonumber
     (n-1) * 
     \frac{3s^2 + s - 2}{2} \\
   & = & \label{AnwDiagMult}
     \frac{1}{2} ( 3 n s^2 + n s - 2n - 3 s^2 - s + 2)
\end{eqnarray}
Anweisungen entsprechen. So hat man bereits das Ergebnis als Element
von $R[[]]$. Um die Determinante als Element von $R$ zu erhalten, m"ussen
nun noch die homogenen Komponenenten bis zum Grad $s$ addiert werden.
Dies kann mit Hilfe von $s$ Anweisungen erfolgen. Abgesehen von diesen 
Additionen ist das Programm homogen im Sinne von \ref{DefHomogen}.
Deshalb ist es von Vorteil, die in \ref{SecVSBR} beschriebene Methode
auf das Programm ohne die letzten Additionen anzuwenden und diese
Additionen mit Hilfe der Bin"arbaummethode nach \ref{SatzAlgBinaerbaum}
durchzuf"uhren. Die Addition der homogenen Komponenten kann so in 
\[ \lc \log(s+1) \rc \] Schritten von \[ \lf \frac{s+1}{2} \rf \]
Prozessoren geleistet werden.

Man erh"alt das Gesamtergebnis
f"ur die Programml"ange ohne die letzten Additionen als
Summe von \equref{AnwNeueElem} und \equref{AnwDiagMult}:
\[ % \label{Gesamt}
%  1 + 1/4*n^3*s^3 + 1/2*n^3*s^2 + 1/12*n^3*s - 1/4*n*s^3 + n*s^2 + 5/12*n*
%  s - n - 3/2*s^2 - 1/2*s;
   \frac{1}{4} 
   \lb
       n^3 s^3 + 2 n^3 s^2 + \frac{1}{3}n^3 s - n s^3 + n s^2 + 
       \frac{5}{3}n s - 4 n - 6 s^2 - 2 s + 4
   \rb
\]
Anweisungen. Dieser Wert wird entsprechend der Terminologie in
\ref{SecVSBR} mit $c$ bezeichnet. Da bei allen Rechnungen nur die
homogenen Komponenten bis zum Grad $s$ beachtet werden, gilt
\[ g(v_c) = s \MyPunkt \]
Aus $c$ und $g(v_c)$ erh"alt man mit Hilfe der Analyseergebnisse
\equref{Equ3VSBRAnalyse} und \equref{Equ4VSBRAnalyse} aus
\ref{SecVSBR} f"ur die in diesem Kapitel beschriebene Methode zur
parallelen Determinantenberechnung einen Aufwand 
von\footnote{Genau genommen mu"s der Wert noch um $1$ erh"oht werden 
f"ur die Berechnung der $a_{i,j}'$ aus den urspr"unglichen 
Matrizenelementen $a_{i,j}$ entsprechend Gleichung 
\equref{EquDefBGHErsetzung}.}
\begin{eqnarray*}
    & & 2\lc \log(s) \rc * (\lc \log(c) \rc + 2) + \lc \log(s+1) \rc \\
    & = &
        2\lc \log(s) \rc \\
    & &
        * \lb
            \lc
            \log\lb
   \frac{1}{4} 
   \lb
       n^3 s^3 + 2 n^3 s^2 + \frac{1}{3}n^3 s - n s^3 + n s^2 + 
       \frac{5}{3}n s - 4 n - 6 s^2 - 2 s + 4
   \rb
            \rb
            \rc + 2
        \rb \\
    & & + \lc \log(s+1) \rc
\end{eqnarray*}
Schritten und
\begin{eqnarray*}
  \lefteqn{ \max \lb c \: , \lf \frac{s}{2} \rf \rb } \\
  & = & c \\
  & = &
   \frac{1}{4}
   \lb
       n^3 s^3 + 2 n^3 s^2 + \frac{1}{3}n^3 s - n s^3 + n s^2 +
       \frac{5}{3}n s - 4 n - 6 s^2 - 2 s + 4
   \rb
\end{eqnarray*}
Prozessoren. Man erkennt an diesen Werte die Bedeutung des Parameters
$s$, dem maximal ber"ucksichtigten Grad der homogenen Komponenten der
Potenzreihen. Betrachtet man $s$ als Konstante, so kann der Algorithmus in
\[ O(\log(n)) \] Schritten von \[ O(n^3) \] Prozessoren bearbeitet werden.

Die Analyse in Unterkapitel \ref{SecGaussOhneDiv} ergibt, da"s $s=n$ zu
setzen ist, so da"s der Algorithmus in
\[ O(\log^2(n)) \] Schritten von \[ O(n^6) \] Prozessoren bearbeitet
werden kann.

Die Aufwandanalyse ergibt, da"s BGH-Alg. insbesondere bei der 
Gr"o"senordnung der Anzahl der Prozessoren deutlich hinter C-Alg., B-Alg.
und P-Alg. zur"uckliegt. Die Konstanten bei der Anzahl der Schritte sind
ebenfalls vergleichsweise schlecht.

In BGH-Alg. werden, wie bei den anderen drei Algorithmen, keine 
Fallunterscheidungen verwendet, was aus den bereits in Unterkapitel
\ref{SecAlgFrame} erw"ahnten Gr"unden beim Entwurf von Schaltkreisen
vorteilhaft ist. 

Betrachtet man die Methodik von BGH-Alg., so ist er P-Alg. am "ahnlichsten.
Beide fassen mehrere auch unabh"angig voneinander bedeutsame Verfahren
zu einem Algorithmus zur Determiantenberechnung zusammen. C-Alg. und B-Alg.
hingegen st"utzen sich jeweils auf bestimmte schon seit 40 bis
50 Jahren bekannte S"atze, die nach einigen Umformungen f"ur einen
parallelen Algorithmus verwendet werden.


%
% Datei: berk.tex (Textteile nach 'Berk84')
%
\MyChapter{Der Algorithmus von Berkowitz}
\label{ChapBerk}

Der in diesem Kapitel vorgestellte Algorithmus \cite{Berk84} berechnet
die Determinante mit Hilfe einer rekursiven Beziehung zwischen den
charakteristischen Polynomen einer Matrix und ihren Untermatrizen. Dabei
wird \ref{SatzDdurchP} ausgenutzt. Auf den Algorithmus wird mit {\em B-Alg.} Bezug
genommen\footnote{vgl. Unterkapitel \ref{SecBez}}. 

Wie in BGH-Alg., werden keine Divisionen
verwendet\footnote{vgl. Erl"auterungen in Unterkapitel \ref{SecAlgFrame}}.

%******************************************************************

\MySection{Toepliz-Matrizen}

Im darzustellenden Algorithmus spielen Toepliz-Matrizen (Definition s. u.)
eine wichtige Rolle und werden deshalb in diesem Unterkapitel behandelt.

Eine Matrix $n \times p$-Matrix $A$ hei"st {\em Toepliz-Matrix}, falls
gilt: \index{Toepliz-Matrizen}
\[ a_{i,j} = a_{i-1,j-1} , \: 1 < i \leq n, \: 1 < j \leq p \MyPunkt \]
Sie hat also folgendes Aussehen:
\[
   \left[ \begin{array}{ccccc}
       a_{1,1} & a_{1,2} & a_{1,3} & a_{1,4} & \cdots \MatStrut \\
       a_{2,1} & a_{1,1} & a_{1,2} & a_{1,3} & \ddots \MatStrut \\
       a_{3,1} & a_{2,1} & a_{1,1} & a_{1,2} & \ddots \MatStrut \\
       a_{4,1} & a_{3,1} & a_{2,1} & a_{1,1} & \ddots \MatStrut \\
       \vdots  & \ddots  & \ddots  & \ddots  & \ddots \MatStrut
   \end{array} \right]
\]

Die folgende Eigenschaft von Toepliz-Matrizen ist f"ur uns wichtig:

\begin{satz}
\label{SatzToeplizMult}
\index{Toepliz-Matrizen!Multiplikation von}
    Sei $A$ eine $n \times p$-Matrix und $B$ eine $p \times m$-Matrix.
    Beide seien untere Dreiecks-Toeplitz-Matrizen.
    Falls f"ur die Matrix $C$ gilt
    \[ C = A * B \MyKomma \]
    dann ist $C$ ebenfalls eine untere Dreiecks-Toeplitz-Matrix.
    Sie kann in \[ \lceil \log(p) \rceil + 1 \] Schritten von
    \begin{eqnarray*} 
        & & \frac{ \min(p,\,m)* (\min(p,\,m) +1) }{2}
            + p * \max(n-p,0) \\
        & \leq & n * p
    \end{eqnarray*} Prozessoren berechnet
    werden. 
\end{satz}
\begin{beweis}
    Es sind drei Eigenschaften von $C$ zu zeigen:
    \begin{enumerate}
        \item $C$ ist eine untere Dreiecksmatrix.
        \item $C$ ist eine Toeplitz-Matrix.
        \item $C$ kann mit dem oben angegebenen Aufwand an Schritten und
              Prozessoren berechnet werden.
    \end{enumerate}
    Dies geschieht in drei entsprechenden Beweisschritten. Dazu ist zu
    beachten, da"s die einzelnen Elemente von $C$ nach der Gleichung f"ur
    die Matrizenmultiplikation berechnet werden:
    \Beq{Berk84Equ4}
        c_{i,j}= \sum_{k=1}^p a_{i,k} b_{k,j}
    \Eeq
    \begin{enumerate}
        \item
            Um zu beweisen, da"s $C$ ebenfalls eine untere Dreiecksmatrix
            darstellt, ist zu zeigen
            \[ i < j \Rightarrow c_{i,j} = 0 \]
            Dies erfolgt durch Fallunterscheidung anhand des Index $k$ in
            Gleichung \equref{Berk84Equ4}. Es gibt zwei F"alle:
            \begin{MyDescription}
                \MyItem{ $i < k$ }
                    Da $A$ nach Voraussetzung eine untere Dreiecksmatrix
                    ist und somit
                    \[ i < j \Rightarrow a_{i,j} = 0 \]
                    gilt, folgt 
                    \[ a_{i,k} = 0 \MyKomma\]
                    wodurch der entsprechende Summand in Gleichung
                    \equref{Berk84Equ4} zu $0$ wird.
                \MyItem{ $i \geq k$ }
                    Nach Voraussetzung gilt \[ i < j \MyKomma \] 
                    da f"ur die
                    Elemente oberhalb der Hauptdiagonalen von $C$ zu zeigen
                    ist, da"s sie gleich $0$ sind. Daraus folgt aber
                    \[ k < j \MyPunkt \]
                    Nach Voraussetzung ist $B$ ebenfalls eine untere
                    Dreiecksmatrix und es gilt somit
                    \[ i < j \Rightarrow b_{i,j} = 0 \]
                    Daraus folgt \[ b_{k,j} = 0 \MyKomma \]
                    wodurch wiederum der
                    entsprechende Summand in Gleichung \equref{Berk84Equ4}
                    zu $0$ wird.
            \end{MyDescription}
            In beiden F"allen sind die betrachteten Summanden von Gleichung
            \equref{Berk84Equ4} gleich $0$. Also ist dann auch
            \[ c_{i,j} = 0 \MyKomma \]
            was zu zeigen war.
        \item
            Damit $C$ eine Toeplitz-Matrix ist, mu"s gelten
            \[ c_{i,j} = c_{i+1,j+1} \MyPunkt \]
            Mit Hilfe von Gleichung \equref{Berk84Equ4} ausgedr"uckt
            bedeutet dies
            \Beq{Berk84Equ5}
                 \sum_{k=1}^p a_{i,k} b_{k,j}
               = \sum_{l=1}^p a_{i+1,l} b_{l,j+1} \MyPunkt
            \Eeq
            Da $C$ eine untere Dreiecksmatrix ist, wie oben bewiesen wurde,
            m"ussen nur \[ c_{i,j} \] betrachtet werden, f"ur die gilt
            \[ i \geq j \MyPunkt \]
            Man kann Fallunterscheidungen anhand der Indizes $k$ und $l$
            durchf"uhren. Es gibt f"ur jeden Index drei F"alle, also
            insgesamt sechs:
            \begin{MyDescription}
                \MyItem{ $k>i$ }
                    Da $A$ nach Voraussetzung eine untere Dreiecksmatrix
                    ist, gilt in diesem Fall \[ a_{i,k}= 0 \MyKomma \]
                    und der
                    entsprechende Summand wird zu $0$.
                \MyItem{ $j>k$ }
                    Da $B$ nach Voraussetzung ebenfalls eine untere
                    Dreiecksmatrix ist, gilt in diesem Fall
                    \[ b_{k,j} = 0 \MyKomma \]
                    und der entsprechende Summand wird zu $0$.
                \MyItem{ $i \geq k \geq j$ }
                    Nur in diesem Fall ergibt sich auf der linken Seite
                    von Gleichung \equref{Berk84Equ5} f"ur den jeweiligen
                    Summand ein von $0$ verschiedener Wert. Deshalb kann man
                    die linke Seite dieser Gleichung auch schreiben als
                    \[ \sum_{k=j}^i a_{i,k} b_{k,j} \MyPunkt \]
                \MyItem{ $l>i+1$ }
                    In diesem Fall gilt, da $A$ eine obere Dreiecksmatrix
                    ist, \[ a_{i+1,l} = 0 \MyPunkt \]
                    Der entsprechende Summand der
                    Summe in Gleichung \equref{Berk84Equ5} wird somit zu
                    $0$ und mu"s nicht l"anger betrachtet werden.
                \MyItem{ $j+1>l$ }
                    In diesem Fall gilt \[ b_{l,j+1} = 0 \MyKomma \]
                    da $B$ eine
                    obere Dreiecksmatrix ist und der entsprechende Summand
                    in Gleichung \equref{Berk84Equ5} mu"s nicht l"anger
                    betrachtet werden.
                \MyItem{ $i+1 \geq l \geq j+1$ }
                    Nur in diesem Fall ergibt sich auf der rechten Seite
                    von Gleichung \equref{Berk84Equ5} ein von $0$
                    verschiedener Wert f"ur den entsprechenden Summanden.
                    Man kann also die rechte Seite dieser Gleichung auch
                    schreiben als
                    \[ \sum_{l=j+1}^{i+1} a_{i+1,l} b_{l,j+1} \]
            \end{MyDescription}
            Nach der Betrachtung dieser sechs F"alle reduziert sich
            Gleichung \equref{Berk84Equ5} also, falls man nur die von
            $0$ verschiedenen Summanden betrachtet, auf die Form
            \[ \sum_{k=j}^i a_{i,k} b_{k,j} =
               \sum_{l=j+1}^{i+1} a_{i+1,l} b_{l,j+1}
            \]
            Anders geschrieben hat diese Gleichung die Form
            \begin{eqnarray*}
            &   a_{i,j} b_{j,j} + a_{i,j+1} b_{j+1,j} + a_{i,j+2} b_{j+2,j}
                + \ldots + a_{i,i} b_{i,j} =
            & \\
            &   a_{i+1,j+1} b_{j+1,j+1} + a_{i+1,j+2} b_{j+2,j+1} +
                a_{i+1,j+3} b_{j+3,j+1} + \ldots + a_{i+1,i+1} b_{i+1,j+1}
            &
            \end{eqnarray*}
            Da $A$ und $B$ Toeplitz-Matrizen sind, haben die beiden
            Seiten dieser Gleichung den gleichen Wert, was zu beweisen war.
        \item
            Da $C$ wiederum eine Toeplitz-Matrix ist,
            m"ussen nur die $c_{i,j}$ mit $j=1$ neu berechnet
            werden. Alle anderen Elemente sind entweder gleich Null oder
            gleich einem $c_{i,1}$. L"a"st man zus"atzlich alle
            Multiplikationen mit Null weg, kommt man
            zur Berechnung von $C$ insgesamt mit
            \[ \lc \log(p) \rc + 1 \] Schritten und
            \begin{eqnarray*}
                & & \sum_{k=1}^{\min(p,\,m)} k + p * \max(n-p,\,0) \\
                & = & \frac{ \min(p,\,m)* (\min(p,\,m) +1) }{2}
                      + p * \max(n-p,0)
            \end{eqnarray*} Prozessoren aus. Einschlie"slich der 
            Multiplikationen mit Null erh"alt man
            \[ \lc \log(p) \rc + 1 \] Schritte und
            \[ n * p \] Prozessoren.
    \end{enumerate}
\end{beweis}

% **************************************************************************

\MySection{Der Satz von Samuelson}
\label{SecSamuelson}

In diesem Unterkapitel wird der theoretische Hintergrund des
darzustellenden Algorithmus behandelt.

Zur Beschreibung des Satzes von Samuelson \cite{Samu42} wird folgende
Schreibweise eingef"uhrt ($A$ ist eine $n \times n$-Matrix):
\label{SeiteRMSSchreibweise}
\begin{itemize}
\item
     Den Vektor $S_i$ erh"alt man aus dem $i$-ten Spaltenvektor von $A$
     durch Entfernen der ersten $i$ Elemente. Er hat also folgendes
     Aussehen:
     \[ \left[
        \begin{array}{c}
            a_{i+1,i} \MatStrut \\
            a_{i+2,i} \MatStrut \\
            \vdots    \MatStrut \\
            a_{n,i}
        \end{array}
        \right]
     \]
\item
     Den Vektor $R_i$ erh"alt man aus dem $i$-ten Zeilenvektor von $A$
     durch Entfernen der ersten $i$ Elemente. Er hat also folgendes
     Aussehen:
     \[ [a_{i,i+1}, a_{i,i+2}, \ldots , a_{i,n} ] \]
\item
     Die Matrix $M_i$ erh"alt man aus der Matrix $A$ durch Entfernen
     der ersten $i$ Zeilen und Spalten. Sie hat also folgendes Aussehen:
     \[ \left[
        \begin{array}{cccc}
            a_{i+1,i+1} & a_{i+1,i+2} & \cdots & a_{i+1,n} \MatStrut \\
            a_{i+2,i+1} & a_{i+2,i+2} & \cdots & a_{i+2,n} \MatStrut \\
            \vdots      & \vdots      & \ddots & \vdots    \MatStrut \\
            a_{n,i+1}   & a_{n,i+2}   & \cdots & a_{n,n}   \MatStrut
        \end{array}
        \right]
     \]
\item
     Statt $S_1$, $R_1$ und $M_1$ wird auch $S$, $R$ und $M$ geschrieben.
\end{itemize}

Die Matrix $A$ l"a"st sich also auch in den Formen
\[
   \left[
   \begin{array}{cc}
       a_{11} & R \MatStrut \\
       S      & M \MatStrut
   \end{array}
   \right]
\]
oder
\[
   \left[
   \begin{array}{ccccc}
       a_{11}     & R_1        & \rightarrow &             & \MatStrut \\
       S_1        & a_{22}     & R_2         & \rightarrow & \MatStrut \\
       \downarrow & S_2        & a_{33}      & R_3         & \rightarrow
                                                             \MatStrut \\
                  & \downarrow & S_3         & \ddots      & \ddots
                                                             \MatStrut \\
                  &            & \downarrow  & \ddots      & \MatStrut
   \end{array}
   \right]
\]
darstellen.

Im folgenden Lemma wird das charakteristische Polynom einer Matrix
mit Hilfe der oben definierten $R$, $S$ und $M$ ausgedr"uckt:

\begin{lemma}
\label{Berk84Satz1}
% $$$ Claim 1
    Sei $p(\lambda)$ das charakteristische Polynom der $n \times n$-Matrix
    $A$. Dann gilt:
    \[
        p(\lambda) = (a_{1,1} - \lambda) * \det(M - \lambda * E_{n-1})
                     - R * \adj(M - \lambda * E_{n-1}) * S
    \]
\end{lemma}
\begin{beweis}
    Es gilt \[ p(\lambda) = \det(A - \lambda * E_n) \]
    Durch Entwicklung nach der ersten Zeile erh"alt man:
    \[
        p(\lambda)= (a_{1,1} - \lambda) * \det(M - \lambda * E_{n-1}) +
        \sum_{j=2}^n (-1)^{1+j} a_{1,j}
        \underline{ \det( (A - \lambda * E_n)_{(1|j)} ) }
    \]
    Nun werden die in der obigen Gleichung unterstrichenen Determinanten
    jeweils nach der ersten Spalte entwickelt:
    \begin{eqnarray*}
        & p(\lambda)=
        & (a_{1,1} - \lambda) * \det(M - \lambda * E_{n-1}) +
    \\  & & \sum_{j=2}^n
            \underbrace{ (-1)^{1+j} a_{1,j} }_{ \mbox{(*1)} }
        \sum_{k=2}^n (-1)^{1+(k-1)} a_{k,1}
            \det(
                \underbrace{
                    (A - \lambda * E_n)_{(1,k|1,j)}
                }_{ \mbox{(*2)} }
            )
    \end{eqnarray*}
    Wenn man in dieser Gleichung (*1) mit der inneren Summe multipliziert
    und (*2) mit Hilfe von $M$ ausdr"uckt erh"alt man:
    \[
        p(\lambda)= (a_{1,1} - \lambda) * \det(M - \lambda * E_{n-1}) +
        \sum_{j=2}^n
        \sum_{k=2}^n (-1)^{1+j+k} \underbrace{ a_{1,j} a_{k,1} }_{ \mbox{(*)} }
            \det( (M - \lambda * E_{n-1})_{(k-1|j-1)} )
    \]
    Hier l"a"st sich (*) mit Hilfe von $R$ und $S$ formulieren:
    \[
        p(\lambda)= (a_{1,1} - \lambda) * \det(M - \lambda * E_{n-1}) +
        \sum_{j=2}^n
        \sum_{k=2}^n (-1)^{1+j+k} r_j s_k
            \det( (M - \lambda * E_{n-1})_{(k-1|j-1)} )
    \]
    Dies wiederum ist in Matrizenschreibweise und mit Hilfe der Adjunkten
    einer Matrix ausgedr"uckt nichts anderes als
    \[
        p(\lambda) = (a_{1,1} - \lambda) * \det(M - \lambda * E_{n-1})
                     - R * \adj(M - \lambda * E_{n-1}) * S \MyKomma
    \]
    was zu beweisen war.
\end{beweis}

Vor Lemma \ref{Berk84Satz2} m"ussen wir hier zun"achst einen wichtigen
Satz behandeln (\cite{MM64} S. 50 f):

\begin{satz}[Cayley und Hamilton]
\label{SatzCayleyHamilton}
\index{Cayley und Hamilton!Satz von}
    Sei $p(\lambda)$ das charakteristische Polynom von $A$. Dann gilt:
    \[ p(A)=0_{n,n} \]
\end{satz}
\begin{beweis}
    Aus Satz \ref{SatzAdj} folgt
    \begin{equation}
    \label{Equ1SatzCayleyHamilton}
        (A - \lambda E_n) \adj(A - \lambda E_n) = p(A) E_n
    \end{equation}
    Da die Elemente von \[ \adj(A - \lambda E_n) \] aus Unterdeterminanten
    von $A$ gewonnen werden, bestehen diese Elemente aus Polynomen "uber
    $\lambda$ vom maximalen Grad \[ n - 1 \] Also gilt f"ur geeignete
    $n \times n$-Matrizen \[ B_j, 1 \leq j \leq n-1 \] die folgende 
    Beziehung:
    \Beq{Equ2SatzCayleyHamilton} 
        \adj(A - \lambda E_n) = 
        B_{n-1} \lambda^{n-1} + \ldots + B_1 \lambda + B_0
    \Eeq
    Au"serdem kann man $p(A)$ schreiben als
    \Beq{Equ3SatzCayleyHamilton}
        p(A) = c_n \lambda^n + \ldots + c_1 \lambda + c_0
    \Eeq
    Dr"uckt man \equref{Equ1SatzCayleyHamilton} mit Hilfe von
    \equref{Equ2SatzCayleyHamilton} und \equref{Equ3SatzCayleyHamilton} aus,
    erh"alt man
    \[
        (A - \lambda E_n)(B_{n-1} \lambda^{n-1} + \ldots
        + B_1 \lambda + B_0)
            =
        (c_n \lambda^n + \ldots + c_1 \lambda + c_0) E_n
    \]
    Multipliziert man die Terme auf beiden Seiten aus und vergleicht die
    Koeffizienten miteinander, erh"alt man folgende Gleichungen:
    \[
        \begin{array}{lllcr}
                     & - & B_{n-1} & =      & c_n E_n
        \\  AB_{n-1} & - & B_{n-2} & =      & c_{n-1} E_n
        \\  AB_{n-2} & - & B_{n-3} & =      & c_{n-2} E_n
        \\           &   &         & \vdots &
        \\  AB_1     & - & B_0     & =      & c_1 E_n
        \\  AB_0     &   &         & =      & c_0 E_n
        \end{array}
    \]
    Multipliziert man beide Seiten der
    ersten dieser Gleichungen mit $A^n$, beide Seiten der zweiten
    mit $A^{n-1}$, allgemein beide Seiten der $j$-ten mit $A^{n-j+1}$, und
    addiert sie, erh"alt man
    \[ 0_{n,n} =
       c_n \lambda^n + c_{n-1} \lambda^{n-1} + \ldots + c_1 A = p(A)
    \]
\end{beweis}

Die Adjunkte in Lemma \ref{Berk84Satz1} l"a"st sich mit Hilfe der
Koeffizienten des charakteristischen Polynoms $q(\lambda)$ von $M$
ausdr"ucken. In Koeffizientendarstellung besitzt $q(\lambda)$ die
Form:
\[ q(\lambda) = q_{n-1} \lambda^{n-1} + q_{n-2} \lambda^{n-2} + \ldots
                + q_1 \lambda + q_0
\]

Es gilt folgende Aussage:

\begin{lemma}
\label{Berk84Satz2}
% $$$ Claim 2
    \begin{equation}
    \label{Berk84Equ1}
        \adj(M - \lambda * E_{n-1}) =
            - \sum_{k=0}^{n-2} \lambda^{k} \sum_{l=k+1}^{n-1} M^{l-k-1} q_l
    \end{equation}
\end{lemma}
\begin{beweis}
    Multipliziert man beide Seiten von \equref{Berk84Equ1} mit
    \[ M - \lambda * E_{n-1} \MyKomma \]
    erh"alt man auf der linken Seite
    \[ \adj(M - \lambda * E_{n-1}) * (M - \lambda * E_{n-1}) \MyPunkt \]
    Dies ist nach Satz \ref{SatzAdj} gleich
    \begin{eqnarray*}
        & & E_{n-1} * \det(M - \lambda * E_{n-1}) \\
        & = & q(\lambda) * E_{n-1} \MyPunkt
    \end{eqnarray*}

    Auf der rechten Seite von Gleichung \equref{Berk84Equ1} erh"alt man
    \[ - ( \underbrace{M}_{\mbox{(*1)}}
           \underbrace{- \lambda * E_{n-1})}_{\mbox{(*2)}}
         )
         \sum_{k=0}^{n-2} \lambda^{k} \sum_{l=k+1}^{n-1} M^{l-k-1} q_l
    \]
    Bei der Multiplikation erh"alt man f"ur (*1) und (*2) im obigen je
    eine Doppelsumme:
    \[ - \sum_{k=0}^{n-2} \lambda^{k} \sum_{l=k+1}^{n-1} M^{l-k} q_l
       + \sum_{k=0}^{n-2} \lambda^{k+1} \sum_{l=k+1}^{n-1} M^{l-k-1} q_l
    \]
    Durch Umordnen der Indizes der zweiten Doppelsumme erh"alt man
    \begin{equation}
    \label{Berk84Equ2}
        - \sum_{k=0}^{n-2} \lambda^{k} \sum_{l=k+1}^{n-1} M^{l-k} q_l
        + \sum_{k=1}^{n-1} \lambda^{k} \sum_{l=k+1}^{n-1} M^{l-k} q_l
    \end{equation}
    Nach Satz \ref{SatzCayleyHamilton} gilt
    \begin{equation}
    \label{Berk84Equ3}
        \sum_{l=0}^{n-1} M^l q_l = 0
    \end{equation}
    Somit kann man die linke Seite von Gleichung \equref{Berk84Equ3} zur
    zweiten Doppelsumme von Term \equref{Berk84Equ2} addieren und erh"alt
    \[
        - \sum_{k=0}^{n-2} \lambda^{k} \sum_{l=k+1}^{n-1} M^{l-k} q_l
        + \sum_{k=0}^{n-1} \lambda^{k} \sum_{l=k+1}^{n-1} M^{l-k} q_l
    \]
    Wenn man nun die Vorzeichen der beiden Doppelsummen sowie
    die benutzten Indizes betrachtet, erkennt man, da"s sich der Gesamtterm
    vereinfacht darstellen l"a"st, da gro"se Teile zusammengenommen $0$
    ergeben. Die Teile, die sich nicht auf diese
    Weise gegenseitig aufheben, lassen sich schreiben als
    \[ \sum_{k=0}^{n-1} \lambda^k E_{n-1} q_{k} \MyKomma \]
    was gleichbedeutend ist mit
    \[ q(\lambda) * E_{n-1} \MyPunkt \]

    Also stimmen die beiden Seiten von Gleichung \equref{Berk84Equ1}
    "uberein.
\end{beweis}

Die beiden Lemmata \ref{Berk84Satz1} und \ref{Berk84Satz2} f"uhren zu
folgendem Satz \cite{Samu42}:

\begin{satz}[Samuelson]
\label{SatzSamuelson}
\index{Samuelson!Satz von}
% $$$ Claim 2 into Claim 1
    \begin{equation}
    \label{EquSatzSamuelson}
        p(\lambda) =
            (a_{1,1} - \lambda) * \det(M - \lambda * E_{n-1})
            + R * \left(
              \sum_{k=0}^{n-2} \lambda^{k} \sum_{l=k+1}^{n-1} M^{l-k-1} q_l
            \right) * S
    \end{equation}
\end{satz}
\begin{beweis}
    Lemma \ref{Berk84Satz2} angewendet auf Lemma \ref{Berk84Satz1}
    ergibt die Behauptung.
\end{beweis}

% **************************************************************************

\MySection{Determinantenberechnung mit Hilfe des Satzes von Samuelson}
\label{SecAlgBerk}
\index{Berkowitz!Algorithmus von}
\index{Algorithmus!von Berkowitz}

Um Satz \ref{SatzSamuelson} zur Determinantenberechnung zu benutzen
\cite{Berk84}, sind weitere "Uberlegungen notwendig, die in diesem
Unterkapitel behandelt werden.

Betrachtet man die Methodik des entstehenden Algorithmus, erkennt man
"Ahnlichkeit zu C-Alg. . Auch dort wird ein schon l"anger bekannter Satz
mit Hilfe von zus"atzlichen "Uberlegungen f"ur eine parallelen Algorithmus
verwendet.

Zu beachten ist, da"s in diesem Unterkapitel f"ur die Multiplikation 
zweier $n \times n$-Matrizen $n^{2+\gamma}$ Prozessoren in Rechnung 
gestellt werden (vgl. S. \pageref{PageAlg2MatMult}).

Benutzt man die Koeffizientendarstellung f"ur die charakteristischen
Polynome von $A$ und $M$, l"a"st sich Gleichung
\equref{EquSatzSamuelson} umformulieren in
\[
   \sum_{i=0}^n p_i \lambda^i =
       (a_{1,1} - \lambda) * \sum_{i=0}^{n-1} q_i \lambda^i
       + R * \left(
         \sum_{k=0}^{n-2} \lambda^{k} \sum_{l=k+1}^{n-1} M^{l-k-1} q_l
       \right) * S  
   \MyPunkt
\]
Vergleicht man die Koeffizienten der $\lambda^i$ auf beiden Seiten
der Gleichung und definiert \[ q_{-1} := 0 \MyKomma \] erh"alt man
\begin{eqnarray}
    p_n     & = & -q_{n-1}                 \label{Equ1Berk84KoeffVergl}
\\  p_{n-1} & = & a_{1,1}q_{n-1} - q_{n-2} \label{Equ2Berk84KoeffVergl}
\\  \forall i=n-2 \ldots 0 : \: p_i & = &  \label{Equ3Berk84KoeffVergl}
        a_{1,1}q_i-q_{i-1}+\sum_{j=i+1}^{n-1}RM^{j-i-1}S q_i
\end{eqnarray}
Die Beziehungen zwischen den Koeffizienten, die diese Gleichungen
beschreiben, kann man auch durch eine Matrizengleichung 
ausdr"ucken. Dazu
wird Matrix $C_t$ definiert als untere Dreiecks-Toeplitz-Matrix der
Gr"o"se $(n-t+2) \times (n-t+1)$. Ihre Elemente werden definiert durch
\[ (c_t)_{i,j} :=
       \left\{
           \begin{array}{lcr}
               -1                       & : & i=1
            \\ a_{t,t}                  & : & i=2
            \\ R_t M_t^{i-3} S_t        & : & i>2
           \end{array}
       \right.
\]
Die Matrix hat also das folgende Aussehen:
\[
    \left[ \begin{array}{ccc}
        -1                   & 0         & \cdots \MatStrut
    \\  a_{t,t}              & -1        & \ddots \MatStrut
    \\  R_t S_t              & a_{t,t}   & \ddots \MatStrut
    \\  R_t M_t S_t          & R_t S_t   & \ddots \MatStrut
    \\  \vdots               & \ddots    & \ddots \MatStrut
    \\  R_t M_t^{n-t-1} S_t  &           &        \MatStrut
    \end{array} \right]
\] \MyPunktA{30em} 
Insbesondere hat $C_n$ die Form
\[ 
    \left[ \begin{array}{c}
    -1 \\ a_{n,n}
    \end{array} \right]
\]

Mit Hilfe dieser Definition erh"alt man aus den Gleichungen
\equref{Equ1Berk84KoeffVergl}, \equref{Equ2Berk84KoeffVergl} und
\equref{Equ3Berk84KoeffVergl} die folgende Matrizengleichung:
\Beq{Berk84Equ19}
   \left[ \begin{array}{c}
       p_n \\
       p_{n-1} \\
       \vdots \\
       p_0
   \end{array} \right]
   =
   C_1
   \left[ \begin{array}{c}
       q_{n-1} \\
       q_{n-2} \\
       \vdots \\
       q_0
   \end{array} \right]
\Eeq
Auf die gleiche Weise, wie man 
Satz \ref{SatzSamuelson} auf die Matrizen $A$ und $M$ anwendet, kann man 
diesen Satz auch auf die Matrizen $M$ und $M_2$, $M_2$ und $M_3$, etc.
anwenden und erh"alt so Matrizengleichungen, die in ihrer Form der 
Gleichung \equref{Berk84Equ19} entsprechen. 

Wendet man diese Matrizengleichungen aufeinander an, erh"alt man:
\Beq{EquProdCi}
   \left[ \begin{array}{c}
       p_n \\
       p_{n-1} \\
       \vdots \\
       p_0
   \end{array} \right]
   =
   \prod_{i=1}^{n} C_i
\Eeq
Um die Koeffizienten des charakteristischen Polynoms von $A$ auf die
geschilderte Weise zu berechnen, mu"s man also die Matrizen $C_i$
berechnen und dann miteinander multiplizieren. Nach
\ref{SatzDdurchP} ist damit auch die Determinante der Matrix $A$ berechnet.

F"ur jede  $(n-i+2) \times (n-i+1)$-Matrix $C_i$ bei ist 
der $(n-i)$-elementige Vektor
\Beq{EquRMSVektor}
    T_t := [ R_i S_i, \, R_i M_i S_i, \, R_i M_i^2 S_i, \, 
    \ldots , \, R_i M_i^m S_i ], \: m:= n-i-1
\Eeq
zu berechnen. Da also $T_n$ keine Elemente enth"alt, ist die Berechnung
der Vektoren $T_1$ bis $T_{n-1}$ erforderlich.

Man kann jeden Exponenten $k$ eines Elementes \[ R_i * M_i^k * S_i \] von 
$T_i$ in der Form \[ k = u + v * \left\lceil \sqrt{m} \right\rceil \] mit
\begin{eqnarray*}
    & 0 \leq u < \left\lceil \sqrt{m} \right\rceil &
\\  & 0 \leq v \leq \left\lfloor \sqrt{m} \right\rfloor &
\end{eqnarray*}
eindeutig darstellen. Man k"onnte statt $\sqrt{m}$ auch einen anderen
Wert zwischen $0$ und $m$ nehmen. Jedoch f"uhrt die Wahl von $\sqrt{m}$ 
dazu, da"s sich die Gr"o"se der Mengen aller $u$ und $v$ um h"ochstens $1$
unterscheidet.

Um $T_i$ effizient zu erhalten, kann man zun"achst die den Mengen der $u$
und $v$ entsprechenden Vektoren
\[
   U_i := \left[ R_i,\, R_i M_i,\, R_i M_i^2,\, 
                 \ldots ,\, R_i M_i^{\lc \sqrt{m} \rc - 1} 
          \right]
\]
und
\[
   V_i := \left[ S_i,\, M_i^{\lc\sqrt{m}\rc} S_i,\, 
                    M_i^{2\lc\sqrt{m}\rc} S_i,\,
                \ldots,\, M_i^{\lf\sqrt{m}\rf \lc\sqrt{m}\rc} S_i
          \right]
\]
berechnen und danach jedes Element des einen Vektors mit jedem Element
des anderen multiplizieren.

Genau genommen werden auf diese Weise einige Werte zuviel berechnet, wie
sich bei noch exakterer Analyse des Algorithmus zeigt. Es sind jedoch
vernachl"assigbar wenige. Die Berechnung dieser Werte kann durch
vernachl"assigbar geringen zus"atzlichen Aufwand verhindert werden. Um die
Darstellung des Algorithmus nicht unn"otig un"ubersichtlich zu machen,
werden diese Werte nicht weiter beachtet.

Vor Beginn der Rechnung wird ein \label{PageWahlEpsilon} 
\[ \epsilon \in \Rationals \, , \: 0 < \epsilon \leq 0.5 \]
festgelegt\footnote{ein Wert $\epsilon>0.5$ ist m"oglich, jedoch von
                     seinen Auswirkungen her uninteressant}.
O. B. d. A. sei $\epsilon$ so gew"ahlt, da"s 
gilt\footnote{erf"ullt $\epsilon$ diese Bedingung nicht, wird dadurch
              die Analyse des Algorithmus unn"otig un"ubersichtlich}
\[
   \exists \, p \in \Nat : \: p * \epsilon = 0.5 \MyPunkt
\]

Die Wahl von $\epsilon$ beeinflu"st das Verh"altnis zwischen der Anzahl
der Schritte und der Anzahl der dabei besch"aftigten Prozessoren.
Dies wird weiter unten durch die Analyse deutlich.

F"ur den Rest dieses Unterkapitels gelte die Vereinbarung, da"s mit
\[ a^b \] der Wert \[ \lc a^b \rc \] gemeint ist.

Mit Hinweis auf die Bemerkungen im Anschlu"s an die Behandlung der
Matrizenmultiplikation in Satz \ref{SatzAlgMatMult} wird im folgenden
f"ur die Multiplikation zweier $n \times n$-Matrizen ein Aufwand von
\[ \gamma_S (\lc \log(n) \rc + 1) \] Schritten und
\[ \gamma_P n^{2+\gamma} \] Prozessoren in Rechnung gestellt.

Im folgenden ist mit $T$, $U$ und $V$ jeweils $T_i$, $U_i$ bzw. $V_i$
gemeint, wobei $1 \leq i < n$ gilt.

Um den Vektor $U$ zu berechnen, benutzen wir folgenden iterativen 
Algorithmus\footnote{zur Vereinfachung der Darstellung werden keine 
ganzzahligen Werte zur Indizierung benutzt}: \nopagebreak[3]
\begin{itemize}
\item
      Der Vektor $Z_\alpha$ wird wie folgt definiert:
      \[ 
         Z_\alpha := \left[ R_i,\, R_i M_i,\, R_i M_i^2,\, 
                           \ldots ,\, R_i M_i^{m^\alpha - 1} 
                     \right]
      \]
      Das bedeutet, es gilt
      \[
          Z_0 = [ R_i ]
      \]
      Das Ziel ist es, $Z_{0.5}= U$ zu berechnen. 
      Der Vektor $Z_0$ ist bekannt, da $R_i$ Teil der Eingabe ist.
\item
      Wenn $Z_{\alpha}$ bekannt ist, im ersten Schleifendurchlauf
      also $Z_0$, dann wird daraus $Z_{\alpha+\epsilon}$ wie folgt 
      berechnet:
      \begin{itemize}
      \item
            Berechne
            \[ Y_{\alpha+\epsilon} :=
               \left[
                   M_i^{m^\alpha},\, M_i^{2m^\alpha},\, M_i^{3m^\alpha},
                      \, \ldots,\,
                   M_i^{m^\epsilon m^\alpha}
               \right]
            \]
            Nach \ref{SatzAlgPraefix} in Verbindung mit \ref{SatzAlgMatMult}
            erh"alt man f"ur die Anzahl der Schritte
            \begin{eqnarray*}
               & & \gamma_S \lceil \log(m^\epsilon) \rceil 
                   (\lceil \log(m) \rceil +1)
            \\ & \leq & %  <  ist hier falsch !
                        \gamma_S \lceil (\epsilon \lceil \log(m) \rceil + 1)
                            (\lceil \log(m) \rceil + 1 )
                     \rceil
            \\ & = & \gamma_S \lc \epsilon \lceil \log(m) \rceil^2 +
                         (\epsilon + 1) \lceil \log(m) \rceil + 1
                     \rc
            \end{eqnarray*}
            und f"ur die Anzahl der
            Prozessoren 
            \[ \gamma_P \lf 0.75 m^\epsilon \rf m^{2+\gamma}
                   <
               \gamma_P m^{2+\gamma+\epsilon} \MyPunkt
            \]
            Die
            daf"ur n"otige Startmatrix $M_i^\alpha$ erh"alt man als
            Nebenergebnis aus der Berechnung von $Y_\alpha$. Die Startmatrix
            f"ur die Berechnung von $Y_\epsilon$ ist $M_i$.
      \item
            Der Vektor $X_{\alpha+\epsilon}$ wird folgenderma"sen definiert:
            \[
               X_{\alpha+\epsilon} := 
               \left[
                   R_i M_i^{m^\alpha}, \,R_i M_i^{m^\alpha + 1}, \,
                   R_i M_i^{m^\alpha + 2}, \, \ldots, 
                   \, R_i M_i^{m^{\alpha+\epsilon}-1}
               \right]
            \]
            Es wird nun $X_{\alpha+\epsilon}$ aus $Z_\alpha$ und 
            $Y_{\alpha+\epsilon}$ berechnet.

            Der Vektor
            $Z_\alpha$ besitzt $m^\alpha$ Elemente, die ihrerseits Vektoren
            der L"ange $m$ darstellen. Sie werden mit
            \[
               z_{\alpha,1},\, z_{\alpha,2},\, \ldots,\, z_{\alpha,m^\alpha}
            \] bezeichnet.
            Der Vektor $Y_{\alpha+\epsilon}$ besitzt $m^\epsilon$
            Elemente. Diese Elemente sind $m \times m$-Matrizen und werden
            mit
            \[ y_{\alpha+\epsilon,1},\, y_{\alpha+\epsilon,2},\, \ldots,\,
               y_{\alpha+\epsilon,m^\epsilon}
            \] bezeichnet.
           
            \begin{tabbing}
                Der Vektor $X_{\alpha+\epsilon}$ wird wie folgt
                berechnet: \\
                    \hspace{1.5em} \= \hspace{1.5em} \= \kill 
                \> Parallel f"ur $i:= 1$ bis $m^\epsilon-1$ : \\
                \> \>  Parallel f"ur $j:= 1$ bis $m^\alpha$:
            \end{tabbing}
            \vspace{-4ex}
            \[
               x_{ \alpha+\epsilon,(i-1)*m^\epsilon+j}
                 := z_{\alpha,j} * y_{\alpha+\epsilon,i}
            \]
            Bei dieser Berechnung f"allt auf, da"s 
            $y_{\alpha+\epsilon,m^\epsilon}$ nicht verwendet wird. Diese 
            Matrix bildet die Startmatrix f"ur die Berechnung von
            $Y_{\alpha+2\epsilon}$ im n"achsten Schleifendurchlauf
            (s. o.).

            F"ur die Analyse des Aufwandes der Berechnung von 
            $X_{\alpha+\epsilon}$ wird $Z_\alpha$ als Matrix betrachtet.
            Die $z_{\alpha,j}$ bilden die Zeilenvektoren dieser Matrix.
            So gesehen sind also $m^\epsilon$ Matrizenmultiplikationen 
            durchzuf"uhren. Dies kann von 
            \[ \gamma_P m^{2+\gamma+\epsilon} \]
            Prozessoren in 
            \[ \gamma_S \lceil \log(m) \rceil + 1 \] 
            Schritten durchgef"uhrt werden.
      \item 
            Die ersten $m^\alpha$ Elemente des in diesem 
            Schleifendurchlauf gesuchten Vektors $Z_{\alpha+\epsilon}$
            werden durch die Elemente des Vektors $Z_\alpha$ gebildet und
            alle weiteren durch die Elemente des soeben berechneten 
            Vektors $X_{\alpha+\epsilon}$. 
      \end{itemize}
      Betrachtet man den Aufwand zur Berechnung von $Y_{\alpha+\epsilon}$
      und $X_{\alpha+\epsilon}$ zusammen, erh"alt man f"ur die
      Berechnung von $Z_{\alpha+\epsilon}$ aus $Z_{\alpha}$
      \[ 
         \gamma_S \lc \epsilon \lceil \log(m) \rceil^2 +
            (\epsilon + 2) \lceil \log(m) \rceil + 2                      
         \rc
      \]
      Schritte und \[ \gamma_P m^{2+\gamma+\epsilon} \] Prozessoren.
\item
      Insgesamt erfolgen \[ \frac{1}{2 \epsilon} \]
      Schleifendurchl"aufe. Der Aufwand zur Berechnung von $U$ betr"agt
      deshalb
      \begin{eqnarray*}
         & & \frac{ 0.5 \gamma_S }{ \epsilon }
         \lc \epsilon \lceil \log(m) \rceil^2 +
            (\epsilon + 2) \lceil \log(m) \rceil + 2                      
         \rc
      \\ 
         & \leq & 0.5 \gamma_S
         \lc \lceil \log(m) \rceil^2 +
            \left( 1 + \frac{2}{\epsilon} \right) \lceil \log(m) \rceil + 
            \frac{2}{\epsilon}
         \rc
      \end{eqnarray*} Schritte und
      \[ \gamma_P m^{2+\gamma+\epsilon} \] Prozessoren.
\end{itemize}

Im Anschlu"s an die Berechnung von $U$ erfolgt
die Berechnung von $V$ auf die gleiche Weise.
Der einzige wesentliche Unterschied zwischen den beiden
Berechnungsvorg"angen ist die andere
Startmatrix zur Berechnung des $Y_{\epsilon}$ entsprechenden Vektors. Hier
wird $M_i^{m^{0.5}}$ statt $M_i$ ben"otigt. Man erh"alt $M_i^{m^{0.5}}$
aus $M_i$ mit Hilfe der Bin"arbaummethode
nach \ref{SatzAlgBinaerbaum}. Dies kann in
\begin{eqnarray*}
    &      & \gamma_S \lc \log(m^{0.5}) \rc (\lc \log(m) \rc + 1)
\\  & \leq & \gamma_S \lc 0.5 \lc \log(m) \rc (\lc \log(m) \rc + 1) \rc
\\  & =    & \gamma_S \lc 0.5 (\lc \log^2(m) \rc + \lc \log(m) \rc + 1)  \rc
\end{eqnarray*}
Schritten von
\begin{eqnarray*}
    &      & \gamma_P \lc 0.5 m^{0.5} m^{2+\gamma} \rc
\\  & \leq & \gamma_P \lc 0.5 m^{2.5+\gamma} \rc
\end{eqnarray*}
Prozessoren geleistet werden.
Ist die Startmatrix berechnet, ist der weitere Aufwand zur Berechnung von
$V$ gleich dem Aufwand zur Berechnung von $U$. Also kann $V$ insgesamt 
in\footnote{Da die Terme, die die Anzahl der Schritte und Prozessoren
    beschreiben, bereits nach oben abgesch"atzt sind, wird bei der
    Zusammenfassung von Termen, die durch Gau"sklammern eingefa"st sind,
    auf eine weitere Absch"atzung verzichtet.}
\begin{eqnarray*}
   & & \gamma_S \lc 0.5 (\lc \log(m) \rc^2 + \lc \log(m) \rc + 1) +
           0.5  \left( \lceil \log(m) \rceil^2 +
           \left( 1 + \frac{2}{\epsilon} \right) \lceil \log(m) \rceil +
           \frac{2}{\epsilon} \right)
       \rc
\\ & = & \gamma_S
     \lc
         \lc \log(m) \rc^2 + 
         \left( 1 + \frac{1}{\epsilon} \right) \lc \log(m) \rc + 
         \frac{1}{\epsilon} + 0.5
     \rc
\end{eqnarray*}
Schritten erledigt werden. Die Anzahl der Prozessoren betr"agt
\begin{eqnarray*}
    \max \left( 
              \underbrace{ \gamma_P m^{2+\gamma+\epsilon} }_{\mbox{Term 1}}
         \, ,
              \underbrace{ \gamma_P \lc 0.5 m^{2.5+\gamma} \rc 
                         }_{\mbox{Term 2}}
         \right) \MyPunkt
\end{eqnarray*}
Da mit steigendem $m$ Term 2 st"arker w"achst als Term 1, wird die 
Analyse mit Term 2 f"ur die Anzahl der Prozessoren fortgesetzt.

Parallel zur Berechnung von $U$ wird zuerst
$M_i^{m^{0.5}}$ und mit Hilfe dieser Matrix dann $V$ berechnet. Der Aufwand
daf"ur betr"agt insgesamt
\[   \gamma_S
     \lc
         \lc \log(m) \rc^2 + 
         \left( 1 + \frac{1}{\epsilon} \right) \lc \log(m) \rc + 
         \frac{1}{\epsilon} + 0.5
     \rc
\] 
Schritte und
\[ 
   \gamma_P \lb m^{2+\gamma+\epsilon} + \lc 0.5 m^{2.5+\gamma} \rc \rb
\] Prozessoren.

Um den nach \equref{EquRMSVektor} gesuchten Vektor $T$ zu erhalten, 
m"ussen 
noch die Elemente der Vektoren $U$ und $V$, die ja ihrerseits wiederum 
Vektoren darstellen, miteinander multipliziert werden.
Die Vektoren $U$ und $V$ besitzen eine L"ange von $m^{0.5}$.
Die Multiplikation zweier Elemente dieser Vektoren k"onnen
analog zur Matrizenmultiplikation in \ref{SatzAlgMatMult} in 
\[ \lc \log(m) \rc + 1 \] Schritten von \[ m \] Prozessoren erledigt 
werden. Insgesamt sind \[ m^{0.5} * m^{0.5} = m \] solcher Multiplikationen
durchzuf"uhren. Die Berechnung von $T$ aus $U$ und $V$ kann
also in 
\[ \lc \log(m) \rc + 1 \] Schritten von \[ m^2 \] Prozessoren 
durchgef"uhrt werden.

Betrachtet man den Gesamtaufwand zur Berechnung von $U$, $V$ und $T$,
kommt man auf
\[ \gamma_S
   \lc
       \lc \log(m) \rc^2
       + \left( 1 + \frac{1}{\gamma_S} 
       + \frac{1}{\epsilon} \right) \lc \log(m) \rc
       + \frac{1}{\epsilon} + \frac{1}{\gamma_S} + 0.5
   \rc
\] 
Schritte und  
\Beq{EquBerkProzT}
   \gamma_P \lb m^{2+\gamma+\epsilon} + \lc 0.5 m^{2.5+\gamma} \rc \rb
      \leq
   \gamma_P \lb m^{2+\gamma+\epsilon} + 0.5 m^{2.5+\gamma} + 1 \rb
\Eeq
Prozessoren. 

Nach der
obigen Analyse der Berechnung einer der Vektoren kann die parallele
Berechnung aller Vektoren $T_1$ bis $T_{n-1}$ in
\Beq{TermBerkSchritte}
   \gamma_S
   \lc
       \lc \log(n-2) \rc^2 
       + \left( 1 + \frac{1}{\gamma_S}
       + \frac{1}{\epsilon} \right) \lc \log(n-2) \rc + 
       \frac{1}{\epsilon} + \frac{1}{\gamma_S} + 0.5
   \rc
\Eeq
Schritten durchgef"uhrt werden. Da die Berechnung eines Vektors $T_i$ f"ur 
$i>1$ bei gleichem $\epsilon$ schneller ist als die Berechnung von 
$T_1$, ist es m"oglich, dadurch Prozessoren zu sparen, da"s man 
$\epsilon$ f"ur jeden Vektor $T_i$ verschieden w"ahlt, und zwar als Funktion
von 
\begin{itemize}
\item
      der Gr"o"se $n$ der Eingabematrix $A$,
\item 
      der L"ange $m+1$ des jeweiligen Vektors $T_i$ und
\item
      dem $\epsilon$, da"s zur Berechnung des Vektors $T_1$
      verwendet wird.
\end{itemize}
Das separat f"ur jeden Vektor $T_i$ zu w"ahlende $\epsilon$ wird 
mit\footnote{ Es wurde bereits definiert: $m:= n-i-1$.}
$\epsilon_m$ bezeichnet.

Da die Vektoren $T$ f"ur $m \leq n-2$ berechnet werden sollen, mu"s f"ur
jedes $\epsilon_m$ mit $\epsilon_m \neq \epsilon$ die 
Bedingung $m \leq n-3$ erf"ullt sein.
Da gleichzeitig $m \geq 1$ erf"ullt sein mu"s, wird f"ur die folgenden
Analysen $n \geq 4$ angenommen. Andernfalls ist die Anwendung der Idee
zur Wahl der $\epsilon_m$ nicht sinnvoll.

Wie $\epsilon_m$ zu w"ahlen ist, ergibt sich aus
Term \equref{TermBerkSchritte}. Es mu"s gelten:
\[
   \lc \log(m) \rc^2 
   + \left( 1 + \frac{1}{\gamma_S} 
   + \frac{1}{\epsilon_m} \right) \lc \log(m) \rc +
   \frac{1}{\epsilon_m}
       \leq
   \lc \log(n-2) \rc^2 
   + \left( 1 + \frac{1}{\gamma_S} 
   + \frac{1}{\epsilon} \right) \lc \log(n-2) \rc +
   \frac{1}{\epsilon}
\]
L"ost man diese Ungleichung nach $\epsilon_m$ auf erh"alt man:
\Beq{EquWaehleEpsilonM}
   \epsilon_m
       \geq
   \frac{
       \lceil \log(m) \rceil + 1
   }{
       \lceil \log(n-2) \rceil^2 +
       \lb 1 + \frac{1}{\gamma_S} 
              + \frac{1}{\epsilon} 
       \rb \lceil \log(n-2) \rceil +
       \frac{1}{\epsilon} -
       \lceil \log(m) \rceil^2 -
       \lb 1 + \frac{1}{\gamma_S} \rb \lceil \log(m) \rceil
   }
\Eeq
Die Gau"sklammern in dieser Ungleichung f"uhren zu einigen wichtigen
Konsequenzen f"ur $n$, $m$ und $\epsilon_m$. Es gelte dazu 
\begin{eqnarray*}
    & k \in \Nat & \\
    & 1 \leq 2^k < m_1 \leq 2^{k+1} \leq n-2 & \\
    & 1 \leq 2^k < m_2 \leq 2^{k+1} \leq n-2 & \MyPunkt
\end{eqnarray*}
Aus \equref{EquWaehleEpsilonM} folgt dann
\[
    \epsilon_{m_1} = \epsilon_{m_2} \MyPunkt
\]
Das bedeutet insbesondere, da"s es u. U. einige $m$ mit 
$m \leq n-3$ gibt, f"ur die gilt 
\[ \epsilon_m = \epsilon \MyPunkt \] Der ung"unstigste Fall tritt f"ur
\[ n-2 = 2^{k+1} \] ein. Bei diesem Fall ist nur f"ur 
\[ m \leq \frac{n-2}{2} \]
die Bedingung \[ \epsilon_m < \epsilon \] erf"ullt.

F"ur die weitere Analyse ist es an dieser Stelle sinnvoll, die Gau"sklammern
im Term auf der rechten Seite von \equref{EquWaehleEpsilonM} zu beseitigen. 
Dazu wird
der Term nach oben abgesch"atzt.

Terme, die in Gau"sklammern eingefa"st sind, kann man mit Hilfe der
Beziehung
\Beq{EquSchaetzeGauss}
    a \leq \lc a \rc \leq a + 1
\Eeq
absch"atzen. Es gilt jedoch
\begin{eqnarray*}
   &      & \lc \log(x) \rc \\
   & \leq & \log(x) + 1     \\
   & =    & \log(2x) \MyPunkt
\end{eqnarray*} 
Zu beachten sind hier die Konsequenzen, wenn die Absch"atzung mit Hilfe
von \equref{EquSchaetzeGauss} vorgenommen werden. Falls $n-2$ eine 
Zweierpotenz ist, ergibt die auf diese Weise abgesch"atzte Ungleichung 
\equref{EquWaehleEpsilonM} nur
f"ur \[ m \leq \frac{n-2}{4} \] Werte f"ur $\epsilon_m$, so da"s 
\[ \epsilon_m < \epsilon \MyPunkt \]
Eine Verbesserung dieser Absch"atzung der Gau"sklammerfunktion ist 
w"unschenswert.

Dazu wird definiert, da"s eine Funktion $f(x)$
\index{konkav} {\em konkav auf einem Intervall I} ist, falls f"ur ihre
zweite Ableitung $f''(x)$ gilt:
\[ \forall x \in I: \: f''(x) \leq 0 \MyPunkt \]

Soll die Gau"sklammer einer konkaven Funktion $h(x)$ gebildet und die
Fl"ache unter der resultierenden Kurve berechnet werden, so l"a"st sich
der Ausdruck auch durch
\[ \int \lc h(x) dx \rc \leq \int (h(x) + 0.5) dx \]
nach oben absch"atzen. In Abbildung \ref{PicKonkav} ist dies verdeutlicht.
Dort ist Fl"ache 1 gr"o"ser als Fl"ache 2.
\begin{figure}[htb]
\begin{center}
    \input{bilder/konkav}
    \caption{Integration der Gau"sklammer einer konkaven Funktion}
    \label{PicKonkav}
\end{center}
\end{figure}
Somit kommen wir auf
\begin{eqnarray}
    & & \int \lc \log(x) \rc dx \nonumber \\
    & \leq & \int (\log(x) + 0.5 ) dx \label{EquLogNullFuenf} \\
    & = & \int \log(\sqrt{2}x) dx \nonumber \MyPunkt
\end{eqnarray}
Ist der abzusch"atzende Gau"sklammerterm Teil einer Funktion 
\[ h_2(\lc h(x) \rc,\, x ) \MyKomma \] so l"a"st sich die beschriebene
Absch"atzung durchf"uhren, falls $h_2$ monoton ist. Diese Bedingung ist bei 
den folgenden Anwendungen erf"ullt.

F"ur die folgenden Untersuchungen wird die Funktion
\[ g : \, \Rationals \rightarrow \Rationals \] eingef"uhrt. Sie sei
{ \em im Riemann'schen Sinne integrierbar } \cite{BS87} (S. 289)
auf dem Intervall \[ ( - \infty, \infty ) \] und
wird als Platzhalter f"ur die Funktion verwendet, die schlie"slich zur 
Absch"atzung der Gau"sklammerfunktion benutzt wird.
Je genauer sie die Gau"sklammerfunktion absch"atzt, umso besser werden
die Analyseergebnisse.

Mit Hilfe von \equref{EquWaehleEpsilonM} wird folgende Funktion zur 
Berechnung von $\epsilon_m$ bei gegebenen $n$ und $\epsilon$ definiert:
\[ f(m):=
   \frac{
       g(\log(m))+ 1
   }{
       g^2(\log(n-2))
       + \lb 1 + \frac{1}{\gamma_S} 
               + \frac{1}{\epsilon} 
         \rb g(\log(n-2))
       + \frac{1}{\epsilon} -
       g^2(\log(m))-
       \lb 1 + \frac{1}{\gamma_S} \rb g(\log(m))
   }
\]
Mit Hilfe dieser Funktion kommt man anhand von \equref{EquBerkProzT}
f"ur die Anzahl der Prozessoren zur Berechnung aller Vektoren $T$ auf:
\begin{eqnarray}
\nonumber
   & & 
   4 + (n-2)^{2+\gamma+\epsilon} + \frac{ (n-2)^{2.5+\gamma} }{ 2 } +
   \sum_{m=2}^{n-3}
       \left( m^{2+\gamma+f(m)} + \frac{ m^{2.5+\gamma} }{ 2 } + 1
       \right)
\\ 
\nonumber
   & \leq & 
   \overbrace{
       4 + (n-2)^{2+\gamma+\epsilon} + \frac{ (n-2)^{2.5+\gamma} }{ 2 }
   }^{t_1:=}
       +
   \int\limits_{2}^{n-2}
       \left( m^{2+\gamma+f(m)} + \frac{ m^{2.5+\gamma} }{ 2 } + 1
       \right) dm
\\ 
%%\label{EquBerkProzInt1} wird nicht ben"otigt
   & = &
       \overbrace{ 
           t_1 + \left. m \right|_2^{n-2} +
           \left. \left(
               \frac{1}{7+2\gamma} m^{3.5+\gamma}
           \right) \right|_2^{n-2} 
       }^{t_2:=}
       +
       \int\limits_2^{n-2}
           m^{2+\gamma+f(m)} dm
\\
\label{EquBerkProzInt2}
   & = & t_2 +
       \int\limits_2^{n-2}
       \overbrace{
           \MathE^{(2+\gamma+f(m))\ln(m)}
       }^{t_3:=} \, dm
\end{eqnarray}
F"ur die Integration von $t_3$ sind 3 Methoden von Bedeutung:
\begin{MyDescription}
\MyItem{Numerische Berechnung}
    Diese Methode ist f"ur gegebene $n$ und $\epsilon$ eine gangbare
    M"oglichkeit \cite{EM88} (Kap. 9). F"ur eine allgemeine 
    Analyse ist sie jedoch nicht geeignet.
\MyItem{Analytische Berechnung}
    Bezeichne $t_4$ die erste Ableitung von 
    \[ (2+\gamma+f(m))\ln(m) \MyPunkt \]
    Bezeichne $t_5'$ die erste Ableitung des zu bestimmenden Terms $t_5$.
    Da $t_3$ eine Exponentialfunktion ist, mu"s die gesuchte Stammfunktion,
    die Form $t_3 t_5$ besitzen.
    Die Ableitung dieses Ausdrucks ergibt 
    \[ (t_3 t_5)'= t_3 t_4 t_5+t_3 t_5' = t_3 \,
       \underline{ (t_4 t_5 + t_5') } \MyPunkt 
    \]
    Da der unterstrichene Teil den Wert 1 besitzen mu"s, ist die 
    Differentialgleichung \[ t_5' = 1 - t_4 t_5 \] zu l"osen. 
    Dies ist eine {\em explizite gew"ohnliche Differentialgleichung 
    erster Ordnung} \cite{BS87} (S. 414 ff.).

    Man gelangt zu der Vermutung, da"s zu $t_3$ keine Stammfunktion 
    existiert, da sowohl die Integration der Differentialgleichung, 
    als auch die Integration von $t_3$ mit Hilfe der Eigenschaften
    unbestimmter Integrale \cite{BS87} (S. 295) nicht zu 
    einem Ergebnis zu f"uhren scheinen. Unterst"utzt wird diese Vermutung
    durch die Tatsache, da"s f"ur \[ \MathE^{m^2} \] keine Stammfunktion 
    existiert.
\MyItem{Absch"atzung der Stammfunktion nach oben}
    Diese Methode ist am besten geeignet und wird im folgenden benutzt.
    Dazu wird der oben erw"ahnte Term $t_5$ so bestimmt, da"s gilt
    \Beq{EquBerkUngleichung}
        t_4 t_5 + t_5' \geq 1 \MyPunkt
    \Eeq 
\end{MyDescription}

Der Nenner von $f(m)$ wird mit $t_6$ bezeichnet, der Z"ahler mit $t_7$.
Die Ableitung von $t_3$ ergibt:
\begin{eqnarray*}
   \lefteqn{ 
       t_3' = t_3 
       \lb
           \frac{2}{m} + \frac{\gamma}{m}
       \rb
   } \\
   & + &
       \frac{
           \lb 
               g'(\log(m)) \frac{\log(m)}{m} + \frac{ g(\log(m)) }{m} 
               + \frac{1}{m} 
           \rb t_6
       }{ t_6^2 
       } \\
   & + &
       \frac{
           t_7 \ln(m) 
           \lb \frac{ 2 g(\log(m)) g'(\log(m)) }{m \ln(2) }
               + \frac{ \lb 1 + \frac{1}{\gamma_S} \rb
                        g'(\log(m)) 
                 }{ m \ln(2) }
           \rb
       }{ t_6^2
       } 
\end{eqnarray*}
Wie bereits beschrieben wurde, ist es an dieser Stelle m"oglich
\Beq{EquDefGaussNullFuenf}
    g(x) := x + 0.5
\Eeq
als obere Absch"atzung der Gau"sklammerfunktion zu verwenden. Es ist zu
beachten, da"s diese Absch"atzung nicht f"ur die Berechnung von $\epsilon_m$
bei der Anwendung des Algorithmus in einer konkreten Situation
benutzt werden darf. Die Benutzung von \equref{EquDefGaussNullFuenf} an
dieser Stelle ist nur zul"assig, weil die Absch"atzung des gesamten 
Integrals in \equref{EquBerkProzInt2} das Ziel ist.

Wie ebenfalls bereits beschrieben wurde, ist darauf zu achten, da"s die
auftretenden Werte f"ur $\epsilon_m$ kleiner oder gleich $\epsilon$ sind.
Damit dies der Fall ist muss wegen der mit \equref{EquDefGaussNullFuenf} 
gew"ahlten Absch"atzung gelten:
\begin{MyEqnArray}
    \MT  \log( \sqrt{2}m ) \MT \leq \MT \lc \log( n - 2 ) \rc 
\MNl
    \Rightarrow \MT \log(m) \MT \leq \MT
        \lc \log \lb \frac{ n - 2 }{ \sqrt{2} } \rb \rc
\MNl
    \Rightarrow \MT m \MT \leq \MT 
        2^{ \lc \log \lb \frac{ n - 2 }{ \sqrt{2} } \rb \rc }
\end{MyEqnArray}

Ungleichung
\equref{EquBerkUngleichung} wird erf"ullt, wenn man 
\[ t_5 = m \] w"ahlt. Die G"ultigkeit dieser Behauptung ergibt
sich insbesondere aus
der Betrachtung der Gr"o"senordnungen der Z"ahler und Nenner in der
Ableitung von $t_3$.

So erh"alt man durch Absch"atzung von \equref{EquBerkProzInt2} nach oben
f"ur die Anzahl der Prozessoren:
\begin{eqnarray*}
    & & t_2 + 
        \int\limits_{ 2^{\lf \log \lb \frac{n-2}{\sqrt{2}} \rb \rf} }^{n-2} 
            m^{2+\gamma+0.5} dm
        + \int\limits_2^{ 2^{\lf \log \lb \frac{n-2}{\sqrt{2}} \rb \rf} }
            \MathE^{ (2+\gamma+f(m)) \ln(m) } dm \\
    & \leq & t_2 + 
        \left. \frac{1}{3.5+\gamma} m^{3.5+\gamma} 
        \right|_{ 2^{\lf \log \lb \frac{n-2}{\sqrt{2}} \rb \rf} }^{n-2}
        + \left. \lb m^{ 2+\gamma+f(m) } m \rb
          \right|_2^{ 2^{\lf \log \lb \frac{n-2}{\sqrt{2}} \rb \rf} } \\
    & = &  4 + (n-2)^{2+\gamma+\epsilon} + 
               \frac{ (n-2)^{2.5+\gamma} }{ 2 } \\
    & & 
         + \left. m \right|_2^{n-2} +
           \left. \left(
               \frac{1}{7+2\gamma} m^{3.5+\gamma}
           \right) \right|_2^{n-2} 
        +  
        \left. \frac{1}{3.5+\gamma} m^{3.5+\gamma} 
        \right|_{ 2^{\lf \log \lb \frac{n-2}{\sqrt{2}} \rb \rf} }^{n-2} \\
    & & +
         \left. m^{3+\gamma+f(m)}
         \right|_2^{ 2^{\lf \log \lb \frac{n-2}{\sqrt{2}} \rb \rf} }
\end{eqnarray*}
Dieser Term wird mit $t_8$ bezeichnet. An ihm erkennt man, da"s die
Anzahl der Prozessoren mit wachsendem $n$ asymptotisch
\[
    \frac{3}{7+2\gamma} (n-2)^{3.5+\gamma}
\]
betr"agt.

Schlie"slich m"ussen noch die Matrizen $C_1$ bis $C_n$ miteinander
multipliziert werden. Dies geschieht mit Hilfe der Bin"arbaummethode
nach \ref{SatzAlgBinaerbaum}. Es wird definiert
\[ n':= \lf \frac{n}{2} \rf \MyPunkt \]
Da nach \ref{SatzToeplizMult}
bei allen Multiplikationen Dreiecks-Toeplitz-Matrizen verkn"upft werden
und $C_i$ eine $(n-i+2) \times (n-i+1)$-Matrix handelt, k"onnen diese
Multiplikationen in
\Beq{TermBerkSchritteB}
   (\lc \log(n+1) \rc + 1) \lc \log(n) \rc
\Eeq Schritten von weniger als
\begin{eqnarray*}
    & & \sum_{k=1}^{n'} (n-2k+2) * (n-2k+1) \\
    & = & \sum_{k=1}^{n'} (2k(2k-1)) \\
    & = & 4 \sum_{k=1}^{n'} k^2 - 2 \sum_{k=1}^{n'} k \\
    & \MyStack{\ref{SatzSumK}, \, \ref{SatzSumK2} }{ = } & 
        4 \frac{n'(n'+1)(2n'+1)}{6} - 2\frac{n'(n'+1)}{2}
\end{eqnarray*}
Prozessoren durchgef"uhrt werden. F"ur ein gegebenes $n$ ist
der Wert dieses Terms ist kleiner als
der Wert von $t_8$.

In Verbindung mit \ref{SatzDdurchP}
ergibt sich als Endergebnis der Analyse, da"s mit Hilfe des in diesem 
Kapitel vorgestellen Algorithmus die Determinante einer $n \times n$-Matrix
in weniger als\footnote{Summe von \equref{TermBerkSchritte} und
 \equref{TermBerkSchritteB}}
\[
   \gamma_S
   \lc
       \lc \log(n-2) \rc^2 
       + \left( 1 + \frac{1}{\gamma_S}
       + \frac{1}{\epsilon} \right) \lc \log(n-2) \rc + 
       \frac{1}{\epsilon} + \frac{1}{\gamma_S} + 0.5
   \rc +
   (\lc \log(n+1) \rc + 1) \lc \log(n) \rc
\]
Schritten von weniger als $t_8$ Prozessoren berechnet werden kann.

Da in der Praxis f"ur die Matrizenmultiplikation Satz 
\ref{SatzAlgMatMult} statt der in \cite{CW90} angegebenen Methode 
benutzt wird, ist f"ur diesen Fall in allen obigen Termen
\[ \gamma = \gamma_S = \gamma_P = 1 \] zu setzen.

Vergleicht man B-Alg. mit C-Alg., BGH-Alg. und P-Alg., f"allt wiederum
das Fehlen von Fallunterscheidungen auf. Weiterhin werden wie bei BGH-Alg.
keine Divisionen verwendet, so da"s B-Alg. auch in Ringen anwendbar ist.

Betrachtet man die Aufwandsanalyse, so erkennt man, da"s B-Alg. 
leicht schlechter ist als C-Alg. und deutlich besser als BGH-Alg. .
In Kapitel \ref{ChapPan} wird P-Alg. in diese Rangfolge eingereiht.


%
% Datei: pan.tex
%
\MyChapter{Der Algorithmus von Pan}
\label{ChapPan}
%\label{SecIteration}
In diesem Kapitel wird der Algorithmus von V. Pan \cite{Pan85} zur 
Determinantenberechnung vorgestellt. Er kommt ebenfalls ohne
Divisionen\footnote{vgl. Bemerkungen in \ref{SecAlgFrame}}
aus und berechnet die Determinante iterativ. Auf diesen Algorithmus wird
mit {\em P-Alg.} Bezug genommen\footnote{vgl. Unterkapitel \ref{SecBez}}.

Man erh"alt insbesondere durch Variation der in den Unterkapitel
\ref{SecGuessInverse} und \ref{SecNewton} dargestellten Inhalte
einige weitere Versionen des Algorithmus.
F"ur eine vollst"andige Darstellung m"ussen alle diese Varianten beschrieben
und auf ihre Effizienz hin untersucht werden\footnote{N"aheres dazu ist 
insbesondere in \cite{PR85a} zu finden}. Da dies jedoch den Rahmen 
dieses Textes sprengt, beschr"anken sich die folgenden Darstellungen auf
die effizienteste Version des Algorithmus.

P-Alg. bietet erheblich mehr Variationsm"oglichkeiten als C-Alg., BGH-Alg.
und B-Alg., die, wie erw"ahnt, nicht alle hier behandelt werden k"onnen.
Vergleicht man P-Alg. von seiner Methodik her mit den drei
anderen, so erkennt man "Ahnlichkeiten zu BGH-Alg. . In beiden Algorithmen
werden mehrere auch separat bedeutsame teilweise schon l"anger bekannte
Verfahren zusammen verwendet. Von diesen Verfahren hebt sich lediglich
die in Unterkapitel \ref{SecGuessInverse} dargestellte Methode zur
Berechnung einer N"aherungsinversen ab. Sie wurde in \cite{Pan85} erstmalig
ver"offentlicht.

% **************************************************************************

\MySection{Diagonalisierbarkeit}

In diesem Unterkapitel wird die Diagonalisierbarkeit von Matrizen behandelt.
Es ist f"ur das Verst"andnis des in diesem Kapitel dargestellten Algorithmus
zur Determiantenberechnung nicht unbedingt erforderlich und kann daher beim 
Lesen auch "ubersprungen werden. Im folgenden werden jedoch einige 
Hintergr"unde der im Unterkapitel \ref{SecKrylov} dargestellten Methode von 
Krylov n"aher beleuchtet, die ein paar Zusammenh"ange klarer werden lassen.

Literatur zu diesem 
Thema ist neben den in Kapitel \ref{ChapBase} genannten Stellen
auch \cite{Zurm64} S. 169 ff .

Zum Problem der Diagonalisierbarkeit\footnote{Definition s. u.} gelangt
man "uber den Begriff der Basis eines Vektorraumes. Da es sich dabei
um Grundlagen der Linearen Algebra handelt, erfolgt die Darstellung 
vergleichsweise oberfl"achlich.

Sei $K$ ein K"orper und $V$ ein $K$-Vektorraum.
Seien $k_1,\, k_2, \, \ldots, \, k_n \in K \backslash \{ 0 \} $ 
und $v_1, \, v_2, \, \ldots, \, v_n \in V$.
Dann wird
\Beq{EquLinKomb}
   k_1 v_1 + k_2 v_2 + \ldots + k_n v_n 
\Eeq
als \index{Linearkombination} {\em Linearkombination} der Vektoren 
$v_1$ bis $v_n$ bezeichnet.
Sei $0_m$ der Nullvektor in $V$.
Falls f"ur die Vektoren die Bedingung
\[ k_1 v_1 + k_2 v_2 + \ldots + k_n v_n = 0_m \Rightarrow
   k_1 = k_2 = \ldots = k_n = 0
\]
erf"ullt ist, werden sie als 
{\em linear unabh"angig} \index{linear unabh{\Mya}ngig} bezeichnet, 
ansonsten als {\em linear abh"angig} \index{linear abh{\Mya}ngig}.
Falls jedes Element von $V$ als Linearkombination der Vektoren 
$v_1,\ldots,v_n$
darstellbar ist, werden diese Vektoren als 
\index{Basis} {\em Basis} bezeichnet.

F"ur alle Basen gilt die Aussage: \nopagebreak
\begin{quote}
    Je zwei Basen eines $K$-Vektorraumes bestehen aus dergleichen 
    Anzahl von Vektoren.
\end{quote}
Sei $B$ die Basis des $K$-Vektorraumes $V$. Dann wird die Anzahl 
der Vektoren, die $B$ bilden, 
als {\em Dimension von $V$} \index{Dimension} , 
kurz $\dim(V)$, bezeichnet.

Zur Darstellung von Elementen eines Vektorraumes $V$ w"ahlt man sich eine
Basis und beschreibt jedes Element des Vektorraumes als Linearkombination
der Elemente der Basis. Sei $n$ die Dimension des Vektorraumes. Dann
kann man auf diese Weise jedes Element von $V$ als $n$-Tupel von Elementen
des zugrunde liegenden K"orpers $K$ betrachten. Man erh"alt die
Vektorschreibweise:
\Beq{EquVektorSchreibweise}
   \left[ \begin{array}{c} k_1 \\k_2\\ \vdots\\ k_n \end{array} \right]
\Eeq
Die Basis der Form
\begin{quote} % $$$$ Formatierung gepr"uft ?
   $\left[ \begin{array}{c} 1\\ 0\\ 0\\ \vdots\\ 0\end{array} \right]$,
   \hspace{0.7em}
   $\left[ \begin{array}{c} 0\\ 1\\ 0\\ \vdots\\ 0\end{array} \right]$,
   $\ldots , $ \hspace{0.7em}
   $\left[ \begin{array}{c} 0\\ 0\\ 0\\ \vdots\\ 1 \end{array} \right]$
\end{quote}
wird als
{\em kanonische Basis} \index{kanonische Basis} bezeichnet.

Zur Betrachtung der Beziehungen verschiedener Basen zueinander werden diese
Basen ihrerseits bzgl. der kanonischen Basis dargestellt. 

Wird ein Vektor $v$ bzgl. einer Basis $B$ dargestellt, so wird dies
folgenderma"sen ausgedr"uckt:
\[ \left[
       \begin{array}{c}
           v_1 \\ \vdots \\ v_n
       \end{array}
   \right]_B
\]
Werden Vektoren zu Matrizen zusammengefa"st, so wird die gleiche 
Schreibweise auch f"ur diese Matrizen verwendet.

Sei $V$ ein $K$-Vektorraum der Dimension $n$ und $W$ ein $K$-Vektorraum
der Dimension $m$. Ein Ergebnis der Linearen Algebra lautet, da"s dann
die Menge aller $K$-Vektorraumhomomorphismen\footnote{also die Menge
aller {\em strukurvertr"aglichen} linearen Abbildungen} $f$
\[ f : \: V \rightarrow W \]
isomorph ist zur Menge aller $m \times n$-Matrizen $A$, wenn man die 
Abbildung definiert als\footnote{Dies entspricht der 
Matrizenmultiplikation (vgl. \ref{SatzAlgMatMult}), wenn man den 
abzubildenden Vektor $v$ als $n \times 1$-Matrix 
betrachtet (vgl. \ref{SatzAlgMatMult}).}
\[ f \lb
         \left[
             \begin{array}{c} v_1\\ v_2\\ \vdots\\ v_n \end{array}
         \right]
     \rb
   :=
      \left[
          \begin{array}{c}
              \sum_{j=1}^n a_{1,j} v_j \LMatStrut \\
              \vdots \LMatStrut                   \\
              \sum_{j=1}^n a_{m,j} v_j \LMatStrut
          \end{array}
      \right]
\]

Die Untersuchung der $K$-Vektorraumhomomorphismen kann man
also anhand der entsprechenden Matrien vornehmen. Im folgenden 
sind nur quadratische Matrizen von Interesse. Deshalb werden in den 
weiteren Ausf"uhrungen nur diese Matrizen beachtet.

Stellt man die Vektoren einer Basis $B_V$ bzgl. einer anderen Basis $B_W$
(normalerweise der kanonischen Basis) dar und betrachtet sie als
Spaltenvektoren einer Matrix \[ [B_V]_{B_W} \MyKomma \]
so erkennt man beim Vergleich von
\equref{EquLinKomb} und \equref{EquVektorSchreibweise} miteinander,
da"s man einen bzgl. $B_V$ dargestellten Vektor $x$ in seine Darstellung
bzgl. $B_W$ umrechnen kann durch
\Beq{EquBasiswechsel}
    [x]_{B_W}= B_V[x] \MyPunkt
\Eeq
Man erkennt also, da"s man eine Basis auch als Vektorraumhomomorphismus
betrachten kann. Die umgekehrte Betrachtungsweise ist nat"urlich nicht
m"oglich.

Um zum Begriff der {\em Diagonalisierbarkeit} zu gelangen, betrachten wir
nun, was passiert, wenn man Basen austauscht und die Darstellungen bzgl. der
neuen Basen vornehmen will.

Seien $V$ und $W$ jeweils $K$-Vektorr"aume sowie $B_V$ und $B_W$ jeweils
Basen dieser Vektorr"aume. Sei \[ f: \: V \rightarrow W \] ein
Vektorraumhomomorphismus und $A$ die entsprechende Matrix. Es gelte
\Beq{EquVonVNachW}
    \left[
        \begin{array}{c} y_1\\ y_2\\ \vdots\\ y_n \end{array}
    \right]_{B_W}
    = 
    A
    \left[
        \begin{array}{c} x_1\\ x_2\\ \vdots\\ x_n \end{array}
    \right]_{B_V}
\Eeq
Wechselt man nun zu den Basen $\tilde{B_V}$ und $\tilde{B_W}$ und stellt 
die neuen Basen bzgl. der alten durch die Matrizen $C$ und $D$ dar, so gilt
entsprechend \equref{EquBasiswechsel}:
\begin{eqnarray*}
    \left[
        \begin{array}{c} x_1\\ x_2\\ \vdots\\ x_n \end{array}
    \right]_{B_V} 
    & = & 
    C
    \left[
        \begin{array}{c} 
            \tilde{x_1}\\ 
            \tilde{x_2}\\ \vdots \\
            \tilde{x_n}
        \end{array}
    \right]_{\tilde{B_V}}
\\
    \left[
        \begin{array}{c} y_1\\ y_2\\ \vdots\\ y_n \end{array}
    \right]_{B_W}
    & = & 
    D
    \left[
        \begin{array}{c} 
            \tilde{y_1}\\ 
            \tilde{y_2}\\ \vdots \\
            \tilde{y_n}
        \end{array}
    \right]_{\tilde{B_W}}
\end{eqnarray*}
Gleichung \equref{EquVonVNachW} bekommt also folgendes Aussehen:
\[
    D
    \left[
        \begin{array}{c} 
            \tilde{y_1}\\ 
            \tilde{y_2}\\ \vdots \\
            \tilde{y_n}
        \end{array}
    \right]_{\tilde{B_W}}
    =
    A \: C
    \left[
        \begin{array}{c} 
            \tilde{x_1}\\ 
            \tilde{x_2}\\ \vdots \\
            \tilde{x_n}
        \end{array}
    \right]_{\tilde{B_V}}
\]
Multipliziert man beide Seiten mit $D^{-1}$, erh"alt man:
\[
    \left[
        \begin{array}{c} 
            \tilde{y_1}\\ 
            \tilde{y_2}\\ \vdots \\
            \tilde{y_n}
        \end{array}
    \right]_{\tilde{B_W}}
    =
    D^{-1} \: A \: C
    \left[
        \begin{array}{c} 
            \tilde{x_1}\\ 
            \tilde{x_2}\\ \vdots \\
            \tilde{x_n}
        \end{array}
    \right]_{\tilde{B_V}}
\]
Die Abbildung $f$ wird bzgl. der neuen Basen $\tilde{B_V}$ und $\tilde{B_W}$
also durch die Matrix $A':=D^{-1}AC$ dargestellt. W"ahlt man die neuen 
Basen geeignet, so ist es immer m"oglich zu erreichen, da"s 
$A'$ die Form
\[ \left[
   \begin{array}{cccc}
       a_{1,1}' & 0        & \cdots & 0        \MatStrut \\
       0        & a_{2,2}' & \ddots & \vdots   \MatStrut \\
       \vdots   & \ddots   & \ddots & 0        \MatStrut \\
       0        & \cdots   & 0      & a_{n,n}' \MatStrut
   \end{array} \right]
\] bekommt. Eine Matrix dieser Form wird als 
{\em Diagonalmatrix} \index{Diagonalmatrix} bezeichnet.

Betrachtet man nun statt einer Abbildung zwischen den
Vektorr"aumen eine Abbildung in $V$ und wechselt die Basis von $V$, so
besitzt die Abbildung bzgl. der neuen Basis die Form $C^{-1}AC$.
Es stellt sich wiederum die Frage, ob es m"oglich ist, die neue Basis
so zu w"ahlen, da"s $C^{-1}AC$ eine Diagonalmatrix ist. Eine Matrix $A$,
f"ur die das m"oglich ist, wird als
{\em diagonalisierbar} \index{diagonalisierbar} bezeichnet.

Um eine Beziehung zur Methode von Krylov (siehe Unterkapitel 
\ref{SecKrylov}) herstellen zu k"onnen, folgt 
eine Charakterisierung der Diagonalisierbarkeit. Dazu greifen wir auf 
den bereits in
\ref{DefUnterraum} definierten Begriff des {\em Unterraumes} zur"uck.
Anhand dieser Definitionen erkennt man, da"s alle zu einem Eigenwert
geh"orenden Eigenvektoren zusammen mit dem Nullvektor einen Unterraum
des zugrunde liegenden Vektorraumes bilden. Er wird als
{\em Eigenraum} \index{Eigenraum} bezeichnet.

An dieser Stelle werden die Eigenwerte mit der Diagonalisierbarkeit
in Verbindung gebracht:
\begin{satz}
\label{SatzEigenDiagonalBasis}
    Eine $n \times n$-Matrix $A$ ist genau dann diagonalisierbar, wenn der 
    zugrunde liegende $K$-Vektorraum $V$ eine Basis aus Eigenvektoren von 
    $A$ besitzt.
\end{satz}
\begin{beweis}
    Wir verzichten hier auf einen ausf"uhrlichen Beweis. Die Ideen f"ur
    die beiden Beweisrichtungen sind:
    \begin{itemize}
    \item 
          Ist eine Matrix $A$ diagonalisierbar und wechselt man die Basis
          so, da"s $A$ Diagonalgestalt bekommt, werden dadurch die
          Vektoren der kanonischen Basis zu Eigenvektoren der Matrix.
    \item
          Bilden die Eigenvektoren einer Matrix $A$ eine Basis von $V$ und
          benutzt man diese Basis zur Darstellung bekommt $A$ die
          Form einer Diagonalmatrix.
    \end{itemize}
\end{beweis}

\begin{satz}
\label{SatzEigenUnabhaengig}
    Seien $\lambda_1,\, \ldots,\, \lambda_k$ paarweise verschiedene
    Eigenwerte der $n \times n$-Matrix $A$. Sei $v_i$ ein Eigenvektor zu
    $\lambda_i$. Dann sind die Eigenvektoren $v_1,\, \ldots,\, v_k$
    linear unabh"angig.
\end{satz}
\begin{beweis}
    Der Beweis erfolgt durch Induktion nach der Anzahl der verschiedenen
    Eigenwerte $k$:
    \begin{MyDescription}
    \MyItem{$k=1$}
        F"ur diesen Fall ist die Behauptung offensichtlich richtig.
    \MyItem{$k>1$}
        Die Behauptung gelte f"ur $k-1$ und sei f"ur $k$ zu zeigen.
        Induktionsvoraussetzung ist also, da"s
        $v_1, \, \ldots, \, v_{k-1}$ linear unabh"angig sind.
        Angenommen $v_1, \, \ldots,\, v_{k}$ sind linear abh"angig.
        Dann existieren eindeutig bestimmte $r_1, \ldots, r_{k-1} \in K$ 
        mit
        \Beq{EquEigenUnabhaengig}
           v_k = r_1 v_1 + \cdots + r_{k-1} v_{k-1} \MyPunkt
        \Eeq
        Da $v_k$ nicht der Nullvektor sein kann, mu"s mindestens einer 
        der Faktoren $r_1, \dots, r_k$ ungleich Null sein, z. B. $r_i$. 
        
        Betrachtet man $A$ als 
        Abbildung und wendet diese Abbildung auf 
        \equref{EquEigenUnabhaengig} an, so kann man, da es sich um
        Eigenvektoren handelt, auch mit den Eigenwerten multiplizieren
        und erh"alt
        \[
           \lambda_k v_k = \lambda_1 r_1 v_1 + \cdots 
                                             + \lambda_k r_k v_k \MyPunkt
        \]
        Ist nun $\lambda_k = 0$, dann mu"s wegen der Verschiedenheit
        der Eigenwerte $\lambda_i \neq 0$ sein und man erh"alt einen 
        Widerspruch zur linearen Unabh"angigkeit von 
        $v_1,\, \ldots, \, v_k$.
        
        Ist $\lambda_k \neq 0$ erh"alt man einen Widerspruch zur 
        Eindeutigkeit der Darstellung von $v_k$.
    \end{MyDescription}
\end{beweis}

Aus \ref{SatzEigenDiagonalBasis} und \ref{SatzEigenUnabhaengig} ergibt sich:

\begin{korollar}
\label{SatzMaxEigen}
    Die maximale Anzahl verschiedener Eigenwerte einer Matrix ist gleich
    der Dimension des zugrunde liegenden Vektorraumes. 
    
    Besitzt eine Matrix maximal viele verschiedene Eigenwerte, so ist
    sie diagonalisierbar.
\end{korollar}

F"ur die zweite Folgerung ben"otigen wird einen weiteren Betriff. Seien
dazu $T$ und $U$ Unterr"aume des $K$-Vektorraumes $V$. Dann wird die Menge
\[
   \{ w \MySetProperty 
      \exists k,l \in K, \, t \in T, \, u \in U: \: w=kt+lu \}
\]
aller Linearkombinationen zweier Vektoren aus $T$ und $U$ als
{\em direkte Summe von $T$ und $U$} \index{direkte Summe} bezeichnet.
Anhand von \ref{DefUnterraum} erkennt man, da"s die direkte Summe
zweier Unterr"aume von $V$ wiederum ein Unterraum von $V$ ist.

Somit erh"alt man aus
\ref{SatzEigenDiagonalBasis} und \ref{SatzEigenUnabhaengig}:

\begin{korollar}
\label{SatzEigenraum}
    Eine Matrix ist genau dann diagonalisierbar, wenn die direkte Summe
    aller Eigenr"aume der Matrix den zugrundliegenden Vektorraum ergibt.
\end{korollar}

Die Bedeutung der in diesem Unterkapitel dargestellten Sachverhalte
wird deutlich, wenn man sie mit den in Unterkapitel
\ref{SecKrylov} erw"ahnten Einschr"ankungen f"ur die
Verwendbarkeit der Methode von Krylov zur Berechnung der Koeffizienten 
des charakteristischen Polynoms vergleicht.

% ... mehrfache Eigenwerte --> Dimension des Eigenraums

% Welche Beziehungen bestehen zwischen Invertierbarkeit und
%     Dialgonalisierbarkeit?

% **************************************************************************

\MySection{Das Minimalpolynom}
\label{SecMinimalpolynom}

Die Methode von Krylov (siehe \ref{SecKrylov}) dient zur Bestimmung der
Koeffizienten des Minimalpolynoms einer Matrix. Deshalb wird hier dieses
Minimalpolynom zun"achst n"aher betrachtet. Eine tiefergreifende 
Behandlung befindet sich z. B. in \cite{Zurm64} S. 233 ff .

In Satz \ref{SatzCayleyHamilton} wird bewiesen, da"s eine
$n \times n$-Matrix $A$ ihre eigene charakteristische Gleichung erf"ullt.
Diese Beobachtung f"uhrt zu: \nopagebreak[3]
\MyBeginDef
\label{DefMinimalpolynom} 
\index{Minimalpolynom} \index{Minimumgleichung}
    Das Polynom $m(\lambda)$ mit dem kleinsten Grad, f"ur das die 
    Gleichung \[ m(A) = 0_n \] erf"ullt ist, wird 
    {\em Minimalpolynom} genannt. Die Gleichung wird als 
    {\em Minimumgleichung} \index{Minimumgleichung} bezeichnet.
\MyEndDef

Um die Methode von Krylov verstehen zu k"onnen, m"ussen wir verschiedene
Eigenschaften des Minimalpolynoms beleuchten:

\begin{satz}
\label{SatzMinimalpolynomVielfaches}
    Sei $A$ eine $n \times n$-Matrix und $f(\lambda)$ ein Polynom. Es
    gelte \[ f(A) = 0_n \MyPunkt \] Dann ist $f(\lambda)$ ein Vielfaches
    des Minimalpolynoms $m(\lambda)$ von $A$.
\end{satz}
\begin{beweis}
    Angenommen die Behauptung ist falsch. Dann entsteht bei der Division
    von $f(\lambda)$ durch $m(\lambda)$ ein Rest $r(\lambda)$ und f"ur
    ein geeignetes Polynom $q(\lambda)$ gilt:
    \[ f(\lambda) = q(\lambda)m(\lambda) + r(\lambda) \MyPunkt \]
    Der Grad von $r(\lambda)$ ist kleiner als der Grad von $m(\lambda)$.
    Wird nun in diese Gleichung $A$ eingesetzt, erh"alt man
    \[ 0_n = 0_n + r(A) \MyPunkt \] Also mu"s auch gelten
    \[ r(A) = 0_n \MyPunkt \] Da jedoch das Minimalpolynom das Polynom mit
    dem kleinsten Grad ist, das diese Bedingung erf"ullt, f"uhrt dies
    zu einem Widerspruch.
\end{beweis}

Aus \ref{SatzCayleyHamilton} und \ref{SatzMinimalpolynomVielfaches} 
ergibt sich:
\begin{korollar}
\label{SatzVielfaches}
    Das charakteristische Polynom ist ein Vielfaches des Minimalpolynoms.    
\end{korollar}

Da wir die Methode von Krylov zur Berechnung des charakteristischen 
Polynoms verwenden wollen, m"ussen wir wissen, unter welchen Umst"anden
es mit dem Minimalpolynom zusammenf"allt. Diese 
Frage beantwortet der folgende Satz:

\begin{satz}
\label{SatzCharMatGGT}
    Sei $A$ eine $n \times n$-Matrix.
    Es wird definiert:
    \[ C := A - \lambda E_n \MyPunkt \] Sei
    \[ p(\lambda) = \det(C) \] das charakteristische Polynom von $A$.
    Es gelte
    \Beq{Equ20MatGGT}
        m(\lambda) = \frac{ p(\lambda) }{ q(\lambda) }
    \Eeq
    \MyPunktA{25em}
    Das Polynom $m(\lambda)$ ist genau dann das Minimalpolynom von $A$,
    wenn $q(\lambda)$ der gr"o"ste gemeinsame Teiler (ggT) der
    Determinanten aller $(n-1) \times (n-1)$-Untermatrizen von $C$ ist.
\end{satz}
\begin{beweis}
    Der Beweis erfolgt in zwei Schritten:
    \begin{itemize}
    \item Sei zun"achst $q(\lambda)$ der ggT Determinanten der 
          Untermatrizen. Es ist
          zu zeigen, da"s dann $m(\lambda)$ das Minimalpolynom ist.

          Mit Hilfe von Satz \ref{SatzEntw} (Zeilen- und Spaltenentwicklung)
          folgt, da"s $p(\lambda)$ durch $q(\lambda)$ teilbar ist. D. h. es
          gibt ein Polynom $m'(\lambda)$, so da"s
          \Beq{Equ2MatGGT}
              p(\lambda) = m'(\lambda) q(\lambda) \MyPunkt
          \Eeq
          Weiterhin gibt es eine $n \times n$-Matrix
          $M$ aus teilerfremden Polynomen "uber $\lambda$, so da"s gilt:
          \Beq{Equ1MatGGT}
              \adj(C) = M q(\lambda) \MyPunkt
          \Eeq
          Nach Satz \ref{SatzAdj} gilt:
          \Beq{Equ5MatGGT}
              C \, \adj(C) = E_n p(\lambda) \MyPunkt
          \Eeq
          Mit \equref{Equ2MatGGT} folgt aus \equref{Equ5MatGGT}:
          \Beq{Equ4MatGGT}
              C \, \adj(C) = E_n m'(\lambda) q(\lambda) \MyPunkt
          \Eeq
          Mit \equref{Equ1MatGGT} folgt aus \equref{Equ5MatGGT}:
          \Beq{Equ3MatGGT}
              C \, \adj(C) = C M q(\lambda) \MyPunkt
          \Eeq
          Aus \equref{Equ3MatGGT} und \equref{Equ4MatGGT} folgt:
          \[ C M = E_n m'(\lambda) \MyPunkt \]
          Benutzt man die Definition von $C$, erh"alt man
          \[ (A - \lambda E_n) M = E_n m'(\lambda) \MyPunkt \]
          Setzt man nun in dieser Gleichung $A$ f"ur $\lambda$ ein,
          ergibt sich
          \[ m'(A) = 0_n \MyPunkt \]
          Nach Satz \ref{SatzMinimalpolynomVielfaches} ist $m'(\lambda)$
          also ein Vielfaches des Minimalpolynoms $m(\lambda)$.
          
          Angenommen es gibt ein Polynom $m''(\lambda)$ mit
          \[ m''(A) = 0_n \MyKomma \] dessen Grad kleiner ist als der
          Grad von $m'(\lambda)$. Da der Grad des charakteristischen
          Polynoms immer $n$ ist, mu"s dann der Grad von $q(\lambda)$ in
          \equref{Equ2MatGGT} und somit auch in \equref{Equ1MatGGT}
          entsprechend gr"o"ser sein, im Widerspruch dazu, da"s
          $q(\lambda)$ der ggT ist. Also ist $m'(\lambda)$ das
          Minimalpolynom.
    \item Sei nun $m(\lambda)$ das Minimalpolynom. Dann ist zu zeigen, da"s
          $q(\lambda)$ der ggT der Unterdeterminanten ist.
          
          Nach \ref{SatzVielfaches} gibt es ein Polynom $q'(\lambda)$, so
          da"s
          \Beq{Equ6MatGGT}
              p(\lambda) = q'(\lambda) m(\lambda) \MyPunkt
          \Eeq
          Benutzt man f"ur das Minimalpolynom die Koeffizientendarstellung,
          erh"alt man mit geeigneten Koeffizienten $b_i$:
          \begin{eqnarray*}
              \lefteqn{- E_n m(\lambda)} \\
              & = & m(A) - E_n m(\lambda) \\
              & = & b_m(A^m - \lambda^m E_n)
                    + b_{m-1}(A^{m-1} - \lambda^{m-1} E_n) + \cdots +
                    b_1(A - \lambda E_n) \MyPunkt \\
          \end{eqnarray*}
          Also ist $m(\lambda)$ durch $(A-\lambda E_n)$ teilbar und es
          gibt eine Matrix $N$ aus Polynomen "uber $\lambda$, so da"s gilt:
          \Beq{Equ7MatGGT}
              m(\lambda) E_n = (A - \lambda E_n) N \MyPunkt
          \Eeq
          Mutipliziert man beide Seiten mit $q'(\lambda)$, erh"alt man
          \[ p(\lambda) E_n = q'(\lambda) (A - \lambda E_n) N \MyPunkt \]
          Subtrahiert man nun auf beiden Seiten
          \begin{eqnarray*}
               & & p(\lambda) E_n \\
               & \MyStack{nach \ref{SatzAdj}}{=} & C \,\adj(C) \\
               & = & (A - \lambda E_n) \adj(C) \MyKomma
          \end{eqnarray*}
          erh"alt man
          \[
             0_{n,n} =
             \underbrace{ (A - \lambda E_n)
                        }_{ \mbox{$(*1)$} }
             \:
             \underbrace{ (q'(\lambda) N - \adj(C))
                        }_{ \mbox{$(*2)$} }
          \]
          In dieser Gleichung ist Term $(*1)$ f"ur ein beliebig gew"ahltes 
          $\lambda$ ungleich 
          der Nullmatix\footnote{abgesehen von einigen Sonderf"allen, deren
          Existenz den Beweis jedoch nicht beeintr"achtigt}. Also mu"s 
          Term $(*2)$ gleich der Nullmatrix sein, so 
          da"s gilt:
          \[ q'(\lambda) N = \adj(C) \]
          Also ist $q'(\lambda)$ Teiler der Elemente von $\adj(C)$.
          
          Angenommen es gibt ein Polynom $q''(\lambda)$, dessen Grad
          gr"o"ser ist als der Grad von $q'(\lambda)$ und das ebenfalls
          Teiler der Elemente von $\adj(C)$ ist. Da der Grad von
          $p(\lambda)$ immer $n$ ist, mu"s dann der Grad von $m(\lambda)$ 
          in \equref{Equ6MatGGT} kleiner sein, im Widerspruch zu der
          Voraussetzung, da"s $m(\lambda)$ das Minimalpolynom ist.
    \end{itemize}
\end{beweis}

Berechnet man in \equref{Equ7MatGGT} auf beiden Seiten die Determinante,
erh"alt man
\Beq{Equ21MatGGT}
   m^n(\lambda) = p(\lambda) \det(N) \MyPunkt
\Eeq
Aus \equref{Equ20MatGGT} und \equref{Equ21MatGGT} ergibt sich:

\begin{korollar}
\label{SatzMinimalNullGenauDann}
    Es ist $\lambda_i$ genau dann Nullstelle von $m(\lambda)$, wenn es auch
    Nullstelle von $p(\lambda)$ ist.
\end{korollar}

Anders ausgedr"uckt: $m(\lambda)$ und $p(\lambda)$ besitzen die gleichen
Nullstellen mit evtl. verschiedenen Vielfachheiten. Das f"uhrt
zu einer weiteren Schlu"sfolgerung:

\begin{korollar}
\label{SatzPaarweiseVerschieden}
    Besitzt eine $n \times n$-Matrix $n$ paarweise verschiedene Eigenwerte,
    so stimmen ihr Minimalpolynom und ihr charakteristisches Polynom
    "uberein.
\end{korollar}

Falls die Eigenwerte nicht paarweise verschieden sind, k"onnen 
Minimalpolynom und charakteristisches Polynom also verschieden sein, was
eine Einschr"ankung f"ur die Methode von Krylov 
(siehe Unterkapitel \ref{SecKrylov}) bedeutet, wenn man sie zur Berechnung
der Koeffizienten des charakteristischen Polynoms verwendet.
Wann die beiden Polynome verschieden sind, zeigt der folgende Satz:

\begin{satz}
\label{SatzMinimalDimEins}
    Das Minimalpolynom $m(\lambda)$ einer Matrix $A$ stimmt genau dann 
    mit dem charakteristischen Polynom $p(\lambda)$ der Matrix "uberein, 
    wenn die Dimension jedes Eigenraumes $1$ 
    betr"agt\footnote{Es ist zu beachten, da"s diese Aussage nicht dazu 
    "aquivalent ist, da"s die direkte Summe der Eigenr"aume den gesamten 
    Vektorraum ergibt.}.
\end{satz}
\begin{beweis}
    Aus \ref{SatzCharMatGGT} folgt, da"s $m(\lambda)$ und $p(\lambda)$
    genau dann "ubereinstimmen, wenn der 
    ggT der Determinanten aller $(n-1) \times (n-1)$-Untermatrizen
    der charakteristischen Matrix von $A$ gleich $1$ ist.
    
    Betrachtet man die beiden Polynome in ihrer Linearfaktorendarstellung,
    mu"s der genannte ggT, falls er ungleich $1$ ist, mit einem Linearfaktor
    von $p(\lambda)$ "ubereinstimmen. F"ur die entsprechende Nullstelle
    von $p(\lambda)$ verschwinden auch alle 
    $(n-1) \times (n-1)$-Unterdeterminanten. Es gibt also f"ur diese
    Nullstelle keine $n-1$ linear unabh"angigen Spaltenvektoren der 
    charakteristischen Matrix. Der Rangabfall der Nullstelle ist also
    gr"o"ser als $1$ und es gibt mehr als einen linear unabh"angigen 
    Eigenvektor zu diesem Eigenwert.
\end{beweis}

Aus \ref{SatzPaarweiseVerschieden} und \ref{SatzMinimalDimEins} erh"alt
man:
\begin{korollar}
\label{SatzMinimalKomplex}
    Falls die Eigenwerte nicht paarweise verschieden sind und das
    Minimalpolynom mit dem charakteristischen Polynom "ubereinstimmt,
    zerf"allt es im K"orper der reellen Zahlen nicht in Linearfaktoren.
\end{korollar}

Wie bereits mehrfach erw"ahnt, erfolgt nun die Anwendung der Ergebnisse 
dieses Unterkapitels auf die Methode von Krylov.

% $$$ Verbesserung der Benutzung der Methode von Krylov:
% (\det(A) gesucht); bestimme Matrix B, so da"s $\det(B)$ bekannt oder
% leicht zu berechnen und f"ur $A*B$ gilt:
%     die Dimension jedes Eigenraumes ist gleich 1; 
%     (es gen"ugt: die Eigenwerte sind paarweise verschieden)
%     w"unschenswert $\det(B)$ ist 'Einheit' im zugrundeliegenden Ring;
% berechne \det(AB); dann eine zus"atzliche Division:
%  \det(A) = \det(AB) / \det(B)

% **************************************************************************

\MySection{Die Methode von Krylov}
\label{SecKrylov}
\index{Krylov!Methode von}

In diesem Unterkapitel wird die Methode von Krylov zur Bestimmung der
Koeffizienten des Minimalpolynoms einer Matrix beschrieben
(siehe z. B. \cite{Zurm64} ab S. 171 oder \cite{Hous64} ab S. 149;
Originalver"offentlichung \cite{Kryl31} ). Wie in Unterkapitel 
\ref{SecMinimalpolynom} beschrieben wird, ist das Minimalpolynom
unter bestimmten Bedingungen mit dem charakteristischen Polynom identisch.
Da sich unter den Koeffizienten des charakteristischen Polynoms auch die
Determinante der zugrunde liegenden Matrix befindet 
(vgl. \ref{SatzDdurchP}), ist es m"oglich, Krylovs Methode zur 
Determinantenberechnung zu verwenden, was im Algorithmus von Pan 
ausgenutzt wird.

Sei $A$ die $n \times n$-Matrix, deren Minimalpolynom zu
berechnen ist. 
Sei $z_0$ ein geeigneter Vektor der L"ange $n$. Wie $z_0$ beschaffen ist, 
wird noch behandelt. Sei $i \in \Nat$ gegeben. Die 
Vektoren \[ z_1,\, \ldots,\, z_i \] erh"alt man durch
\begin{eqnarray}
    z_1 & := & A z_0 \nonumber \\
    z_2 & := & A z_1 = A^2 z_0 \nonumber \\
    \vdots \nonumber \\
    z_i & := & A z_{i-1} = A^i z_0 \label{EquDefZI} \MyPunkt
\end{eqnarray}
Die Vektoren \[ z_0,\, \ldots ,\, z_i \] werden als {\em iterierte Vektoren}
bezeichnet.
Betrachtet man die iterierten Vektoren als Spaltenvektoren einer Matrix,
erh"alt man eine sogenannte {\em Krylov}-Matrix:
\[ K(A,z_0,i):= [ z_0, z_1, z_2, \ldots, z_{i-1} ] \MyPunkt \]
Zwischen den iterierten Vektoren besteht eine lineare Abh"angigkeit
besonderer Form, die von Krylov \cite{Kryl31} f"ur das hier zu
beschreibende Verfahren entdeckt wurde.

Das Minimalpolynom von $A$ wird mit $m(\lambda)$ bezeichnet.
Es gilt also
\Beq{EquKrylovPolynom}
    m(A) = 0_{n,n} \MyPunkt
\Eeq Das Polynom habe den Grad $j$.
Seien $c_0,\, \ldots\, c_{j-1}$ die Koeffizienten des Polynoms. 
Definiert man
\[
   c := \left[
        \begin{array}{c} c_0 \\ c_1 \\ \vdots \\ c_{j-1} \end{array}
        \right]
\]
dann ergibt sich
\begin{eqnarray} % $$$$ Formatierung gepr"uft ?
    m(A) & = & 0_{n,n} \nonumber
\\ \Leftrightarrow \hspace{1.2mm} \nonumber
    A^j + c_{j-1} A^{j-1} + \ldots + c_1 A + c_0 E_n & = & 0_{n,n}
\\ \Leftrightarrow \hspace{9.1mm} \nonumber
    c_{j-1} A^{j-1} + \ldots + c_1 A + c_0 E_n & = & - A^j
\\ \Leftrightarrow \nonumber
    c_0 E_n z_0 + c_1 A z_0 + \ldots + c_{j-1} A^{j-1} z_0 &
                                                         = & - A^j z_0
\\ \Leftrightarrow \hspace{3.6cm} \label{EquKrylovEqu}
    K(A,z_0,j) c & = & - A^j z_0
\end{eqnarray}
Gleichung \equref{EquKrylovEqu} kann man als lineares Gleichungssystem
in Matrizenschreibweise betrachten. Multipliziert man die rechte Seite
dieser Gleichung aus, erh"alt man einen Vektor der L"ange $n$.
Nach \ref{SatzRangGleich} ist das entsprechende Gleichungssystem
genau dann l"osbar, wenn
\[ \rg(K(A,z_0,j) = \rg([K(A,z_0,j),\, A^j z_0]) \MyPunkt \]
Wir suchen eine eindeutige L"osung des Gleichungssystems und erhalten diese
durch Verwendung von \ref{SatzGenauEine}, wonach der bis hierhin
nicht n"aher beschriebene Vektor $z_0$ so gew"ahlt werden mu"s,
da"s die Spaltenvektoren von $K(A,z_0,j)$ linear unabh"angig und
die Spaltenvektoren von $[K(A,z_0,j),\, A^j z_0]$ linear abh"angig sind.

An dieser Stelle wird deutlich, da"s $m(\lambda)$ das Polynom mit dem
kleinsten Grad sein mu"s, das \equref{EquKrylovPolynom} erf"ullt, damit
\equref{EquKrylovEqu} eine eindeutige L"osung besitzt. Falls ein
Polynom $m_1(\lambda)$ existiert, da"s \equref{EquKrylovPolynom} erf"ullt
und dessen Grad kleiner ist als der Grad von $m(\lambda)$, dann ist die
lineare Abh"angigkeit unabh"angig von der Wahl von $z_0$ bereits f"ur
weniger als $j$ iterierte Vektoren gegeben.

Da nach \equref{EquKrylovEqu} die ersten $j+1$ iterierten Vektoren linear
abh"angig sind, bleibt zu zeigen, da"s die ersten $j$ dieser Vektoren linear
unabh"angig sind. Dazu betrachten wir das $s \in \Nat$ mit der Eigenschaft,
da"s $s$ paarweise verschiedene Eigenvektoren von $A$ immer linear 
unabh"angig und $s+1$ von ihnen immer linear abh"angig sind. Aus den 
Grundlagen "uber Eigenvektoren und lineare Unabh"angigkeit geht hervor, 
da"s ein solches $s$ gibt.

Seien somit
\[ x_1,\, \ldots ,\, x_s \] linear 
unabh"angige Eigenvektoren von $A$.
Sei der iterierte Vektor $z_0$ darstellbar als Linearkombination einer 
maximalen Anzahl (also $s$) von Eigenvektoren von $A$. 
Demnach gilt f"ur geeignete 
\[ d_1,\, \ldots,\, d_s \MyKomma \] ungleich Null\footnote{Eine 
ausf"uhrliche Diskussion der Eigenschaften der 
iterierten Vektoren, insbesondere ihrer Beziehung zu den Eigenvektoren, 
befindet sich in \cite{Bode59} (Teil 2, Kapitel 2).}:
\Beq{EquDefZNull}
    z_0 = d_1 x_1 + \cdots + d_s x_s \MyPunkt
\Eeq 
Eine lineare Abh"angigkeit zwischen den ersten $j$ iterierten Vektoren
hat die Form
\Beq{EquIteriertLinAbh}
    h(z_0):= e_0 z_0 + e_1 z_1 + \cdots + e_{j-1} z_{j-1} = 0_n \MyKomma
\Eeq
wobei nicht alle $e_i$ gleich Null sind. Mit Hilfe der
Gleichungen \equref{EquDefZI} und \equref{EquDefZNull} in Verbindung mit
den Eigenwertgleichungen\footnote{vgl. Gleichung \equref{EquEigenMotiv}}
\[ A x_i = \lambda_i x_i \]
erh"alt man:
\begin{eqnarray*} % $$$$ Formatierung gepr"uft ?
    z_0 & = & \left. d_1 x_1 + \cdots + d_s x_s 
                        \: \hspace{14.2mm} \right| * e_0 \\
    z_1 & = & 
        \left. \lambda_1 d_1 x_1 + \cdots + 
                   \lambda_s d_s x_s \: \hspace{7.1mm} \right| * e_1 \\
    z_2 & = & 
         \left. \lambda_1^2 d_1 x_1 + \cdots + \lambda_s^2 d_s x_s 
              \: \hspace{7mm} \right| * e_2 \\
    & \vdots \\
    z_{j-1} & = &
         \left. \lambda_1^{j-1} d_1 x_1 
                       + \cdots + \lambda_s^{j-1} d_s x_s \: \right| * 1
\end{eqnarray*}
Diese Gleichungen f"ur die $z_i$ werden mit den am rechten Rand angegebenen 
Werten (vgl. \equref{EquIteriertLinAbh}) multipliziert und anschlie"send 
addiert. Definiert man
\[
   g(\lambda):= e_0 + e_1 \lambda + \cdots + e_{j-1} \lambda^{j-1} \MyKomma
\]
so lautet das Ergebnis in Verbindung mit \equref{EquIteriertLinAbh} :
\[
    h(z_0)= g(\lambda_1) d_1 x_1 + \cdots + g(\lambda_s) d_s x_s
          = 0_n \MyPunkt
\]
Da die $x_k$ nach Voraussetzung linear unabh"angig sind, mu"s also gelten
\[ \forall 1 \leq k \leq s :\: g(\lambda_k)d_k = 0 \MyPunkt \]
Da wiederum nach Voraussetzung $z_0$ eine Linearkombination aller $s$
linear unabh"angigen Eigenvektoren $x_i$ (s. o.) ist, folgt
\Beq{Equ2BewLinUnabh}
    \forall 1 \leq i \leq s :\: g(\lambda_i)= 0 \MyPunkt
\Eeq
Da die Dimension jedes Eigenraumes mindestens $1$ betr"agt, folgt mit Hilfe
von \ref{SatzMinimalNullGenauDann}, da"s
die maximale Anzahl linear unabh"angiger Eigenvektoren $s$ mindestens
so gro"s ist wie der Grad des Minimalpolynoms $j$.

Da $g(\lambda)$ als Polynom vom Grad $j-1$
nur maximal $j-1$ Nullstellen besitzen kann, folgt aus 
\equref{Equ2BewLinUnabh}
\[ e_0 = e_1 = \cdots = e_{j-1} = 0 \MyKomma \]
Daraus wiederum folgt mit \equref{EquIteriertLinAbh},
da"s die iterierten Vektoren linear unabh"angig sind.

Der bis hierhin gef"uhrte Beweis der linearen Unabh"angigkeit
iterierter Vektoren verwendet die maximale Anzahl $s$ linear unabh"angiger
Eigenvektoren. Betrachten wir deshalb diesen Wert $s$ genauer.
Es gibt zwei F"alle:
\begin{itemize}
\item
      Minimalpolynom und charakteristisches Polynom sind identisch.
      Satz \ref{SatzMinimalDimEins} f"uhrt zu zwei Unterf"allen:
      \begin{itemize}
      \item
            Die Eigenwerte sind paarweise verschieden. In diesem Fall
            ist $s$ gleich dem Grad $n$ des charakteristischen Polynoms
            und somit gleich $j$.
      \item
            Die Eigenwerte sind nicht paarweise verschieden. Nach
            \ref{SatzMinimalDimEins} ist $s$ gleich der Anzahl verschiedener
            Eigenwerte und damit in $\Rationals$ kleiner als $n$ und 
            mindestens $1$. In diesem Fall l"a"st sich die lineare
            Unabh"angigkeit nur f"ur weniger als $j$ iterierte Vektoren
            beweisen und die Methode von Krylov ist zur Bestimmung der
            Koeffizienten des Minimalpolynoms nicht anwendbar.
      \end{itemize}
\item 
      Minimalpolynom und charakteristisches Polynom sind nicht identisch.
      Es gibt wiederum zwei Unterf"alle:
      \begin{itemize}
      \item 
            Die direkte Summe der Eigenr"aume ergibt den gesamten 
            Vektorraum. In diesem Fall ist $s = n >j$.
      \item
            Die direkte Summe der Eigenr"aume ergibt nicht den gesamten
            Vektorraum. In diesem Fall ist $s \geq j$.
      \end{itemize}
\end{itemize}
W"ahlt man also $z_0$ als Linearkombination einer maximalen Anzahl linear
unabh"angiger Eigenvektoren, so sind die ersten $j$ iterierten Vektoren
linear unabh"angig, es sei denn, Minimalpolynom und charakteristisches
Polynom sind identisch und die Eigenwerte nicht paarweise verschieden.

Das nun noch verbliebene Problem ist die Wahl von $z_0$ f"ur eine gegebene 
Matrix $A$, da im Normalfall die Eigenvektoren nicht bekannt sind. Diese
Schwierigkeit
kann dadurch "uberwunden werden, da"s man die Methode von Krylov mit 
verschiedenen Vektoren $z_0$ auf die Matrix $A$ anwendet und dabei die
Vektoren $z_0$ so ausw"ahlt, da"s mindestens einer unter ihnen eine 
Linearkombination aller Eigenvektoren $x_1$ bis $x_s$ ist.

W"ahlt man eine Basis des zugrunde liegenden Vektorraumes, bestehend aus $n$
Vektoren, sowie einen
weiteren Vektor so aus, da"s je $n$ dieser $n+1$ Vektoren linear unabh"angig
sind und der $n+1$-te jeweils eine Linearkombination aller $n$ anderen ist,
so besitzt mindestens einer dieser Vektoren die geforderte Eigenschaft.
Dies erkennt man durch folgende "Uberlegung:
Stellt man die $n$ Vektoren als Linearkombinationen der Eigenvektoren dar,
so wird jeder Eigenvektor mindestens einmal ben"otigt. Da der $n+1$-te
Vektor eine Linearkombination der anderen $n$ ist, ist er also auch eine
Linearkombination aller beteiligen Eigenvektoren. Ein Beispiel f"ur 
$n+1$ Vektoren, die offensichtlich diese Eigenschaft besitzen, sind die
Vektoren der kanonische Basis und deren Summe:
\[ 
   \left[
       \begin{array}{c}
           1 \\ 0 \\ 0 \\ \vdots \\ 0
       \end{array}
   \right]
   \left[
       \begin{array}{c}
           0 \\ 1 \\ 0 \\ \vdots \\ 0
       \end{array}
   \right] \: \ldots  \:
   \left[
       \begin{array}{c}
           0 \\ 0 \\ \vdots \\ 0 \\ 1
       \end{array}
   \right] \: 
   \left[
       \begin{array}{c}
           1 \\ 1 \\ 1 \\ \vdots \\ 1
       \end{array}
   \right]  
\]

Uns interessiert die Berechnung der Determinante mit Hilfe der 
Methode von Krylov. Zusammengefa"st sieht das Vorgehen dazu folgenderma"sen
aus:
\begin{itemize}
\item 
      Zun"achst werden auf die soeben beschriebene Weise $n+1$ f"ur den 
      iterierten Vektor $z_0$ bestimmt. In der praktischen Anwendung 
      beschr"ankt man
      sich in der Regel auf einen Vektor, da die Anzahl der F"alle, in 
      denen dieser Vektor nicht ausreicht, so gering ist, da"s diese F"alle
      vernachl"assigt werden k"onnen.

      Die folgenden Schritte werden mit jedem Vektor $z_0$ parallel 
      durchgef"uhrt. 
      
      Falls mehrere der parallelen Zweige ein Ergebnis 
      liefern, m"ussen diese Ergebisse gleich sein. Falls sie nicht 
      gleich sind, wurde die Rechnung nicht korrekt durchgef"uhrt.

      Falls keiner der Zweige ein Ergebnis liefert, sind entweder die 
      Eigenwerte der zugrunde liegenden Matrix nicht paarweise verschieden,
      oder die Matrix ist nicht invertierbar. Diese beiden Unterf"alle 
      k"onnen mit dem hier dargestellten Algorithmus nicht unterschieden 
      werden.
\item 
      Es werden die iterierten Vektoren von $z_1$ bis $z_n$ berechnet.
\item
      Das Gleichungssystem \equref{EquKrylovEqu} wird dadurch gel"ost, 
      da"s die aus den iterierten Vektoren bestehende Krylov-Matrix
      inveritert wird. Ist die Krylov-Matrix nicht invertierbar, so ist 
      der Berechnungsversuch aus den bereits beschriebenen Gr"unden
      ein Fehlschlag und wird abgebrochen.
\item
      Die Koeffizienten des charakteristischen Polynoms, und somit auch
      die Determinante, werden dadurch berechnet, da"s die Inverse
      der Krylov-Matrix mit dem iterierten Vektor $z_n$ multipliziert wird
      (vgl. \equref{EquKrylovEqu}).
\end{itemize}

% mehr iterierte Vektoren, als der Grad des Minimalpolynoms angibt sind
% immer linear abh"angig!!!! (--> L"osbarkeit linearer Gleichungssysteme);

% **************************************************************************

\MySection{Vektor- und Matrixnormen}
\label{SecNorm}

Die Darstellungen der Wahl einer N"aherungsinversen in Unterkapitel 
\ref{SecGuessInverse} und der iterativen Matrizeninvertierung 
in Unterkapitel \ref{SecNewton} benutzen Normen von Matrizen.
Deshalb werden im vorliegenden Unterkapitel die Normen eingef"uhrt, die
dort zur Beschreibung erforderlich sind. 
Literatur dazu ist z. B. \cite{GL83} ab S. 12 
oder \cite{Isaa73} ab S. 3.

Umgangssprachlich formuliert, stellt der Begriff der {\em Norm} eine 
Verallgemeinerung des Begriffs der {\em L"ange} dar. Um zu Matrixnormen 
zu gelangen, kl"aren wir zun"achst, was eine Vektornorm ist:
\MyBeginDef
\index{Norm!eines Vektors}
\label{DefVektorNorm}
    Eine Funktion
    \[ f: \Rationals^n \rightarrow \Rationals \MyKomma \]
    die die Bedingungen
    \begin{enumerate} % $$$$ Formatierung geprueft ?
    \item
         $ \forall x \in \Rationals: f(x) \geq 0, 
            f(x) = 0 \Leftrightarrow x = 0_n 
         $
    \item
         $ \forall x,y \in \Rationals: f(x + y) \leq f(x) + f(y) $
    \item
         $ \forall a \in \Rationals, x \in \Rationals^n: f(ax)= |a| f(x) $
    \end{enumerate}
    erf"ullt, hei"st {\em Norm "uber $\Rationals^n$}. 
\MyEndDef

\MyBeginDef
\label{DefPNorm}
\index{H{\Myo}ldernorm} \index{p-Norm}
    Sei \[ p \in \Nat \] 
    fest gew"ahlt und \[ x \in \Rationals^n \] beliebig.
    \[ ||x||_p := (|x_1|^p + \ldots + |x_n|^p)^{1/p} \]  
    Die so definierte Funktion hei"st { \em $p$-Norm }.

    \[ ||x||_{\infty}:= \max_{1\leq i \leq n} |x_i| \]
    Diese Funktion wird mit {\em $\infty$-Norm} bezeichnet.

    Die $p$-Normen sowie die $\infty$-Norm 
    werden als { \em H"oldernormen } bezeichnet.
\MyEndDef

Mit der vorangegangenen Definition werden zwar einige Begriffe angegeben,
es ist jedoch nicht selbstverst"andlich, da"s es sich bei den Funktionen
auch wirklich um Normen handelt:
\begin{satz}
\label{SatzPNorm}
    Die in \ref{DefPNorm} definierten Funktionen 
    sind Normen "uber $\Rationals^n$.
\end{satz}
\begin{beweis}
    Die Funktionen erf"ullen die Bedingungen aus \ref{DefVektorNorm}.

    Der Beweis dieser Behauptung ist f"ur
    die $1$-Norm, $2$-Norm und $\infty$-Norm in
    \cite{Isaa73} ab S. 4 und
    f"ur die anderen Normen in \cite{Achi67} S. 4-7 angegeben.
    % (Isaa73 -> [30]) Achi67; BM b260/Achi
\end{beweis}

Den Begriff der Norm kann man auch auf Matrizen ausdehnen. F"ur uns 
gen"ugt die Betrachtung quadratischer Matrizen.
\MyBeginDef
\index{Norm!einer Matrix} \index{Matrixnorm}
\label{DefMatrixNorm}
    Eine Funktion 
    \[ f: \Rationals^{n^2} \rightarrow \Rationals \MyKomma \]
    die die Bedingungen
    \begin{eqnarray*}
        \forall A \in \Rationals^{n^2} & : & f(A) \geq 0,
             f(A) = 0 \leftrightarrow A=0_n \\
        \forall A,B \in \Rationals^{n^2} & : & f(A+B) \leq f(A) + f(B) \\
        \forall c \in \Rationals, A \in \Rationals^{n^2} & : &
             f(cA) = |c|f(A) \\
        \forall A,B \in \Rationals^{n^2} & : & f(AB) \leq f(A) * f(B) \\
    \end{eqnarray*}
    erf"ullt, hei"st {\em Matrixnorm "uber $\Rationals^{n^2}$ }.
\MyEndDef
Die vierte der obigen Bedingungen wird in der Literatur nicht immer f"ur 
Matrixnormen gefordert. In solchen F"allen wird unterschieden zwischen
Matrixnormen, die diese Bedingungen erf"ullen, und solchen, die diese
Bedingung nicht erf"ullen (vgl. \cite{Isaa73} S. 8 und \cite{GL83}
S. 14). F"ur uns sind diese Unterschiede nicht von Bedeutung.

Die von uns benutzten Matrixnormen sind folgenderma"sen definiert:
\MyBeginDef
\label{DefInduzierteNorm}
\index{Operatornorm} \index{Matrixnorm!induzierte}
\index{Matrixnorm!nat{\Myu}rliche} \index{Norm!nat{\Myu}rliche}
    Sei \[ A \in \Rationals^{n^2} \] Sei \[ x \in \Rationals^n \] Es gelte
    \[ ||x|| = 1 \] f"ur eine fest gew"ahlte Vektornorm.
    Die Funktion \[ ||A||:= ||Ax|| \] hei"st dann
    {\em durch die Vektornorm induzierte Matrixnorm }.
\MyEndDef
Sie ist in der Literatur auch noch unter den Namen {\em nat"urliche Norm}
und {\em Operatornorm} bekannt und wird h"aufig noch anders definiert
(vgl. \cite{Isaa73} S. 8). Das hat jedoch f"ur unsere Anwendungen keine
Bedeutung.

\begin{satz}
\label{SatzInduzierteNorm}
    Die in \ref{DefInduzierteNorm} definierte Funktion ist eine
    Matrixnorm.
\end{satz}
\begin{beweis}
    Die Funktion erf"ullt die Bedingungen in \ref{DefMatrixNorm}
    (\cite{Isaa73} ab S. 8 ).
\end{beweis}

Es folgen Beispiele f"ur induzierte
Matrixnormen, die im weiteren Text benutzt werden. Dazu ben"otigen wir
vorher noch einen weiteren Begriff:

Seien $\lambda_1, \, \ldots, \, \lambda_n$ die Eigenwerte von $A$. Dann
wird der Spektralradius \index{Spektralradius} $\rho(A)$ definiert als
\[ \rho(A) := \max\{ |\lambda_1|,\, \ldots,\, |\lambda_n| \} \MyPunkt \]

Durch
Indizes wird jeweils kenntlich gemacht, durch welche
Vektornorm die jeweilige Matrixnorm induziert wird\footnote{Bei der
Vielzahl der in diesem Unterkapitel auftauchenden Normen, sollte man
nicht vergessen, da"s $|x|$ f"ur einen Skalar $x$ einfach nur den
Absolutwert bezeichnet.}.
\begin{eqnarray}
    ||A||_1 & = & \max_j\sum_{k=1}^n |a_{k,j}| \label{EquMatNormEins} \\
    ||A||_2 & = & \sqrt{\rho(A * A)} \label{EquMatNormZwei} \\
    ||A||_\infty & = & \max_i\sum_{k=1}^n |a_{i,k}| 
                                              \label{EquMatNormInfty} \\
\end{eqnarray}
Die Beweise der Gleichungen \equref{EquMatNormEins},
\equref{EquMatNormZwei} und \equref{EquMatNormInfty} sind in
\cite{Isaa73} ab S. 9 zu finden.

Falls es nicht im Einzelfall anders festgelegt ist, gilt im weiteren 
Text $||\:||=||\:||_2$.

% $$$$ hier Satz "uber maximale Gr"o"se der Eigenwerte  ( <--> Normen )
%       (\det = \prod \lambda !!!)
%      
% ... sehr interessant, jedoch nicht 100-prozentig erforderlich
% ... zu finden in Isaa73 (S. 12)

% **************************************************************************

\MySection{Wahl einer N"aherungsinversen}
\label{SecGuessInverse}
\index{Inverse!einer Matrix} \index{Matrizeninvertierung}
\index{N{\Mya}herungsinverse}

In dem in Kapitel \ref{ChapPan} vorzustellenden Algorithmus wird die
Krylov-Matrix dadurch invertiert, da"s eine N"aherungsinverse berechnet
und diese dann iterativ verbessert wird. In diesem Unterkapitel
wird die Wahl der N"aherungsinversen beschrieben.
Literatur dazu sind \cite{PR85} und \cite{PR85a}.

F"ur die weiteren Betrachtungen ben"otigen wir:
\begin{eqnarray}
    t & := & \frac{1}{ ||A^TA||_1 } \label{EquPanDefT} \\
    B & := & t A^T \label{EquPanDefB} \\
    R(B) & := & E_n - B A \label{EquDefResidual} 
\end{eqnarray}

Im folgenden wird in mehreren Schritten eine Ungleichung f"ur
$||R(B)||$ bewiesen, aus der folgt, da"s das in Unterkapitel \ref{SecNewton}
beschriebene Verfahren effizient auf $B$ angewendet werden kann, um eine
Inverse von $A$ mit zufriedenstellender N"aherung zu erhalten.

\MyBeginDef
\label{DefSymmetrisch}
\index{symmetrisch} \index{Matrix!symmetrische}
    Gilt f"ur eine Matrix $A$ \[ A = A^T \MyKomma \] dann wird sie als
    {\em symmetrisch} bezeichnet.
\MyEndDef

Da die beiden folgenden Lemmata aus den Grundlagen "uber Normen stammen,
werden die Beweise auf einen Literaturverweis beschr"ankt. Sie sind
in \cite{Atki78} ab S. 416 zu finden.

\begin{lemma}
\label{SatzSymmetrischSpektral}
    F"ur jede symmetrische Matrix $A$ gilt:
    \[ ||A||_2 = \rho(A) \MyPunkt \]
\end{lemma}
            
\begin{lemma}
\label{SatzAtkinson}
\[
    ||A^T A||_2 = \rho(A^T A) = ||A||^2 \leq 
    ||A^T A||_1 \leq \max_i \sum_j |a_{i,j}| \max_j \sum_i |a_{i,j}| 
    \leq n ||A^T A||
\]
\end{lemma}

\begin{lemma}
\label{SatzInverseEigenwert}
    Sei $\lambda$ ein Eigenwert der invertierbaren $n \times n$-Matrix $A$. 
    Dann ist $1/\lambda$ ein Eigenwert von $A^{-1}$.
\end{lemma}
\begin{beweis}
    Da $A$ invertierbar ist, ist $\lambda \neq 0$. Sei $v$ ein Eigenvektor
    zu $\lambda$. Dann gilt $v \neq 0_n$, sowie
    \begin{eqnarray*}
        Av & \neq & 0_n \\
        Av & = & \lambda v \MyPunkt
    \end{eqnarray*}
    Damit ergibt sich:
    \[ A^{-1}(Av) = v = (1/\lambda) \lambda v = (1/\lambda) A v \MyKomma\]
    woraus die Behauptung folgt.
\end{beweis}

\begin{lemma}
\label{SatzEigenUngleichung}
    Sei $A$ invertierbar. Sei $\lambda$ ein Eigenwert von $A^T A$.
    Dann gilt:
    \Beq{EquEigenUngleichung}
        \frac{ 1 }{  ||A^{-1}||^2 }
             \leq \lambda \leq ||A||^2
             \MyPunkt
    \Eeq
\end{lemma}
\begin{beweis}
    Die rechte Ungleichung von \equref{EquEigenUngleichung} folgt aus
    \ref{SatzAtkinson}.
    
    Die linke Ungleichung von \equref{EquEigenUngleichung} folgt aus
    \ref{SatzInverseEigenwert} in Verbindung mit \ref{SatzAtkinson}.
\end{beweis}

\begin{satz}
\label{SatzResidualEigenwert}
    Seien $t$, $B$ und $R(B)$ entsprechend der Gleichungen
    \equref{EquPanDefB}, \equref{EquPanDefT} und \equref{EquDefResidual}
    definiert. Sei $\mu$ ein Eigenwert von $R(B)$. Dann
    gilt
    \[
        0 \leq \mu \leq
        1- \frac{ 1
                }{ ||A^T A||_1 ||A^{-1}||^2
                } \MyPunkt
    \]
\end{satz}
\begin{beweis}
    Sei $v$ Eigenvektor von $\mu$ (d. h. $v$ ist nicht der Nullvektor).
    Dann gilt:
    \[ R(B)v = \mu v \MyPunkt \]
    Mit Hilfe von \equref{EquDefResidual} und \equref{EquPanDefB}
    erh"alt man:
    \[ (E_n - t A^T A)v = v - tA^TA v = \mu v \MyPunkt \]
    Daraus folgt
    \[ A^T A v = \lambda v, \: \lambda = \frac{ 1-\mu }{ t } \]
    Also ist $\lambda$ ein Eigenwert von $A^T A$ und mit
    \ref{SatzEigenUngleichung} folgt
    \begin{eqnarray*}
        \frac{ 1 }{ ||A^{-1}||^2
             } \leq \lambda =
            \frac{ 1 - \mu }{ t } \leq ||A||^2 \\
        \Leftrightarrow
            1 - t||A||^2 \leq \mu \leq 1 - \frac{ t }{ ||A^{-1}||^2 }
    \end{eqnarray*}
    Mit Hilfe von Lemma \ref{SatzAtkinson} und \equref{EquPanDefT}
    ergibt sich die Behauptung.
\end{beweis}

Da die Matrix \[ R(B) = E_n - tA^T A \] symmetrisch ist, folgt aus
\ref{SatzSymmetrischSpektral} und \ref{SatzResidualEigenwert} eine
Ungleichung, die es erlaubt, die in \equref{EquPanDefB} definierte Matrix 
$B$ f"ur unsere Zwecke zu verwenden:
\begin{korollar}
\label{SatzNormNaheInvers}
    \[
        ||R(B)|| \leq
            1-\frac{ 1
                   }{ ||A^T A||_1 ||A^{-1}||^2
                   }
    \]
\end{korollar}
Die Benutzung dieser Folgerung wird in Unterkapitel \ref{SecNewton}
beschrieben.

% **************************************************************************

\MySection{Iterative Matrizeninvertierung}
\label{SecNewton}
\index{Iterationsverfahren} \index{Matrizeninvertierung} 
\index{Inverse!einer Matrix}
\index{Newton!Iterationsverfahren von}

In diesem Unterkapitel wird beschrieben, wie man eine gegebene 
N"aherungsinverse $B$ einer invertierbaren Matrix $A$ iterativ schrittweise
verbessern kann. Diese Methode ist in der Literatur als
{\em Newton-Verfahren} bekannt und wird in
\cite{PR85} und \cite{PR85a} sowie in \cite{Hous64} ab S. 64
behandelt. 

Um zu einem Iterationsverfahren zu gelangen, nehmen wir einige Umformungen
an einer Gleichung vor, in der das mit \equref{EquDefResidual} $R(B)$ 
vorkommt:
\begin{MyEqnArray}
                    \MT R(B)          \MT = \MT 0_{n,n} \MNl
    \Leftrightarrow \MT R(B) + E_n    \MT = \MT E_n \MNl
    \Leftrightarrow \MT (R(B) + E_n)B \MT = \MT B \MNl
    \Leftrightarrow \MT (2E_n- BA)B   \MT = \MT B
\end{MyEqnArray}
Man definiert mit Hilfe der letzten dieser Gleichungen die Iteration
\begin{eqnarray}
    B_0 & := & B \nonumber \\
    B_i & := & (2E_n - B_{i-1}A)B_{i-1} \label{EquPanDefNewton} \MyPunkt
\end{eqnarray}
Betrachtet man nun \equref{EquDefResidual} f"ur $B_i$ statt $B$, bekommt
man eine Aussage "uber die St"arke der Konvergenz der Iteration:
\begin{eqnarray*}
    R(B_i) & = & E_n - B_iA \\
           & = & E_n - (2E_n - B_{i-1}A)B_{i-1}A \\
           & = & E_n - 2B_{i-1}A + (B_{i-1}A)^2 \\
           & = & (R(B_{i-1}))^2
\end{eqnarray*}
Daraus folgt, da"s f"ur jede Matrixnorm gilt:
\[ ||R(B_i)|| = ||(R(B_{i-1}))^2|| \MyPunkt \]

Unter der Voraussetzung \Beq{EquResidualLessOne} ||R(B)|| < 1 \Eeq
konvergiert $||R(B_i)||$ also quadratisch gegen Null und die
$B_i$ entsprechend gegen $A^{-1}$.

Um das Iterationsverfahren einsetzen zu k"onnen, m"ussen wir noch 
feststellen, wieviele Iterationen erforderlich sind, um eine bestimmte
Genauigkeit zu erreichen. Dazu definieren wir als Abk"urzung
\[ q:= ||R(B)|| , \: q_i:= ||R(B_i)||, \: q_0 := q \MyPunkt \]
Damit das Verfahren "uberhaupt in praktisch nutzbarer Weise konvergiert,
darf $q$ nicht beliebig nahe bei $1$ liegen. Sei $c \in \Nat$ gegeben. 
Dann nehmen wir an, da"s f"ur $q$ gilt:
\Beq{EquSupposeQ} 
    q = 1 - \frac{1}{n^c} \MyPunkt 
\Eeq
Da das Verfahren quadratisch konvergiert, gilt f"ur $k \in \Nat_0$:
\[
    q_k = q^{2^k} \MyPunkt
\]
Mit \equref{EquSupposeQ} erhalten wir:
\[
    q_k = \lb 1 - \frac{1}{n^c} \rb^{2^k} \MyPunkt
\]
F"ur unsere Anwendung sei
\[ q_k < \frac{1}{d} , \: d \in \Nat \] ausreichend. Mit 
\[ 
    k := e \log(n), \: e \in \Rationals_{>0} 
\]  erhalten wir:
\[ 
    \lb 1 - \frac{1}{n^c} \rb^{n^e} = \frac{1}{d} \MyPunkt
\]
Da uns die Anzahl der Iterationen interessiert, l"osen wir diese Gleichung
nach $e$ auf und nehmen dazu $n \rightarrow \infty$ an:
\begin{MyEqnArray}
    \MT \lb 1 - \frac{1}{n^c} \rb^{n^e} \MT = \MT \frac{1}{d} \MNl
    \Rightarrow \MT
    \lb \lb 1 - \frac{1}{n^c} \rb^{n^c} \rb^{n^e / n^c} \MT = \MT 
                                                       \frac{1}{d} \MNl
    \Rightarrow \MT
    \MathE^{- (n^e / n^c) } \MT \approx \MT \frac{1}{d} \MNl
    \Rightarrow \MT
    \ln(d) \MT \approx \MT n^{e-c} \MNl
    \Rightarrow \MT
    \frac{ \ln(\ln(d)) }{ \ln(n) } + c \MT \approx \MT e    
\end{MyEqnArray}
Wir k"onnen also f"ur wachsende $n$ annehmen, da"s gilt
\[ c \approx e \MyPunkt \]
Um die --- bei gegebenem $n$ --- von $e$ bestimmte Anzahl der Iterationen
zu erhalten, m"ussen wir $c$ n"aher bestimmen. Vergleichen wir dazu
\equref{EquSupposeQ} mit der Aussage von \ref{SatzNormNaheInvers}.
Aus Grundlagen "uber Normen (z. B. \cite{Isaa73} ab S. 12) erhalten wir
f"ur eine beliebige Matrix $W$:
\[
    ||W||_2 \leq \sqrt{||W^2||_1} \MyPunkt
\]
Wir k"onnen also die Berechnung von $c$ auf die Bestimmung der 1-Norm, also
der maximalen Betragsspaltensumme, einer Matrix zur"uckf"uhren und erkennen,
da"s $c$ von der Gr"o"se der Matrizenelemente abh"angt.

Somit erkennen wir eine weitere Einschr"ankung f"ur die Anwendbarkeit des
Algorithmus. Die Matrix mu"s die Bedingung
\Beq{EquWellConditioned}
    ||A^T A||_1 ||A^{-1}||^2 \leq n^c
\Eeq f"ur ein $c \in \Nat$ erf"ullen. Die Wahl von $c$ h"angt davon ab, 
auf welche Matrizen man
das Verfahren anwendet. Um P-Alg. mit den anderen Algorithmen 
vergleichen zu k"onnen, nehmen wir im weiteren Verlauf des Textes $c=1$ an.

% **************************************************************************

\MySection{Determinatenber. mit Hilfe der Methoden von Krylov und Newton}
\label{SecAlgPan}

In diesem Unterkapitel werden die in den vorangegangenen Abschnitten
dargestellten Methoden zu einem Algorithmus zur Determinantenberechnung
zusammengefa"st, auf den mit {\em P-Alg.} Bezug genommen wird\footnote{
vgl. Unterkapitel \ref{SecBez}}. Weiterhin wird die Anzahl der Schritte und
der Prozessoren analysiert. 

Im folgenden ist mit $n^x$ jeweils $\lc n^x \rc$ gemeint.

Im folgenden wird mit Hinweis auf die Bemerkungen auf S.
\pageref{PageAlg2MatMult} f"ur die Multplikation zweier 
$n \times n$-Matrizen ein Aufwand von \[ \gamma_S (\lc\log(n)\rc + 1) \]
Schritten und \[ \gamma_P n^{2+\gamma} \] Prozessoren in Rechnung gestellt.

Es sei die Determinante der $n \times n$-Matrix $A$ zu berechnen.
Zuerst wird die Krylov-Matrix entsprechend der Darstellung in Abschnitt
\ref{SecKrylov} berechnet. Es ist "ublich, f"ur den dort erw"ahnten
Vektor $z$ zur Berechnung der iterierten Vektoren den Einheitsvektor
zu verwenden, dessen s"amtliche Elemente gleich $1$ sind. Aufgrund der
aus theoretischer Sicht ohnehin bereits eingeschr"ankten Verwendbarkeit
der Methode von Krylov f"ur unsere Zwecke, bedeutet die Beschr"ankung auf
diesen Vektor nur eine unbedeutende Verschlechterung des Algorithmus.

Zuerst sind die iterierten Vektoren anhand des Vektors $z$ und der Matrix
$A$ zu berechnen. Dies geschieht anhand der nachstehenden Gleichungen
(\cite{Pan85}, \cite{BM75} S. 128, \cite{Kell82}). Auf den rechten Seiten 
werden nur Ergebnisse vorangegangener Gleichungen oder $z$ bzw. $A$ 
verwendet. Auf den linken Seiten stehen neu berechnete Ergebnisse. Die Terme 
auf den rechten Seiten beschreiben Matrizenmultiplikationen und die linken 
Seiten deren Ergebnisse.
\begin{eqnarray*}
    [ A^3v,\, A^2v ] & = & A^2[ Av,\,v ] \\ \relax
    [ A^7v,\, A^6v,\, A^5v,\, A^4v ] & = & 
                                      A^4[ A^3v,\, A^2v,\, Av,\, v ] 
                                       \\ \relax
    & \vdots & \\ \relax
    [ A^{2*2^h}v,\, \ldots,\, A^{2^h}v ] & = & 
                              A^{2^h}[ A^{2^h-1}v,\, \ldots,\,v ] 
\end{eqnarray*}
Um die jeweils n"achste Gleichung mit Hilfe der schon bekannten Ergebnisse
aufzustellen, sind zwei Matrizenmultiplikationen erforderlich. Mit der
ersten wird die n"achste ben"otigte Potenz von $A$ berechnet. Mit der 
zweiten wird der jeweilige Term der rechten Seite der Gleichung ausgewertet.
Die $n$ gesuchten iterierten Vektoren k"onnen auf diese Weise in
\[ 
    \gamma_S (\lc\log(n)\rc + 1) \lc\log(n)\rc
\] 
Schritten von
\[ 
    \gamma_P n^{2+\gamma}
\] 
Prozessoren berechnet werden.

Anschlie"send ist f"ur die aus iterierten Vektoren bestehende Krylov-Matrix
eine N"aherungsinverse entsprechend der Ausf"uhrungen in Unterkapitel
\ref{SecGuessInverse} zu berechnen. Dazu werden die Gleichungen
\equref{EquPanDefB} und \equref{EquPanDefT} verwendet. Betrachtet man 
diese Gleichungen, erkennt man, da"s zur Berechnung der dortigen Matrix $B$
aus der Matrix $A$ eine Matrizenmultiplikation, eine Berechnung der
$1$-Norm einer Matrix, eine Division durch einen Skalar sowie eine 
Matrix--Skalar--Multiplikation erforderlich ist. An dieser Stelle wird
deutlich, da"s der Algorithmus nicht ohne Divisionen auskommt.

Um die $1$-Norm zu erhalten, sind $n$ Summen von je $n$-Matrizenelementen
zu berechnen und miteinander zu vergleichen. Dies kann mit Hilfe der 
Bin"arbaummethode nach Satz \ref{SatzAlgBinaerbaum} in
\[
    2\lc\log(n)\rc
\] Schritten von
\[ 
    n \lf \frac{n}{2} \rf
\] Prozessoren durchgef"uhrt werden.

Insgesamt kann die Berechnung der N"aherungsinversen also in
\[
    \gamma_S (\lc\log(n)\rc+1) + 2\lc\log(n)\rc + 2
\] Schritten von
\[
    \gamma_P n^{2+\gamma}
\] Prozessoren geleistet werden.

Nachdem sie zur Verf"ugung steht, kann sie mit Hilfe des in Unterkapitel
\ref{SecNewton} beschriebenen Verfahrens verbessert werden.

Ein Iterationsschritt mit Hilfe von Gleichung \equref{EquPanDefNewton}
erfordert zwei Multiplikationen und eine Addtition von Matrizen. Setzt man
f"ur die Anzahl der durchzuf"uhrenden Iterationen zun"achst die Unbestimmte
$I$ ein, kann das Iterationsverfahren in
\[
    I * 2 \gamma_S (\lc\log(n)\rc + 1) + 1
\] Schritten von
\[
    \gamma_P n^{2+\gamma}
\] Prozessoren auf die N"aherungsinverse angewendet werden um diese
bis auf eine ausreichende Genauigkeit an die gesuchte Inverse anzun"ahern.

Schlie"slich ist noch eine Matrix--Vektor--Multiplikation durchzuf"uhren,
um die Methode von Krylov zur Berechnung der Koeffizienten des
charakteristischen Polynoms zu vollenden. Betrachtet man den Vektor wiederum
als Matrix, kann dies in
\[
    \gamma_S(\lc\log(n)\rc+1)
\] Schritten von
\[
    \gamma_P n^{2+\gamma}
\] Prozessoren erledigt werden. Nach \ref{SatzDdurchP} hat man 
mit den Koeffizienten auch die Determinante berechnet.

Hier wird eine Einschr"ankung f"ur den Algorithmus deutlich. Da ein 
N"aherungsverfahren verwendet wird, ist es nicht m"oglich, die Determinante
f"ur Matrizen mit Elementen aus $\Rationals$ genau zu berechnen. Diese
m"ussen deshalb aus $\Integers$ stammen, denn in diesem Fall sind
die Koeffizienten des charakteristischen Polynoms ebenfalls ganzzahlig und
die Determinante kann bei gen"ugend genau durchgef"uhrtem Newton-Verfahren
durch Rundung genau ermittelt werden.

Betrachtet man die Einschr"ankungen f"ur die Verwendbarkeit der Methode
von Krylov (vgl. Unterkaptel \ref{SecKrylov}) sowie die Bedingung
\equref{EquWellConditioned} auf S. \pageref{EquWellConditioned}, erkennt
man, da"s insgesamt nicht unerhebliche Anforderungen an die Matrix,
deren Determianten zu berechnen ist, gestellt werden m"ussen, damit der
in diesem Kapitel dargestellte Algorithmus anwendbar ist.

Als Gesamtaufwand f"ur den Algorithmus erh"alt man f"ur die
Anzahl der Schritte:
\begin{eqnarray*}
   & & \gamma_S (\lc\log(n)\rc + 1) \lc\log(n)\rc \\
   & + & \gamma_S (\lc\log(n)\rc +1 ) + 2\lc\log(n)\rc + 2 \\
   & + & I * 2 \gamma_S (\lc\log(n)\rc + 1) + 1 \\
   & + & \gamma_S(\lc\log(n)\rc+1) \\
   & = &
      \gamma_S \lb \lc\log(n)\rc^2 + 5 \lc\log(n)\rc
    + I \lb 2 \lc\log(n)\rc + 2 \rb + \frac{3}{\gamma_S} + 2 \rb
\end{eqnarray*}
Mit Verweis auf Unterkapitel \ref{SecNewton} nehmen wir $I= \log(n)$ an.
Dadurch lautet der Term f"ur die Anzahl der Schritte:
\[
    \gamma_S \lb 3 \lc\log(n)\rc^2 + 7 \lc\log(n)\rc
    + \frac{3}{\gamma_S} + 2 \rb \MyPunkt
\]
Vergleicht man die Annahme f"ur $I$ mit den Ausf"uhrungen in Unterkapitel
\ref{SecNewton}, erkennt man, da"s dies der beste mit dem Algorithmus zu
erreichende Wert ist. Bei schlechteren Randbedingungen erh"alt man f"ur
die Schritte einen entsprechend schlechteren Wert.

F"ur die Prozessoren ergibt sich:
\[ 
    \gamma_P n^{2+\gamma} \MyPunkt
\]
Da die Matrizenmultiplikation nach \cite{CW90} nur f"ur sehr gro"se $n$
Vorteile bringt, wird \[ \gamma_P = \gamma_S = \gamma = 1 \] gesetzt,
um eine f"ur die Anwendung des Algorithmus realistische 
Vergleichsm"oglichkeit mit den anderen Algorithmen zu bekommen.

Vergleicht man P-Alg. mit C-Alg., BGH-Alg. und B-Alg., so erkennt man, da"s
die Anzahl der Prozessoren in P-Alg. gleich ist mit der Anzahl der
Prozessoren f"ur die Matrizenmultiplikation ist und somit um eine Potenz
geringer als beim besten der drei anderen Algorithmen.

Ein gravierender Nachteil von P-Alg. sind die Einschr"ankungen
f"ur die Benutzbarkeit. Die Bedingung, da"s die Matrizenelemente ganzzahlig
sein m"ussen, l"a"st sich durch Ausnutzung der Eigenschaften einer 
Determinaten (siehe Definition \ref{DefDet}) ausgleichen. Dazu werden
die Matrizenelemente durch einen geeigneten Faktor multipliziert, so da"s
die resultierende Matrix nur ganzzahlige Elemente enth"alt. Nachdem der
Algorithmus auf diese Matrix angewendet worden ist, kann man dann aus
dem Ergebnis auf die Determinate der urspr"unglichen Matrix schlie"sen.

Durch die in Unterkapitel \ref{SecKrylov} beschriebenen Einschr"ankungen,
kann f"ur P-Alg. nicht garantiert werden, da"s er f"ur jede invertierbare
Matrix eine Determinante ungleich Null liefert. Die Wahrscheinlichkeit f"ur
einen solchen Fall ist zwar gering, jedoch unterscheidet diese 
Beschr"ankung P-Alg. von den anderen drei Algorithmen.

Weiterhin mu"s man bei P-Alg. nat"urlich wiederum auf die Existenz von 
Divisionen und das Fehlen von Fallunterscheidungen hinweisen.


%
% Datei: implemen.tex
%
\MyChapter{Implementierung}
\label{ChapImplemen}

In diesem Kapitel wird die Implementierung der in den Kapitel 
\ref{ChapCsanky} bis \ref{ChapPan} behandelten Algorithmen 
beschrieben\footnote{Der Algorithmus nach dem Entwicklungssatz von
Laplace (siehe Unterkapitel \ref{SecDivCon}) wird hier nicht 
ber"ucksichtigt.}.

\MySection{Erf"ullte Anforderungen}
\label{SecAnford}

In diesem Unterkapitel werden die wesentlichen Eigenschaften des
implementierten Programms beschrieben um einen "Uberblick "uber dessen
Leistungsmerkmale zu geben.

Die Algorithmen sind in Modula-2 auf einem Rechner mit einem Prozessor
implementiert\footnote{Megamax Modula-2 auf einem ATARI ST}. 
Als Literatur "uber die Programmiersprache Modula-2 ist
z. B. \cite{DCLR86} empfehlenswert.

Alle Algorithmusteile, die parallel ausgef"uhrt werden sollen,
werden mit Hilfe von Schleifen nacheinander ausgef"uhrt. Mit Hilfe von
Z"ahlprozeduren (siehe Modul `Pram') wird w"ahrend der Programmausf"uhrung
ermittelt, in wieviel Schritten und mit Hilfe von wieviel Prozessoren der
Algorithmus durch eine PRAM abgearbeitet werden kann.

Das Programm erm"oglicht es, Matrizen anhand von Parametervorgaben 
zuf"allig zu erzeugen. Die Parameter sind
\begin{itemize}
\item Gr"o"se der Matrix,
\item
      Rang (um auch Matrizen mit der Determinante Null gezielt 
      erzeugen zu k"onnen),
\item
      Wahl einer der Mengen\footnote{durch geeignete Parameterwahl mit dem 
      Befehl `param'}  $\Nat$, $\Integers$, $\Rationals$ oder
      $\Rationals^+$ \footnote{alle Elemente aus $\Rationals$, die
      gr"o"ser oder gleich Null sind} f"ur die Matrizenelemente,
      da Algorithmen evtl. nicht auf jede dieser Mengen anwendbar sind,
      und
\item 
      Vielfachheiten der Eigenwerte (wichtig f"ur den Algorithmus von Pan).
\end{itemize}

F"ur erzeugte Matrizen kann jeder der implementierten Algorithmen aufgerufen
werden. Eine Matrix wird zusammen mit den durch die verschiedenen 
Algorithmen berechneten Determinanten und den Me"sergebnissen der 
Z"ahlprozeduren unter einem Namen auf dem Hintergrundspeicher abgelegt. 
Diese Verwaltung geschieht automatisch. Dazu mu"s der Benutzer vor jedem
neuen Anlegen eines aus den obigen Daten bestehenden Datensatzes einen
Namen f"ur diesen Datensatz angeben.

Das Programm ist kommandoorientiert. D. h. nach dem Start wird der 
Benutzer aufgefordert, Befehle einzugeben, die nacheinander ausgef"uhrt 
werden.

Die Lesbarkeit des Quelltextes wird durch h"aufige Kommentare gesteigert.

% **************************************************************************

\MySection{Bedienung des Programms}
\label{SecBedienung}

In diesem Unterkapitel wird die Benutzung des Programms beschrieben.

Nach dem Programmstart mu"s zun"achst mit Hilfe des Befehls {\em find}
ein Datensatz festgelegt werden, der bearbeitet werden soll. Dies kann
ein bereits existierender oder ein neu anzulegender sein.

Bei einem neu angelegten Datensatz m"ussen danach mit Hilfe des Befehls 
{\em param} die Parameter f"ur die zu erzeugende Matrix festgelegt werden.

Anschlie"send kann man mit {\em gen} eine neue Matrix erzeugen lassen.
Nachdem mit {\em param} die Parameter festgelegt sind, kann mit mit {\em gen}
zu jeder Zeit eine neue Matrix generieren lassen, wodurch jedoch die 
vorangegangene Matrix verloren geht.

Wenn eine Matrix erzeugt worden ist, kann man mit Hilfe der Befehle
{\em berk, bgh, csanky} und {\em pan} die entsprechenden Algorithmen auf 
die generierte Matrix anwenden.

Mit {\em show} kann man sich zu jeder Zeit den aktuellen mit {\em find}
bestimmten Datensatz auf dem Bildschirm anzeigen lassen. Die generierte
Matrix wird nur ausgegeben, wenn eine Matrixzeile in eine Bildschirmzeile 
pa"st. Gr"o"sere Matrizen kann man mit {\em mshow} trotzdem ausgeben 
lassen.

Falls man einen weiteren Datensatz anlegen oder einen bereits vorhandenen
wieder bearbeiten will, benutzt man erneut den Befehl {\em find}. Die
Speicherung des alten Datensatzes geschieht automatisch.

Weitere Befehle, die man nach dem Programmstart zu jedem Zeitpunkt angeben
kann, sind: {\em del, exit, h, help, hilfe, ?, ls} und {\em q} . Ihre
Bedeutung ist in der folgenden Liste aller erlaubten Befehle erkl"art:

\begin{MyDescription}
\MyItem{\em berk}
    Der Algorithmus von Berkowitz aus Kapitel \ref{ChapBerk} wird auf
    die Matrix des aktuellen Datensatzes angewendet. Die berechnete
    Determinante sowie die Ergebisse der Z"ahlprozeduren 
    (siehe Modul `Pram') werden im aktuellen Datensatz abgelegt.
\MyItem{\em bgh}
    Die Wirkung dieses Befehls ist analog zu der des Befehls {\em berk},
    jedoch bezogen auf den Algorithmus von Borodin, von zur Gathen und
    Hopcroft aus Kapitel \ref{ChapBGH}.
\MyItem{\em csanky}
    Die Wirkung dieses Befehls ist analog zu der des Befehls {\em berk},
    jedoch bezogen auf den Algorithmus von Csanky aus Unterkapitel 
    \ref{SecAlgFrame}.
\MyItem{\em del}
    Es wird nach einem Datensatznamen gefragt. Der zugeh"orige Datensatz
    wird im Hauptspeicher und auf dem Hintergrundspeicher gel"oscht.
\MyItem{\em exit}
    Das Programm wird beendet. Der aktuelle mit {\em find} festgelegte
    Datensatz wird automatisch auf dem Hintergrundspeicher abgelegt.
\MyItem{\em find}
    Es wird nach einem Datensatznamen gefragt. Falls ein Datensatz mit 
    diesem Namen bereits im Hauptspeicher abgelegt ist, wird dieser erneut
    zum aktuellen Datensatz. Falls dies nicht der Fall ist und auf dem 
    Hintergrundspeicher ein Datensatz mit diesem Namen abgelegt ist,
    wird dieser in den Hauptspeicher geladen und zum aktuellen Datensatz.
    Falls beide F"alle nicht zutreffen, wird ein neuer Datensatz mit dem 
    angegebenen Namen im Hauptspeicher angelegt und als aktueller 
    Datensatz betrachtet.
\MyItem{\em gen}
    Entsprechend der mit {\em param} festgelegten Parameter wird eine neue
    Matrix f"ur den aktuellen Datensatz generiert. Die dort durch die
    Befehle {\em berk, bgh, csanky} und {\em pan} abgelegten Daten werden
    gel"oscht.
\MyItem{\em h, help, hilfe, ?}
    Durch diese Befehle wird eine Kurzbeschreibung aller erlaubten Befehle
    auf den Bildschirm ausgegeben.
\MyItem{\em ls}
    Auf dem Bildschirm wird eine Liste der Namen der im Hauptspeicher
    befindlichen Datens"atze ausgegeben.
\MyItem{\em mshow}
    Die Matrix des aktuellen Datensatzes wird auf dem Bildschirm ausgegeben.
\MyItem{\em pan}
    Die Wirkung dieses Befehls ist analog zu der des Befehls {\em berk},
    jedoch bezogen auf den Algorithmus von Pan aus Kapitel \ref{ChapPan}.
\MyItem{\em param}
    Es wird nach den Parametern f"ur die mit Hilfe von {\em gen} zu
    generierende Matrix gefragt. Alle zuvor im aktuellen Datensatz
    abgelegten Daten werden gel"oscht.
\MyItem{\em q}
    Die Wirkung dieses Befehls ist mit der des Befehls {\em exit} identisch.
\MyItem{\em show}
    Der aktuelle Datensatz wird auf dem Bildschirm ausgegeben. Die 
    Matrix wird nur ausgegeben, falls eine Matrixzeile in eine 
    Bildschirmzeile pa"st. Mit dem Befehl {\em mshow} kann die Matrix 
    dennoch ausgegeben werden.
\end{MyDescription}

% **************************************************************************

\MySection{Die Modulstruktur}
\label{SecModule}

In diesem Unterkapitel wird die Struktur des implementierten Programms 
beschrieben. Dazu wird zu jedem Modul dessen Aufgabe und evtl. dessen
Beziehung zu anderen Modulen angegeben. Alle Beschreibungen von Details
der Implementierung, die f"ur die Benutzung des Programms und f"ur 
das Verst"andnis von dessen Gesamtstruktur unwichtig sind, erfolgen durch 
Kommentare innerhalb des Quelltextes (siehe Anhang).

Eine Beschreibung des Programmoduls {\em main} entspricht einer Erkl"arung
der Benutzung des Programms. Diese ist in Unterkapitel \ref{SecBedienung}
zu finden.

Die Beschreibung der Programmodule zum Test einzelner Teile des 
Gesamtprogramms ist von untergeordnetem Interesse und beschr"ankt sich 
deshalb auf kurze Bemerkungen "uber ihren Zweck ihm Rahmen der 
alphabetischen Auflistung (s. u.).

Die vorrangige Aufmerksamkeit des an der Implementierung der Algorithmen
Interessierten sollte sich auf die Module {\em Det} und {\em Pram} sowie
auf das Programmodul {\em algtest} richten. 

Die genannten vorrangig interessanten Module sind im Anhang 
\ref{ChapImplDet} zusammen mit dem Modul {\em main} gesammelt. Die
weniger interessanten Testprogrammodule sind im Anhang \ref{ChapTest}
aufgef"uhrt. Alle weiteren Module sind in alphabetischer Reihenfolge in
Anhang \ref{ChapSupport} zu finden.

Zun"achst wird anhand eines {\em reduzierten Ebenenstrukturbildes}
(Erkl"arung s. u.) ein "Uberblick "uber die Programmstruktur geben. 
Anschlie"send erfolgt eine alphabetische Auflistung der Module und 
ihrer Erkl"arungen.

In das erw"ahnte Strukturbild sind alle Module nach folgenden Regeln
eingetragen:
\begin{itemize}
\item
      Die Eintragung erfolgt ebenenweise. Die niedrigste Ebene ist im
      Bild unten zu finden und die h"ochste oben. Jedes Modul geh"ort 
      genau einer Ebene an.
\item
      Jedes Modul wird im Rahmen der Ma"sgaben durch die anderen Regeln
      in einer m"oglichst niedrigen Ebene eingetragen.
\item
      Von jedem Modul A aus, das ein Modul B benutzt, z. B. durch Aufruf von
      Prozeduren des Moduls B, wird im Rahmen der Einschr"ankungen durch
      andere Regeln ein Pfeil auf dieses Modul $B$ gerichtet.
\item 
      Jedes Modul wird so eingetragen, da"s kein Pfeil von ihm auf ein 
      Modul auf der gleichen oder einer h"oheren Ebene gerichtet ist.
\item
      Alle Pfeile von einem Modul aus auf Module, die nicht genau eine
      Ebene tiefer angeordnet sind, werden weggelassen.
\end{itemize}
Durch die letzte Regel gewinnt das Strukturbild erheblich an
"Ubersichtlichkeit, ohne wesentlichen Informationsgehalt zu verlieren.
Aus dem Quelltext jedes Moduls ist zu entnehmen, welche Module, au"ser
den im Bild angegebenen, sonst noch benutzt werden. Die aufgef"uhrten 
Regeln liefern das in Abbildung \ref{PicModule} angegebene Bild.

\begin{figure}[htb]
\begin{center}
    \input{bilder/module}
    \caption{reduziertes Ebenenstrukturbild}
    \label{PicModule}
\end{center}
\end{figure}

Es folgen die Kurzbeschreibungen der einzelnen Module in alphabetischer 
Reihenfolge.

\begin{MyDescription}
\MyItem{\bf algtest (Programmodul)}
    Dieses Modul dient zum Test der Algorithmen zur 
    Determinantenberechnung ohne Behinderung durch Anforderungen 
    irgendwelcher Art, insbesondere
    ohne Beachtung der Parallelisierung und der Ma"sgabe, Matrizen
    beliebiger Gr"o"se zu verarbeiten.
\MyItem{Cali (CArdinal LIst)}
    In diesem Modul sind lineare Listen positiver ganzer Zahlen 
    implementiert. Es st"utzt sich auf das Modul `List'.
\MyItem{Data}
    Das Modul `Data' dient der Verwaltung der Datens"atze bestehend aus 
    Matrizen und ihren Parametern, sowie der berechneten Determinanten 
    und der dabei gez"ahlten Schritte und Prozessoren.
\MyItem{Det}
    In diesem Modul sind die Algorithmen zur parallelen
    Determinantenberechnung implementiert.
\MyItem{Frag (array FRAGments)}
    Im Modul `Frag' sind Felder beliebiger variabler L"ange und beliebigen
    Inhalts implementiert. Das Modul ist erforderlich, um Matrizen 
    verarbeiten zu k"onnen, deren Gr"o"se durch den Benutzer erst w"ahrend 
    der Laufzeit des Programms festgelegt wird.
    
    Das Modul profitiert von der Verwaltung von 
    Elementen beliebiger Typen durch das Modul `Type'.
\MyItem{Func (FUNCtions)}
    In diesem Modul sind verschiedene Prozeduren und Funktionen 
    insbesondere f"ur mathematische Zwecke zusammengefa"st.
\MyItem{Hash}
    Durch dieses Modul werden Prozeduren zur Streuspeicherung, auch unter
    dem Namen `Hashing' bekannt, zur Verf"ugung gestellt. Das Modul wird 
    durch den Algorithmus von Borodin, von zur Gathen und Hopcroft im 
    Modul `Det' ben"otigt, um Zwischenergebnisse bei der parallelen 
    Berechnung von Termen zu speichern.
    
    Das Modul `Hash' erlaubt es, beliebige Daten zu speichern. Dabei wird
    auf das Modul `Type' zur Verwaltung von Elementen beliebiger Typen
    zur"uckgegriffen.
\MyItem{Inli (INteger LIst)}
    In diesem Modul sind lineare Listen ganzer Zahlen 
    implementiert. Es st"utzt sich auf das Modul `List'.
\MyItem{List}
    Das Modul `List' stellt Prozeduren zur Verwaltung von linearen 
    doppelt verketteten Listen bliebiger Elemente zur Verf"ugung. Analog
    zu den Modulen `Frag', `Hash' und `Mat' benutzt `List' das Modul
    'Type` zur Verwaltung von Elementen beliebiger Typen.
    
    Auf dem Modul `List' bauen verschiedene Module zur Implementierung von
    Listen spezieller Typen auf.
\MyItem{\bf listtest (Programmodul)}
    Dieses Programmodul dient zum Test des Moduls `List'. Es verwendet
    dazu das Modul `Cali'.
\MyItem{\bf main (Programmodul)}
    Dies ist das Hauptmodul des gesamten Programms. Es nimmt die Befehle
    des Benutzers entgegen und ruft die entsprechenden Prozeduren auf.
    Die Benutzung ist in Unterkapitel \ref{SecBedienung} beschrieben.
\MyItem{Mali (MAtrix LIst)}
    In diesem Modul sind lineare Listen von Matrizen
    implementiert. Es st"utzt sich auf die Module `List' und `Mat'.
\MyItem{Mat (MATrix)}
    Dieses Modul stellt Prozeduren zur Verwaltung von zweidimensionalen 
    Matrizen beliebiger Gr"o"se f"ur beliebige Elemente zur Verf"ugung.
    Es st"utzt sich auf das Modul `Frag' zur Verwaltung der Felder
    beliebiger Gr"o"se und auf das Modul `Type' zur Verwaltung von 
    Elementen beliebiger Typen.
\MyItem{Pram}
    Das Modul `Pram' stellt die Z"ahlprozeduren zur Verf"ugung, die zur 
    Ermittlung des Aufwandes f"ur
    eine PRAM zur Abarbeitung der verschiedenen Algorithmen zur 
    Determiantenberechnung erforderlich sind. Das Modul wird durch die
    Module 'Det' und 'Rema' benutzt und verwendet seinerseits insbesondere
    das Modul `Cali' f"ur Verwaltungsaufgaben.
\MyItem{\bf pramtest (Programmodul)}
    Dieses Programmodul dient zum Test des Modul `Pram'.
\MyItem{Reli (REal LIst)}
    In diesem Modul sind lineare Listen von Flie"skommazahlen
    implementiert. Es st"utzt sich auf das Modul `List'.
\MyItem{Rema (REal MAtrix)}
    Dieses Modul implementiert Matrizen aus Flie"skommazahlen. Es st"utzt
    sich dazu auf das Modul \nopagebreak[3] `Mat'.
\MyItem{Rnd  (RaNDomize)}
    Das Modul 'Rnd' erlaubt es, Zufallszahlen nach der linearen 
    Kongruenzmethode zu erzeugen. Es wird vom Modul `Rema' dazu benutzt,
    anhand von verschiedenen Parametern zuf"allige Matrizen zu generieren.
\MyItem{\bf rndtest (Programmodul)}
    Dieses Programmodul dient zum Test des Moduls `Rnd'.
\MyItem{simptype (SIMPle TYPE)} \sloppy
    Dieses Modul stellt Verwaltungsprozeduren f"ur die einfachen 
    Datentypen \newline[3] `LONGCARD', `LONGINT' und `LONGREAL' zur 
    Verf"ugung, damit
    sie in Verbindung mit dem Modul `Type' verwendet werden k"onnen.
    \fussy
\MyItem{Str (STRing)}
    Im Modul `Str' sind diverse Prozeduren zur Verarbeitung von 
    Zeichenketten implementiert.
\MyItem{\bf strtest (Programmodul)}
    Dieses Programmodul dient zum Test des Moduls `Str'.
\MyItem{Sys}
    Dieses Modul stellt Prozeduren zum Ablegen von Daten auf dem 
    Hintergrundspeicher zur Verf"ugung. Da die Behandlung der Massenspeicher
    auf den verschiendenen Rechnersystemen unterschiedlich ist, mu"s
    das Modul 'Sys' bei der Portierung des Programms auf einen
    anderen Rechner neu implementiert werden. 
\MyItem{SysMath}
    Die zur Verf"ugung gestellten mathematischen Funktionen sind von 
    System zu System unterschiedlich. Deshalb sind im Modul `SysMath' die
    Funktionen gesammelt, die im Programm benutzt werden. Bei der Portierung
    des Programms auf ein anderes Computersystem mu"s dieses Modul evtl.
    angepa"st werden.
\MyItem{Type}
    Dieses Modul dient der Verwaltung von Elementen beliebiger Datentypen.
    Ein neuer Typ wird im Rahmen dieses Moduls durch die Angabe 
    verschiedener Verwaltungsprozeduren definiert. Das Modul "ubernimmt 
    auf diese Weise die Sammlung der Eigenschaften verschiedener Typen, um 
    so die "Ubersichtlichkeit zu steigern. Ohne dieses Modul mu"s jedes 
    der Module `Frag', `Hash', `List' und `Mat' eine entsprechende 
    Verwaltung separat enthalten.
\MyItem{\bf typetest (Programmodul)}
    Dieses Programmodul dient zum Test des Moduls `Type'.
\end{MyDescription}

% **************************************************************************

\MySection{Anmerkungen zur Implementierung}

An dieser Stelle werden einige praktische Gesichtspunkte der
Implementierung kommentiert.

Vergleicht man das Modul `algtest' mit dem Rest des Quelltextes, so erkennt 
man, da"s insbesondere die Anforderung der {\em Pseudoparallelisierung}
die L"ange des Quelltextes stark vergr"o"sert. Bei den Algorithmen im
Modul `Det' handelt es sich ungef"ahr um eine Vergr"o"serung 
um den Faktor 5.

Weiterhin zeigt sich, da"s eine flexible auch nachtr"aglich 
erweiterbare Programmstruktur, die unter dem Gesichtspunkt sich evtl. 
anschlie"sender Arbeiten w"unschenswert ist, nicht unerheblichen Aufwand
bedeutet. So machen die eigentlich interessierenden Programmteile nur
ca. 30 Prozent des Quelltextes\footnote{Gesamtl"ange ca. 9500 Zeilen} aus. 
Der gesamte weitere Aufwand ergibt sich
einerseits aus verschiedenen Anforderungen an Leistungen und Struktur des 
Programms, andererseits aus der Notwendigkeit, Datentypen zu implementieren,
die die verwendete Sprache nicht standardm"a"sig zur Verf"ugung stellt.

Auf eine Implementierung auf leistungsf"ahigeren Rechnern wurde verzichtet,
da der ben"otigte Speicherplatz quadratisch mit der Anzahl der Zeilen und
Spalten der Matrizen w"achst. Der zus"atzliche Speicherplatz f"uhrt nicht
zu einer Steigerung der Matrizengr"o"se von weitreichendem Interesse.
Diese Beschr"ankung erlaubt es, mit der relativ geringen Rechenleistung
eine ATARI ST auszukommen.

Gr"o"sere Matrizen sind auch aus einem weiteren Grund nicht ohne
erheblichen weiteren Aufwand sinnvoll. Die Standardarithmetiken
verschiedener Implementierungen von Programmiersprachen erlauben eine
Rechengenauigkeit von typischerweise ca. 19 Stellen. Dies reicht nicht
aus, um Determinanten
gr"o"serer Matrizen "uberhaupt darzustellen. Deshalb ist es f"ur eine
deutliche Steigerung der Matrizengr"o"se erforderlich, eine eigene
Flie"skommaarithmetik zu implementieren, die es erm"oglicht, mit
beliebiger Genauigkeit\footnote{im Rahmen der physikalischen Grenzen}
zu rechnen.

Die praktischen Erfahrungen in verschiedenen Bereichen der angewandten
Informatik zeigen, da"s in bestehenden in der Regel zufriedenstellend
laufenden Programmen eine Restquote an Programmierfehlern im
Quelltext von ca. einem Fehler pro 1000 Zeilen Quelltext existiert.
Beim gegenw"artigen Stand der Technik ist es nicht m"oglich, Programme
wesentlich fehlerfreier zu bekommen.

Besonders bei mathematischen Programmen ist es in der Regel erforderlich,
trotzdem nahezu Fehlerfreiheit zu erreichen, was die Implementierung
solcher Programme zus"atzlich erschwert. Ein praktisches Beispiel
f"ur diese Probleme sind die implementierten Algorithmen zur
Determiantenberechnung. Die pseudoparallelen Algorithmen im Modul
`Det' besitzen eine Gesamtl"ange von ca. 2300 Zeilen, erheblich mehr
als die Implementierungen im Programmodul 'algtest'.

Da die Dauer einer Fehlersuche schwer abzusch"atzen ist und nur begrenzte
Zeit zur Verf"ugung stand, haben die genannten Schwierigkeiten dazu
gef"uhrt, da"s zwar alle Algorithmen lauff"ahig sind, jedoch
die im Anhang zu findenden Implementierungen leider keine
Determinanten berechnen:
\begin{itemize}
\item P-Alg. im Programmodul `algtest' sowie
\item BGH-Alg., B-Alg. und P-Alg. im Modul `Det'.
\end{itemize}
Durch umfangreiche Testl"aufe kann ausgeschlossen werden, da"s die Fehler
au"serhalb der Module zu suchen sind. Es mu"s sich jeweils um fehlerhafte
Implementierung der Algorithmusbeschreibungen in den jeweiligen Kapiteln
handeln (z. B. Vorzeichenfehler oder falsche Indizes).


%
% Datei: endbem.tex
%
\MyChapter{Nachbetrachtungen}
\label{ChapEndbem}

In diesem Kapitel werden die Ergebnisse dieser Arbeit abschlie"send 
betrachtet und bewertet.

\MySection{Vergleich der Algorithmen}
% Csanky: schnellster, am einfachsten zu implementieren, jedoch Divisionen
% BGH: relativ gesehen sehr langsam, Implementierung sehr aufw"andig,
%      nur f"ur positive ganze Zahlen (mit Tricks: ...)
% Berk: etwas langsamer als 'Csanky', etwas schwieriger zu implementieren,
%       keine Divisionen!
% Pan: nicht f"ur alle Matrizen verwendbar; nur ganze Zahlen

In diesem Unterkapitel werden die vier haupts"achlich in dieser Arbeit 
behandelten Algorithmen zusammenfassend miteinander verglichen.

Auf die Algorithmen wird, wie in Unterkapitel \ref{SecBez} definiert 
wird, mit Hilfe der Anfangsbuchstaben ihrer Autoren
Bezug genommen: C-Alg., BGH-Alg., B-Alg. und P-Alg..

Der Vergleich wird nach folgenden Kriterien durchgef"uhrt:
\begin{itemize}
\item
      Werden Divisionen benutzt?
\item
      Werden Fallunterscheidungen durchgef"uhrt?
\item
      Wie gro"s ist die Anzahl der Schritte?
\item
      Wie gro"s ist die Anzahl der Prozessoren?
\item
      Welche Einschr"ankungen f"ur die Anwendbarkeit gibt es?
\item
      Wie aufwendig ist die Implementierung?
\item
      Welche "Ahnlichkeiten in der Methodik gibt es? 
\end{itemize}
Im Anschlu"s an diesen Vergleich wird eine Bewertung der Algorithmen 
anhand der genannten Gesichtspunkte vorgenommen.

Zun"achst sei angemerkt, da"s keiner der Algorithmen Fallunterscheidungen
verwendet. Dies vereinfacht eine Betrachtung aus der Sicht des 
Schaltkreisentwurfs.

Es kommen lediglich B-Alg. und BGH-Alg. ohne Divisionen aus und k"onnen 
somit in beliebigen Ringen angewendet werden. C-Alg. und P-Alg. k"onnen nur
in K"orpern verwendet werden, falls man exakte Ergebnisse verlangt

P-Alg. darf nur auf ganzzahlige Matrizen
angewendet werden. Da P-Alg. Divisionen verwendet, ist es jedoch auch dann
nicht gew"ahrleistet, da"s er die Determinante exakt liefert. Hinzu kommt
bei P-Alg. als erheblicher Nachteil die eingeschr"ankte Verwendbarkeit, auch 
wenn sich dies in der praktischen Anwendung nicht sehr stark auswirkt. 
Diese Eigenschaft von P-Alg. ist mit dem Laufzeitverhalten von `Quicksort'
vergleichbar, das im {\em Average Case} sehr gut, jedoch im {\em Worst 
Case} sehr schlecht ist. P-Alg. ist theoretisch nur eingeschr"ankt 
verwendbar, praktisch jedoch nahezu uneingeschr"ankt, da zuf"allig
zusammengestellte Matrizen mit sehr hoher Wahrscheinlichkeit invertierbar
sind und die Bedingungen f"ur P-Alg. erf"ullen.

Betrachtet man die Anzahl der Schritte, stellt man fest, da"s C-Alg. 
am schnellsten ist, gefolgt von B-Alg. und P-Alg.. Das Schlu"slicht 
bildet mit einigem Abstand BGH-Alg. 

Bez"uglich der Anzahl der Prozessoren hat P-Alg. die Nase vorn, gefolgt
von C-Alg. und B-Alg.. Das Schlu"slicht bildet BGH-Alg. wiederum mit
Abstand. Zu beachten ist der sehr gute Wert f"ur die Prozessoren bei P-Alg.,
der sich vermutlich kaum weiter verbessern l"a"st.

Auf einem Rechner mit einem Prozessor ist P-Alg. der effizienteste, da
er insgesamt die wenigsten Operationen ben"otigt. Er wird gefolgt von 
C-Alg. B-Alg. liegt hier an dritter Stelle gefolgt von BGH-Alg..

Betrachtet man den Aufwand f"ur die Implementierung gemessen in Anzahl der
Quelltextzeilen\footnote{Dies besitzt in der Praxis in Verbindung mit
den anderen Eigenschaften der Algorithmen eine nicht zu untersch"atzende 
Bedeutung.}, stellt man fest, da"s C-Alg. bei weitem am einfachsten 
zu implementieren ist. Ihm folgt P-Alg. dichtauf. Die Implementierung
von B-Alg. ist bereits etwas aufwendiger, aber noch zumutbar. BGH-Alg.
bildet auch in diesem Punkt mit relativ gro"sem Abstand das Schlu"slicht.

Bei der Betrachtung der Verfahren, die die einzelnen Algorithmen verwenden,
erkennt man Parallelen zwischen C-Alg. und B-Alg. sowie zwischen 
BGH-Alg. und P-Alg.. Sowohl in C-Alg. als auch in B-Alg. wird jeweils
ein Satz verwendet, der zum Zeitpunkt der Ver"offentlichung der Algorithmen
bereits seit mehreren Jahrzehnte bekannt war. Die beiden S"atze sind 
durch diverse Umformungen f"ur die parallele Determinanteberechnung nutzbar
gemacht worden. In BGH-Alg. und P-Alg. hingegen werden jeweils mehrere
auch separat bedeutsame Verfahren zu einem Algorithmus in Verbindung 
gebracht.

Unter dem Gesichtspunkt, da"s BGH-Alg. nicht der einzige divisionsfreie
Algorithmus mit polynomiellem Aufwand ist, besitzt er wegen des
erforderlichen Implementierungsaufwandes und seiner relativen Langsamkeit
nur theoretisches Interesse. 

C-Alg. ist am brauchbarsten f"ur schnelle Erstellung einer Implementierung,
falls das Vorhandensein von Divisionen nicht weiter st"ort.

P-Alg. sollte verwendet werden, falls es im wesentlichen auf Geschwindigkeit
ankommt und die Einschr"ankungen f"ur die Anwendbarkeit sowie die
Existenz von Divisionen nicht st"oren.

B-Alg. ist der Algorithmus unter den vieren mit den ausgewogensten
Leistungsmerkmalen und geht insgesamt als Sieger aus dem Vergleich hervor.
Er besitzt eine ausreichende Effizienz, kommt ohne Divisionen aus und
unterliegt keinen sonstigen Einschr"ankungen, wie z. B. P-Alg.. Im
Zweifelsfall sollte immer B-Alg. verwendet werden.

Um zum Abschlu"s einen Eindruck von der Effizienz der
Algorithmen im direkten Vergleich
zu liefern, sind f"ur Tabelle \ref{TabVergleich} die aus den
Aufwandsanalysen hervorgegangene Terme beispielhaft ausgewertet worden.
In der Tabelle werden bei den Anzahlen der Prozessoren die guten
Werte f"ur P-Alg. und die auff"allig schlechten Werte f"ur BGH-Alg.
besonders deutlich.

\begin{table}[htb]
    \begin{center}
    \begin{tabular}{|c||r|r||r|r||r|r||r|r|}
        \hline
        n & \multicolumn{2}{|c||}{C-Alg.}   & 
            \multicolumn{2}{|c||}{BGH-Alg.} &
            \multicolumn{2}{|c||}{B-Alg.}   &
            \multicolumn{2}{|c|}{P-Alg.} \\
        \cline{2-9}
          & Schr. & Proz. & Schr. & Proz. &
            Schr. & Proz. & Schr. & Proz. \\
        \hline
        2 & 9 & 8  & 16 & 24  & 12 & 4  & 15 & 8 \\ \hline
        4  & 16 & 128  & 55 & 1486  & 19 & 27  & 31 & 64 \\ \hline
        6  & 25 & 648  & 99 & 15343  & 27 & 361  & 53 & 216 \\ \hline
        8  & 25 & 2048  & 118 & 81284  & 36 & 2234  & 53 & 512 \\ \hline
        10  & 36 & 5000  & 172 & 298460  & 41 & 6417  & 81 & 1000 \\ \hline
        12  & 36 & 10368  & 180 & 867715  & 49 & 15670  & 81 & 1728 \\ \hline
        14  & 36 & 19208  & 196 & 2145346  & 49 & 41922  & 81 & 2744 \\ \hline
        16  & 36 & 32768  & 205 & 4708392  & 53 & 72317  & 81 & 4096 \\ \hline 
        18  & 49 & 52488  & 265 & 9431425  & 59 & 120961  & 115 & 5832 \\ \hline
        20  & 49 & 80000  & 275 & 17574870  & 69 & 194580  & 115 & 8000 \\ \hline
    \end{tabular}
    \end{center}
    \caption{Vergleich der Algorithmen}
    \label{TabVergleich}
\end{table}

% $$$ Reihenentwicklung Grad >n fuer BGH-Alg. sch"adlich
%     Algorithmen auch ohne Parallelisierung interessant

\MySection{Ausblick}

Zum Schlu"s folgt noch eine kurze Liste weiterer Themen, auf die man bei der 
Bearbeitung der vorliegenden Arbeit st"o"st:

\begin{itemize}
\item Analyse von Schaltkreisen f"ur die beschriebenen Algorithmen
      % \cite{ ... Ingos Buch}
\item Betrachtung von Matrizen mit Elementen aus $\Complex$
      % Diagonalisierbarkeit in $\Complex$ !? --> Methode von Krylov
\item Analyse verschiedener Varianten der beschriebenen Algorithmen
      % Pan: 3 Verfahren zur iterativen Invertierung
      %      etliche Verfahren zur Bestimmung der N"aherungsinversen
      % Berk: Wahl von \epsilon; Wahl der Parallelisierung von
      %       Algorithmusteilen
      % BGH: etliche M"oglichkeiten, die Konvergenz der Potenzreihen
      %      sicherzustellen
\item Analyse des Aufwandes f"ur die Aufgabenverteilung zwischen mehreren
      Prozessoren
\item Betrachtung anderer Rechnermodelle (insbesondere Rechnermodelle
      ohne gemeinsamen Speicher f"ur die Prozessoren)
      % CREW, Arten der Zugriffsregelung
      % kein gemeinsamer Speicher (Analyse des Kommunikationsaufwandes)
% \item Suche weiterer Algorithmen  % selbstverst"andlich ...
\item Analyse des Speicherplatzverbrauchs
%\item Implementierung von Flie"skommazahlen mit einer beliebigen Anzahl von
%      Stellen
% Implementierung:  Zahlen beliebiger L"ange 
% $$$$$            (--> Satz "uber maximale Gr"o"se der Eigenwerte)
\end{itemize}

Es bleibt also einiges zu tun... . F"ur dieses Mal soll das jedoch alles 
sein.
\vspace{5ex}

Wenn die Gedanken wieder leichter flie"sen...
\begin{verse}
    Himmlische Stille rauscht durch die Nacht \\
    Samtenes Schweigen str"omt durch die Luft \\
    glitzernd breitet sich das Tal in der Ferne \\[2ex]
    
    kein Laut \\
    ohne Hast l"a"st die Ruhe verbreiten ihr Gl"uck
\end{verse}


%
% Datei: index.tex (Literaturliste, Stichwortverzeichnis)
% 
\markboth{}{}
\addcontentsline{toc}{chapter}{Literatur}
\bibliography{diplom}
\addcontentsline{toc}{chapter}{Stichwortverzeichnis}
\printindex


%
% Datei: anhang.tex (Listings)
%
\begin{appendix}
\sloppy
\thispagestyle{empty}
\vspace*{6cm}
\begin{center}
   \LARGE
   Algorithmen zur parallelen Determinantenberechnung \\[1cm]
   \Large
   Holger Burbach \\[1cm]
   Oktober 1992 \\[1.5cm]
   \LARGE Anhang (Quelltexte)
\end{center}

\MyChapter{Implementierung der parallelen Determinantenberechnung}
\label{ChapImplDet}

In diesem Kapitel sind alle Module gesammelt, denen das vorrangige
Interesse der Implementierung gilt.

\input{listings/main.m}
\input{listings/det.d}
\input{listings/det.i}
\input{listings/pram.d}
\input{listings/pram.i}
\input{listings/algtest.m}

% **************************************************************************

\MyChapter{Unterst"utzungsmodule}
\label{ChapSupport}

In diesem Kapitel sind alle Module alphabetisch sortiert gesammelt,
denen nicht das Hauptinteresse der Implementierung gilt, die jedoch zur
Implementierung der Module in Kapitel \ref{ChapImplDet} erforderlich
sind.

\input{listings/cali.d}
\input{listings/cali.i}
\input{listings/data.d}
\input{listings/data.i}
\input{listings/frag.d}
\input{listings/frag.i}
\input{listings/func.d}
\input{listings/func.i}
\input{listings/hash.d}
\input{listings/hash.i}
\input{listings/inli.d}
\input{listings/inli.i}
\input{listings/list.d}
\input{listings/list.i}
\input{listings/mali.d}
\input{listings/mali.i}
\input{listings/mat.d}
\input{listings/mat.i}
\input{listings/reli.d}
\input{listings/reli.i}
\input{listings/rema.d}
\input{listings/rema.i}
\input{listings/rnd.d}
\input{listings/rnd.i}
\input{listings/simptype.d}
\input{listings/simptype.i}
\input{listings/str.d}
\input{listings/str.i}
\input{listings/sys.d}
\input{listings/sys.i}
\input{listings/sysmath.d}
\input{listings/sysmath.i}
\input{listings/type.d}
\input{listings/type.i}

%***************************************************************************

\MyChapter{Testprogramme}
\label{ChapTest}

In diesem Kapitel sind alle Programmodule gesammelt, die zum Test von
Teilen der Gesamtimplementierung erforderlich sind.

\input{listings/listtest.m}
\input{listings/pramtest.m}
\input{listings/rndtest.m}
\input{listings/strtest.m}
\input{listings/typetest.m}

\end{appendix}


%%
\end{document}


\input{listings/det.d}
\input{listings/det.i}
\input{listings/pram.d}
\input{listings/pram.i}
\input{listings/algtest.m}

% **************************************************************************

\MyChapter{Unterst"utzungsmodule}
\label{ChapSupport}

In diesem Kapitel sind alle Module alphabetisch sortiert gesammelt,
denen nicht das Hauptinteresse der Implementierung gilt, die jedoch zur
Implementierung der Module in Kapitel \ref{ChapImplDet} erforderlich
sind.

\input{listings/cali.d}
\input{listings/cali.i}
\input{listings/data.d}
\input{listings/data.i}
\input{listings/frag.d}
\input{listings/frag.i}
\input{listings/func.d}
\input{listings/func.i}
\input{listings/hash.d}
\input{listings/hash.i}
\input{listings/inli.d}
\input{listings/inli.i}
\input{listings/list.d}
\input{listings/list.i}
\input{listings/mali.d}
\input{listings/mali.i}
\input{listings/mat.d}
\input{listings/mat.i}
\input{listings/reli.d}
\input{listings/reli.i}
\input{listings/rema.d}
\input{listings/rema.i}
\input{listings/rnd.d}
\input{listings/rnd.i}
\input{listings/simptype.d}
\input{listings/simptype.i}
\input{listings/str.d}
\input{listings/str.i}
\input{listings/sys.d}
\input{listings/sys.i}
\input{listings/sysmath.d}
\input{listings/sysmath.i}
\input{listings/type.d}
\input{listings/type.i}

%***************************************************************************

\MyChapter{Testprogramme}
\label{ChapTest}

In diesem Kapitel sind alle Programmodule gesammelt, die zum Test von
Teilen der Gesamtimplementierung erforderlich sind.

\input{listings/listtest.m}
\input{listings/pramtest.m}
\input{listings/rndtest.m}
\input{listings/strtest.m}
\input{listings/typetest.m}

\end{appendix}


%%
\end{document}

