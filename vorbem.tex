%
% Datei: vorbem.tex
%
\MyChapter{Vorbemerkungen}
\label{ChapIntro}

% **************************************************************************

\MySection{Einf"uhrung}

Die Berechnung der Determinante einer quadratischen Matrix ist ein
Problem,
dessen effiziente L"osung in vielen Bereichen von Interesse ist, in der
Informatik z. B. in der Computergrafik und der Kodierungstheorie.

Ein Problem in der analytischen Geometrie ist es, die 
Lage von geometrischen Objekten zueinander festzustellen und 
Schnittpunkte oder -ebenen zu berechnen. Ein Teilproblem dabei ist die
Pr"ufung der linearen Unabh"angigkeit von Vektoren. Es
l"a"st sich auf die Berechnung einer Determinante zur"uckf"uhren.

Ein weiterer Bereich, in dem die Determiantenberechnung angewendet wird,
ist
die theoretische Physik. Dort wird in vielen Theorien auf Matrizen zu
Beschreibung der verschiedenen Sachverhalte zur"uckgegriffen. In der
Einstein'sche Relativit"atstheorie z. B. wird die {\em Tersorrechnung},
die die Eigenschaften sich von Koordinatensystem zu Koordinatensystem
"andernder Ma"szahlen untersucht, ausgibig verwendet. Die Determinante ist
eine solche Ma"szahl. Beim Studium von Literatur, die diese Thematiken
behandelt (z. B. \cite{BS86} ab S. 70), st"o"st man immer wieder auf
Matrizen und ihre Determinanten.

Seitdem
sich die Forschung im Bereich der Informatik zunehmend mit 
Parallelrechnern
be\-sch"af\-tigt, werden f"ur alle bekannten Probleme Algorithmen gesucht, 
die die Tatsache, da"s auf einem Parallelrechner mehrere Prozessoren 
gleichzeitig an der L"osung desselben Problems arbeiten, besonders 
effizient ausnutzen.

Betrachtet man eine Matrix aus der Sicht der Informatik als Datenstruktur,
so dr"angt sich die Benutzung dieser Datenstruktur in Parallelrechnern 
geradezu auf, denn intuitiv, ohne zun"achst alle Probleme ausgearbeitet 
zu haben, kann man auf die Idee kommen, die Matrizenelemente jeweils
einzelnen Prozessoren oder Gruppen von Prozessoren zuzuordnen, die das 
zugrundeliegende Problem f"ur dieses Matrizenelement bearbeiten. 
Selbstverst"andlich ist die praktische Verwendung dieser Idee nicht in 
jedem Fall ganz so einfach.

So war die effiziente Parallelisierung der Determinantenberechnung lange
Zeit ein ungel"ostes Problem, bis 1976, als Laszlo Csanky einen in jenen 
Tagen "uberraschenden Algorithmus ver"offentlichte \cite{Csan76}. Es 
folgten eine Reihe weiterer Algorithmen verschiedener Autoren mit 
vergleichbaren Leistungsmerkmalen.

In all diesen Ver"offentlichungen wird vorrangig die Gr"o"senordnung 
der Laufzeiten und Anzahlen der Prozessoren betrachtet. Es werden in
einzelnen Ver"offentlichungen auch bereits einige Vergleiche mit den 
anderen Algorithmen durchgef"uhrt. So ist es w"unschenswert einen 
"Uberblick "uber die existierenden Algorithmen zu bekommen und sie 
insgesamt miteinander zu vergleichen.

Die vorliegende Diplomarbeit behandelt vier Algorithmen zur 
parallelen De\-ter\-mi\-nan\-ten\-be\-rech\-nung\footnote{ \cite{Csan76}, 
\cite{BGH82}, \cite{Berk84} und \cite{Pan85}}. Dabei wird auf die 
Verwendung von Gr"o"senordnungen ( O-Notation ) in Aufwandsanalysen 
weitgehend verzichtet.
Um den Einsteig in das Thema zu 
erleichtern, wird zus"atzlich noch der Entwicklungssatz von
Laplace zur Berechnung der Determinante erw"ahnt.

Die Darstellung der vier Algorithmen in den zugeh"origen Kapiteln
\ref{ChapCsanky} bis \ref{ChapPan} umfa"st neben den Grundlagen und der 
Algorithmen selbst, jeweils eine Analyse der Rechenzeit und des Grades der
Parallelisierung\footnote{Diese Begriffe sind in Kapitel \ref{SecBez}
definiert.}. Da in der Praxis die Gr"o"se des ben"otigen 
Speicherplatzes kein vorrangiges Problem mehr darstellt, wird dieser 
Wert nicht analysiert. Matrizen mit Elementen aus $\Complex$ werden nicht
betrachtet. In diesen vier Kapitel wird versucht, auf Unterschiede
und Gemeinsamkeiten der Algorithmen einzugehen.

Im Anschlu"s an die Darstellung der Algorithmen wird in Kapitel 
\ref{ChapImplemen} ihre Implementierung beschrieben. Die Quelltexte 
sind im Anhang zu finden.
Schlie"slich erfolgt in Kapitel \ref{ChapEndbem}
ein zusammenfassender Vergleich der Algorithmen.

Der Text soll es erm"oglichen, die Algorithmen ohne weitere Literatur 
anhand
von Grundkenntnissen aus der Mathematik und Informatik zu verstehen. 
Aus diesem Grund und um einheitliche Bezeichnungen zu vereinbaren sind 
Grundlagen, insbesondere aus der Linearen Algebra, an den ben"otigten 
Stellen aufgef"uhrt. F"ur den Fall, da"s die im Text enthaltenen
Informationen nicht ausreichen, sind die benutzten Quellen an den
jeweiligen Stellen angegeben.

Alle Betrachtungen abstrahieren von technischen Problemen bei der 
Konstruktion von Parallelrechnern. Dazu wird das Rechnermodell der
PRAM benutzt. Die Beschreibung dieses Modells erfolgt in 
Kapitel \ref{SecModell}. 

Vor anderen Teilen dieser Diplomarbeit sollten zun"achst die Kapitel 
\ref{SecModell} und \ref{SecBez} gelesen werden. Alle 
weiteren Teile von Kapitel \ref{ChapIntro} sowie Kapitel \ref{ChapBase}
sind als Sammlung von Grundlagen zu verstehen, auf die bei Bedarf 
zur"uckgegriffen werden kann\footnote{Im Text kommen h"aufig 
Punkte und Kommata als Satzzeichen direkt im Anschlu"s an
abgesetzte Gleichungen vor. An einigen Stellen, besonders hinter
Vektoren und Matrizen, fehlen diese Satzzeichen aus technischen 
Gr"unden.}.

% **************************************************************************

\MySection{Das Berechnungsmodell}
\label{SecModell}
\index{Berechnungsmodell} \index{PRAM} \index{CRCW}
\index{Modellrechner}
In diesem Kapitel wird der f"ur Komplexit"atsbetrachtungen verwendete
Modellrechner beschrieben. Es ist die 
{\em Arbitrary Concurrent Read Concurrent Write Parallel Random Access 
  Machine ( arbitrary CRCW PRAM) }.
Sie besteht aus
gleichen Prozessoren, die alle auf denselben Arbeitsspeicher zugreifen.
Innerhalb einer Zeiteinheit k"onnen diese
Prozessoren, und zwar alle gleichzeitig,
zwei Operanden aus dem Speicher lesen, eine der in Tabelle \ref{Csan76Tab2}
aufgef"uhrten Operationen ausf"uhren und
das Ergebnis wieder im Speicher ablegen. Falls beim
Schreiben mehrere Prozessoren auf eine Speicherzelle zugreifen, mu"s der
Algorithmus unabh"angig davon korrekt sein, welcher Prozessor seinen
Schreibzugriff tats"achlich ausf"uhrt.
\begin{table}[htb]
    \begin{center}
    \begin{tabular}{|p{4cm}|c|}
        \hline
        Operation & Symbol \\
        \hline
        \hline
        Addition & $+$ \\
        \hline
        Subtraktion & $-$ \\
        \hline
        Multiplikation & $*$ \\
        \hline
        Division (Ergebnis in $\Rationals$) & $/$ \\
        \hline
        Division (Ergebnis in $\Integers$) & div \\
        \hline
        $ x - (x \; \mbox{div} \; y) * y $ & $x$ mod $y$ \\
        \hline
    \end{tabular}
    \end{center}
    \caption{Operationen des Modellrechners}
    \label{Csan76Tab2}
\end{table}
Da die
Komplexit"at der vier haupts"achlich interessierenden 
Algorithmen nicht nur auf ihre Gr"o"senordnung hin
untersucht wird, sondern die Ausdr"ucke zur Beschreibung der Komplexit"at
genau angegeben werden sollen, ist es erforderlich, von Details
der Implementierung, die die Konstanten
beeinflussen, zu abstrahieren, so da"s die Aussagen allgemeing"ultig sind.
Aus diesem Grund wird
\begin{itemize}
    \item f"ur die Verarbeitung von Schleifenbedingungen,
    \item f"ur die Verarbeitung von Verzweigungsbedingungen,
    \item f"ur die Ein- und Ausgabe von Daten,
    \item f"ur die Initialisierung von Speicherbereichen,
    \item und f"ur komplexe Adressierungsarten
          (z. B. indirekte Adressierung) beim Zugriff auf Speicherbereiche
\end{itemize}
kein zus"atzlicher Aufwand in Rechnung gestellt. Es werden also nur die
arithmetischen Operationen gez"ahlt.

Zu beachten ist, da"s bei einer PRAM jeder Aufwand zur Verteilung von
Aufgaben auf verschiedene Prozessoren vernachl"assigt wird. Diese 
Eigenschaft bietet die M"oglichkeit zur Kritik, da so jedes
Problem deutlich vereinfacht wird, jedoch in der Praxis die Organisation der 
Aufgabenverteilung nicht unerheblichen Aufwand erfordert.
Eine genaue Analyse der 
Auswirkungen dieser Vernachl"assigung ist umfangreich und nicht 
Thema des vorliegenden Textes.

% **************************************************************************

\MySection{Bezeichnungen}
\label{SecBez}

In diesem Kapitel werden die verwendeten Begriffe und Symbole 
definiert.

Um auf die in der Arbeit haupts"achlich behandelten Algorithmen 
einfach Bezug
nehmen zu k"onnen, werden mit Hilfe der Namen ihrer Autoren die
folgenden Abk"urzungen vereinbart:
\begin{itemize}
\item
      C-Alg. steht f"ur den Algorithmus von Csanky 
      (Unterkapitel \ref{SecAlgFrame} ab S. \pageref{SecAlgFrame}).
\item
      BGH-Alg. steht f"ur den Algorithmus von Borodin, von zur Gathen 
      und Hopcroft (Unterkapitel \ref{SecAlgBGH} ab 
      S. \pageref{SecAlgBGH}).
\item
      B-Alg. steht f"ur den Algorithmus von Berkowitz 
      (Unterkapitel \ref{SecAlgBerk} ab S. \pageref{SecAlgBerk}).
\item
      P-Alg. steht f"ur den Algorithmus von Pan
      (Unterkapitel \ref{SecAlgPan} ab S. \pageref{SecAlgPan}).
\end{itemize}

\index{Bezeichnungen!nat{\Myu}rliche Zahlen}
Die Menge der positiven ganzen Zahl {\em ohne Null} wird mit \[ \Nat \]
bezeichnet. Die Menge der
positiven ganzen Zahl einschlie"slich der Null wird mit \[ \Nat_0 \]
bezeichnet. Falls nicht im Einzelfall anders festgelegt erfolgen alle
Darstellungen von Zahlen zur Basis $10$.

\index{Bezeichnungen!Schritt}
Der Vorgang, in dem beliebig viele Prozessoren gleichzeitig je
zwei Operanden aus dem Arbeitsspeicher lesen, aus diesen Operanden
ein Ergebnis berechnen und dieses Ergebnis wieder im
Arbeitsspeicher ablegen, wird als ein {\em Schritt} bezeichnet.

\index{Zeitkomplexit{\Mya}t}
\index{Parallelisierungsgrad}
Die {\em parallele Zeitkomplexit"at eines Algorithmus} bezeichnet die
Anzahl der Schritte, die dieser ben"otigt, um die L"osung\footnote{Die
von uns betrachteten Probleme besitzen nur eine L"osung.}
f"ur das
zugrunde liegende Problem zu berechnen.
Die maximale Anzahl der Prozessoren, die dabei gleichzeitig
besch"aftigt werden, wird mit {\em Parallelisierungsgrad des
Algorithmus} bezeichnet.

Falls nicht im Einzelfall anders festgelegt, gilt folgende 
Regelung: Gro"sbuchstaben bezeichnen
Matrizen und Kleinbuchstaben Zahlen oder Vektoren, $A$ bezeichnet eine
$n \times n$-Matrix, indizierte
Kleinbuchstaben beziehen sich auf die Elemente der mit dem zugeh"origen 
Gro"sbuchstaben bezeichneten
Matrix.

Die in Tabelle \ref{Csan76Tab1} aufgelisteten Schreibweisen werden
benutzt.
\index{Bezeichnungen!Indizierung}
\index{Einheitsmatrix} \index{Nullmatrix}
\index{Permutation}
\begin{table}[htb]
    \begin{center}
    \begin{tabular}{|p{10cm}|c|}
        \hline
            Begriff & Schreibweise \\
        \hline\hline
            Element in Zeile $i$, Spalte $j$ von $A$ & $ a_{i,j} $ \\
        \hline
            $i$-tes Element des Vektors $v$ & $v_i$ \\
        \hline
            Matrix, die aus A durch Streichen der Zeilen $v$ und der
            Spalten $w$ entsteht (dabei seien $v$ und $w$
            echte Teilmengen der Menge der Zahlen von $1$ bis $n$; diese
            Mengen werden hier als durch Kommata getrennt Zahlenfolge
            geschrieben) & $ {A}_{(v|w)} $ \\
        \hline
            Einheitsmatrix (Elemente der Haupt\-di\-ago\-na\-len gleich $1$;
            alle anderen Elemente gleich $0$)
            mit $n$ Zeilen und Spalten & $E_n$ \\
        \hline
            Einheitsmatrix (Anzahl der
            Zeilen und Spalten aus dem Zusammenhang klar) & $E$ \\
        \hline
            Nullmatrix (alle Elemente sind gleich 0) mit $m$ Zeilen und
            $n$ Spalten & $0_{m,n}$ \\
        \hline
            Nullvektor (alle Elemente sind gleich 0) der L"ange $m$ &
            $0_m$ \\
        \hline
            Logarithmus von $x$ zur Basis $2$      & $\log(x)$ \\
        \hline
            Menge aller $n$-stelligen Permutationen & $\permut_n$ \\
        \hline
            \begin{minipage}{10em}
                \begin{math} \displaystyle
                     \lim_{n\rightarrow \infty}
                     \left( 1 + \frac{1}{n} \right)^n
                \end{math}
            \end{minipage} \LMatStrut
            $(\: = 2.718281\ldots)$ &  \MathE  \\
        \hline
            Logarithmus von $x$ zur Basis $\MathE$ & $\ln(x)$ \\
        \hline
            Anzahl der Elemente der Menge $M$ & $|M|$ \\
        \hline
    \end{tabular}
    \end{center}
    \caption{Bezeichnungen}
    \label{Csan76Tab1}
\end{table}

% **************************************************************************

\MySection{Das Pr"afixproblem}
Von einer effizienten L"osung des Pr"afixproblems wird an verschiedenen
Stellen Gebrauch gemacht. Es ist also von "ubergreifendem Interesse und
wird deshalb hier behandelt (\cite{LF80},
\cite{Wege89} S. 83 ff.). Es l"a"st sich folgenderma"sen formulieren:
\begin{quote}
\index{Pr{\Mya}fixproblem}
\label{PagePraefixproblem}
    Gegeben sei die Halbgruppe \[ (M,\circ) \MyPunkt \] 
    D. h. die Verkn"upfung 
    $\circ$ ist assoziativ auf $M$. Weiterhin seien 
    \[ x_1,x_2,x_3,\ldots,x_{n} \]
    Elemente aus $M$. Es wird definiert
    \[ p_i := x_1 \circ x_2 \circ x_3 \circ \ldots \circ x_i \MyPunkt \]
    Das Pr"afixproblem besteht darin, alle Elemente der Menge
    \[ \{ p_i | 1 \leq i \leq n \} \] zu berechnen.
\end{quote}
Es sind u. a. zwei M"oglichkeiten\footnote{die sich zu einer dritten 
zusammenfassen lassen (Satz \ref{SatzAlgPraefix})} denkbar, dies mit 
parallelen Algorithmen zu erreichen.
\begin{itemize}
    \item Die erste M"oglichkeit:
        \begin{enumerate}
            \item L"ose das Pr"afixproblem parallel f"ur
                  \[ x_1,\ldots,x_{\lceil n/2 \rceil} \]
                  und
                  \[ x_{\lceil n/2+1 \rceil},\ldots,x_n \MyKomma \]
                  so da"s nach diesem Schritt
                  \[ p_1,\ldots,p_{\lceil n/2 \rceil} \]
                  bereits berechnet sind.
            \item
                  Berechne aus \[ p_{\lceil n/2 \rceil} \] und
                  der L"osung des Problems f"ur
                  \[ x_{\lceil n/2+1 \rceil},\ldots,x_n \]
                  parallel in einem weiteren Schritt
                  \[ p_{\lceil n/2+1 \rceil},\ldots,p_n \]
        \end{enumerate}
    \item
        Die zweite M"oglichkeit, die hier kurz dargestellt werden soll,
        sieht folgenderma"sen aus (o. B. d. A. sei $n$ eine Zweierpotenz):
        \begin{enumerate}
            \item
                Berechne parallel in einem Schritt
                \[ x_1 \circ x_2, x_3 \circ x_4, \ldots,
                   x_{n-1} \circ x_n \]
            \item
                L"ose das Pr"afixproblem f"ur diese
                $n/2$ Werte. Damit werden alle $p_i$ mit geradem $i$
                berechnet.
            \item
                Die noch fehlenden $p_i$ f"ur ungerade $i$ k"onnen nun
                parallel in einem weiteren Schritt aus der L"osung f"ur die
                $n/2$ Werte und den $x_i$ mit ungeradem $i$ berechnet
                werden.
        \end{enumerate}
\end{itemize}
Diese beiden M"oglichkeiten k"onnen zu einem Algorithmus zusammengefa"st
werden:
\begin{satz}
\label{SatzAlgPraefix}
\index{Algorithmus!Pr{\Mya}fixproblem}
    Gegeben sei die Halbgruppe \[ (M,\circ) \MyPunkt \] 
    Das Pr"afixproblem
    f"ur $n$ Elemente \[ x_1,x_2,\ldots,x_n \] von $M$ l"a"st sich
    von \[ \lf \frac{3}{4}n \rf \] Prozessoren in
    \[ \lceil \log(n) \rceil \] Schritten l"osen.
\end{satz}
\begin{beweis}
    O. B. d. A. sei $n$ eine Zweierpotenz. In dem Fall, da"s $n$ keine
    Zweierpotenz ist, wird $n$ durch die n"achst h"ohere Zweierpotenz $n'$
    ersetzt und alle Verkn"upfungen mit Elementen $x_i$ von $M$ f"ur
    \[ i>n \] werden nicht durchgef"uhrt. 

    Benutze folgenden Algorithmus:
    \begin{enumerate}
    \item
          Wenn \[ n=1 \] dann ist $x_1$ das Ergebnis.
    \item
          Wenn \[ n=2 \] dann ist \[ x_1, x_1 \circ x_2 \]
          das Ergebnis.
    \item
          Schritte \ref{StepPraefix3a} und \ref{StepPraefix3b} parallel:
          \begin{enumerate}
          \item \label{StepPraefix3a}
                \begin{enumerate}
                \item \label{StepPraefix3a1}
                      Berechne parallel in einem Schritt
                      \[ x_1 \circ x_2,\ldots,
                         x_{n/2-1} \circ x_{n/2}
                      \]
                \item \label{StepPraefix3a2}
                      Benutze den Algorithmus rekursiv zur L"osung des
                      Problems f"ur die in Schritt \ref{StepPraefix3a1}
                      erhaltenen $n/4$ Werte. Auf diese Weise sind die
                      $p_i$ f"ur
                      \[ 1 \leq i \leq n/2 \] mit geraden $i$, u. a.
                      auch $p_{n/2}$, bereits berechnet.
                \end{enumerate}
          \item \label{StepPraefix3b}
                Benutze den Algorithmus rekursiv zur L"osung des Problems
                f"ur \[ x_{n/2+1},\ldots,x_n \]
          \end{enumerate}
    \item Schritte \ref{StepPraefix4a} und \ref{StepPraefix4b} parallel:
          \begin{enumerate}
          \item \label{StepPraefix4a}
               F"ur $i$ gelte \[ 1 \leq i \leq n/2 \MyPunkt \]
               Wenn $n/2 > 2$, dann
               berechne parallel in einem Schritt mit Hilfe der $p_i$
               aus \ref{StepPraefix3a2} und der $x_i$ mit ungeradem $i$
               die fehlenden $p_i$ mit ungeradem $i$.
          \item \label{StepPraefix4b}
                Berechne aus $p_{n/2}$ und den Ergebnissen von 
                \ref{StepPraefix3b} die $p_i$ mit
                \[ n/2+1 \leq i \leq n \MyPunkt \]
          \end{enumerate}
    \end{enumerate}
    Zur Analyse des Algorithmus bezeichnet $s(n)$ die Anzahl der Schritte,
    die er ben"otigt, um das Pr"afixproblem f"ur $n$ Eingabewerte zu
    l"osen, und $p(n)$ die Anzahl der Prozessoren, die dabei besch"aftigt
    werden k"onnen.
    \begin{itemize}
    \item Hier wird zun"achst die Anzahl der Schritte betrachtet.
          Es gilt \[ s(1) = 0,\,s(2) = 1,\, s(4) = 2 \MyPunkt \]
          Bei der Betrachtung des Algorithmus erkennt man, da"s folgende
          Rekursionsgleichung G"ultigkeit besitzt:
          \Beq{Berk84Equ6}
              \forall n>4: \: s(n) = \max(s(n/4)+1,\,s(n/2)) + 1 \MyPunkt
          \Eeq
          Wenn man diese Formel auf $s(n/2)$ anwendet und das Ergebnis in
          die obige Formel einsetzt, erh"alt man
          \[ \forall n>4: \: s(n) = 
                 \max(\: s(n/4)+1,\, \max(s(n/8)+1,\, s(n/4))+1 \:) \:+ 1 
          \]
          Aufgrund der Assoziativit"at der $\max$-Funktion ist dies
          gleichbedeutend mit
          \begin{MyEqnArray}
             \MT \forall n>4: \: s(n) \MT = \MT
                 \max(s(n/4)+1,\, s(n/8)+2,\, s(n/4)+ 1) + 1 \MNl
             \Rightarrow 
             \MT \forall n>4: \: s(n) \MT = \MT s(n/4)+2 
          \end{MyEqnArray}
          Es gilt also f"ur jedes $i$:
          \[ \forall n>4: \: s(n) = s(n/2^{2i}) + 2i \]
          Mit \[ i = \frac{\log(n)}{2} \] erh"alt man als Endergebnis 
          \[ s(n) = \log(n) \MyPunkt \]
    \item F"ur die Anzahl der besch"aftigten Prozessoren $p(n)$ gilt:
          \[ p(1) = 0, p(2) = 1, p(4)= 2 \]
          Ferner gilt offensichtlich folgende Rekursionsgleichung:
          \Beq{Berk84Equ8}
              \forall n>4:\: p(n)= 
              \max(n/2+n/4,\, p(n/4)+p(n/2),\, n/4+p(n/2))
          \Eeq
          Beim Ausrechnen der Werte von $p(8)$, $p(16)$ und $p(32)$
          mit Hilfe dieser Rekursionsgleichung gelangt man zu der
          Vermutung, da"s gilt:
          \Beq{Berk84Equ7}
              \forall n>4: \: p(n) = \frac{3}{4} n 
          \Eeq
          Dies wird durch Induktion bewiesen. Zu beachten
          ist, da"s nach Voraussetzung nur die Potenzen von $2$ als Werte
          f"ur $n$ in Frage kommen. 
          
          Sei also nun \[ n>4 \] und es gelte
          \begin{eqnarray*}
              p(n) & = & \frac{3}{4}n \\
              p(n/2) & = & \frac{3}{8}n \MyPunkt
          \end{eqnarray*}
          Es ist zu zeigen, da"s dann auch
          \[ p(2n)= \frac{3}{2}n \]
          richtig ist.
          Nach der Rekursionsgleichung \equref{Berk84Equ8} gilt
          \[ p(2n) = \max(3/2*n, p(n/2)+p(n), n/2 + p(n)) \]
          Die Anwendung der Induktionsvoraussetzung f"uhrt zu
          \[ p(2n) = \max(3/2*n, 3/8*n + 3/4*n, n/2 + 3/4*n) \] und somit zu
          \[ p(2n) = \frac{3}{2}n \] was zu zeigen war. F"ur die Anzahl der 
          ben"otigen Prozessoren gilt also
          \[ \forall n > 4: \: p(n) = \frac{3}{4}n \]
          Da die L"osung des Problems f"ur \[ n \leq 4 \] einfach ist, wird
          die Quantifizierung nicht weiter beachtet.
    \end{itemize}
    Damit die Aussagen nicht nur f"ur Zweierpotenzen, werden die Werte
    f"ur $s(n)$ und $p(n)$ mit Gau"sklammern versehen. F"ur $p(n)$ ist
    dies ohne weitere Begr"undung problematisch. Betrachtet man den 
    Algorithmus jedoch genauer, stellt man fest, da"s beim ersten 
    ausgef"uhrten Schritt die meisten Prozessoren besch"aftigt werden. 
    Die Anzahl dieser Prozessoren gibt der Term in Gau"sklammern an.
\end{beweis}

% **************************************************************************

\MySection{L"osungen grundlegender Probleme}
In diesem Kapitel werden Algorithmen zur L"osung einiger
grundlegender Probleme angegeben und auf ihre Komplexit"at hin untersucht.

\begin{satz}[Bin"arbaummethode]  % $$$ wird benutzt (nicht loeschen)
\label{SatzAlgBinaerbaum}
    Wird das Pr"afixproblem (siehe Beschreibung Seite
    \pageref{PagePraefixproblem}) dahingehend vereinfacht, da"s nur
    $p_n$ zu berechnen ist, so l"a"st sich dieses vereinfachte Problem in
    \[ \lc \log(n) \rc \] Schritten von \[ \lf \frac{n}{2} \rf \] 
    Prozessoren l"osen.
\end{satz}
\begin{beweis}
    Verkn"upfe die $x_i$ nach dem Schema in Abbildung \ref{PicBinBaum}.
    \begin{figure}[htb]
    \begin{center}
        %
% You need 'epic.sty'
%
\setlength{\unitlength}{1mm}
\makeatletter
\def\Thicklines{\let\@linefnt\tenlnw \let\@circlefnt\tencircw
\@wholewidth4\fontdimen8\tenln \@halfwidth .5\@wholewidth}
\makeatother
\begin{picture}(110.00,54.00)
\put(1.00,49.00){$x_1$}
\put(1.00,45.00){$x_2$}
\put(1.00,41.00){$x_3$}
\put(1.00,37.00){$x_4$}
\put(1.00,33.00){$x_5$}
\put(1.00,29.00){$x_6$}
\put(1.00,25.00){$x_7$}
\put(1.00,21.00){$x_8$}
\put(2.25,16.50){$\vdots$}
\put(1.00,13.00){$x_{n-3}$}
\put(1.00,9.00){$x_{n-2}$}
\put(1.00,5.00){$x_{n-1}$}
\put(1.00,1.00){$x_n$}
\drawline(7.25,50.00)(15.25,48.00)
\drawline(7.25,46.00)(15.25,48.00)
\drawline(7.25,42.00)(15.25,40.00)
\drawline(7.25,38.00)(15.25,40.00)
\drawline(7.25,34.00)(15.25,32.00)
\drawline(7.25,30.00)(15.25,32.00)
\drawline(7.25,26.00)(15.25,24.00)
\drawline(7.25,22.00)(15.25,24.00)
\drawline(7.00,15.00)(15.00,13.00)
\drawline(7.00,11.00)(15.00,13.00)
\drawline(7.00,7.00)(15.00,5.00)
\drawline(7.00,3.00)(15.00,5.00)
\put(17.25,47.25){$\circ$}
\put(17.25,39.25){$\circ$}
\put(17.25,31.25){$\circ$}
\put(17.25,23.25){$\circ$}
\put(17.25,11.25){$\circ$}
\put(17.25,3.25){$\circ$}
\drawline(21.00,13.00)(33.00,9.00)
\drawline(21.00,5.00)(33.00,9.00)
\drawline(21.00,49.00)(33.00,45.00)
\drawline(21.00,41.00)(33.00,45.00)
\drawline(21.00,33.00)(33.00,29.00)
\drawline(21.00,25.00)(33.00,29.00)
\put(34.00,8.00){$\cdots$}
\put(35.25,27.50){$\circ$}
\put(35.25,43.50){$\circ$}
\drawline(41.00,45.00)(55.00,37.00)
\drawline(41.00,29.00)(55.00,37.00)
\put(99.00,21.00){$p_n$}
\put(79.25,7.25){$\circ$}
\drawline(77.00,9.00)(71.00,15.00)
\drawline(77.00,9.00)(71.00,3.00)
\put(79.25,35.25){$\circ$}
\drawline(77.00,37.00)(71.00,43.00)
\drawline(77.00,37.00)(71.00,31.00)
\drawline(97.00,23.00)(83.00,37.00)
\drawline(97.00,23.00)(83.00,9.00)
\put(65.00,41.50){$\cdots$}
\put(65.00,29.50){$\cdots$}
\put(65.00,13.50){$\cdots$}
\put(65.00,1.50){$\cdots$}
\put(56.00,36.00){$\cdots$}
\end{picture}
%

        \caption{Bin"arbaummethode}
        \label{PicBinBaum}
    \end{center}
    \end{figure}
    Falls $n$ keine Zweierpotenz ist, werden die Verkn"upfungen mit den 
    $x_j$, f"ur die gilt \[ n < j \leq 2^{\lc \log(n) \rc} \MyKomma \]
    nicht durchgef"uhrt. Die L"osung des Problems erfordert offensichtlich
    den angegebenen Aufwand.
    \mbox{ \hspace{4em} \hfill }
\end{beweis}

\begin{korollar}[Parallele Grundrechenarten]
\label{SatzAlgRechnen}             % $$$ wird benutzt (nicht loeschen)
\index{Algorithmus!parallele Grundrechenarten}
    Seien $n$ Zahlen durch die gleiche Rechenoperation miteinander zu
    verkn"upfen. Diese Rechenoperation sei eine der Grundrechenarten
    Addition oder Multiplikation. Die Verkn"upfung kann in
    \[ \lc \log(n) \rc \] Schritten von
    \[ \lf \frac{n}{2} \rf \] Prozessoren durchgef"uhrt
    werden.
\end{korollar}
\begin{beweis}
    Aufgrund der Assoziativit"at der Rechenoperationen folgt
    dies direkt aus Satz \ref{SatzAlgBinaerbaum}. 
    \mbox{ \hspace{4em} \hfill }
\end{beweis}

\begin{satz}[Parallele Matrizenmultiplikation]
\label{SatzAlgMatMult}            % $$$ wird benutzt (nicht loeschen)
\index{Algorithmus!parallele Matrizenmultiplikation}
    Sei $A$ eine $m \times p$-Matrix und $B$ eine $p \times n$-Matrix.
    Sie lassen sich in
    \[ \lceil \log(p) \rceil + 1 \] Schritten von
    \[ m * n * p \] Prozessoren miteinander
    multiplizieren.
\end{satz}
\begin{beweis}
    Sei $C$ die $m \times n$-Ergebnismatrix. 
    Sie wird mit Hilfe der Gleichung
    \[ c_{i,j} = \sum_{k=1}^p a_{i,k} b_{k,j} \]
    berechnet. Dazu werden zuerst parallel in einem Schritt
    \[ d_{i,k,j} := a_{i,k} b_{k,j} \] mit
    \begin{eqnarray*}
         1 \leq & i & \leq m \\
         1 \leq & j & \leq n \\
         1 \leq & k & \leq p 
    \end{eqnarray*} von
    \[ m * n * p \] Prozessoren berechnet.
    Die Ergebnismatrix erh"alt man dann nach der Gleichung
    \[ c_{i,j} = \sum_{k=1}^p d_{i,k,j} \]
    Die Berechnung der Matrix $C$ aus den $d_{i,k,j}$ kann nach
    \ref{SatzAlgRechnen} f"ur ein Matrizenelement in
    \[ \lceil \log(p) \rceil \] Schritten von
    \[ \lf \frac{p}{2} \rf \] Prozessoren durchgef"uhrt
    werden, also f"ur
    die gesamte Matrix in genauso vielen Schritten von
    \[ m * n * \lf \frac{p}{2} \rf \] Prozessoren.
    Die Werte
    f"ur Schritte und Prozessoren zusammengenommen ergeben die Behauptung.
\end{beweis}

Zwei $n \times n$-Matrizen lassen sich also in 
\[ \lceil \log(n) \rceil + 1 \] Schritten von
\[ n^3 \] Prozessoren miteinander multiplizieren.

Die \label{PageAlg2MatMult}
Matrizenmultiplikation l"a"st sich asymptotisch, d. h. f"ur $n \to \infty$
auch mit
\[ O(n^{2+\gamma}), \: \gamma = 0.376 \]
Prozessoren durchf"uhren \cite{CW90}. Gegen"uber \ref{SatzAlgMatMult} 
ergibt sich wegen des erheblichen konstanten Aufwandes nur f"ur gro"se $n$ 
eine Verbesserung. Es wird jeweils gesondert darauf hingewiesen, falls
auf diese M"oglichkeit zur"uckgegriffen wird.

% **************************************************************************
% **************************************************************************
% **************************************************************************

\MyChapter{Grundlagen aus der Linearen Algebra}
\label{ChapBase}

In diesem Kapitel werden die f"ur den gesamten weiteren Text wichtigen 
Begriffe und S"atze aus der Linearen Algebra behandelt. Falls in sp"ateren
Kapiteln an einzelnen Stellen weitergehende Grundlagen insbesondere aus 
anderen Bereichen n"otig sind, werden diese an den jeweiligen Stellen 
behandelt.

Da es sich bei dem Inhalt dieses Kapitels um
Grundlagen handelt, sind einige Beweise etwas oberfl"achlicher bzw. 
fehlen ganz.

Literatur:
\begin{itemize}
\item
      \cite{MM64} Kapitel 1 und 2
\item
      \cite{Doer77} Kapitel 6, 9 und 12
\item
      \cite{BS87} ab Seite 148
\end{itemize}

Im folgenden sind $A$ und $B$ $n \times n$-Matrizen. F"ur uns reichen
Betrachtungen im K"orper der rationalen Zahlen aus.

% **************************************************************************

\MySection{Matrizen und Determinanten}
\label{SecMatUndDet}

In diesem Kapitel werden die grundlegendsten Begriffe "uber Matrizen
und Determinanten aufgef"uhrt, um eine Grundlage f"ur den weiteren 
Text zu vereinbaren. 

\MyBeginDef
\index{invertierbar}
\label{DefInvertierbar}
    $A$ hei"st {\em invertierbar}, wenn es eine
    Matrix $B$ gibt, so da"s \[ AB = BA = E_n \]
    In diesem Fall hei"st $B$ {\em Inverse von $A$} und wird auch mit
    \[ A^{-1} \] bezeichnet.
\MyEndDef

\MyBeginDef
\index{Transponierte}
    Falls f"ur die Matrizen $A$ und $B$ gilt
    \[ b_{i,j} = a_{j,i} \] so hei"st $B$ { \em Transponierte von $A$}.
    F"ur die Transponierte von $A$ wird auch $A^T$ geschrieben.
\MyEndDef

\MyBeginDef
\label{DefTr}
\index{Spur}
    \[ \tr(A) := \sum_{i=1}^n a_{i,i} \]
    hei"st {\em Spur der Matrix $A$}.
\MyEndDef

\MyBeginDef
\index{Permutation}
\index{Inversion}
    Eine bijektive Abbildung
    \[ f : \{1,\ldots,n \} \rightarrow \{1, \ldots, n \} \]
    hei"st {\em $n$-Permutation}.
    Sei \[ 1 \leq i < j \leq n \] Falls gilt
    \[ f(i) > f(j) \MyKomma \] so hei"st diese Bedingung {\em Inversion der
    $n$-Permutation}.
\MyEndDef

\[ \permut_n \] bezeichnet die Menge aller $n$-Permutationen. Zusammen mit
der Konkatenation von Abbildungen bildet sie eine Gruppe, die
{\em symmetrische Gruppe $\permut_n$}.

\MyBeginDef
\index{Signatur einer Permutation}
\label{DefSig}
    Sei $f$ eine $n$-Permutation. Dann hei"st
    \begin{equation}
    \label{EquDefSig}
        \sig(f) := \prod_{1 \leq i < j \leq n} \frac{f(i)-f(j)}{i - j}
    \end{equation}
    {\em Signatur von $f$}.
\MyEndDef

Die so definierte Signatur besitzt folgende Eigenschaften:
\begin{itemize}
\item 
      Es gilt
      \[ \forall f \in \permut_n: \sig(f) \in \{ 1,-1 \} \]
      Aus der Permutationseigenschaft ergibt sich, da"s es f"ur jede 
      Differenz, die als Faktor im Z"ahler von \equref{EquDefSig} auftaucht,
      eine Differenz im Nenner mit dem gleichen Betrag existiert, so da"s 
      der Wert des gesamten Produktes den Betrag $1$ besitzt. Das Vorzeichen
      wird durch die Anzahl der Inversionen beeinflu"st.
\item
      Falls die Anzahl der Inversionen der $n$-Permutation $f$ gerade ist,
      so gilt \[ \sig(f)= 1 \MyKomma \] andernfalls gilt
      \[ \sig(f) = -1 \]
\item

      Die Anzahl der $n$-Permutationen $f$ mit \[ \sig(f)=1 \MyKomma \]
      ist gleich
      der Anzahl der $n$-Permutationen $g$ mit \[ \sig(g)=-1 \MyPunkt \]
      (\cite{Doer77} Seite 196)
\end{itemize}

Die Permutationen mit \[ \sig(f)=1 \] nennt man {\em gerade}, die anderen 
{\em ungerade}.

\MyBeginDef
\index{Determinante!Definition}
\label{DefDet}
    Seien \[ A,B \in \Rationals^{n^2} \]
    Sei \[ \det : \: \Rationals^{n^2} \rightarrow \Rationals \]
    eine Abbildung mit folgenden Eigenschaften:
    \begin{MyDescription}
    \MyItem{D1}
        Entsteht $B$ aus $A$ durch Multiplikation einer Zeile mit
        \[ r \in \Rationals \] so gilt:
        \[ \det(B) = r \det(A) \]
    \MyItem{D2}
        Enth"alt $A$ zwei gleiche Zeilen, so gilt:
        \[ \det(A) = 0 \]
    \MyItem{D3}
        Entsteht $B$ aus $A$ durch Addition des $r$-fachen einer Zeile zu
        einer anderen, so gilt:
        \[ \det(B) = \det(A) \]
    \MyItem{D4} F"ur die Einheitsmatrix gilt:
        \[ \det(E_n) = 1 \]
    \end{MyDescription}
    Dann hei"st
    \[ \det(A) \] {\em Determinante der Matrix $A$}.
\MyEndDef

\MyBeginDef
\label{DefZeilenOp}
\index{Zeilenoperationen!elementare}
\index{Spaltenoperationen!elementare}
    Die auf einer Matrix definierten Operationen
    \begin{itemize}
    \item 
          Vertauschung zweier Zeilen 
    \item 
          Multiplikation einer Zeile mit einem 
          Faktor\footnote{vgl. D1 in \ref{DefDet}}
    \item 
          Addition des Vielfachen einer Zeile zu einer 
          anderen\footnote{vgl. D3 in \ref{DefDet}}
    \end{itemize}
    werden {\em elementare Zeilenoperationen} genannt. Die entsprechenden
    Operationen auf Matrizenspalten werden 
    {\em elementare Spaltenoperationen} genannt.
\MyEndDef

\begin{satz}
\label{SatzDetPermut}
    \begin{equation}
    \label{EquDet}
       g(A) =
       \sum_{f \in \permut_n} 
           \sig(f) a_{1,f(1)} a_{2,f(2)} \ldots a_{n,f(n)}
    \end{equation}
    besitzt die Eigenschaften aus \ref{DefDet}.
\end{satz}
\begin{beweis}
    Da es sich hier um Grundlagen handelt, wird der Beweis weniger
    ausf"uhrlich angegeben:
    \begin{MyDescription}
    \MyItem{D1}
         F"ur jede Permutation kommt im entsprechenden Summanden in
         \equref{EquDet} aus jeder Zeile der Matrix genau ein Element als
         Faktor vor. Falls eine Zeile mit $r$ multipliziert wurde, kann
         man also aus jedem Summanden $r$ ausklammern und erh"alt die
         Behauptung.
    \MyItem{D2}
         Seien Zeile $i$ und Zeile $j$ gleich $(i \neq j)$.
         Berechne die
         Summe in \equref{EquDet} getrennt f"ur die ungeraden und die
         geraden Permutationen.

         Die
         $n$-Permutation $g$ vertausche $i$ mit $j$ und lasse alles andere
         gleich. Aus den Grundlagen der Theorie der
         Halbgruppen und Gruppen ergibt sich, da"s man die ungeraden
         $n$-Permutationen erh"alt, indem man die geraden
         $n$-Permutationen jeweils einzeln mit der Permutation $g$
         zusammen ausf"uhrt.

         Deshalb entsprechen sich die Summanden der beiden Teilsummen
         paarweise und unterscheiden sich nur durch das Vorzeichen. Der
         Gesamtausdruck besitzt also den Wert $0$.
    \MyItem{D3}
         Es werde das $r$-fache von Zeile $i$ zu Zeile $j$ addiert. Dadurch
         enth"alt jeder Summand in \equref{EquDet} genau einen Faktor,
         der seinerseits wieder die Summe zweier Matrizenelemente ist.
         Deshalb kann man die gesamte Summe in zwei Summen aufteilen.
         Die eine entspricht genau der Summe in \equref{EquDet}, die andere
         enth"alt in jedem Summanden zwei gleiche
         Faktoren, sowie den Faktor $r$. 
         Diesen Faktor kann man ausklammern (mit Hilfe
         von \mbox{D1}). Nach \mbox{D2} ist der Wert dieser Summe dann
         gleich Null, und nur die andere bleibt "ubrig.
    \MyItem{D4}
         Au"ser f"ur die identische Abbildung enth"alt jeder Summand
         der entsprechenden Permutation in \equref{EquDet} mindestens zwei
         Nullen als Faktoren und ist deshalb gleich Null. Der Summand, der
         der identischen Abbildung entspricht, hat den Wert $1$.
    \end{MyDescription}
    \nopagebreak
\end{beweis}

% **************************************************************************

\MySection{Der Rang einer Matrix}
\label{SecRang}
\index{Rang}

Zum Verst"andnis des weiteren Textes wird in diesem Kapitel der
Begriff des {\em Rangs} einer Matrix eingef"uhrt. Da dieser Begriff
f"ur die Untersuchung linearer Gleichungssysteme wichtig ist, werden
teilweise auch nichtquadratische Matrizen betrachtet\footnote{ Literatur: 
siehe \ref{SecMatUndDet} } . 

Da in diesem Kapitel 
wiederum Grundlagen aus der Linearen Algebra behandelt werden, ist
die Darstellung auf die f"ur uns wichtigen Aspekte beschr"ankt.

% $$$ Kenntnis der Begriffe 'linear unabh"angig' und 'Linearkombination'
%     wird hier vorausgesetzt
\MyBeginDef
    Die maximale Anzahl linear unab"angiger Spaltenvektoren einer Matrix
    wird mit {\em Spaltenrang} \index{Spaltenrang} bezeichet.
    
    Die maximale Anzahl linear unabh"angiger Zeilenvektoren einer Matrix
    wird mit {\em Zeilenrang} \index{Zeilenrang} bezeichnet.
\MyEndDef

Die folgenden Betrachtungen gelten f"ur den Zeilenrang analog.

Die $m \times n$-Matrix $A$ habe den Spaltenrang $r$. Es gilt also
\[ 0 \leq r \leq n \MyPunkt \] Die Spaltenvektoren von $A$ werden mit
\[ a_1, \, \ldots , \, a_n \] bezeichnet. Seien die Spaltenvektoren
\[a_{i_1}, \, \ldots, \, a_{i_r}\] linear unabh"angig. Das bedeutet, aus
\begin{eqnarray}
  & & d_1 a_{i_1} + \cdots + d_r a_{i_r} \nonumber \\
  & = & \left[
            \begin{array}{c}
                d_1 a_{1,{i_1}} \\ \vdots \\ d_1 a_{n,{i_1}}
            \end{array}
        \right]
        + \cdots +
        \left[
            \begin{array}{c}
                d_r a_{1,{i_r}} \\ \vdots \\ d_r a_{n,{i_r}}
            \end{array}
        \right]
        \nonumber \\
  & = & \left[
             \begin{array}{c}
                 d_1 a_{1,{i_1}} + \cdots + d_r a_{1,{i_r}} \\
                 \vdots \\
                 d_1 a_{n,{i_1}} + \cdots + d_r a_{n,{i_r}}
             \end{array}
        \right] \label{EquRangZeilentausch} \\
  & = & 0_n \nonumber
\end{eqnarray}
folgt
\[ d_1 = \ldots = d_r = 0 \MyPunkt \]
Vertauscht man zwei Elemente des Vektors \equref{EquRangZeilentausch},
bleibt die G"ultigkeit dieser Bedingung davon unber"uhrt.
Daraus folgt, da"s die Vertauschung zweier Matrixzeilen den Spaltenrang
der Matrix unber"uhrt l"a"st. 

Ebenso verh"alt es sich mit der Multiplikation einer Zeile der Matrix
mit einem Faktor $c \neq 0$.
Analog zur Argumentation bei der Vertauschung zweier Zeilen erh"alt man
\[
        \left[
             \begin{array}{c}
                 c (d_1 a_{1,{i_1}} + \cdots + d_r a_{1,{i_r}}) \\
                 \vdots \\
                 (d_1 a_{n,{i_1}} + \cdots + d_r a_{n,{i_r}})
             \end{array}
        \right]
\]
\MyPunktA{35em}
Die Bedingung $d_1 = \ldots = d_r = 0$ bleibt durch den Faktor unber"uhrt.

Das gleiche Ergebnis erh"alt man f"ur die Addition des Vielfachen einer
Zeile zu einer anderen.

Die elementaren Zeilenoperationen\footnote{siehe \ref{DefZeilenOp}}
lassen den Spaltenrang also unver"andert.
Analog verh"alt es sich mit den elementare Spaltenoperationen und
dem Zeilenrang.

Durch die Zeilen- und Spaltenoperationen l"a"st sich jede Matrix in
Diagonalform "uberf"uhren\footnote{nicht zu verwechseln mit
Diagonalisierbarkeit!}, ohne da"s dadurch der Rang ver"andert wird.

F"ur eine Matrix, bei der nur die Elemente der Hauptdiagonalen von Null
verschieden sein k"onnen, stimmen Zeilen- und Spaltenrang offensichtlich
"uberein. Da die Zeilen und Spaltenoperationen den Rang unver"andert
lassen, erh"alt man:
\begin{korollar}
\label{SatzRang}
    F"ur jede Matrix stimmen Zeilen- und Spaltenrang "uberein.
\end{korollar}
Aus diesem Grund ist die folgende Definition sinnvoll:

\MyBeginDef
\label{DefRang} \index{Rang}
    Die Anzahl linear unabh"angiger Spalten einer Matrix $A$ wird als
    {\em Rang von $A$} bezeichnet, kurz
    \[ \rg(A) \MyPunkt \]
\MyEndDef

Es gilt offensichtlich:
\[ rg(A) \leq \min(m,n) \MyPunkt \]

An dieser Stelle k"onnen wir drei wichtige Begriffe zueinander in
Beziehung setzen:

\begin{satz}
\label{SatzRgDetInv}
    F"ur eine $n \times n$-Matrix $A$ sind folgende Aussagen "aquivalent:
    \begin{itemize}
    \item
         Matrix $A$ ist invertierbar.
    \item
         F"ur die Determinante gilt:
         \[ \det(A) \neq 0 \]
    \item
         F"ur den Rang gilt:
         \[ \rg(A) = n \]
    \end{itemize}
\end{satz}
\begin{beweis}
    Es ist zu beachten, da"s jede Matrix durch elementare Zeilen-
    und Spaltenoperationen in eine Diagonalmatrix "uberf"uhrt werden kann,
    ohne da"s sich die Invertierbarkeitseigenschaft, der Betrag der 
    Determinante oder der Rang dadurch ver"anderen. Man kann also 
    o. B. d. A. davon ausgehen, da"s $A$ Diagonalform besitzt.
    
    Durch diese "Uberlegung wird die G"ultigkeit der Aussage offensichtlich.
\end{beweis}

% **************************************************************************

\MySection{L"osbarkeit linearer Gleichungssysteme}
\label{SecLinEqu}
\index{Gleichungssystem}

Da lineare Gleichungssysteme eine wichtige Rolle beim Verst"andnis
noch folgender Ausf"uhrungen spielen, werden sie in diesem Kapitel
n"aher betrachtet. Die nun folgenden Grundlagen aus der Linearen 
Algebra sind in der in \ref{SecMatUndDet} aufgelisteten Literatur
ausf"uhrlich behandelt. Wir beschr"anken uns hier auf die f"ur
den nachfolgenden Text wichtigen Sachverhalte.

F"ur die weiteren Beschreibungen drei grundlegende Begriffe aus der 
Linearen Algebra von Bedeutung, deren Definitionen deshalb hier
angegeben sind:

Sei $K$ ein K"orper. Eine Menge $V$ zusammen mit zwei Verkn"upfungen
\begin{eqnarray}
    + & : & V \times V \rightarrow V \\
    * & : & K \times V \rightarrow V \MyKomma
\end{eqnarray}
die die folgenden Bedingungen erf"ullt:
\begin{itemize}
\item Die Menge $V$ in Verbindung mit $+$ ist eine Gruppe.
\item F"ur alle $v,w \in V$ und alle $r,s \in K$ gelten die 
      Gleichungen
      \begin{eqnarray*}
          (r+s)v = rv +sv \\
          r(v+w) = rv + rw \\
          (rs)v = r(sv) \\
          1v = v \MyPunkt 
      \end{eqnarray*}
\end{itemize}
wird als $K$-Vektorraum bezeichnet.

\MyBeginDef
\label{DefUnterraum}
    Eine nichtleere Teilmenge $U$ eines $K$-Vektorraumes $V$ wird als 
    {\em Unterraum} \index{Unterraum} von $V$ bezeichnet, falls gilt
    \[
        \forall u,v \in U, \, r \in K : \: 
        \left\{
            \begin{array}{rcl}
                u + v & \in & U \\
                r u & \in & U \MyPunkt 
            \end{array}
        \right.
    \]
\MyEndDef

\MyBeginDef
\label{DefKern}
\index{Kern!einer linearen Abbildung}
    Die Menge aller Vektoren $x$,
    f"ur die bei einer gegebenen $m \times n$-Matrix $A$ gilt
    \Beq{EquKern}
        A x = 0_n
    \Eeq
    wird als {\em Kern der Matrix A} , kurz $\MyKer(A)$,
    bezeichnet\footnote{ In der Literatur wird h"aufig an dieser
    Stelle der Kern einer linearen Abbildung betrachtet. Da die 
    theoretischen Hintergr"unde hier nicht von Interesse sind, ist in
    unser Betrachtung von Matrizen die Rede.} .
\MyEndDef

F"ur unsere Zwecke ben"otigen wir noch zwei weitere Feststellungen:

\begin{bemerkung}
\label{SatzKernUnterraum}
    Betrachtet man \ref{DefUnterraum}, erkennt man,
    da"s alle $x$, die Gleichung \equref{EquKern} er\-f"ul\-len,
    einen Unterraum bilden.
\end{bemerkung}

Der Rang einer Matrix und die Dimension von $\MyKer(A)$ h"angen auf
eine f"ur uns wichtige Weise zusammen:

\begin{lemma}
\label{SatzDimKer}
    Sei $V$ ein $K$-Vektorraum der Dimension $n$ und
    $W$ ein $K$-Vektorraum der Dimension $m$. Sei
    \[ f: V \rightarrow W \] eine lineare Abbildung mit der
    entsprechenden $m \times n$-Matrix $A$. Dann gilt:
    \[ \rg(A) + \MyDim(\MyKer(A)) = \MyDim(V) \MyPunkt \]
\end{lemma}
\begin{beweis}
    Zum Beweis der obigen Gleichung wird die Dimension von 
    $\MyKer(A)$ in Abh"angigkeit vom Rang von $A$ betrachtet.

    Die Matrix $A$ habe den Rang $r$. Die maximale Anzahl linear
    unabh"angiger Spaltenvektoren betr"agt also $r$. O. B. d. A. seien
    die ersten $r$ Spaltenvektoren linear unabh"angig.
    
    Es wird in zwei Schritten gezeigt, da"s in $\MyKer(A)$ die maximale 
    Anzahl der linear unabh"angigen Vektoren, also die Dimension von $A$,
    $n-r$ betr"agt.
    
    \begin{itemize}
    \item Bezeichne $a_i$ den $i$-ten Spaltenvektor von $A$. Der Kern 
          von $A$ ist die Menge
          \begin{eqnarray*}
              & & \{ x \in V \MySetProperty A x = 0_m \} \\
              & = & \{ x \in V \MySetProperty
               a_1 x_1 + a_2 x_2 + \cdots + a_n x_n = 0_m \} \MyPunkt
          \end{eqnarray*}
          Die Dimension dieser Menge ist die maximale Anzahl linear
          unabh"angiger Vektoren $x$, die in ihr enthalten sind. Die
          Elemente eines Vektors $x$ kann man, wie an der obigen
          Darstellung ersichtlich ist, als Faktoren in einer
          Linearkombination von Spaltenvektoren von $A$ betrachten.
          Anhand der Voraussetzungen, die f"ur die Spaltenvektoren
          gelten, lassen sich Aussagen f"ur die Vektoren $x$ machen.
          
          Da $r+1$
          Spaltenvektoren von $A$ immer linear abh"angig sind, ist es
          m"oglich, durch Linearkombination von jeweils $r+1$ 
          Spaltenvektoren den Nullvektor zu erhalten. Sind die "ubrigen 
          $n-(r+1)$ Spaltenvektoren nicht an einer Linearkombination 
          beteiligt, entspricht das einer Null als entsprechendes Element 
          von $x$.
          
          Es gibt $n-r$ M"oglichkeiten,
          zu den ersten $r$ linear unabh"angigen Spaltenvektoren von $A$
          genau einen weiteren
          auszusuchen. Aufgrund der Verteilung der Nullen in den 
          entsprechenden Vektoren $x$ mu"s es mindestens $n-r$ linear 
          unabh"angige Vektoren im Kern von $A$ geben.
    \item 
          Angenommen, die Anzahl der linear unabh"angigen Vektoren 
          in $\MyKer(A)$ ist gr"o"ser als $n-r$. O. B. d. A. seien die 
          ersten $n-r$ dieser Vektoren
          so gew"ahlt, da"s bei jedem dieser Vektoren unter den
          Vektorelementen $x_{r+1}$ bis $x_n$ genau eines ungleich Null ist.
          Aus den vorangegangenen Ausf"uhrungen folgt, da"s solche
          Vektoren existieren m"ussen. Die Vektoren werden mit 
          $v_{r+1}, \, \ldots, \, v_n$ bezeichnet. F"ur den Vektor $v_i$
          sei das $i$-te Vektorelement ungleich Null.
          
          Da die ersten $r$ Spaltenvektoren von $A$ linear unabh"angig sind,
          gibt es f"ur die ersten $r$ Elemente jedes Vektors $v_i$
          keine Wahlm"oglichkeit, sobald das $i$-te Element im
          Rahmen der Randbedingungen dem Wert nach festliegt.

          Sei $w$ ein weiterer Vektor aus $\MyKer(A)$, der zu den Vektoren
          $v$ linear unabh"angig ist. Wird das $j$-te Element eines Vektors
          $v_i$ mit $v_{i,j}$ bezeichnet, und das $k$-te Element von $w$
          mit $w_k$, dann gibt es Faktoren $d_{r+1}, \, \ldots , \, d_n$,
          so da"s
          \[ \forall r+1 \leq k \leq n : \: w_k = d_k * v_{k,k} \MyPunkt \]
          
          Ebenso wie f"ur die Vektoren $v$ liegen damit auch die
          ersten $r$ Elemente des Vektors $w$ fest. Sie k"onnen anhand
          der obigen Beziehung aus den Elementen $r+1$ bis $n$ der
          Vektoren $v$ und $w$ berechnet werden, sofern $w$ "uberhaupt die
          Rahmenbedinungen erf"ullt und in $\MyKer(A)$ ist.

          Damit $w$ dennoch zu den Vektoren $v$ linear unabh"angig ist,
          mu"s es einen $r+1$-ten Spaltenvektor von $A$ geben, der zu den
          ersten $r$ Spaltenvektoren linear unabh"angig ist und der
          bei der Zusammenstellung der Vektoren $v$ mit Hilfe der
          Betrachtung von Linearkombinationen (s. o.) noch nicht benutzt
          wurde. Dies ist jedoch im Widerspruch zu den Voraussetzungen.
    \end{itemize}
    Somit folgt:
    \[ \rg(A) + \dim(\MyKer(A)) = r + (n-r) = n = \dim(V) \MyPunkt \]
\end{beweis}
Mit Hilfe von \ref{SatzKernUnterraum} und \ref{SatzDimKer} k"onnen wir
die f"ur uns wichtigen Eigenschaften linearer Gleichungssysteme
betrachten.

Ein lineares Gleichungssystem besteht aus $n$ Gleichungen mit $m$
Unbekannten:
\begin{eqnarray*}
    a_{1,1} x_1 + a_{1,2} x_2 + \cdots + a_{1,n} x_n & = & b_1 \\
    \vdots & \vdots & \vdots \\
    a_{n,1} x_1 + a_{n,2} x_2 + \cdots + a_{m,n} x_n & = & b_m
\end{eqnarray*}
Die $x_i$ sind die zu bestimmenden Unbekannten. Betrachtet man die
gegebenen Konstanten $a_{i,j}$ und $b_i$ als Elemente einer Matrix
bzw. eines Vektors, kann man das Gleichungssystem auch kompakter
darstellen:
\[ A x = b \MyPunkt \]
Dabei ist $A$ ein $n \times m$-Matrix und $b$ ein Vektor der L"ange $m$.
Ist $b= 0_m$, so bezeichnet man das Gleichungssystem als
homogen \index{Gleichungssystem!homogenes}.
Ist $b \neq 0_m$, so bezeichnet man das Gleichungssystem als
inhomogen \index{Gleichungssystem!inhomogenes}.

Uns interessieren die L"osungsmengen eines solchen Gleichungssystems.

\begin{satz}
\label{SatzLoesungsraum}
    Sei $V$ der zugrundeliegende $K$-Vektorraum der Dimension $n$.
    Die L"osungsmenge $L(A,0_m)$ des homogenen Gleichungssystems
    \[ A x = 0_m \] ist ein Unterraum von $V$ der Dimension
    \[ n - \rg(A) \MyPunkt \]
\end{satz}
\begin{beweis}
    Die L"osungsmenge $L(A,0_m)$ ist der Kern von $A$. Damit folgt
    die Behauptung aus \ref{SatzKernUnterraum} und
    \ref{SatzDimKer}.
\end{beweis}

Ist $\rg(A)=n$, gilt also $L(A,0_m) = \{ 0_n \}$.

Uns interessieren auch die L"osungen inhomogener Gleichungssysteme.
Dazu betrachtet man die erweiterte Matrix $[A,b]$, die aus der Matrix
$A$ und dem Vektor $b$ als $n+1$-te Spalte besteht:

\begin{satz}
\label{SatzRangGleich}
    Die Gleichung
    \Beq{EquInhomogen}
        A x = b \MyPunkt
    \Eeq
    ist genau dann l"osbar, wenn gilt
    \Beq{EquRangGleich}
        \rg(A) = \rg([A,b]) \MyPunkt
    \Eeq
\end{satz}
\begin{beweis}
    Man kann die linke Seite von \equref{EquInhomogen} als 
    Linearkombination von Spaltenvektoren von $A$ betrachten. Die Faktoren 
    der Linearkombination bilden die Elemente des L"osungsvektors $x$.

    Der Vektor $b$ ist genau dann als Linearkombination der
    Spaltenvektoren von $A$ darstellbar, wenn die maximale Anzahl linear
    unabh"angiger Vektoren in $A$ und $[A,b]$ gleich ist. Dies ist
    gleichbedeutend mit der G"ultigkeit von \equref{EquRangGleich}.
\end{beweis}

Dieser Satz f"uhrt zu einer f"ur uns bedeutsamen Charakterisierung
der L"osbarkeit:

\begin{korollar}
\label{SatzGenauEine}
    Ist \equref{EquInhomogen} l"osbar und $\rg(A) = n$,
    also $n \leq m$, dann sind die Spaltenvektoren von $A$
    linear unabh"angig
    und werden alle f"ur die Linearkombination zur Darstellung von $b$
    ben"otigt. Es gibt in diesem Fall genau einen Vektor $x$, der
    \equref{EquInhomogen} l"ost.
\end{korollar}

Ist $\rg(A) = m$, also $m \leq n$, dann ist \equref{EquInhomogen} f"ur
jedes $b$ l"osbar, denn es gilt allgemein
\[ \rg(A) \leq \rg([A,b]) \leq m \MyPunkt \] F"ur $\rg(A) = m$ folgt
daraus mit Hilfe der vorangegangenen Bemerkungen die L"osbarkeit.

% $$$ hier evtl. noch Wegener Satz 7.4 und Hilfssatz 7.7

% **************************************************************************

\MySection{Das charakteristische Polynom}
\label{ChapCharPoly}

Betrachtet \label{PageEigenMotiv}
man $A$ als lineare Abbildung im $n$-dimensionalen Vektorraum,
so lautet eine interessante Fragestellung:
\begin{quote}
    Gibt es einen Vektor $x$ ungleich dem Nullvektor sowie einen Skalar 
    $\lambda$, so da"s gilt
    \Beq{EquEigenMotiv}
        Ax=\lambda x \mbox{\hspace{1em}?}
    \Eeq
\end{quote}
Man kann \equref{EquEigenMotiv} umformen in
\[ (A - \lambda E_n)x = 0 \MyPunkt \] Dies ist eine homogenes lineares
Gleichungssystem mit $n$ Gleichungen in $n$ Unbekannten. 
Aus \ref{SatzLoesungsraum} und \ref{SatzRgDetInv} folgt, da"s es genau
dann eine nichttriviale L"osung besitzt, wenn gilt
\[ \det(A - \lambda E_n) = 0 \MyPunkt \]
Dies motiviert die folgende Definition:

\MyBeginDef
\label{DefCharPoly}
\index{Polynom!charakteristisches}
\index{Gleichung!charakteristische}
    \[ \det(A-\lambda E_n) \] hei"st\footnote{Die Form
    $\det(\lambda E_n - A)$ ist in der Literatur auch gebr"auchlich. Welche
    Form gew"ahlt wird, h"angt von den Vorlieben im Umgang mit den
    Vorzeichen ab. Bis auf die Vorzeichen bleibt die G"ultigkeit von
    Aussagen unber"uhrt. F"ur uns ist die in der Definition angegebene
    Form bequemer.} {\em charakteristisches Polynom der Matrix $A$}. Die
    obige Darstellung als Determinante einer Matrix 
    hei"st \index{Matrizendarstellung!des charakteristischen Polynoms} 
    {\em Matrizendarstellung} des charakteristischen Polynoms.
    \[ \det(A-\lambda E_n) = 0 \] hei"st {\em charakteristische Gleichung
    der Matrix $A$}.
    Die Matrix \[ A - \lambda E_n \] wird als 
    {\em charakteristische Matrix} \index{Matrix!charakteristische}
    bezeichnet.
\MyEndDef

Das charakteristische Polynom einer Matrix l"a"st sich auch in der
\index{Koeffizientendarstellung} {\em Koeffizientendarstellung} 
\Beq{EquKoeffDarst}
    c_n \lambda^n + c_{n-1} \lambda^{n-1} + \cdots + c_1 \lambda + c_0
\Eeq
angeben.

In der Literatur sind als
Vorzeichen von $c_i$ nicht nur $+$, sondern auch $-$ oder $(-1)^i$
gebr"auchlich. Mancherorts herrscht auch die Gewohnheit,
die Indizierung der Koeffizienten in der Form
$c_0 \lambda^n + c_1 \lambda^{n-1} + \cdots + c_{n-1} \lambda + c_n$
durchzuf"uhren. Da diese Vielfalt der Gebr"auche in der Literatur nicht 
nur auf die Koeffizientendarstellung beschr"ankt ist, kann es u. U. sein, 
da"s $c_n$ sowohl das Vorzeichen $+$ als auch immer den Wert $1$ besitzt.
In diesen F"allen wird $c_n$ auch h"aufig weggelassen.
F"ur uns ist die gew"ahlte Form \equref{EquKoeffDarst} der Darstellung
am sinnvollsten.
% $$$ in Verbindung mit meinem anderen Krempel: c_n = (-1)^n

\MyBeginDef 
\label{DefEigenwert}
\index{Eigenwert} \index{Eigenvektor}
    Die Nullstellen des charakteristischen Polynoms einer Matrix hei"sen
    {\em Eigenwerte} der Matrix. Ein Vektor $x$ der
    Gleichung \equref{EquEigenMotiv} zusammen mit einem Eigenwert
    $\lambda$ erf"ullt, hei"st {\em Eigenvektor}. Der Nullvektor ist
    als Eigenvektor nicht zugelassen.
\MyEndDef

\MyBeginDef
    Sei $\lambda_i$ ein Eigenwert der $n \times n$-Matrix $A$. Dann wird
    \[ \rg(A) - \rg(A - \lambda_i E_n) \] als
    {\em Rangabfall} \index{Rangabfall} des Eigenwertes $\lambda_i$ 
    bezeichnet.
\MyEndDef

Die Determinante und die Spur findet man unter den Koeffizienten des
charakteristischen Polynoms wieder. Besonders f"ur die Berechnung der
Determinante ist dies eine wichtige Beobachtung:
\begin{bemerkung}
\label{SatzDdurchP}
    \begin{eqnarray}
       \det(A) & = & c_0               \nonumber
    \\ \tr(A) & = & (-1)^{n-1} c_{n-1} \label{EquTrCoefficient}
    \end{eqnarray}
\end{bemerkung}
Wenn $p(\lambda)$ das charakteristische Polynom der Matrix $A$ ist, gilt
also: \[ \det(A)= p(0) \MyPunkt \]

Rechnet man die Matrizendarstellung des charakteristischen Polynoms um in
die Koeffizientendarstellung, wird dabei die G"ultigkeit von
\equref{EquTrCoefficient} deutlich. Der Wert von $c_{n-1}$ wird nur durch
die Matrixelemente in der Hauptdiagonalen beeinflu"st. Das Produkt dieser
Elemente besteht aus $n$ Linearfaktoren der Form
\[ a_{i,i} - \lambda \MyPunkt \] Berechnet man den Wert dieses Produktes,
so wird $c_n$ durch alle Summanden der Form
\[ (-1)^{n-1} a_{i,i}\lambda^{n-1} \] beeinflu"st. Insgesamt gilt also
\[ c_{n-1} = (-1)^{n-1} \sum_{i=1}^n a_{i,i} \MyKomma \] so da"s man mit
\equref{EquTrCoefficient} die Spur erh"alt.

Ein Algorithmus zur Berechnung der Koeffizienten des charakteristischen
Polynoms einer Matrix berechnet somit u. a. auch deren Determinante und
deren Spur.


