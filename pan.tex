%
% Datei: pan.tex
%
\MyChapter{Der Algorithmus von Pan}
\label{ChapPan}
%\label{SecIteration}
In diesem Kapitel wird der Algorithmus von V. Pan \cite{Pan85} zur 
Determinantenberechnung vorgestellt. Er kommt ebenfalls ohne
Divisionen\footnote{vgl. Bemerkungen in \ref{SecAlgFrame}}
aus und berechnet die Determinante iterativ. Auf diesen Algorithmus wird
mit {\em P-Alg.} Bezug genommen\footnote{vgl. Unterkapitel \ref{SecBez}}.

Man erh"alt insbesondere durch Variation der in den Unterkapitel
\ref{SecGuessInverse} und \ref{SecNewton} dargestellten Inhalte
einige weitere Versionen des Algorithmus.
F"ur eine vollst"andige Darstellung m"ussen alle diese Varianten beschrieben
und auf ihre Effizienz hin untersucht werden\footnote{N"aheres dazu ist 
insbesondere in \cite{PR85a} zu finden}. Da dies jedoch den Rahmen 
dieses Textes sprengt, beschr"anken sich die folgenden Darstellungen auf
die effizienteste Version des Algorithmus.

P-Alg. bietet erheblich mehr Variationsm"oglichkeiten als C-Alg., BGH-Alg.
und B-Alg., die, wie erw"ahnt, nicht alle hier behandelt werden k"onnen.
Vergleicht man P-Alg. von seiner Methodik her mit den drei
anderen, so erkennt man "Ahnlichkeiten zu BGH-Alg. . In beiden Algorithmen
werden mehrere auch separat bedeutsame teilweise schon l"anger bekannte
Verfahren zusammen verwendet. Von diesen Verfahren hebt sich lediglich
die in Unterkapitel \ref{SecGuessInverse} dargestellte Methode zur
Berechnung einer N"aherungsinversen ab. Sie wurde in \cite{Pan85} erstmalig
ver"offentlicht.

% **************************************************************************

\MySection{Diagonalisierbarkeit}

In diesem Unterkapitel wird die Diagonalisierbarkeit von Matrizen behandelt.
Es ist f"ur das Verst"andnis des in diesem Kapitel dargestellten Algorithmus
zur Determiantenberechnung nicht unbedingt erforderlich und kann daher beim 
Lesen auch "ubersprungen werden. Im folgenden werden jedoch einige 
Hintergr"unde der im Unterkapitel \ref{SecKrylov} dargestellten Methode von 
Krylov n"aher beleuchtet, die ein paar Zusammenh"ange klarer werden lassen.

Literatur zu diesem 
Thema ist neben den in Kapitel \ref{ChapBase} genannten Stellen
auch \cite{Zurm64} S. 169 ff .

Zum Problem der Diagonalisierbarkeit\footnote{Definition s. u.} gelangt
man "uber den Begriff der Basis eines Vektorraumes. Da es sich dabei
um Grundlagen der Linearen Algebra handelt, erfolgt die Darstellung 
vergleichsweise oberfl"achlich.

Sei $K$ ein K"orper und $V$ ein $K$-Vektorraum.
Seien $k_1,\, k_2, \, \ldots, \, k_n \in K \backslash \{ 0 \} $ 
und $v_1, \, v_2, \, \ldots, \, v_n \in V$.
Dann wird
\Beq{EquLinKomb}
   k_1 v_1 + k_2 v_2 + \ldots + k_n v_n 
\Eeq
als \index{Linearkombination} {\em Linearkombination} der Vektoren 
$v_1$ bis $v_n$ bezeichnet.
Sei $0_m$ der Nullvektor in $V$.
Falls f"ur die Vektoren die Bedingung
\[ k_1 v_1 + k_2 v_2 + \ldots + k_n v_n = 0_m \Rightarrow
   k_1 = k_2 = \ldots = k_n = 0
\]
erf"ullt ist, werden sie als 
{\em linear unabh"angig} \index{linear unabh{\Mya}ngig} bezeichnet, 
ansonsten als {\em linear abh"angig} \index{linear abh{\Mya}ngig}.
Falls jedes Element von $V$ als Linearkombination der Vektoren 
$v_1,\ldots,v_n$
darstellbar ist, werden diese Vektoren als 
\index{Basis} {\em Basis} bezeichnet.

F"ur alle Basen gilt die Aussage: \nopagebreak
\begin{quote}
    Je zwei Basen eines $K$-Vektorraumes bestehen aus dergleichen 
    Anzahl von Vektoren.
\end{quote}
Sei $B$ die Basis des $K$-Vektorraumes $V$. Dann wird die Anzahl 
der Vektoren, die $B$ bilden, 
als {\em Dimension von $V$} \index{Dimension} , 
kurz $\dim(V)$, bezeichnet.

Zur Darstellung von Elementen eines Vektorraumes $V$ w"ahlt man sich eine
Basis und beschreibt jedes Element des Vektorraumes als Linearkombination
der Elemente der Basis. Sei $n$ die Dimension des Vektorraumes. Dann
kann man auf diese Weise jedes Element von $V$ als $n$-Tupel von Elementen
des zugrunde liegenden K"orpers $K$ betrachten. Man erh"alt die
Vektorschreibweise:
\Beq{EquVektorSchreibweise}
   \left[ \begin{array}{c} k_1 \\k_2\\ \vdots\\ k_n \end{array} \right]
\Eeq
Die Basis der Form
\begin{quote} % $$$$ Formatierung gepr"uft ?
   $\left[ \begin{array}{c} 1\\ 0\\ 0\\ \vdots\\ 0\end{array} \right]$,
   \hspace{0.7em}
   $\left[ \begin{array}{c} 0\\ 1\\ 0\\ \vdots\\ 0\end{array} \right]$,
   $\ldots , $ \hspace{0.7em}
   $\left[ \begin{array}{c} 0\\ 0\\ 0\\ \vdots\\ 1 \end{array} \right]$
\end{quote}
wird als
{\em kanonische Basis} \index{kanonische Basis} bezeichnet.

Zur Betrachtung der Beziehungen verschiedener Basen zueinander werden diese
Basen ihrerseits bzgl. der kanonischen Basis dargestellt. 

Wird ein Vektor $v$ bzgl. einer Basis $B$ dargestellt, so wird dies
folgenderma"sen ausgedr"uckt:
\[ \left[
       \begin{array}{c}
           v_1 \\ \vdots \\ v_n
       \end{array}
   \right]_B
\]
Werden Vektoren zu Matrizen zusammengefa"st, so wird die gleiche 
Schreibweise auch f"ur diese Matrizen verwendet.

Sei $V$ ein $K$-Vektorraum der Dimension $n$ und $W$ ein $K$-Vektorraum
der Dimension $m$. Ein Ergebnis der Linearen Algebra lautet, da"s dann
die Menge aller $K$-Vektorraumhomomorphismen\footnote{also die Menge
aller {\em strukurvertr"aglichen} linearen Abbildungen} $f$
\[ f : \: V \rightarrow W \]
isomorph ist zur Menge aller $m \times n$-Matrizen $A$, wenn man die 
Abbildung definiert als\footnote{Dies entspricht der 
Matrizenmultiplikation (vgl. \ref{SatzAlgMatMult}), wenn man den 
abzubildenden Vektor $v$ als $n \times 1$-Matrix 
betrachtet (vgl. \ref{SatzAlgMatMult}).}
\[ f \lb
         \left[
             \begin{array}{c} v_1\\ v_2\\ \vdots\\ v_n \end{array}
         \right]
     \rb
   :=
      \left[
          \begin{array}{c}
              \sum_{j=1}^n a_{1,j} v_j \LMatStrut \\
              \vdots \LMatStrut                   \\
              \sum_{j=1}^n a_{m,j} v_j \LMatStrut
          \end{array}
      \right]
\]

Die Untersuchung der $K$-Vektorraumhomomorphismen kann man
also anhand der entsprechenden Matrien vornehmen. Im folgenden 
sind nur quadratische Matrizen von Interesse. Deshalb werden in den 
weiteren Ausf"uhrungen nur diese Matrizen beachtet.

Stellt man die Vektoren einer Basis $B_V$ bzgl. einer anderen Basis $B_W$
(normalerweise der kanonischen Basis) dar und betrachtet sie als
Spaltenvektoren einer Matrix \[ [B_V]_{B_W} \MyKomma \]
so erkennt man beim Vergleich von
\equref{EquLinKomb} und \equref{EquVektorSchreibweise} miteinander,
da"s man einen bzgl. $B_V$ dargestellten Vektor $x$ in seine Darstellung
bzgl. $B_W$ umrechnen kann durch
\Beq{EquBasiswechsel}
    [x]_{B_W}= B_V[x] \MyPunkt
\Eeq
Man erkennt also, da"s man eine Basis auch als Vektorraumhomomorphismus
betrachten kann. Die umgekehrte Betrachtungsweise ist nat"urlich nicht
m"oglich.

Um zum Begriff der {\em Diagonalisierbarkeit} zu gelangen, betrachten wir
nun, was passiert, wenn man Basen austauscht und die Darstellungen bzgl. der
neuen Basen vornehmen will.

Seien $V$ und $W$ jeweils $K$-Vektorr"aume sowie $B_V$ und $B_W$ jeweils
Basen dieser Vektorr"aume. Sei \[ f: \: V \rightarrow W \] ein
Vektorraumhomomorphismus und $A$ die entsprechende Matrix. Es gelte
\Beq{EquVonVNachW}
    \left[
        \begin{array}{c} y_1\\ y_2\\ \vdots\\ y_n \end{array}
    \right]_{B_W}
    = 
    A
    \left[
        \begin{array}{c} x_1\\ x_2\\ \vdots\\ x_n \end{array}
    \right]_{B_V}
\Eeq
Wechselt man nun zu den Basen $\tilde{B_V}$ und $\tilde{B_W}$ und stellt 
die neuen Basen bzgl. der alten durch die Matrizen $C$ und $D$ dar, so gilt
entsprechend \equref{EquBasiswechsel}:
\begin{eqnarray*}
    \left[
        \begin{array}{c} x_1\\ x_2\\ \vdots\\ x_n \end{array}
    \right]_{B_V} 
    & = & 
    C
    \left[
        \begin{array}{c} 
            \tilde{x_1}\\ 
            \tilde{x_2}\\ \vdots \\
            \tilde{x_n}
        \end{array}
    \right]_{\tilde{B_V}}
\\
    \left[
        \begin{array}{c} y_1\\ y_2\\ \vdots\\ y_n \end{array}
    \right]_{B_W}
    & = & 
    D
    \left[
        \begin{array}{c} 
            \tilde{y_1}\\ 
            \tilde{y_2}\\ \vdots \\
            \tilde{y_n}
        \end{array}
    \right]_{\tilde{B_W}}
\end{eqnarray*}
Gleichung \equref{EquVonVNachW} bekommt also folgendes Aussehen:
\[
    D
    \left[
        \begin{array}{c} 
            \tilde{y_1}\\ 
            \tilde{y_2}\\ \vdots \\
            \tilde{y_n}
        \end{array}
    \right]_{\tilde{B_W}}
    =
    A \: C
    \left[
        \begin{array}{c} 
            \tilde{x_1}\\ 
            \tilde{x_2}\\ \vdots \\
            \tilde{x_n}
        \end{array}
    \right]_{\tilde{B_V}}
\]
Multipliziert man beide Seiten mit $D^{-1}$, erh"alt man:
\[
    \left[
        \begin{array}{c} 
            \tilde{y_1}\\ 
            \tilde{y_2}\\ \vdots \\
            \tilde{y_n}
        \end{array}
    \right]_{\tilde{B_W}}
    =
    D^{-1} \: A \: C
    \left[
        \begin{array}{c} 
            \tilde{x_1}\\ 
            \tilde{x_2}\\ \vdots \\
            \tilde{x_n}
        \end{array}
    \right]_{\tilde{B_V}}
\]
Die Abbildung $f$ wird bzgl. der neuen Basen $\tilde{B_V}$ und $\tilde{B_W}$
also durch die Matrix $A':=D^{-1}AC$ dargestellt. W"ahlt man die neuen 
Basen geeignet, so ist es immer m"oglich zu erreichen, da"s 
$A'$ die Form
\[ \left[
   \begin{array}{cccc}
       a_{1,1}' & 0        & \cdots & 0        \MatStrut \\
       0        & a_{2,2}' & \ddots & \vdots   \MatStrut \\
       \vdots   & \ddots   & \ddots & 0        \MatStrut \\
       0        & \cdots   & 0      & a_{n,n}' \MatStrut
   \end{array} \right]
\] bekommt. Eine Matrix dieser Form wird als 
{\em Diagonalmatrix} \index{Diagonalmatrix} bezeichnet.

Betrachtet man nun statt einer Abbildung zwischen den
Vektorr"aumen eine Abbildung in $V$ und wechselt die Basis von $V$, so
besitzt die Abbildung bzgl. der neuen Basis die Form $C^{-1}AC$.
Es stellt sich wiederum die Frage, ob es m"oglich ist, die neue Basis
so zu w"ahlen, da"s $C^{-1}AC$ eine Diagonalmatrix ist. Eine Matrix $A$,
f"ur die das m"oglich ist, wird als
{\em diagonalisierbar} \index{diagonalisierbar} bezeichnet.

Um eine Beziehung zur Methode von Krylov (siehe Unterkapitel 
\ref{SecKrylov}) herstellen zu k"onnen, folgt 
eine Charakterisierung der Diagonalisierbarkeit. Dazu greifen wir auf 
den bereits in
\ref{DefUnterraum} definierten Begriff des {\em Unterraumes} zur"uck.
Anhand dieser Definitionen erkennt man, da"s alle zu einem Eigenwert
geh"orenden Eigenvektoren zusammen mit dem Nullvektor einen Unterraum
des zugrunde liegenden Vektorraumes bilden. Er wird als
{\em Eigenraum} \index{Eigenraum} bezeichnet.

An dieser Stelle werden die Eigenwerte mit der Diagonalisierbarkeit
in Verbindung gebracht:
\begin{satz}
\label{SatzEigenDiagonalBasis}
    Eine $n \times n$-Matrix $A$ ist genau dann diagonalisierbar, wenn der 
    zugrunde liegende $K$-Vektorraum $V$ eine Basis aus Eigenvektoren von 
    $A$ besitzt.
\end{satz}
\begin{beweis}
    Wir verzichten hier auf einen ausf"uhrlichen Beweis. Die Ideen f"ur
    die beiden Beweisrichtungen sind:
    \begin{itemize}
    \item 
          Ist eine Matrix $A$ diagonalisierbar und wechselt man die Basis
          so, da"s $A$ Diagonalgestalt bekommt, werden dadurch die
          Vektoren der kanonischen Basis zu Eigenvektoren der Matrix.
    \item
          Bilden die Eigenvektoren einer Matrix $A$ eine Basis von $V$ und
          benutzt man diese Basis zur Darstellung bekommt $A$ die
          Form einer Diagonalmatrix.
    \end{itemize}
\end{beweis}

\begin{satz}
\label{SatzEigenUnabhaengig}
    Seien $\lambda_1,\, \ldots,\, \lambda_k$ paarweise verschiedene
    Eigenwerte der $n \times n$-Matrix $A$. Sei $v_i$ ein Eigenvektor zu
    $\lambda_i$. Dann sind die Eigenvektoren $v_1,\, \ldots,\, v_k$
    linear unabh"angig.
\end{satz}
\begin{beweis}
    Der Beweis erfolgt durch Induktion nach der Anzahl der verschiedenen
    Eigenwerte $k$:
    \begin{MyDescription}
    \MyItem{$k=1$}
        F"ur diesen Fall ist die Behauptung offensichtlich richtig.
    \MyItem{$k>1$}
        Die Behauptung gelte f"ur $k-1$ und sei f"ur $k$ zu zeigen.
        Induktionsvoraussetzung ist also, da"s
        $v_1, \, \ldots, \, v_{k-1}$ linear unabh"angig sind.
        Angenommen $v_1, \, \ldots,\, v_{k}$ sind linear abh"angig.
        Dann existieren eindeutig bestimmte $r_1, \ldots, r_{k-1} \in K$ 
        mit
        \Beq{EquEigenUnabhaengig}
           v_k = r_1 v_1 + \cdots + r_{k-1} v_{k-1} \MyPunkt
        \Eeq
        Da $v_k$ nicht der Nullvektor sein kann, mu"s mindestens einer 
        der Faktoren $r_1, \dots, r_k$ ungleich Null sein, z. B. $r_i$. 
        
        Betrachtet man $A$ als 
        Abbildung und wendet diese Abbildung auf 
        \equref{EquEigenUnabhaengig} an, so kann man, da es sich um
        Eigenvektoren handelt, auch mit den Eigenwerten multiplizieren
        und erh"alt
        \[
           \lambda_k v_k = \lambda_1 r_1 v_1 + \cdots 
                                             + \lambda_k r_k v_k \MyPunkt
        \]
        Ist nun $\lambda_k = 0$, dann mu"s wegen der Verschiedenheit
        der Eigenwerte $\lambda_i \neq 0$ sein und man erh"alt einen 
        Widerspruch zur linearen Unabh"angigkeit von 
        $v_1,\, \ldots, \, v_k$.
        
        Ist $\lambda_k \neq 0$ erh"alt man einen Widerspruch zur 
        Eindeutigkeit der Darstellung von $v_k$.
    \end{MyDescription}
\end{beweis}

Aus \ref{SatzEigenDiagonalBasis} und \ref{SatzEigenUnabhaengig} ergibt sich:

\begin{korollar}
\label{SatzMaxEigen}
    Die maximale Anzahl verschiedener Eigenwerte einer Matrix ist gleich
    der Dimension des zugrunde liegenden Vektorraumes. 
    
    Besitzt eine Matrix maximal viele verschiedene Eigenwerte, so ist
    sie diagonalisierbar.
\end{korollar}

F"ur die zweite Folgerung ben"otigen wird einen weiteren Betriff. Seien
dazu $T$ und $U$ Unterr"aume des $K$-Vektorraumes $V$. Dann wird die Menge
\[
   \{ w \MySetProperty 
      \exists k,l \in K, \, t \in T, \, u \in U: \: w=kt+lu \}
\]
aller Linearkombinationen zweier Vektoren aus $T$ und $U$ als
{\em direkte Summe von $T$ und $U$} \index{direkte Summe} bezeichnet.
Anhand von \ref{DefUnterraum} erkennt man, da"s die direkte Summe
zweier Unterr"aume von $V$ wiederum ein Unterraum von $V$ ist.

Somit erh"alt man aus
\ref{SatzEigenDiagonalBasis} und \ref{SatzEigenUnabhaengig}:

\begin{korollar}
\label{SatzEigenraum}
    Eine Matrix ist genau dann diagonalisierbar, wenn die direkte Summe
    aller Eigenr"aume der Matrix den zugrundliegenden Vektorraum ergibt.
\end{korollar}

Die Bedeutung der in diesem Unterkapitel dargestellten Sachverhalte
wird deutlich, wenn man sie mit den in Unterkapitel
\ref{SecKrylov} erw"ahnten Einschr"ankungen f"ur die
Verwendbarkeit der Methode von Krylov zur Berechnung der Koeffizienten 
des charakteristischen Polynoms vergleicht.

% ... mehrfache Eigenwerte --> Dimension des Eigenraums

% Welche Beziehungen bestehen zwischen Invertierbarkeit und
%     Dialgonalisierbarkeit?

% **************************************************************************

\MySection{Das Minimalpolynom}
\label{SecMinimalpolynom}

Die Methode von Krylov (siehe \ref{SecKrylov}) dient zur Bestimmung der
Koeffizienten des Minimalpolynoms einer Matrix. Deshalb wird hier dieses
Minimalpolynom zun"achst n"aher betrachtet. Eine tiefergreifende 
Behandlung befindet sich z. B. in \cite{Zurm64} S. 233 ff .

In Satz \ref{SatzCayleyHamilton} wird bewiesen, da"s eine
$n \times n$-Matrix $A$ ihre eigene charakteristische Gleichung erf"ullt.
Diese Beobachtung f"uhrt zu: \nopagebreak[3]
\MyBeginDef
\label{DefMinimalpolynom} 
\index{Minimalpolynom} \index{Minimumgleichung}
    Das Polynom $m(\lambda)$ mit dem kleinsten Grad, f"ur das die 
    Gleichung \[ m(A) = 0_n \] erf"ullt ist, wird 
    {\em Minimalpolynom} genannt. Die Gleichung wird als 
    {\em Minimumgleichung} \index{Minimumgleichung} bezeichnet.
\MyEndDef

Um die Methode von Krylov verstehen zu k"onnen, m"ussen wir verschiedene
Eigenschaften des Minimalpolynoms beleuchten:

\begin{satz}
\label{SatzMinimalpolynomVielfaches}
    Sei $A$ eine $n \times n$-Matrix und $f(\lambda)$ ein Polynom. Es
    gelte \[ f(A) = 0_n \MyPunkt \] Dann ist $f(\lambda)$ ein Vielfaches
    des Minimalpolynoms $m(\lambda)$ von $A$.
\end{satz}
\begin{beweis}
    Angenommen die Behauptung ist falsch. Dann entsteht bei der Division
    von $f(\lambda)$ durch $m(\lambda)$ ein Rest $r(\lambda)$ und f"ur
    ein geeignetes Polynom $q(\lambda)$ gilt:
    \[ f(\lambda) = q(\lambda)m(\lambda) + r(\lambda) \MyPunkt \]
    Der Grad von $r(\lambda)$ ist kleiner als der Grad von $m(\lambda)$.
    Wird nun in diese Gleichung $A$ eingesetzt, erh"alt man
    \[ 0_n = 0_n + r(A) \MyPunkt \] Also mu"s auch gelten
    \[ r(A) = 0_n \MyPunkt \] Da jedoch das Minimalpolynom das Polynom mit
    dem kleinsten Grad ist, das diese Bedingung erf"ullt, f"uhrt dies
    zu einem Widerspruch.
\end{beweis}

Aus \ref{SatzCayleyHamilton} und \ref{SatzMinimalpolynomVielfaches} 
ergibt sich:
\begin{korollar}
\label{SatzVielfaches}
    Das charakteristische Polynom ist ein Vielfaches des Minimalpolynoms.    
\end{korollar}

Da wir die Methode von Krylov zur Berechnung des charakteristischen 
Polynoms verwenden wollen, m"ussen wir wissen, unter welchen Umst"anden
es mit dem Minimalpolynom zusammenf"allt. Diese 
Frage beantwortet der folgende Satz:

\begin{satz}
\label{SatzCharMatGGT}
    Sei $A$ eine $n \times n$-Matrix.
    Es wird definiert:
    \[ C := A - \lambda E_n \MyPunkt \] Sei
    \[ p(\lambda) = \det(C) \] das charakteristische Polynom von $A$.
    Es gelte
    \Beq{Equ20MatGGT}
        m(\lambda) = \frac{ p(\lambda) }{ q(\lambda) }
    \Eeq
    \MyPunktA{25em}
    Das Polynom $m(\lambda)$ ist genau dann das Minimalpolynom von $A$,
    wenn $q(\lambda)$ der gr"o"ste gemeinsame Teiler (ggT) der
    Determinanten aller $(n-1) \times (n-1)$-Untermatrizen von $C$ ist.
\end{satz}
\begin{beweis}
    Der Beweis erfolgt in zwei Schritten:
    \begin{itemize}
    \item Sei zun"achst $q(\lambda)$ der ggT Determinanten der 
          Untermatrizen. Es ist
          zu zeigen, da"s dann $m(\lambda)$ das Minimalpolynom ist.

          Mit Hilfe von Satz \ref{SatzEntw} (Zeilen- und Spaltenentwicklung)
          folgt, da"s $p(\lambda)$ durch $q(\lambda)$ teilbar ist. D. h. es
          gibt ein Polynom $m'(\lambda)$, so da"s
          \Beq{Equ2MatGGT}
              p(\lambda) = m'(\lambda) q(\lambda) \MyPunkt
          \Eeq
          Weiterhin gibt es eine $n \times n$-Matrix
          $M$ aus teilerfremden Polynomen "uber $\lambda$, so da"s gilt:
          \Beq{Equ1MatGGT}
              \adj(C) = M q(\lambda) \MyPunkt
          \Eeq
          Nach Satz \ref{SatzAdj} gilt:
          \Beq{Equ5MatGGT}
              C \, \adj(C) = E_n p(\lambda) \MyPunkt
          \Eeq
          Mit \equref{Equ2MatGGT} folgt aus \equref{Equ5MatGGT}:
          \Beq{Equ4MatGGT}
              C \, \adj(C) = E_n m'(\lambda) q(\lambda) \MyPunkt
          \Eeq
          Mit \equref{Equ1MatGGT} folgt aus \equref{Equ5MatGGT}:
          \Beq{Equ3MatGGT}
              C \, \adj(C) = C M q(\lambda) \MyPunkt
          \Eeq
          Aus \equref{Equ3MatGGT} und \equref{Equ4MatGGT} folgt:
          \[ C M = E_n m'(\lambda) \MyPunkt \]
          Benutzt man die Definition von $C$, erh"alt man
          \[ (A - \lambda E_n) M = E_n m'(\lambda) \MyPunkt \]
          Setzt man nun in dieser Gleichung $A$ f"ur $\lambda$ ein,
          ergibt sich
          \[ m'(A) = 0_n \MyPunkt \]
          Nach Satz \ref{SatzMinimalpolynomVielfaches} ist $m'(\lambda)$
          also ein Vielfaches des Minimalpolynoms $m(\lambda)$.
          
          Angenommen es gibt ein Polynom $m''(\lambda)$ mit
          \[ m''(A) = 0_n \MyKomma \] dessen Grad kleiner ist als der
          Grad von $m'(\lambda)$. Da der Grad des charakteristischen
          Polynoms immer $n$ ist, mu"s dann der Grad von $q(\lambda)$ in
          \equref{Equ2MatGGT} und somit auch in \equref{Equ1MatGGT}
          entsprechend gr"o"ser sein, im Widerspruch dazu, da"s
          $q(\lambda)$ der ggT ist. Also ist $m'(\lambda)$ das
          Minimalpolynom.
    \item Sei nun $m(\lambda)$ das Minimalpolynom. Dann ist zu zeigen, da"s
          $q(\lambda)$ der ggT der Unterdeterminanten ist.
          
          Nach \ref{SatzVielfaches} gibt es ein Polynom $q'(\lambda)$, so
          da"s
          \Beq{Equ6MatGGT}
              p(\lambda) = q'(\lambda) m(\lambda) \MyPunkt
          \Eeq
          Benutzt man f"ur das Minimalpolynom die Koeffizientendarstellung,
          erh"alt man mit geeigneten Koeffizienten $b_i$:
          \begin{eqnarray*}
              \lefteqn{- E_n m(\lambda)} \\
              & = & m(A) - E_n m(\lambda) \\
              & = & b_m(A^m - \lambda^m E_n)
                    + b_{m-1}(A^{m-1} - \lambda^{m-1} E_n) + \cdots +
                    b_1(A - \lambda E_n) \MyPunkt \\
          \end{eqnarray*}
          Also ist $m(\lambda)$ durch $(A-\lambda E_n)$ teilbar und es
          gibt eine Matrix $N$ aus Polynomen "uber $\lambda$, so da"s gilt:
          \Beq{Equ7MatGGT}
              m(\lambda) E_n = (A - \lambda E_n) N \MyPunkt
          \Eeq
          Mutipliziert man beide Seiten mit $q'(\lambda)$, erh"alt man
          \[ p(\lambda) E_n = q'(\lambda) (A - \lambda E_n) N \MyPunkt \]
          Subtrahiert man nun auf beiden Seiten
          \begin{eqnarray*}
               & & p(\lambda) E_n \\
               & \MyStack{nach \ref{SatzAdj}}{=} & C \,\adj(C) \\
               & = & (A - \lambda E_n) \adj(C) \MyKomma
          \end{eqnarray*}
          erh"alt man
          \[
             0_{n,n} =
             \underbrace{ (A - \lambda E_n)
                        }_{ \mbox{$(*1)$} }
             \:
             \underbrace{ (q'(\lambda) N - \adj(C))
                        }_{ \mbox{$(*2)$} }
          \]
          In dieser Gleichung ist Term $(*1)$ f"ur ein beliebig gew"ahltes 
          $\lambda$ ungleich 
          der Nullmatix\footnote{abgesehen von einigen Sonderf"allen, deren
          Existenz den Beweis jedoch nicht beeintr"achtigt}. Also mu"s 
          Term $(*2)$ gleich der Nullmatrix sein, so 
          da"s gilt:
          \[ q'(\lambda) N = \adj(C) \]
          Also ist $q'(\lambda)$ Teiler der Elemente von $\adj(C)$.
          
          Angenommen es gibt ein Polynom $q''(\lambda)$, dessen Grad
          gr"o"ser ist als der Grad von $q'(\lambda)$ und das ebenfalls
          Teiler der Elemente von $\adj(C)$ ist. Da der Grad von
          $p(\lambda)$ immer $n$ ist, mu"s dann der Grad von $m(\lambda)$ 
          in \equref{Equ6MatGGT} kleiner sein, im Widerspruch zu der
          Voraussetzung, da"s $m(\lambda)$ das Minimalpolynom ist.
    \end{itemize}
\end{beweis}

Berechnet man in \equref{Equ7MatGGT} auf beiden Seiten die Determinante,
erh"alt man
\Beq{Equ21MatGGT}
   m^n(\lambda) = p(\lambda) \det(N) \MyPunkt
\Eeq
Aus \equref{Equ20MatGGT} und \equref{Equ21MatGGT} ergibt sich:

\begin{korollar}
\label{SatzMinimalNullGenauDann}
    Es ist $\lambda_i$ genau dann Nullstelle von $m(\lambda)$, wenn es auch
    Nullstelle von $p(\lambda)$ ist.
\end{korollar}

Anders ausgedr"uckt: $m(\lambda)$ und $p(\lambda)$ besitzen die gleichen
Nullstellen mit evtl. verschiedenen Vielfachheiten. Das f"uhrt
zu einer weiteren Schlu"sfolgerung:

\begin{korollar}
\label{SatzPaarweiseVerschieden}
    Besitzt eine $n \times n$-Matrix $n$ paarweise verschiedene Eigenwerte,
    so stimmen ihr Minimalpolynom und ihr charakteristisches Polynom
    "uberein.
\end{korollar}

Falls die Eigenwerte nicht paarweise verschieden sind, k"onnen 
Minimalpolynom und charakteristisches Polynom also verschieden sein, was
eine Einschr"ankung f"ur die Methode von Krylov 
(siehe Unterkapitel \ref{SecKrylov}) bedeutet, wenn man sie zur Berechnung
der Koeffizienten des charakteristischen Polynoms verwendet.
Wann die beiden Polynome verschieden sind, zeigt der folgende Satz:

\begin{satz}
\label{SatzMinimalDimEins}
    Das Minimalpolynom $m(\lambda)$ einer Matrix $A$ stimmt genau dann 
    mit dem charakteristischen Polynom $p(\lambda)$ der Matrix "uberein, 
    wenn die Dimension jedes Eigenraumes $1$ 
    betr"agt\footnote{Es ist zu beachten, da"s diese Aussage nicht dazu 
    "aquivalent ist, da"s die direkte Summe der Eigenr"aume den gesamten 
    Vektorraum ergibt.}.
\end{satz}
\begin{beweis}
    Aus \ref{SatzCharMatGGT} folgt, da"s $m(\lambda)$ und $p(\lambda)$
    genau dann "ubereinstimmen, wenn der 
    ggT der Determinanten aller $(n-1) \times (n-1)$-Untermatrizen
    der charakteristischen Matrix von $A$ gleich $1$ ist.
    
    Betrachtet man die beiden Polynome in ihrer Linearfaktorendarstellung,
    mu"s der genannte ggT, falls er ungleich $1$ ist, mit einem Linearfaktor
    von $p(\lambda)$ "ubereinstimmen. F"ur die entsprechende Nullstelle
    von $p(\lambda)$ verschwinden auch alle 
    $(n-1) \times (n-1)$-Unterdeterminanten. Es gibt also f"ur diese
    Nullstelle keine $n-1$ linear unabh"angigen Spaltenvektoren der 
    charakteristischen Matrix. Der Rangabfall der Nullstelle ist also
    gr"o"ser als $1$ und es gibt mehr als einen linear unabh"angigen 
    Eigenvektor zu diesem Eigenwert.
\end{beweis}

Aus \ref{SatzPaarweiseVerschieden} und \ref{SatzMinimalDimEins} erh"alt
man:
\begin{korollar}
\label{SatzMinimalKomplex}
    Falls die Eigenwerte nicht paarweise verschieden sind und das
    Minimalpolynom mit dem charakteristischen Polynom "ubereinstimmt,
    zerf"allt es im K"orper der reellen Zahlen nicht in Linearfaktoren.
\end{korollar}

Wie bereits mehrfach erw"ahnt, erfolgt nun die Anwendung der Ergebnisse 
dieses Unterkapitels auf die Methode von Krylov.

% $$$ Verbesserung der Benutzung der Methode von Krylov:
% (\det(A) gesucht); bestimme Matrix B, so da"s $\det(B)$ bekannt oder
% leicht zu berechnen und f"ur $A*B$ gilt:
%     die Dimension jedes Eigenraumes ist gleich 1; 
%     (es gen"ugt: die Eigenwerte sind paarweise verschieden)
%     w"unschenswert $\det(B)$ ist 'Einheit' im zugrundeliegenden Ring;
% berechne \det(AB); dann eine zus"atzliche Division:
%  \det(A) = \det(AB) / \det(B)

% **************************************************************************

\MySection{Die Methode von Krylov}
\label{SecKrylov}
\index{Krylov!Methode von}

In diesem Unterkapitel wird die Methode von Krylov zur Bestimmung der
Koeffizienten des Minimalpolynoms einer Matrix beschrieben
(siehe z. B. \cite{Zurm64} ab S. 171 oder \cite{Hous64} ab S. 149;
Originalver"offentlichung \cite{Kryl31} ). Wie in Unterkapitel 
\ref{SecMinimalpolynom} beschrieben wird, ist das Minimalpolynom
unter bestimmten Bedingungen mit dem charakteristischen Polynom identisch.
Da sich unter den Koeffizienten des charakteristischen Polynoms auch die
Determinante der zugrunde liegenden Matrix befindet 
(vgl. \ref{SatzDdurchP}), ist es m"oglich, Krylovs Methode zur 
Determinantenberechnung zu verwenden, was im Algorithmus von Pan 
ausgenutzt wird.

Sei $A$ die $n \times n$-Matrix, deren Minimalpolynom zu
berechnen ist. 
Sei $z_0$ ein geeigneter Vektor der L"ange $n$. Wie $z_0$ beschaffen ist, 
wird noch behandelt. Sei $i \in \Nat$ gegeben. Die 
Vektoren \[ z_1,\, \ldots,\, z_i \] erh"alt man durch
\begin{eqnarray}
    z_1 & := & A z_0 \nonumber \\
    z_2 & := & A z_1 = A^2 z_0 \nonumber \\
    \vdots \nonumber \\
    z_i & := & A z_{i-1} = A^i z_0 \label{EquDefZI} \MyPunkt
\end{eqnarray}
Die Vektoren \[ z_0,\, \ldots ,\, z_i \] werden als {\em iterierte Vektoren}
bezeichnet.
Betrachtet man die iterierten Vektoren als Spaltenvektoren einer Matrix,
erh"alt man eine sogenannte {\em Krylov}-Matrix:
\[ K(A,z_0,i):= [ z_0, z_1, z_2, \ldots, z_{i-1} ] \MyPunkt \]
Zwischen den iterierten Vektoren besteht eine lineare Abh"angigkeit
besonderer Form, die von Krylov \cite{Kryl31} f"ur das hier zu
beschreibende Verfahren entdeckt wurde.

Das Minimalpolynom von $A$ wird mit $m(\lambda)$ bezeichnet.
Es gilt also
\Beq{EquKrylovPolynom}
    m(A) = 0_{n,n} \MyPunkt
\Eeq Das Polynom habe den Grad $j$.
Seien $c_0,\, \ldots\, c_{j-1}$ die Koeffizienten des Polynoms. 
Definiert man
\[
   c := \left[
        \begin{array}{c} c_0 \\ c_1 \\ \vdots \\ c_{j-1} \end{array}
        \right]
\]
dann ergibt sich
\begin{eqnarray} % $$$$ Formatierung gepr"uft ?
    m(A) & = & 0_{n,n} \nonumber
\\ \Leftrightarrow \hspace{1.2mm} \nonumber
    A^j + c_{j-1} A^{j-1} + \ldots + c_1 A + c_0 E_n & = & 0_{n,n}
\\ \Leftrightarrow \hspace{9.1mm} \nonumber
    c_{j-1} A^{j-1} + \ldots + c_1 A + c_0 E_n & = & - A^j
\\ \Leftrightarrow \nonumber
    c_0 E_n z_0 + c_1 A z_0 + \ldots + c_{j-1} A^{j-1} z_0 &
                                                         = & - A^j z_0
\\ \Leftrightarrow \hspace{3.6cm} \label{EquKrylovEqu}
    K(A,z_0,j) c & = & - A^j z_0
\end{eqnarray}
Gleichung \equref{EquKrylovEqu} kann man als lineares Gleichungssystem
in Matrizenschreibweise betrachten. Multipliziert man die rechte Seite
dieser Gleichung aus, erh"alt man einen Vektor der L"ange $n$.
Nach \ref{SatzRangGleich} ist das entsprechende Gleichungssystem
genau dann l"osbar, wenn
\[ \rg(K(A,z_0,j) = \rg([K(A,z_0,j),\, A^j z_0]) \MyPunkt \]
Wir suchen eine eindeutige L"osung des Gleichungssystems und erhalten diese
durch Verwendung von \ref{SatzGenauEine}, wonach der bis hierhin
nicht n"aher beschriebene Vektor $z_0$ so gew"ahlt werden mu"s,
da"s die Spaltenvektoren von $K(A,z_0,j)$ linear unabh"angig und
die Spaltenvektoren von $[K(A,z_0,j),\, A^j z_0]$ linear abh"angig sind.

An dieser Stelle wird deutlich, da"s $m(\lambda)$ das Polynom mit dem
kleinsten Grad sein mu"s, das \equref{EquKrylovPolynom} erf"ullt, damit
\equref{EquKrylovEqu} eine eindeutige L"osung besitzt. Falls ein
Polynom $m_1(\lambda)$ existiert, da"s \equref{EquKrylovPolynom} erf"ullt
und dessen Grad kleiner ist als der Grad von $m(\lambda)$, dann ist die
lineare Abh"angigkeit unabh"angig von der Wahl von $z_0$ bereits f"ur
weniger als $j$ iterierte Vektoren gegeben.

Da nach \equref{EquKrylovEqu} die ersten $j+1$ iterierten Vektoren linear
abh"angig sind, bleibt zu zeigen, da"s die ersten $j$ dieser Vektoren linear
unabh"angig sind. Dazu betrachten wir das $s \in \Nat$ mit der Eigenschaft,
da"s $s$ paarweise verschiedene Eigenvektoren von $A$ immer linear 
unabh"angig und $s+1$ von ihnen immer linear abh"angig sind. Aus den 
Grundlagen "uber Eigenvektoren und lineare Unabh"angigkeit geht hervor, 
da"s ein solches $s$ gibt.

Seien somit
\[ x_1,\, \ldots ,\, x_s \] linear 
unabh"angige Eigenvektoren von $A$.
Sei der iterierte Vektor $z_0$ darstellbar als Linearkombination einer 
maximalen Anzahl (also $s$) von Eigenvektoren von $A$. 
Demnach gilt f"ur geeignete 
\[ d_1,\, \ldots,\, d_s \MyKomma \] ungleich Null\footnote{Eine 
ausf"uhrliche Diskussion der Eigenschaften der 
iterierten Vektoren, insbesondere ihrer Beziehung zu den Eigenvektoren, 
befindet sich in \cite{Bode59} (Teil 2, Kapitel 2).}:
\Beq{EquDefZNull}
    z_0 = d_1 x_1 + \cdots + d_s x_s \MyPunkt
\Eeq 
Eine lineare Abh"angigkeit zwischen den ersten $j$ iterierten Vektoren
hat die Form
\Beq{EquIteriertLinAbh}
    h(z_0):= e_0 z_0 + e_1 z_1 + \cdots + e_{j-1} z_{j-1} = 0_n \MyKomma
\Eeq
wobei nicht alle $e_i$ gleich Null sind. Mit Hilfe der
Gleichungen \equref{EquDefZI} und \equref{EquDefZNull} in Verbindung mit
den Eigenwertgleichungen\footnote{vgl. Gleichung \equref{EquEigenMotiv}}
\[ A x_i = \lambda_i x_i \]
erh"alt man:
\begin{eqnarray*} % $$$$ Formatierung gepr"uft ?
    z_0 & = & \left. d_1 x_1 + \cdots + d_s x_s 
                        \: \hspace{14.2mm} \right| * e_0 \\
    z_1 & = & 
        \left. \lambda_1 d_1 x_1 + \cdots + 
                   \lambda_s d_s x_s \: \hspace{7.1mm} \right| * e_1 \\
    z_2 & = & 
         \left. \lambda_1^2 d_1 x_1 + \cdots + \lambda_s^2 d_s x_s 
              \: \hspace{7mm} \right| * e_2 \\
    & \vdots \\
    z_{j-1} & = &
         \left. \lambda_1^{j-1} d_1 x_1 
                       + \cdots + \lambda_s^{j-1} d_s x_s \: \right| * 1
\end{eqnarray*}
Diese Gleichungen f"ur die $z_i$ werden mit den am rechten Rand angegebenen 
Werten (vgl. \equref{EquIteriertLinAbh}) multipliziert und anschlie"send 
addiert. Definiert man
\[
   g(\lambda):= e_0 + e_1 \lambda + \cdots + e_{j-1} \lambda^{j-1} \MyKomma
\]
so lautet das Ergebnis in Verbindung mit \equref{EquIteriertLinAbh} :
\[
    h(z_0)= g(\lambda_1) d_1 x_1 + \cdots + g(\lambda_s) d_s x_s
          = 0_n \MyPunkt
\]
Da die $x_k$ nach Voraussetzung linear unabh"angig sind, mu"s also gelten
\[ \forall 1 \leq k \leq s :\: g(\lambda_k)d_k = 0 \MyPunkt \]
Da wiederum nach Voraussetzung $z_0$ eine Linearkombination aller $s$
linear unabh"angigen Eigenvektoren $x_i$ (s. o.) ist, folgt
\Beq{Equ2BewLinUnabh}
    \forall 1 \leq i \leq s :\: g(\lambda_i)= 0 \MyPunkt
\Eeq
Da die Dimension jedes Eigenraumes mindestens $1$ betr"agt, folgt mit Hilfe
von \ref{SatzMinimalNullGenauDann}, da"s
die maximale Anzahl linear unabh"angiger Eigenvektoren $s$ mindestens
so gro"s ist wie der Grad des Minimalpolynoms $j$.

Da $g(\lambda)$ als Polynom vom Grad $j-1$
nur maximal $j-1$ Nullstellen besitzen kann, folgt aus 
\equref{Equ2BewLinUnabh}
\[ e_0 = e_1 = \cdots = e_{j-1} = 0 \MyKomma \]
Daraus wiederum folgt mit \equref{EquIteriertLinAbh},
da"s die iterierten Vektoren linear unabh"angig sind.

Der bis hierhin gef"uhrte Beweis der linearen Unabh"angigkeit
iterierter Vektoren verwendet die maximale Anzahl $s$ linear unabh"angiger
Eigenvektoren. Betrachten wir deshalb diesen Wert $s$ genauer.
Es gibt zwei F"alle:
\begin{itemize}
\item
      Minimalpolynom und charakteristisches Polynom sind identisch.
      Satz \ref{SatzMinimalDimEins} f"uhrt zu zwei Unterf"allen:
      \begin{itemize}
      \item
            Die Eigenwerte sind paarweise verschieden. In diesem Fall
            ist $s$ gleich dem Grad $n$ des charakteristischen Polynoms
            und somit gleich $j$.
      \item
            Die Eigenwerte sind nicht paarweise verschieden. Nach
            \ref{SatzMinimalDimEins} ist $s$ gleich der Anzahl verschiedener
            Eigenwerte und damit in $\Rationals$ kleiner als $n$ und 
            mindestens $1$. In diesem Fall l"a"st sich die lineare
            Unabh"angigkeit nur f"ur weniger als $j$ iterierte Vektoren
            beweisen und die Methode von Krylov ist zur Bestimmung der
            Koeffizienten des Minimalpolynoms nicht anwendbar.
      \end{itemize}
\item 
      Minimalpolynom und charakteristisches Polynom sind nicht identisch.
      Es gibt wiederum zwei Unterf"alle:
      \begin{itemize}
      \item 
            Die direkte Summe der Eigenr"aume ergibt den gesamten 
            Vektorraum. In diesem Fall ist $s = n >j$.
      \item
            Die direkte Summe der Eigenr"aume ergibt nicht den gesamten
            Vektorraum. In diesem Fall ist $s \geq j$.
      \end{itemize}
\end{itemize}
W"ahlt man also $z_0$ als Linearkombination einer maximalen Anzahl linear
unabh"angiger Eigenvektoren, so sind die ersten $j$ iterierten Vektoren
linear unabh"angig, es sei denn, Minimalpolynom und charakteristisches
Polynom sind identisch und die Eigenwerte nicht paarweise verschieden.

Das nun noch verbliebene Problem ist die Wahl von $z_0$ f"ur eine gegebene 
Matrix $A$, da im Normalfall die Eigenvektoren nicht bekannt sind. Diese
Schwierigkeit
kann dadurch "uberwunden werden, da"s man die Methode von Krylov mit 
verschiedenen Vektoren $z_0$ auf die Matrix $A$ anwendet und dabei die
Vektoren $z_0$ so ausw"ahlt, da"s mindestens einer unter ihnen eine 
Linearkombination aller Eigenvektoren $x_1$ bis $x_s$ ist.

W"ahlt man eine Basis des zugrunde liegenden Vektorraumes, bestehend aus $n$
Vektoren, sowie einen
weiteren Vektor so aus, da"s je $n$ dieser $n+1$ Vektoren linear unabh"angig
sind und der $n+1$-te jeweils eine Linearkombination aller $n$ anderen ist,
so besitzt mindestens einer dieser Vektoren die geforderte Eigenschaft.
Dies erkennt man durch folgende "Uberlegung:
Stellt man die $n$ Vektoren als Linearkombinationen der Eigenvektoren dar,
so wird jeder Eigenvektor mindestens einmal ben"otigt. Da der $n+1$-te
Vektor eine Linearkombination der anderen $n$ ist, ist er also auch eine
Linearkombination aller beteiligen Eigenvektoren. Ein Beispiel f"ur 
$n+1$ Vektoren, die offensichtlich diese Eigenschaft besitzen, sind die
Vektoren der kanonische Basis und deren Summe:
\[ 
   \left[
       \begin{array}{c}
           1 \\ 0 \\ 0 \\ \vdots \\ 0
       \end{array}
   \right]
   \left[
       \begin{array}{c}
           0 \\ 1 \\ 0 \\ \vdots \\ 0
       \end{array}
   \right] \: \ldots  \:
   \left[
       \begin{array}{c}
           0 \\ 0 \\ \vdots \\ 0 \\ 1
       \end{array}
   \right] \: 
   \left[
       \begin{array}{c}
           1 \\ 1 \\ 1 \\ \vdots \\ 1
       \end{array}
   \right]  
\]

Uns interessiert die Berechnung der Determinante mit Hilfe der 
Methode von Krylov. Zusammengefa"st sieht das Vorgehen dazu folgenderma"sen
aus:
\begin{itemize}
\item 
      Zun"achst werden auf die soeben beschriebene Weise $n+1$ f"ur den 
      iterierten Vektor $z_0$ bestimmt. In der praktischen Anwendung 
      beschr"ankt man
      sich in der Regel auf einen Vektor, da die Anzahl der F"alle, in 
      denen dieser Vektor nicht ausreicht, so gering ist, da"s diese F"alle
      vernachl"assigt werden k"onnen.

      Die folgenden Schritte werden mit jedem Vektor $z_0$ parallel 
      durchgef"uhrt. 
      
      Falls mehrere der parallelen Zweige ein Ergebnis 
      liefern, m"ussen diese Ergebisse gleich sein. Falls sie nicht 
      gleich sind, wurde die Rechnung nicht korrekt durchgef"uhrt.

      Falls keiner der Zweige ein Ergebnis liefert, sind entweder die 
      Eigenwerte der zugrunde liegenden Matrix nicht paarweise verschieden,
      oder die Matrix ist nicht invertierbar. Diese beiden Unterf"alle 
      k"onnen mit dem hier dargestellten Algorithmus nicht unterschieden 
      werden.
\item 
      Es werden die iterierten Vektoren von $z_1$ bis $z_n$ berechnet.
\item
      Das Gleichungssystem \equref{EquKrylovEqu} wird dadurch gel"ost, 
      da"s die aus den iterierten Vektoren bestehende Krylov-Matrix
      inveritert wird. Ist die Krylov-Matrix nicht invertierbar, so ist 
      der Berechnungsversuch aus den bereits beschriebenen Gr"unden
      ein Fehlschlag und wird abgebrochen.
\item
      Die Koeffizienten des charakteristischen Polynoms, und somit auch
      die Determinante, werden dadurch berechnet, da"s die Inverse
      der Krylov-Matrix mit dem iterierten Vektor $z_n$ multipliziert wird
      (vgl. \equref{EquKrylovEqu}).
\end{itemize}

% mehr iterierte Vektoren, als der Grad des Minimalpolynoms angibt sind
% immer linear abh"angig!!!! (--> L"osbarkeit linearer Gleichungssysteme);

% **************************************************************************

\MySection{Vektor- und Matrixnormen}
\label{SecNorm}

Die Darstellungen der Wahl einer N"aherungsinversen in Unterkapitel 
\ref{SecGuessInverse} und der iterativen Matrizeninvertierung 
in Unterkapitel \ref{SecNewton} benutzen Normen von Matrizen.
Deshalb werden im vorliegenden Unterkapitel die Normen eingef"uhrt, die
dort zur Beschreibung erforderlich sind. 
Literatur dazu ist z. B. \cite{GL83} ab S. 12 
oder \cite{Isaa73} ab S. 3.

Umgangssprachlich formuliert, stellt der Begriff der {\em Norm} eine 
Verallgemeinerung des Begriffs der {\em L"ange} dar. Um zu Matrixnormen 
zu gelangen, kl"aren wir zun"achst, was eine Vektornorm ist:
\MyBeginDef
\index{Norm!eines Vektors}
\label{DefVektorNorm}
    Eine Funktion
    \[ f: \Rationals^n \rightarrow \Rationals \MyKomma \]
    die die Bedingungen
    \begin{enumerate} % $$$$ Formatierung geprueft ?
    \item
         $ \forall x \in \Rationals: f(x) \geq 0, 
            f(x) = 0 \Leftrightarrow x = 0_n 
         $
    \item
         $ \forall x,y \in \Rationals: f(x + y) \leq f(x) + f(y) $
    \item
         $ \forall a \in \Rationals, x \in \Rationals^n: f(ax)= |a| f(x) $
    \end{enumerate}
    erf"ullt, hei"st {\em Norm "uber $\Rationals^n$}. 
\MyEndDef

\MyBeginDef
\label{DefPNorm}
\index{H{\Myo}ldernorm} \index{p-Norm}
    Sei \[ p \in \Nat \] 
    fest gew"ahlt und \[ x \in \Rationals^n \] beliebig.
    \[ ||x||_p := (|x_1|^p + \ldots + |x_n|^p)^{1/p} \]  
    Die so definierte Funktion hei"st { \em $p$-Norm }.

    \[ ||x||_{\infty}:= \max_{1\leq i \leq n} |x_i| \]
    Diese Funktion wird mit {\em $\infty$-Norm} bezeichnet.

    Die $p$-Normen sowie die $\infty$-Norm 
    werden als { \em H"oldernormen } bezeichnet.
\MyEndDef

Mit der vorangegangenen Definition werden zwar einige Begriffe angegeben,
es ist jedoch nicht selbstverst"andlich, da"s es sich bei den Funktionen
auch wirklich um Normen handelt:
\begin{satz}
\label{SatzPNorm}
    Die in \ref{DefPNorm} definierten Funktionen 
    sind Normen "uber $\Rationals^n$.
\end{satz}
\begin{beweis}
    Die Funktionen erf"ullen die Bedingungen aus \ref{DefVektorNorm}.

    Der Beweis dieser Behauptung ist f"ur
    die $1$-Norm, $2$-Norm und $\infty$-Norm in
    \cite{Isaa73} ab S. 4 und
    f"ur die anderen Normen in \cite{Achi67} S. 4-7 angegeben.
    % (Isaa73 -> [30]) Achi67; BM b260/Achi
\end{beweis}

Den Begriff der Norm kann man auch auf Matrizen ausdehnen. F"ur uns 
gen"ugt die Betrachtung quadratischer Matrizen.
\MyBeginDef
\index{Norm!einer Matrix} \index{Matrixnorm}
\label{DefMatrixNorm}
    Eine Funktion 
    \[ f: \Rationals^{n^2} \rightarrow \Rationals \MyKomma \]
    die die Bedingungen
    \begin{eqnarray*}
        \forall A \in \Rationals^{n^2} & : & f(A) \geq 0,
             f(A) = 0 \leftrightarrow A=0_n \\
        \forall A,B \in \Rationals^{n^2} & : & f(A+B) \leq f(A) + f(B) \\
        \forall c \in \Rationals, A \in \Rationals^{n^2} & : &
             f(cA) = |c|f(A) \\
        \forall A,B \in \Rationals^{n^2} & : & f(AB) \leq f(A) * f(B) \\
    \end{eqnarray*}
    erf"ullt, hei"st {\em Matrixnorm "uber $\Rationals^{n^2}$ }.
\MyEndDef
Die vierte der obigen Bedingungen wird in der Literatur nicht immer f"ur 
Matrixnormen gefordert. In solchen F"allen wird unterschieden zwischen
Matrixnormen, die diese Bedingungen erf"ullen, und solchen, die diese
Bedingung nicht erf"ullen (vgl. \cite{Isaa73} S. 8 und \cite{GL83}
S. 14). F"ur uns sind diese Unterschiede nicht von Bedeutung.

Die von uns benutzten Matrixnormen sind folgenderma"sen definiert:
\MyBeginDef
\label{DefInduzierteNorm}
\index{Operatornorm} \index{Matrixnorm!induzierte}
\index{Matrixnorm!nat{\Myu}rliche} \index{Norm!nat{\Myu}rliche}
    Sei \[ A \in \Rationals^{n^2} \] Sei \[ x \in \Rationals^n \] Es gelte
    \[ ||x|| = 1 \] f"ur eine fest gew"ahlte Vektornorm.
    Die Funktion \[ ||A||:= ||Ax|| \] hei"st dann
    {\em durch die Vektornorm induzierte Matrixnorm }.
\MyEndDef
Sie ist in der Literatur auch noch unter den Namen {\em nat"urliche Norm}
und {\em Operatornorm} bekannt und wird h"aufig noch anders definiert
(vgl. \cite{Isaa73} S. 8). Das hat jedoch f"ur unsere Anwendungen keine
Bedeutung.

\begin{satz}
\label{SatzInduzierteNorm}
    Die in \ref{DefInduzierteNorm} definierte Funktion ist eine
    Matrixnorm.
\end{satz}
\begin{beweis}
    Die Funktion erf"ullt die Bedingungen in \ref{DefMatrixNorm}
    (\cite{Isaa73} ab S. 8 ).
\end{beweis}

Es folgen Beispiele f"ur induzierte
Matrixnormen, die im weiteren Text benutzt werden. Dazu ben"otigen wir
vorher noch einen weiteren Begriff:

Seien $\lambda_1, \, \ldots, \, \lambda_n$ die Eigenwerte von $A$. Dann
wird der Spektralradius \index{Spektralradius} $\rho(A)$ definiert als
\[ \rho(A) := \max\{ |\lambda_1|,\, \ldots,\, |\lambda_n| \} \MyPunkt \]

Durch
Indizes wird jeweils kenntlich gemacht, durch welche
Vektornorm die jeweilige Matrixnorm induziert wird\footnote{Bei der
Vielzahl der in diesem Unterkapitel auftauchenden Normen, sollte man
nicht vergessen, da"s $|x|$ f"ur einen Skalar $x$ einfach nur den
Absolutwert bezeichnet.}.
\begin{eqnarray}
    ||A||_1 & = & \max_j\sum_{k=1}^n |a_{k,j}| \label{EquMatNormEins} \\
    ||A||_2 & = & \sqrt{\rho(A * A)} \label{EquMatNormZwei} \\
    ||A||_\infty & = & \max_i\sum_{k=1}^n |a_{i,k}| 
                                              \label{EquMatNormInfty} \\
\end{eqnarray}
Die Beweise der Gleichungen \equref{EquMatNormEins},
\equref{EquMatNormZwei} und \equref{EquMatNormInfty} sind in
\cite{Isaa73} ab S. 9 zu finden.

Falls es nicht im Einzelfall anders festgelegt ist, gilt im weiteren 
Text $||\:||=||\:||_2$.

% $$$$ hier Satz "uber maximale Gr"o"se der Eigenwerte  ( <--> Normen )
%       (\det = \prod \lambda !!!)
%      
% ... sehr interessant, jedoch nicht 100-prozentig erforderlich
% ... zu finden in Isaa73 (S. 12)

% **************************************************************************

\MySection{Wahl einer N"aherungsinversen}
\label{SecGuessInverse}
\index{Inverse!einer Matrix} \index{Matrizeninvertierung}
\index{N{\Mya}herungsinverse}

In dem in Kapitel \ref{ChapPan} vorzustellenden Algorithmus wird die
Krylov-Matrix dadurch invertiert, da"s eine N"aherungsinverse berechnet
und diese dann iterativ verbessert wird. In diesem Unterkapitel
wird die Wahl der N"aherungsinversen beschrieben.
Literatur dazu sind \cite{PR85} und \cite{PR85a}.

F"ur die weiteren Betrachtungen ben"otigen wir:
\begin{eqnarray}
    t & := & \frac{1}{ ||A^TA||_1 } \label{EquPanDefT} \\
    B & := & t A^T \label{EquPanDefB} \\
    R(B) & := & E_n - B A \label{EquDefResidual} 
\end{eqnarray}

Im folgenden wird in mehreren Schritten eine Ungleichung f"ur
$||R(B)||$ bewiesen, aus der folgt, da"s das in Unterkapitel \ref{SecNewton}
beschriebene Verfahren effizient auf $B$ angewendet werden kann, um eine
Inverse von $A$ mit zufriedenstellender N"aherung zu erhalten.

\MyBeginDef
\label{DefSymmetrisch}
\index{symmetrisch} \index{Matrix!symmetrische}
    Gilt f"ur eine Matrix $A$ \[ A = A^T \MyKomma \] dann wird sie als
    {\em symmetrisch} bezeichnet.
\MyEndDef

Da die beiden folgenden Lemmata aus den Grundlagen "uber Normen stammen,
werden die Beweise auf einen Literaturverweis beschr"ankt. Sie sind
in \cite{Atki78} ab S. 416 zu finden.

\begin{lemma}
\label{SatzSymmetrischSpektral}
    F"ur jede symmetrische Matrix $A$ gilt:
    \[ ||A||_2 = \rho(A) \MyPunkt \]
\end{lemma}
            
\begin{lemma}
\label{SatzAtkinson}
\[
    ||A^T A||_2 = \rho(A^T A) = ||A||^2 \leq 
    ||A^T A||_1 \leq \max_i \sum_j |a_{i,j}| \max_j \sum_i |a_{i,j}| 
    \leq n ||A^T A||
\]
\end{lemma}

\begin{lemma}
\label{SatzInverseEigenwert}
    Sei $\lambda$ ein Eigenwert der invertierbaren $n \times n$-Matrix $A$. 
    Dann ist $1/\lambda$ ein Eigenwert von $A^{-1}$.
\end{lemma}
\begin{beweis}
    Da $A$ invertierbar ist, ist $\lambda \neq 0$. Sei $v$ ein Eigenvektor
    zu $\lambda$. Dann gilt $v \neq 0_n$, sowie
    \begin{eqnarray*}
        Av & \neq & 0_n \\
        Av & = & \lambda v \MyPunkt
    \end{eqnarray*}
    Damit ergibt sich:
    \[ A^{-1}(Av) = v = (1/\lambda) \lambda v = (1/\lambda) A v \MyKomma\]
    woraus die Behauptung folgt.
\end{beweis}

\begin{lemma}
\label{SatzEigenUngleichung}
    Sei $A$ invertierbar. Sei $\lambda$ ein Eigenwert von $A^T A$.
    Dann gilt:
    \Beq{EquEigenUngleichung}
        \frac{ 1 }{  ||A^{-1}||^2 }
             \leq \lambda \leq ||A||^2
             \MyPunkt
    \Eeq
\end{lemma}
\begin{beweis}
    Die rechte Ungleichung von \equref{EquEigenUngleichung} folgt aus
    \ref{SatzAtkinson}.
    
    Die linke Ungleichung von \equref{EquEigenUngleichung} folgt aus
    \ref{SatzInverseEigenwert} in Verbindung mit \ref{SatzAtkinson}.
\end{beweis}

\begin{satz}
\label{SatzResidualEigenwert}
    Seien $t$, $B$ und $R(B)$ entsprechend der Gleichungen
    \equref{EquPanDefB}, \equref{EquPanDefT} und \equref{EquDefResidual}
    definiert. Sei $\mu$ ein Eigenwert von $R(B)$. Dann
    gilt
    \[
        0 \leq \mu \leq
        1- \frac{ 1
                }{ ||A^T A||_1 ||A^{-1}||^2
                } \MyPunkt
    \]
\end{satz}
\begin{beweis}
    Sei $v$ Eigenvektor von $\mu$ (d. h. $v$ ist nicht der Nullvektor).
    Dann gilt:
    \[ R(B)v = \mu v \MyPunkt \]
    Mit Hilfe von \equref{EquDefResidual} und \equref{EquPanDefB}
    erh"alt man:
    \[ (E_n - t A^T A)v = v - tA^TA v = \mu v \MyPunkt \]
    Daraus folgt
    \[ A^T A v = \lambda v, \: \lambda = \frac{ 1-\mu }{ t } \]
    Also ist $\lambda$ ein Eigenwert von $A^T A$ und mit
    \ref{SatzEigenUngleichung} folgt
    \begin{eqnarray*}
        \frac{ 1 }{ ||A^{-1}||^2
             } \leq \lambda =
            \frac{ 1 - \mu }{ t } \leq ||A||^2 \\
        \Leftrightarrow
            1 - t||A||^2 \leq \mu \leq 1 - \frac{ t }{ ||A^{-1}||^2 }
    \end{eqnarray*}
    Mit Hilfe von Lemma \ref{SatzAtkinson} und \equref{EquPanDefT}
    ergibt sich die Behauptung.
\end{beweis}

Da die Matrix \[ R(B) = E_n - tA^T A \] symmetrisch ist, folgt aus
\ref{SatzSymmetrischSpektral} und \ref{SatzResidualEigenwert} eine
Ungleichung, die es erlaubt, die in \equref{EquPanDefB} definierte Matrix 
$B$ f"ur unsere Zwecke zu verwenden:
\begin{korollar}
\label{SatzNormNaheInvers}
    \[
        ||R(B)|| \leq
            1-\frac{ 1
                   }{ ||A^T A||_1 ||A^{-1}||^2
                   }
    \]
\end{korollar}
Die Benutzung dieser Folgerung wird in Unterkapitel \ref{SecNewton}
beschrieben.

% **************************************************************************

\MySection{Iterative Matrizeninvertierung}
\label{SecNewton}
\index{Iterationsverfahren} \index{Matrizeninvertierung} 
\index{Inverse!einer Matrix}
\index{Newton!Iterationsverfahren von}

In diesem Unterkapitel wird beschrieben, wie man eine gegebene 
N"aherungsinverse $B$ einer invertierbaren Matrix $A$ iterativ schrittweise
verbessern kann. Diese Methode ist in der Literatur als
{\em Newton-Verfahren} bekannt und wird in
\cite{PR85} und \cite{PR85a} sowie in \cite{Hous64} ab S. 64
behandelt. 

Um zu einem Iterationsverfahren zu gelangen, nehmen wir einige Umformungen
an einer Gleichung vor, in der das mit \equref{EquDefResidual} $R(B)$ 
vorkommt:
\begin{MyEqnArray}
                    \MT R(B)          \MT = \MT 0_{n,n} \MNl
    \Leftrightarrow \MT R(B) + E_n    \MT = \MT E_n \MNl
    \Leftrightarrow \MT (R(B) + E_n)B \MT = \MT B \MNl
    \Leftrightarrow \MT (2E_n- BA)B   \MT = \MT B
\end{MyEqnArray}
Man definiert mit Hilfe der letzten dieser Gleichungen die Iteration
\begin{eqnarray}
    B_0 & := & B \nonumber \\
    B_i & := & (2E_n - B_{i-1}A)B_{i-1} \label{EquPanDefNewton} \MyPunkt
\end{eqnarray}
Betrachtet man nun \equref{EquDefResidual} f"ur $B_i$ statt $B$, bekommt
man eine Aussage "uber die St"arke der Konvergenz der Iteration:
\begin{eqnarray*}
    R(B_i) & = & E_n - B_iA \\
           & = & E_n - (2E_n - B_{i-1}A)B_{i-1}A \\
           & = & E_n - 2B_{i-1}A + (B_{i-1}A)^2 \\
           & = & (R(B_{i-1}))^2
\end{eqnarray*}
Daraus folgt, da"s f"ur jede Matrixnorm gilt:
\[ ||R(B_i)|| = ||(R(B_{i-1}))^2|| \MyPunkt \]

Unter der Voraussetzung \Beq{EquResidualLessOne} ||R(B)|| < 1 \Eeq
konvergiert $||R(B_i)||$ also quadratisch gegen Null und die
$B_i$ entsprechend gegen $A^{-1}$.

Um das Iterationsverfahren einsetzen zu k"onnen, m"ussen wir noch 
feststellen, wieviele Iterationen erforderlich sind, um eine bestimmte
Genauigkeit zu erreichen. Dazu definieren wir als Abk"urzung
\[ q:= ||R(B)|| , \: q_i:= ||R(B_i)||, \: q_0 := q \MyPunkt \]
Damit das Verfahren "uberhaupt in praktisch nutzbarer Weise konvergiert,
darf $q$ nicht beliebig nahe bei $1$ liegen. Sei $c \in \Nat$ gegeben. 
Dann nehmen wir an, da"s f"ur $q$ gilt:
\Beq{EquSupposeQ} 
    q = 1 - \frac{1}{n^c} \MyPunkt 
\Eeq
Da das Verfahren quadratisch konvergiert, gilt f"ur $k \in \Nat_0$:
\[
    q_k = q^{2^k} \MyPunkt
\]
Mit \equref{EquSupposeQ} erhalten wir:
\[
    q_k = \lb 1 - \frac{1}{n^c} \rb^{2^k} \MyPunkt
\]
F"ur unsere Anwendung sei
\[ q_k < \frac{1}{d} , \: d \in \Nat \] ausreichend. Mit 
\[ 
    k := e \log(n), \: e \in \Rationals_{>0} 
\]  erhalten wir:
\[ 
    \lb 1 - \frac{1}{n^c} \rb^{n^e} = \frac{1}{d} \MyPunkt
\]
Da uns die Anzahl der Iterationen interessiert, l"osen wir diese Gleichung
nach $e$ auf und nehmen dazu $n \rightarrow \infty$ an:
\begin{MyEqnArray}
    \MT \lb 1 - \frac{1}{n^c} \rb^{n^e} \MT = \MT \frac{1}{d} \MNl
    \Rightarrow \MT
    \lb \lb 1 - \frac{1}{n^c} \rb^{n^c} \rb^{n^e / n^c} \MT = \MT 
                                                       \frac{1}{d} \MNl
    \Rightarrow \MT
    \MathE^{- (n^e / n^c) } \MT \approx \MT \frac{1}{d} \MNl
    \Rightarrow \MT
    \ln(d) \MT \approx \MT n^{e-c} \MNl
    \Rightarrow \MT
    \frac{ \ln(\ln(d)) }{ \ln(n) } + c \MT \approx \MT e    
\end{MyEqnArray}
Wir k"onnen also f"ur wachsende $n$ annehmen, da"s gilt
\[ c \approx e \MyPunkt \]
Um die --- bei gegebenem $n$ --- von $e$ bestimmte Anzahl der Iterationen
zu erhalten, m"ussen wir $c$ n"aher bestimmen. Vergleichen wir dazu
\equref{EquSupposeQ} mit der Aussage von \ref{SatzNormNaheInvers}.
Aus Grundlagen "uber Normen (z. B. \cite{Isaa73} ab S. 12) erhalten wir
f"ur eine beliebige Matrix $W$:
\[
    ||W||_2 \leq \sqrt{||W^2||_1} \MyPunkt
\]
Wir k"onnen also die Berechnung von $c$ auf die Bestimmung der 1-Norm, also
der maximalen Betragsspaltensumme, einer Matrix zur"uckf"uhren und erkennen,
da"s $c$ von der Gr"o"se der Matrizenelemente abh"angt.

Somit erkennen wir eine weitere Einschr"ankung f"ur die Anwendbarkeit des
Algorithmus. Die Matrix mu"s die Bedingung
\Beq{EquWellConditioned}
    ||A^T A||_1 ||A^{-1}||^2 \leq n^c
\Eeq f"ur ein $c \in \Nat$ erf"ullen. Die Wahl von $c$ h"angt davon ab, 
auf welche Matrizen man
das Verfahren anwendet. Um P-Alg. mit den anderen Algorithmen 
vergleichen zu k"onnen, nehmen wir im weiteren Verlauf des Textes $c=1$ an.

% **************************************************************************

\MySection{Determinatenber. mit Hilfe der Methoden von Krylov und Newton}
\label{SecAlgPan}

In diesem Unterkapitel werden die in den vorangegangenen Abschnitten
dargestellten Methoden zu einem Algorithmus zur Determinantenberechnung
zusammengefa"st, auf den mit {\em P-Alg.} Bezug genommen wird\footnote{
vgl. Unterkapitel \ref{SecBez}}. Weiterhin wird die Anzahl der Schritte und
der Prozessoren analysiert. 

Im folgenden ist mit $n^x$ jeweils $\lc n^x \rc$ gemeint.

Im folgenden wird mit Hinweis auf die Bemerkungen auf S.
\pageref{PageAlg2MatMult} f"ur die Multplikation zweier 
$n \times n$-Matrizen ein Aufwand von \[ \gamma_S (\lc\log(n)\rc + 1) \]
Schritten und \[ \gamma_P n^{2+\gamma} \] Prozessoren in Rechnung gestellt.

Es sei die Determinante der $n \times n$-Matrix $A$ zu berechnen.
Zuerst wird die Krylov-Matrix entsprechend der Darstellung in Abschnitt
\ref{SecKrylov} berechnet. Es ist "ublich, f"ur den dort erw"ahnten
Vektor $z$ zur Berechnung der iterierten Vektoren den Einheitsvektor
zu verwenden, dessen s"amtliche Elemente gleich $1$ sind. Aufgrund der
aus theoretischer Sicht ohnehin bereits eingeschr"ankten Verwendbarkeit
der Methode von Krylov f"ur unsere Zwecke, bedeutet die Beschr"ankung auf
diesen Vektor nur eine unbedeutende Verschlechterung des Algorithmus.

Zuerst sind die iterierten Vektoren anhand des Vektors $z$ und der Matrix
$A$ zu berechnen. Dies geschieht anhand der nachstehenden Gleichungen
(\cite{Pan85}, \cite{BM75} S. 128, \cite{Kell82}). Auf den rechten Seiten 
werden nur Ergebnisse vorangegangener Gleichungen oder $z$ bzw. $A$ 
verwendet. Auf den linken Seiten stehen neu berechnete Ergebnisse. Die Terme 
auf den rechten Seiten beschreiben Matrizenmultiplikationen und die linken 
Seiten deren Ergebnisse.
\begin{eqnarray*}
    [ A^3v,\, A^2v ] & = & A^2[ Av,\,v ] \\ \relax
    [ A^7v,\, A^6v,\, A^5v,\, A^4v ] & = & 
                                      A^4[ A^3v,\, A^2v,\, Av,\, v ] 
                                       \\ \relax
    & \vdots & \\ \relax
    [ A^{2*2^h}v,\, \ldots,\, A^{2^h}v ] & = & 
                              A^{2^h}[ A^{2^h-1}v,\, \ldots,\,v ] 
\end{eqnarray*}
Um die jeweils n"achste Gleichung mit Hilfe der schon bekannten Ergebnisse
aufzustellen, sind zwei Matrizenmultiplikationen erforderlich. Mit der
ersten wird die n"achste ben"otigte Potenz von $A$ berechnet. Mit der 
zweiten wird der jeweilige Term der rechten Seite der Gleichung ausgewertet.
Die $n$ gesuchten iterierten Vektoren k"onnen auf diese Weise in
\[ 
    \gamma_S (\lc\log(n)\rc + 1) \lc\log(n)\rc
\] 
Schritten von
\[ 
    \gamma_P n^{2+\gamma}
\] 
Prozessoren berechnet werden.

Anschlie"send ist f"ur die aus iterierten Vektoren bestehende Krylov-Matrix
eine N"aherungsinverse entsprechend der Ausf"uhrungen in Unterkapitel
\ref{SecGuessInverse} zu berechnen. Dazu werden die Gleichungen
\equref{EquPanDefB} und \equref{EquPanDefT} verwendet. Betrachtet man 
diese Gleichungen, erkennt man, da"s zur Berechnung der dortigen Matrix $B$
aus der Matrix $A$ eine Matrizenmultiplikation, eine Berechnung der
$1$-Norm einer Matrix, eine Division durch einen Skalar sowie eine 
Matrix--Skalar--Multiplikation erforderlich ist. An dieser Stelle wird
deutlich, da"s der Algorithmus nicht ohne Divisionen auskommt.

Um die $1$-Norm zu erhalten, sind $n$ Summen von je $n$-Matrizenelementen
zu berechnen und miteinander zu vergleichen. Dies kann mit Hilfe der 
Bin"arbaummethode nach Satz \ref{SatzAlgBinaerbaum} in
\[
    2\lc\log(n)\rc
\] Schritten von
\[ 
    n \lf \frac{n}{2} \rf
\] Prozessoren durchgef"uhrt werden.

Insgesamt kann die Berechnung der N"aherungsinversen also in
\[
    \gamma_S (\lc\log(n)\rc+1) + 2\lc\log(n)\rc + 2
\] Schritten von
\[
    \gamma_P n^{2+\gamma}
\] Prozessoren geleistet werden.

Nachdem sie zur Verf"ugung steht, kann sie mit Hilfe des in Unterkapitel
\ref{SecNewton} beschriebenen Verfahrens verbessert werden.

Ein Iterationsschritt mit Hilfe von Gleichung \equref{EquPanDefNewton}
erfordert zwei Multiplikationen und eine Addtition von Matrizen. Setzt man
f"ur die Anzahl der durchzuf"uhrenden Iterationen zun"achst die Unbestimmte
$I$ ein, kann das Iterationsverfahren in
\[
    I * 2 \gamma_S (\lc\log(n)\rc + 1) + 1
\] Schritten von
\[
    \gamma_P n^{2+\gamma}
\] Prozessoren auf die N"aherungsinverse angewendet werden um diese
bis auf eine ausreichende Genauigkeit an die gesuchte Inverse anzun"ahern.

Schlie"slich ist noch eine Matrix--Vektor--Multiplikation durchzuf"uhren,
um die Methode von Krylov zur Berechnung der Koeffizienten des
charakteristischen Polynoms zu vollenden. Betrachtet man den Vektor wiederum
als Matrix, kann dies in
\[
    \gamma_S(\lc\log(n)\rc+1)
\] Schritten von
\[
    \gamma_P n^{2+\gamma}
\] Prozessoren erledigt werden. Nach \ref{SatzDdurchP} hat man 
mit den Koeffizienten auch die Determinante berechnet.

Hier wird eine Einschr"ankung f"ur den Algorithmus deutlich. Da ein 
N"aherungsverfahren verwendet wird, ist es nicht m"oglich, die Determinante
f"ur Matrizen mit Elementen aus $\Rationals$ genau zu berechnen. Diese
m"ussen deshalb aus $\Integers$ stammen, denn in diesem Fall sind
die Koeffizienten des charakteristischen Polynoms ebenfalls ganzzahlig und
die Determinante kann bei gen"ugend genau durchgef"uhrtem Newton-Verfahren
durch Rundung genau ermittelt werden.

Betrachtet man die Einschr"ankungen f"ur die Verwendbarkeit der Methode
von Krylov (vgl. Unterkaptel \ref{SecKrylov}) sowie die Bedingung
\equref{EquWellConditioned} auf S. \pageref{EquWellConditioned}, erkennt
man, da"s insgesamt nicht unerhebliche Anforderungen an die Matrix,
deren Determianten zu berechnen ist, gestellt werden m"ussen, damit der
in diesem Kapitel dargestellte Algorithmus anwendbar ist.

Als Gesamtaufwand f"ur den Algorithmus erh"alt man f"ur die
Anzahl der Schritte:
\begin{eqnarray*}
   & & \gamma_S (\lc\log(n)\rc + 1) \lc\log(n)\rc \\
   & + & \gamma_S (\lc\log(n)\rc +1 ) + 2\lc\log(n)\rc + 2 \\
   & + & I * 2 \gamma_S (\lc\log(n)\rc + 1) + 1 \\
   & + & \gamma_S(\lc\log(n)\rc+1) \\
   & = &
      \gamma_S \lb \lc\log(n)\rc^2 + 5 \lc\log(n)\rc
    + I \lb 2 \lc\log(n)\rc + 2 \rb + \frac{3}{\gamma_S} + 2 \rb
\end{eqnarray*}
Mit Verweis auf Unterkapitel \ref{SecNewton} nehmen wir $I= \log(n)$ an.
Dadurch lautet der Term f"ur die Anzahl der Schritte:
\[
    \gamma_S \lb 3 \lc\log(n)\rc^2 + 7 \lc\log(n)\rc
    + \frac{3}{\gamma_S} + 2 \rb \MyPunkt
\]
Vergleicht man die Annahme f"ur $I$ mit den Ausf"uhrungen in Unterkapitel
\ref{SecNewton}, erkennt man, da"s dies der beste mit dem Algorithmus zu
erreichende Wert ist. Bei schlechteren Randbedingungen erh"alt man f"ur
die Schritte einen entsprechend schlechteren Wert.

F"ur die Prozessoren ergibt sich:
\[ 
    \gamma_P n^{2+\gamma} \MyPunkt
\]
Da die Matrizenmultiplikation nach \cite{CW90} nur f"ur sehr gro"se $n$
Vorteile bringt, wird \[ \gamma_P = \gamma_S = \gamma = 1 \] gesetzt,
um eine f"ur die Anwendung des Algorithmus realistische 
Vergleichsm"oglichkeit mit den anderen Algorithmen zu bekommen.

Vergleicht man P-Alg. mit C-Alg., BGH-Alg. und B-Alg., so erkennt man, da"s
die Anzahl der Prozessoren in P-Alg. gleich ist mit der Anzahl der
Prozessoren f"ur die Matrizenmultiplikation ist und somit um eine Potenz
geringer als beim besten der drei anderen Algorithmen.

Ein gravierender Nachteil von P-Alg. sind die Einschr"ankungen
f"ur die Benutzbarkeit. Die Bedingung, da"s die Matrizenelemente ganzzahlig
sein m"ussen, l"a"st sich durch Ausnutzung der Eigenschaften einer 
Determinaten (siehe Definition \ref{DefDet}) ausgleichen. Dazu werden
die Matrizenelemente durch einen geeigneten Faktor multipliziert, so da"s
die resultierende Matrix nur ganzzahlige Elemente enth"alt. Nachdem der
Algorithmus auf diese Matrix angewendet worden ist, kann man dann aus
dem Ergebnis auf die Determinate der urspr"unglichen Matrix schlie"sen.

Durch die in Unterkapitel \ref{SecKrylov} beschriebenen Einschr"ankungen,
kann f"ur P-Alg. nicht garantiert werden, da"s er f"ur jede invertierbare
Matrix eine Determinante ungleich Null liefert. Die Wahrscheinlichkeit f"ur
einen solchen Fall ist zwar gering, jedoch unterscheidet diese 
Beschr"ankung P-Alg. von den anderen drei Algorithmen.

Weiterhin mu"s man bei P-Alg. nat"urlich wiederum auf die Existenz von 
Divisionen und das Fehlen von Fallunterscheidungen hinweisen.

